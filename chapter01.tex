% -*- coding: utf-8 -*-
\documentclass[twoside,letterpaper,openright]{rapport3}

% -*- coding: utf-8 -*-

\usepackage[b5paper,text={5in,8in},centering]{geometry}

\usepackage[CJKchecksingle]{xeCJK}
\setmainfont[Mapping=tex-text]{TeX Gyre Schola}
%\setsansfont{URW Gothic L Book}
%\setmonofont{Nimbus Mono L}
\setCJKmainfont[BoldFont=FandolHei,ItalicFont=FandolKai]{FandolSong}
\setCJKsansfont{FandolHei}
\setCJKmonofont{FandolFang}
\xeCJKsetup{PunctStyle = kaiming}

\linespread{1.25}
\setlength{\parindent}{2em}

\usepackage{xcolor}
\definecolor{myblue}{rgb}{0,0.2,0.6}

\usepackage{titlesec}
\titleformat{\chapter}
    {\normalfont\Huge\sffamily\color{myblue}}
    {第\thechapter 章}
    {1em}
    {}
%\titlespacing{\chapter}{0pt}{50pt}{40pt}
\titleformat{\section}
    {\normalfont\Large\sffamily\color{myblue}}
    {\thesection}
    {1em}
    {}
%\titlespacing{\section}{0pt}{3.5ex plus 1ex minus .2ex}{2.3ex plus .2ex}
\titleformat{\subsection}
    {\normalfont\large\sffamily\color{myblue}}
    {\thesubsection}
    {1em}
    {}
%\titlespacing{\subsection}{0pt}{3.25ex plus 1ex minus .2ex}{1.5ex plus .2ex}
%
\newpagestyle{special}[\small\sffamily]{
  \headrule
  \sethead[\usepage][][\chaptertitle]
  {\chaptertitle}{}{\usepage}}
\newpagestyle{main}[\small\sffamily]{
  \headrule
  \sethead[\usepage][][第\thechapter 章\quad\chaptertitle]
  {\thesection\quad\sectiontitle}{}{\usepage}}

\usepackage{titletoc}
%\setcounter{tocdepth}{1}
%\titlecontents{标题层次}[左间距]{上间距和整体格式}{标题序号}{标题内容}{指引线和页码}[下间距]
\titlecontents{chapter}[1.5em]{\vspace{.5em}\bfseries\sffamily}{\color{myblue}\contentslabel{1.5em}}{}
    {\titlerule*[20pt]{$\cdot$}\contentspage}[]
\titlecontents{section}[4.5em]{\sffamily}{\color{myblue}\contentslabel{3em}}{}
    {\titlerule*[20pt]{$\cdot$}\contentspage}[]
%\titlecontents{subsection}[8.5em]{\sffamily}{\contentslabel{4em}}{}
%    {\titlerule*[20pt]{$\cdot$}\contentspage}

\usepackage{enumitem}
\setlist{topsep=2pt,itemsep=2pt,parsep=1pt,leftmargin=\parindent}

\usepackage{fancyvrb}
\DefineVerbatimEnvironment{verbatim}{Verbatim}
  {xleftmargin=2em,baselinestretch=1,formatcom=\color{teal}\upshape}

\usepackage{etoolbox}
\makeatletter
\preto{\FV@ListVSpace}{\topsep=2pt \partopsep=0pt }
\makeatother

\usepackage[colorlinks,plainpages,pagebackref]{hyperref}
\hypersetup{
   pdfstartview={FitH},
   citecolor=teal,
   linkcolor=myblue,
   urlcolor=black,
   bookmarksnumbered
}

\usepackage{comment,makeidx,multicol}

%\usepackage{german}
%% german
%\righthyphenmin=3
%\mdqoff
%\captionsenglish
\usepackage[english]{babel}
{\catcode`"=13 \gdef"#1{\ifx#1"\discretionary{}{}{}\fi\relax}}
\def\mdqon{\catcode`"=13\relax}
\def\mdqoff{\catcode`"=12\relax}
\makeindex
\hyphenation{ex-em-pli-fies}

\newdimen\tempdima \newdimen\tempdimb

% these are fine
\def\handbreak{\\ \message{^^JManual break!!!!^^J}}
\def\nl{\protect\\}\def\n#1{{\tt #1}}
\def\cs#1{\texorpdfstring{{\tt\char`\\#1}}{\textbackslash#1}} %\def\cs#1{{\tt\char`\\#1}}
\let\csc\cs
\def\lb{{\tt\char`\{}}\def\rb{{\tt\char`\}}}
\def\gr#1{\texorpdfstring{$\langle$#1$\rangle$}{<#1>}} %\def\gr#1{$\langle$#1$\rangle$}
\def\key#1{{\tt#1}}
\def\alt{}\def\altt{}%this way in manstijl
\def\ldash{\unskip\ --\nobreak\ \ignorespaces}
\def\rdash{\unskip\nobreak\ --\ \ignorespaces}
% check these
\def\hex{{\tt"}}
\def\ascii{{\sc ascii}}
\def\ebcdic{{\sc ebcdic}}
\def\IniTeX{Ini\TeX}\def\LamsTeX{LAMS\TeX}\def\VirTeX{Vir\TeX}
\def\AmsTeX{Ams\TeX}
\def\TeXbook{the \TeX\ book}\def\web{{\sc web}}
% needs major thinking
\newenvironment{myquote}{\list{}{%
    \topsep=2pt \partopsep=0pt%
    \leftmargin=\parindent \rightmargin=\parindent
    }\item[]}{\endlist}
\newenvironment{disp}{\begin{myquote}}{\end{myquote}}
\newenvironment{Disp}{\begin{myquote}}{\end{myquote}}
\newenvironment{tdisp}{\begin{myquote}}{\end{myquote}}
\newenvironment{example}{\begin{myquote}\noindent\itshape 例子:}{\end{myquote}}
\newenvironment{inventory}{\begin{description}\raggedright}{\end{description}}
\newenvironment{glossinventory}{\begin{description}}{\end{description}}
\def\gram#1{\gr{#1}}%???
%
% index
%
\def\indexterm#1{\emph{#1}\index{#1}}
\def\indextermsub#1#2{\emph{#1 #2}\index{#1!#2}}
\def\indextermbus#1#2{\emph{#1 #2}\index{#2!#1}}
\def\term#1\par{\index{#1}}
\def\howto#1\par{}
\def\cstoidx#1\par{\index{#1@\cs{#1}@}}
\def\thecstoidx#1\par{\index{#1@\protect\csname #1\endcsname}}
\def\thecstoidxsub#1#2{\index{#1, #2@\protect\csname #1\endcsname, #2}\ignorespaces}
\def\csterm#1\par{\cstoidx #1\par\cs{#1}}
\def\csidx#1{\cstoidx #1\par\cs{#1}}

\def\tmc{\tracingmacros=2 \tracingcommands\tracingmacros}

%%%%%%%%%%%%%%%%%%%
\makeatletter
\def\snugbox{\hbox\bgroup\setbox\z@\vbox\bgroup
    \leftskip\z@
    \bgroup\aftergroup\make@snug
    \let\next=}
\def\make@snug{\par\sn@gify\egroup \box\z@\egroup}
\def\sn@gify
   {\skip\z@=\lastskip \unskip
    \advance\skip\z@\lastskip \unskip
    \unpenalty
    \setbox\z@\lastbox
    \ifvoid\z@ \nointerlineskip \else {\sn@gify} \fi
    \hbox{\unhbox\z@}\nointerlineskip
    \vskip\skip\z@
    }

\newdimen\fbh \fbh=60pt % dimension for easy scaling:
\newdimen\fbw \fbw=60pt % height and width of character box

\newdimen\dh \newdimen\dw % height and width of current character box
\newdimen\lh % height of previous character box
\newdimen\lw \lw=.4pt % line weight, instead of default .4pt

\def\hdotfill{\noindent
    \leaders\hbox{\vrule width 1pt height\lw
                  \kern4pt
                  \vrule width.5pt height\lw}\hfill\hbox{}
    \par}
\def\hlinefill{\noindent
    \leaders\hbox{\vrule width 5.5pt height\lw         }\hfill\hbox{}
    \par}
\def\stippel{$\qquad\qquad\qquad\qquad$}
\makeatother
%%%%%%%%%%%%%%%%%%%

%\def\SansSerif{\Typeface:macHelvetica }
%\def\SerifFont{\Typeface:macTimes }
%\def\SansSerif{\Typeface:bsGillSans }
%\def\SerifFont{\Typeface:bsBaskerville }
\let\SansSerif\relax \def\italic{\it}
\let\SerifFont\relax \def\MainFont{\rm}
\let\SansSerif\relax
\let\SerifFont\relax
\let\PopIndentLevel\relax \let\PushIndentLevel\relax
\let\ToVerso\relax \let\ToRecto\relax

%\def\stop@command@suffix{stop}
%\let\PopListLevel\PopIndentLevel
%\let\FlushRight\relax
%\let\flushright\FlushRight
%\let\SetListIndent\LevelIndent
%\def\awp{\ifhmode\vadjust{\penalty-10000 }\else
%    \penalty-10000 \fi}
\let\awp\relax
\let\PopIndentLevel\relax \let\PopListLevel\relax

\showboxdepth=-1

%\input figs
\def\endofchapter{\vfill\noindent}


\begin{document}

%\chapter{The Structure of the \TeX\ Processor}
\chapter{The Structure of the \TeX\ Processor}

%This book treats the various aspects of \TeX\ in chapters
%that are concerned with relatively small, well-delineated,
%topics. In this chapter, therefore, 
%a global picture of the way \TeX\ operates will be given.
%Of necessity, many details will be omitted here, but all of
%these are treated in later chapters. On the other hand,
%the few examples given in this chapter will be repeated
%in the appropriate places later on; they are included here
%to make this chapter self-contained.
This book treats the various aspects of \TeX\ in chapters
that are concerned with relatively small, well-delineated,
topics. In this chapter, therefore, 
a global picture of the way \TeX\ operates will be given.
Of necessity, many details will be omitted here, but all of
these are treated in later chapters. On the other hand,
the few examples given in this chapter will be repeated
in the appropriate places later on; they are included here
to make this chapter self-contained.

%\section{Four \TeX\protect\ processors}
\section{Four \TeX\protect\ processors}

%The way \TeX\ processes its input can be viewed as
%happening on four levels. One might  say that
%the \TeX\ processor is split into four separate units,
%each one accepting the output of the previous stage, and
%delivering the input for the next stage. The input of
%the first stage is then the \n{.tex} input file; the output
%of the last stage is a \n{.dvi} file.
The way \TeX\ processes its input can be viewed as
happening on four levels. One might  say that
the \TeX\ processor is split into four separate units,
each one accepting the output of the previous stage, and
delivering the input for the next stage. The input of
the first stage is then the \n{.tex} input file; the output
of the last stage is a \n{.dvi} file.

%For many purposes it is most convenient, and most insightful,
%to consider these four levels of processing as happening
%after one another, each one accepting the {\em completed\/}
%output of the previous level. In reality this is not true:
%all levels are simultaneously
%active, and there is interaction between them.
For many purposes it is most convenient, and most insightful,
to consider these four levels of processing as happening
after one another, each one accepting the {\em completed\/}
output of the previous level. In reality this is not true:
all levels are simultaneously
active, and there is interaction between them.

%The four levels are (corresponding roughly
%to the `eyes', `mouth', `stomach', and `bowels' respectively
%in Knuth's original terminology) as follows.
%\begin{enumerate}\item
%The input processor. This is the piece of \TeX\ that
%accepts input lines from the file system of whatever computer
%\TeX\ runs on, and turns them into tokens.
%Tokens are the internal objects of \TeX:
%there are character tokens that constitute the typeset
%text, and control sequence tokens that are commands 
%to be processed by the next two levels.
%\item The expansion processor. 
%Some but not all of the tokens generated in the first level
%\ldash macros, conditionals, and a number
%of primitive \TeX\ commands \rdash  are subject to expansion.
%Expansion is the process that replaces some (sequences of)
%tokens by other (or no) tokens.
%\item The execution processor. 
%Control sequences that are not expandable are executable,
%and this execution takes place on the third level of the
%\TeX\ processor.
The four levels are (corresponding roughly
to the `eyes', `mouth', `stomach', and `bowels' respectively
in Knuth's original terminology) as follows.
\begin{enumerate}\item
The input processor. This is the piece of \TeX\ that
accepts input lines from the file system of whatever computer
\TeX\ runs on, and turns them into tokens.
Tokens are the internal objects of \TeX:
there are character tokens that constitute the typeset
text, and control sequence tokens that are commands 
to be processed by the next two levels.
\item The expansion processor. 
Some but not all of the tokens generated in the first level
\ldash macros, conditionals, and a number
of primitive \TeX\ commands \rdash  are subject to expansion.
Expansion is the process that replaces some (sequences of)
tokens by other (or no) tokens.
\item The execution processor. 
Control sequences that are not expandable are executable,
and this execution takes place on the third level of the
\TeX\ processor.

%One part of the activity here concerns changes to
%\TeX's internal state: assignments (including
%macro definitions) are typical activities in this
%category. The other major thing happening on this level
%is the construction of horizontal, vertical, and
%mathematical lists.
%\item The visual processor. 
%In the final level of processing
%the visual part of \TeX\ processing is performed. Here
%horizontal lists are broken into paragraphs, 
%vertical lists are broken into pages,
%and  formulas are built out of math lists. 
%Also the output to the \n{dvi} file takes place on this level.
%The algorithms working here are not accessible to the user,
%but they can be influenced by a number of parameters.
%\end{enumerate}
One part of the activity here concerns changes to
\TeX's internal state: assignments (including
macro definitions) are typical activities in this
category. The other major thing happening on this level
is the construction of horizontal, vertical, and
mathematical lists.
\item The visual processor. 
In the final level of processing
the visual part of \TeX\ processing is performed. Here
horizontal lists are broken into paragraphs, 
vertical lists are broken into pages,
and  formulas are built out of math lists. 
Also the output to the \n{dvi} file takes place on this level.
The algorithms working here are not accessible to the user,
but they can be influenced by a number of parameters.
\end{enumerate}

%\section{The input processor}
\section{The input processor}

%The input processor of \TeX\ is that part of \TeX\ that
%translates whatever characters it gets from the input file
%into tokens. The output of this processor is a stream
%of tokens: a token list. Most tokens fall into one of two categories:
%character tokens and control sequence tokens. 
%The remaining category is that of the parameter tokens;
%these will not be treated in this chapter.
The input processor of \TeX\ is that part of \TeX\ that
translates whatever characters it gets from the input file
into tokens. The output of this processor is a stream
of tokens: a token list. Most tokens fall into one of two categories:
character tokens and control sequence tokens. 
The remaining category is that of the parameter tokens;
these will not be treated in this chapter.

%\subsection{Character input}
\subsection{Character input}

%For simple input text, characters are made into
%character tokens. However, \TeX\ can ignore input characters:
%a row of spaces in the input is usually equivalent to just one
%space. Also, \TeX\ itself can insert tokens that do not correspond
%to any character in the input, for instance the space token
%at the end of the line, or the \cs{par} token after an empty line.
For simple input text, characters are made into
character tokens. However, \TeX\ can ignore input characters:
a row of spaces in the input is usually equivalent to just one
space. Also, \TeX\ itself can insert tokens that do not correspond
to any character in the input, for instance the space token
at the end of the line, or the \cs{par} token after an empty line.

%Not all character tokens signify characters to be typeset.
%\altt
%Characters fall into sixteen categories \ldash each one
%specifying a certain function that a character can have \rdash 
%of which only two contain the characters that will be
%typeset. The other categories contain such characters 
%as~\n{\char`\{}, \n{\char`\}}, 
%\n\&, and~\n\#. A~character token can be considered
%as a pair of numbers: the character code \ldash typically the \ascii\
%code \rdash  and the category code.
%It is possible to change
%the category code that is associated with a particular
%character code.
Not all character tokens signify characters to be typeset.
\altt
Characters fall into sixteen categories \ldash each one
specifying a certain function that a character can have \rdash 
of which only two contain the characters that will be
typeset. The other categories contain such characters 
as~\n{\char`\{}, \n{\char`\}}, 
\n\&, and~\n\#. A~character token can be considered
as a pair of numbers: the character code \ldash typically the \ascii\
code \rdash  and the category code.
It is possible to change
the category code that is associated with a particular
character code.

%When the escape character (by default~\cs{}$\,$) appears in the input,
%\TeX's behaviour in forming tokens is more complicated. 
%Basically,
%\TeX\ builds a control sequence by taking a number of characters
%from the input and lumping them together into a single token.
When the escape character (by default~\cs{}$\,$) appears in the input,
\TeX's behaviour in forming tokens is more complicated. 
Basically,
\TeX\ builds a control sequence by taking a number of characters
from the input and lumping them together into a single token.

%The behaviour with which \TeX's input processor 
%reacts to category codes can be described
%as a machine that switches between three internal states:
%$N$,~new line; $M$,~middle of line; $S$,~skipping spaces.
%These states and the transitions between them are treated
%in Chapter~\ref{mouth}.
The behaviour with which \TeX's input processor 
reacts to category codes can be described
as a machine that switches between three internal states:
$N$,~new line; $M$,~middle of line; $S$,~skipping spaces.
These states and the transitions between them are treated
in Chapter~\ref{mouth}.

%\subsection{Two-level input processing}
\subsection{Two-level input processing}

%\TeX's input processor is in fact itself a two-level processor.
%Because of limitations of the terminal, the editor, or the operating
%system, the user may not be able to input certain desired characters.
%Therefore, \TeX\ provides a mechanism to access
%with two superscript characters all of the available character
%positions. This may be considered
%a separate stage of \TeX\ processing, taking place
%prior to the three-state machine mentioned above.
\TeX's input processor is in fact itself a two-level processor.
Because of limitations of the terminal, the editor, or the operating
system, the user may not be able to input certain desired characters.
Therefore, \TeX\ provides a mechanism to access
with two superscript characters all of the available character
positions. This may be considered
a separate stage of \TeX\ processing, taking place
prior to the three-state machine mentioned above.

%For instance, the sequence \verb>^^+> is replaced by~\n{k} because
%the \ascii{} codes of \n k and \n + differ by~64. 
%Since this replacement takes place before tokens are formed,
%writing \verb>\vs^^+ip 5cm> has the same effect as
%\verb>\vskip 5cm>. Examples more useful than this exist.
For instance, the sequence \verb>^^+> is replaced by~\n{k} because
the \ascii{} codes of \n k and \n + differ by~64. 
Since this replacement takes place before tokens are formed,
writing \verb>\vs^^+ip 5cm> has the same effect as
\verb>\vskip 5cm>. Examples more useful than this exist.

%Note that this first stage is a transformation from
%characters to characters, without considering category
%codes. These come into play only in the second phase
%of input processing where characters are converted
%to character tokens by coupling the category code
%to the character code.
Note that this first stage is a transformation from
characters to characters, without considering category
codes. These come into play only in the second phase
of input processing where characters are converted
to character tokens by coupling the category code
to the character code.

%\section{The expansion processor}
\section{The expansion processor}

%\TeX's  expansion processor accepts a stream of tokens
%and, if possible, 
%expands the tokens in this stream one by one
%until only unexpandable tokens remain.
%Macro expansion is the clearest example of this:
%if a control sequence is a macro name, it is replaced
%(together possibly with parameter tokens) by 
%the definition text of the macro.
\TeX's  expansion processor accepts a stream of tokens
and, if possible, 
expands the tokens in this stream one by one
until only unexpandable tokens remain.
Macro expansion is the clearest example of this:
if a control sequence is a macro name, it is replaced
(together possibly with parameter tokens) by 
the definition text of the macro.

%Input for the expansion processor is provided mainly
%by the input processor. The stream of tokens coming
%from the first stage of \TeX\ processing is subject
%to the expansion process, and the result is a stream
%of unexpandable tokens which is fed to the execution processor.
Input for the expansion processor is provided mainly
by the input processor. The stream of tokens coming
from the first stage of \TeX\ processing is subject
to the expansion process, and the result is a stream
of unexpandable tokens which is fed to the execution processor.

%However, the expansion processor comes into play 
%also when (among others) an \cs{edef} or \cs{write} is processed.
%The parameter token list of these commands is
%expanded very much as if the lists had been
%on the top level, instead of the argument to a command.
However, the expansion processor comes into play 
also when (among others) an \cs{edef} or \cs{write} is processed.
The parameter token list of these commands is
expanded very much as if the lists had been
on the top level, instead of the argument to a command.

%\subsection{The process of expansion}
\subsection{The process of expansion}

%Expanding a token consists of the following steps:
%\begin{enumerate}
%\item See whether the token is expandable. 
%\item If the token is unexpandable, pass it to the token
%      list currently being built, and take on the next token. 
Expanding a token consists of the following steps:
\begin{enumerate}
\item See whether the token is expandable. 
\item If the token is unexpandable, pass it to the token
      list currently being built, and take on the next token. 

%\item If the token is expandable, replace it by its expansion.
%      For macros without parameters, and a few primitive commands
%      such as \cs{jobname}, this is indeed a simple replacement.
%      Usually, however, \TeX\ needs to absorb some argument tokens from
%      the stream in order to be able to form the replacement
%      of the current token.
%      For instance, if the token was a macro with parameters,
%      sufficiently many tokens need to be absorbed to form
%      the arguments corresponding to  these parameters.
\item If the token is expandable, replace it by its expansion.
      For macros without parameters, and a few primitive commands
      such as \cs{jobname}, this is indeed a simple replacement.
      Usually, however, \TeX\ needs to absorb some argument tokens from
      the stream in order to be able to form the replacement
      of the current token.
      For instance, if the token was a macro with parameters,
      sufficiently many tokens need to be absorbed to form
      the arguments corresponding to  these parameters.

%\item Go on expanding, starting with the first token of the
%      expansion. 
%\end{enumerate}
%%
%Deciding whether a token is expandable is
%a simple decision. Macros and active characters, 
%conditionals, and a number of primitive \TeX\ commands
%(see the list on page~\pageref{expand:lijst})
%are expandable, other tokens are not.
%Thus the expansion processor replaces macros by their expansion,
%it evaluates conditionals and eliminates any irrelevant parts of 
%these, but tokens such as \cs{vskip} and character tokens,
%including characters such as dollars and braces, are passed untouched.
\item Go on expanding, starting with the first token of the
      expansion. 
\end{enumerate}
%
Deciding whether a token is expandable is
a simple decision. Macros and active characters, 
conditionals, and a number of primitive \TeX\ commands
(see the list on page~\pageref{expand:lijst})
are expandable, other tokens are not.
Thus the expansion processor replaces macros by their expansion,
it evaluates conditionals and eliminates any irrelevant parts of 
these, but tokens such as \cs{vskip} and character tokens,
including characters such as dollars and braces, are passed untouched.

%\subsection{Special cases: \cs{expandafter}, \cs{noexpand}, and \cs{the}}
\subsection{Special cases: \cs{expandafter}, \cs{noexpand}, and \cs{the}}

%As stated above,
%after a token has been expanded, \TeX\ will start expanding
%the resulting tokens. At first sight the \cs{expandafter}
%command would seem to be an exception to this rule, because
%it expands only one step. What actually happens is that
%the sequence \begin{disp}\cs{expandafter}\gr{token$_1$}\gr{token$_2$}\end{disp}
%is replaced by 
%\begin{disp}\gr{token$_1$}\gr{\italic expansion of token$_2$}\end{disp}
%and this replacement is in fact reexamined by the expansion
%processor.
As stated above,
after a token has been expanded, \TeX\ will start expanding
the resulting tokens. At first sight the \cs{expandafter}
command would seem to be an exception to this rule, because
it expands only one step. What actually happens is that
the sequence \begin{disp}\cs{expandafter}\gr{token$_1$}\gr{token$_2$}\end{disp}
is replaced by 
\begin{disp}\gr{token$_1$}\gr{\italic expansion of token$_2$}\end{disp}
and this replacement is in fact reexamined by the expansion
processor.

%Real exceptions do exist, however. If the 
%current token is the \cs{noexpand} command, the next
%token is considered for the moment to be unexpandable:
%it is handled as if it were \cs{relax}, and it is
%passed to the token list being built.
Real exceptions do exist, however. If the 
current token is the \cs{noexpand} command, the next
token is considered for the moment to be unexpandable:
it is handled as if it were \cs{relax}, and it is
passed to the token list being built.

%For example,
%in the macro definition
%\begin{verbatim}
%\edef\a{\noexpand\b}
%\end{verbatim}
%the replacement text \verb>\noexpand\b> is expanded at definition 
%time. The expansion of \cs{noexpand} is the next token, with
%a temporary meaning of \cs{relax}. Thus, when the expansion
%processor tackles the next token, the~\cs{b}, it will consider
%that to be unexpandable, and just pass it to the token list
%being built, which is the replacement text of the macro.
For example,
in the macro definition
\begin{verbatim}
\edef\a{\noexpand\b}
\end{verbatim}
the replacement text \verb>\noexpand\b> is expanded at definition 
time. The expansion of \cs{noexpand} is the next token, with
a temporary meaning of \cs{relax}. Thus, when the expansion
processor tackles the next token, the~\cs{b}, it will consider
that to be unexpandable, and just pass it to the token list
being built, which is the replacement text of the macro.

%Another exception is that the tokens
%resulting from \cs{the}\gr{token variable}
%are not expanded further if this statement occurs
%inside an \cs{edef} macro definition.
Another exception is that the tokens
resulting from \cs{the}\gr{token variable}
are not expanded further if this statement occurs
inside an \cs{edef} macro definition.

%\subsection{Braces in the expansion processor}
\subsection{Braces in the expansion processor}

%Above, it was said that braces are passed as unexpandable
%character tokens. In general this is true. For instance,
%the \cs{romannumeral} command is handled by the expansion
%processor; when confronted with 
%\begin{verbatim}
%\romannumeral1\number\count2 3{4 ...
%\end{verbatim} 
%\TeX\ will expand until the brace is encountered:
%if \cs{count2} has the value of zero, the result will be
%the roman numeral representation of~\n{103}.
Above, it was said that braces are passed as unexpandable
character tokens. In general this is true. For instance,
the \cs{romannumeral} command is handled by the expansion
processor; when confronted with 
\begin{verbatim}
\romannumeral1\number\count2 3{4 ...
\end{verbatim} 
\TeX\ will expand until the brace is encountered:
if \cs{count2} has the value of zero, the result will be
the roman numeral representation of~\n{103}.

%As another example, \begin{verbatim}
%\iftrue {\else }\fi
%\end{verbatim}
%is handled by the expansion processor 
%completely analogous to
%\begin{disp}\cs{iftrue} {\italic a}\cs{else} {\italic b}\cs{fi}\end{disp}
%The result is a character token, independent of its category.
As another example, \begin{verbatim}
\iftrue {\else }\fi
\end{verbatim}
is handled by the expansion processor 
completely analogous to
\begin{disp}\cs{iftrue} {\italic a}\cs{else} {\italic b}\cs{fi}\end{disp}
The result is a character token, independent of its category.

%However, in the context of macro expansion 
%the expansion  processor will 
%recognize braces. 
%First of all, a balanced pair of braces marks off a group of tokens
%to be passed as one argument.
%If a macro has an argument \begin{verbatim}
%\def\macro#1{ ... }
%\end{verbatim}
%one can call it with a single token, as in
%\begin{verbatim}
%\macro 1 \macro \$
%\end{verbatim}
%or with a group of tokens, surrounded by braces
%\begin{verbatim}
%\macro {abc} \macro {d{ef}g}
%\end{verbatim}
However, in the context of macro expansion 
the expansion  processor will 
recognize braces. 
First of all, a balanced pair of braces marks off a group of tokens
to be passed as one argument.
If a macro has an argument \begin{verbatim}
\def\macro#1{ ... }
\end{verbatim}
one can call it with a single token, as in
\begin{verbatim}
\macro 1 \macro \$
\end{verbatim}
or with a group of tokens, surrounded by braces
\begin{verbatim}
\macro {abc} \macro {d{ef}g}
\end{verbatim}


%Secondly, when the arguments for a macro with
%parameters are read, no expressions with unbalanced braces
%are accepted. In 
%\begin{verbatim}
%\def\a#1\stop{ ... }
%\end{verbatim}
%the argument consists of all
%tokens up to the first occurrence of \cs{stop}
%that is not in braces: in
%\begin{verbatim}
%\a bc{d\stop}e\stop
%\end{verbatim}
%the argument of~\cs{a} is \verb>bc{d\stop}e>.
%Only balanced expressions
%are accepted here.
Secondly, when the arguments for a macro with
parameters are read, no expressions with unbalanced braces
are accepted. In 
\begin{verbatim}
\def\a#1\stop{ ... }
\end{verbatim}
the argument consists of all
tokens up to the first occurrence of \cs{stop}
that is not in braces: in
\begin{verbatim}
\a bc{d\stop}e\stop
\end{verbatim}
the argument of~\cs{a} is \verb>bc{d\stop}e>.
Only balanced expressions
are accepted here.

%\section{The execution processor}
\section{The execution processor}

%The execution processor builds lists: horizontal, vertical,
%and math lists. Corresponding to these lists, it works
%in horizontal, vertical, or math mode. Of these three modes
%`internal' and `external' variants exist.
%In addition to building lists, this part of the \TeX\ processor
%also performs mode-independent processing, such as
%assignments.
The execution processor builds lists: horizontal, vertical,
and math lists. Corresponding to these lists, it works
in horizontal, vertical, or math mode. Of these three modes
`internal' and `external' variants exist.
In addition to building lists, this part of the \TeX\ processor
also performs mode-independent processing, such as
assignments.

%Coming out of the expansion processor is a stream of
%unexpandable tokens to be processed by
%the execution processor. 
%\relax From the point of view of the execution processor, this
%stream contains two types of tokens:
%\begin{itemize}
%\item Tokens signalling an assignment (this includes
%      macro definitions), and
%      other tokens signalling actions
%      that are independent of the mode, such
%      as \cs{show} and \cs{aftergroup}.
%\item Tokens that build lists:
%      characters, boxes, and glue. The way they are handled
%      depends on the current mode.
%\end{itemize}
Coming out of the expansion processor is a stream of
unexpandable tokens to be processed by
the execution processor. 
\relax From the point of view of the execution processor, this
stream contains two types of tokens:
\begin{itemize}
\item Tokens signalling an assignment (this includes
      macro definitions), and
      other tokens signalling actions
      that are independent of the mode, such
      as \cs{show} and \cs{aftergroup}.
\item Tokens that build lists:
      characters, boxes, and glue. The way they are handled
      depends on the current mode.
\end{itemize}

%Some objects can be used in any mode; for instance boxes
%can appear in horizontal, vertical, and math lists.
%The effect of such an object will of course still depend on the mode.
%Other objects are  specific for one mode.
%For instance, characters (to be more precise:
%character tokens of categories 11 and~12), 
%are intimately connected to horizontal mode:
%if the execution processor 
%is in vertical mode when it encounters a character, it will
%switch to horizontal mode.
Some objects can be used in any mode; for instance boxes
can appear in horizontal, vertical, and math lists.
The effect of such an object will of course still depend on the mode.
Other objects are  specific for one mode.
For instance, characters (to be more precise:
character tokens of categories 11 and~12), 
are intimately connected to horizontal mode:
if the execution processor 
is in vertical mode when it encounters a character, it will
switch to horizontal mode.

%Not all character tokens signal characters to be typeset:
%the execution processor can also encounter math shift
%characters (by default~\n{\char`\$}) and beginning/end of group
%characters (by default \n{\char`\{} and~\n{\char`\}}).
%Math shift characters let \TeX\ enter or exit
%math mode, and braces let it enter or exit a~new level of
%grouping.
Not all character tokens signal characters to be typeset:
the execution processor can also encounter math shift
characters (by default~\n{\char`\$}) and beginning/end of group
characters (by default \n{\char`\{} and~\n{\char`\}}).
Math shift characters let \TeX\ enter or exit
math mode, and braces let it enter or exit a~new level of
grouping.

%One control sequence handled by the execution processor 
%deserves special mention: \cs{relax}.
%This control sequence is not expandable, but the execution
%is to do nothing. Compare the effect of \cs{relax} in
%\begin{verbatim}
%\count0=1\relax 2
%\end{verbatim}
%with that of \cs{empty}
%defined by
%\begin{verbatim}
%\def\empty{}
%\end{verbatim}
%in 
%\begin{verbatim}
%\count0=1\empty 2
%\end{verbatim}
%In the first case the expansion
%process that is forming the number stops at \cs{relax} and
%the number {\tt 1} is assigned; in the second case 
%\cs{empty} expands to nothing, so {\tt 12} is assigned.
One control sequence handled by the execution processor 
deserves special mention: \cs{relax}.
This control sequence is not expandable, but the execution
is to do nothing. Compare the effect of \cs{relax} in
\begin{verbatim}
\count0=1\relax 2
\end{verbatim}
with that of \cs{empty}
defined by
\begin{verbatim}
\def\empty{}
\end{verbatim}
in 
\begin{verbatim}
\count0=1\empty 2
\end{verbatim}
In the first case the expansion
process that is forming the number stops at \cs{relax} and
the number {\tt 1} is assigned; in the second case 
\cs{empty} expands to nothing, so {\tt 12} is assigned.

%\section{The visual processor}
\section{The visual processor}

%\TeX's output processor encompasses those algorithms that
%are outside direct user control: paragraph breaking,
%alignment, page breaking, math typesetting, and \n{dvi} file
%generation. Various parameters control the operation
%of these parts of \TeX.
\TeX's output processor encompasses those algorithms that
are outside direct user control: paragraph breaking,
alignment, page breaking, math typesetting, and \n{dvi} file
generation. Various parameters control the operation
of these parts of \TeX.

%Some of these algorithms return their results in a form that
%can be handled by the execution processor. For instance,
%a paragraph that has been broken into lines is added to
%the main vertical list as a sequence of horizontal boxes
%with intermediate glue and penalties. Also, the page breaking
%algorithm stores its result in \cs{box255}, so output
%routines can dissect it. On the other hand, a math formula
%can not be broken into pieces, and, naturally, 
%shipping a box to the \n{dvi} file is irreversible.
Some of these algorithms return their results in a form that
can be handled by the execution processor. For instance,
a paragraph that has been broken into lines is added to
the main vertical list as a sequence of horizontal boxes
with intermediate glue and penalties. Also, the page breaking
algorithm stores its result in \cs{box255}, so output
routines can dissect it. On the other hand, a math formula
can not be broken into pieces, and, naturally, 
shipping a box to the \n{dvi} file is irreversible.

%\section{Examples}
\section{Examples}

%\subsection{Skipped spaces}
\subsection{Skipped spaces}

%Skipped spaces provide an illustration of the view that
%\TeX's levels of processing accept the completed input
%of the previous level. Consider the commands
%\begin{verbatim}
%\def\a{\penalty200}
%\a 0
%\end{verbatim} 
%This is {\italic not\/} equivalent to
%\begin{verbatim}
%\penalty200 0
%\end{verbatim} 
%which would place a penalty of \n{200}, and
%typeset the digit~\n0. Instead it expands to
%\begin{verbatim}
%\penalty2000
%\end{verbatim}
%because the space after \cs{a} is skipped in the
%input processor. Later stages of processing then receive
%the sequence \begin{verbatim}
%\a0
%\end{verbatim}
Skipped spaces provide an illustration of the view that
\TeX's levels of processing accept the completed input
of the previous level. Consider the commands
\begin{verbatim}
\def\a{\penalty200}
\a 0
\end{verbatim} 
This is {\italic not\/} equivalent to
\begin{verbatim}
\penalty200 0
\end{verbatim} 
which would place a penalty of \n{200}, and
typeset the digit~\n0. Instead it expands to
\begin{verbatim}
\penalty2000
\end{verbatim}
because the space after \cs{a} is skipped in the
input processor. Later stages of processing then receive
the sequence \begin{verbatim}
\a0
\end{verbatim}

%\subsection{Internal quantities and their representations}
\subsection{Internal quantities and their representations}

%\TeX\ uses various sorts of internal quantities,
%such as integers and dimensions. These internal
%quantities have an external representation,
%which is a string of characters, such as 
%\n{4711} or~\n{91.44cm}.
\TeX\ uses various sorts of internal quantities,
such as integers and dimensions. These internal
quantities have an external representation,
which is a string of characters, such as 
\n{4711} or~\n{91.44cm}.

%Conversions between the internal value and the external
%representation take place on two different levels,
%depending on what direction the conversion goes.
%A~string of characters is converted to an internal
%value in assignments such as
%\begin{verbatim}
%\pageno=12 \baselineskip=13pt
%\end{verbatim}
%or statements such as
%\begin{verbatim}
%\vskip 5.71pt
%\end{verbatim}
%and all of these statements are handled by the execution
%processor.
Conversions between the internal value and the external
representation take place on two different levels,
depending on what direction the conversion goes.
A~string of characters is converted to an internal
value in assignments such as
\begin{verbatim}
\pageno=12 \baselineskip=13pt
\end{verbatim}
or statements such as
\begin{verbatim}
\vskip 5.71pt
\end{verbatim}
and all of these statements are handled by the execution
processor.

%On the other hand, the conversion of the internal
%values into a representation as a string of
%characters is handled by the expansion processor.
%For instance, \begin{verbatim}
%\number\pageno \romannumeral\year
%\the\baselineskip
%\end{verbatim}
%are all processed by expansion.
On the other hand, the conversion of the internal
values into a representation as a string of
characters is handled by the expansion processor.
For instance, \begin{verbatim}
\number\pageno \romannumeral\year
\the\baselineskip
\end{verbatim}
are all processed by expansion.

%As a final example, suppose \verb>\count2=45>, and
%consider the statement
%\begin{verbatim}
%\count0=1\number\count2 3
%\end{verbatim}
%The expansion processor tackles \verb>\number\count2>
%to give the characters \n{45}, and the space after
%the \n 2 does not end the number being assigned:
%it only serves as a delimiter
%of the number of the \cs{count} register.
%In the next stage of processing, the execution processor
%will then see the statement
%\begin{verbatim}
%\count0=1453
%\end{verbatim}
%and execute this.
As a final example, suppose \verb>\count2=45>, and
consider the statement
\begin{verbatim}
\count0=1\number\count2 3
\end{verbatim}
The expansion processor tackles \verb>\number\count2>
to give the characters \n{45}, and the space after
the \n 2 does not end the number being assigned:
it only serves as a delimiter
of the number of the \cs{count} register.
In the next stage of processing, the execution processor
will then see the statement
\begin{verbatim}
\count0=1453
\end{verbatim}
and execute this.

%%\endinput
%\endinput

%\endofchapter
%%%%% end of input file [bigpic]
\endofchapter
%%%% end of input file [bigpic]

\end{document}
