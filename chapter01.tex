% -*- coding: utf-8 -*-
% This file is part of TeX by Topic
% Copyright 2007-2014 Victor Eijkhout
% Translated by LiYanrui@bbs.ctex.org
% Translated by zoho@bbs.ctex.org
% Translated by Liam0205@bbs.ctex.org
\documentclass{book}

% -*- coding: utf-8 -*-

\usepackage[b5paper,text={5in,8in},centering]{geometry}

\usepackage[CJKchecksingle]{xeCJK}
\setmainfont[Mapping=tex-text]{TeX Gyre Schola}
%\setsansfont{URW Gothic L Book}
%\setmonofont{Nimbus Mono L}
\setCJKmainfont[BoldFont=FandolHei,ItalicFont=FandolKai]{FandolSong}
\setCJKsansfont{FandolHei}
\setCJKmonofont{FandolFang}
\xeCJKsetup{PunctStyle = kaiming}

\linespread{1.25}
\setlength{\parindent}{2em}

\usepackage{xcolor}
\definecolor{myblue}{rgb}{0,0.2,0.6}

\usepackage{titlesec}
\titleformat{\chapter}
    {\normalfont\Huge\sffamily\color{myblue}}
    {第\thechapter 章}
    {1em}
    {}
%\titlespacing{\chapter}{0pt}{50pt}{40pt}
\titleformat{\section}
    {\normalfont\Large\sffamily\color{myblue}}
    {\thesection}
    {1em}
    {}
%\titlespacing{\section}{0pt}{3.5ex plus 1ex minus .2ex}{2.3ex plus .2ex}
\titleformat{\subsection}
    {\normalfont\large\sffamily\color{myblue}}
    {\thesubsection}
    {1em}
    {}
%\titlespacing{\subsection}{0pt}{3.25ex plus 1ex minus .2ex}{1.5ex plus .2ex}
%
\newpagestyle{special}[\small\sffamily]{
  \headrule
  \sethead[\usepage][][\chaptertitle]
  {\chaptertitle}{}{\usepage}}
\newpagestyle{main}[\small\sffamily]{
  \headrule
  \sethead[\usepage][][第\thechapter 章\quad\chaptertitle]
  {\thesection\quad\sectiontitle}{}{\usepage}}

\usepackage{titletoc}
%\setcounter{tocdepth}{1}
%\titlecontents{标题层次}[左间距]{上间距和整体格式}{标题序号}{标题内容}{指引线和页码}[下间距]
\titlecontents{chapter}[1.5em]{\vspace{.5em}\bfseries\sffamily}{\color{myblue}\contentslabel{1.5em}}{}
    {\titlerule*[20pt]{$\cdot$}\contentspage}[]
\titlecontents{section}[4.5em]{\sffamily}{\color{myblue}\contentslabel{3em}}{}
    {\titlerule*[20pt]{$\cdot$}\contentspage}[]
%\titlecontents{subsection}[8.5em]{\sffamily}{\contentslabel{4em}}{}
%    {\titlerule*[20pt]{$\cdot$}\contentspage}

\usepackage{enumitem}
\setlist{topsep=2pt,itemsep=2pt,parsep=1pt,leftmargin=\parindent}

\usepackage{fancyvrb}
\DefineVerbatimEnvironment{verbatim}{Verbatim}
  {xleftmargin=2em,baselinestretch=1,formatcom=\color{teal}\upshape}

\usepackage{etoolbox}
\makeatletter
\preto{\FV@ListVSpace}{\topsep=2pt \partopsep=0pt }
\makeatother

\usepackage[colorlinks,plainpages,pagebackref]{hyperref}
\hypersetup{
   pdfstartview={FitH},
   citecolor=teal,
   linkcolor=myblue,
   urlcolor=black,
   bookmarksnumbered
}

\usepackage{comment,makeidx,multicol}

%\usepackage{german}
%% german
%\righthyphenmin=3
%\mdqoff
%\captionsenglish
\usepackage[english]{babel}
{\catcode`"=13 \gdef"#1{\ifx#1"\discretionary{}{}{}\fi\relax}}
\def\mdqon{\catcode`"=13\relax}
\def\mdqoff{\catcode`"=12\relax}
\makeindex
\hyphenation{ex-em-pli-fies}

\newdimen\tempdima \newdimen\tempdimb

% these are fine
\def\handbreak{\\ \message{^^JManual break!!!!^^J}}
\def\nl{\protect\\}\def\n#1{{\tt #1}}
\def\cs#1{\texorpdfstring{{\tt\char`\\#1}}{\textbackslash#1}} %\def\cs#1{{\tt\char`\\#1}}
\let\csc\cs
\def\lb{{\tt\char`\{}}\def\rb{{\tt\char`\}}}
\def\gr#1{\texorpdfstring{$\langle$#1$\rangle$}{<#1>}} %\def\gr#1{$\langle$#1$\rangle$}
\def\key#1{{\tt#1}}
\def\alt{}\def\altt{}%this way in manstijl
\def\ldash{\unskip\ --\nobreak\ \ignorespaces}
\def\rdash{\unskip\nobreak\ --\ \ignorespaces}
% check these
\def\hex{{\tt"}}
\def\ascii{{\sc ascii}}
\def\ebcdic{{\sc ebcdic}}
\def\IniTeX{Ini\TeX}\def\LamsTeX{LAMS\TeX}\def\VirTeX{Vir\TeX}
\def\AmsTeX{Ams\TeX}
\def\TeXbook{the \TeX\ book}\def\web{{\sc web}}
% needs major thinking
\newenvironment{myquote}{\list{}{%
    \topsep=2pt \partopsep=0pt%
    \leftmargin=\parindent \rightmargin=\parindent
    }\item[]}{\endlist}
\newenvironment{disp}{\begin{myquote}}{\end{myquote}}
\newenvironment{Disp}{\begin{myquote}}{\end{myquote}}
\newenvironment{tdisp}{\begin{myquote}}{\end{myquote}}
\newenvironment{example}{\begin{myquote}\noindent\itshape 例子:}{\end{myquote}}
\newenvironment{inventory}{\begin{description}\raggedright}{\end{description}}
\newenvironment{glossinventory}{\begin{description}}{\end{description}}
\def\gram#1{\gr{#1}}%???
%
% index
%
\def\indexterm#1{\emph{#1}\index{#1}}
\def\indextermsub#1#2{\emph{#1 #2}\index{#1!#2}}
\def\indextermbus#1#2{\emph{#1 #2}\index{#2!#1}}
\def\term#1\par{\index{#1}}
\def\howto#1\par{}
\def\cstoidx#1\par{\index{#1@\cs{#1}@}}
\def\thecstoidx#1\par{\index{#1@\protect\csname #1\endcsname}}
\def\thecstoidxsub#1#2{\index{#1, #2@\protect\csname #1\endcsname, #2}\ignorespaces}
\def\csterm#1\par{\cstoidx #1\par\cs{#1}}
\def\csidx#1{\cstoidx #1\par\cs{#1}}

\def\tmc{\tracingmacros=2 \tracingcommands\tracingmacros}

%%%%%%%%%%%%%%%%%%%
\makeatletter
\def\snugbox{\hbox\bgroup\setbox\z@\vbox\bgroup
    \leftskip\z@
    \bgroup\aftergroup\make@snug
    \let\next=}
\def\make@snug{\par\sn@gify\egroup \box\z@\egroup}
\def\sn@gify
   {\skip\z@=\lastskip \unskip
    \advance\skip\z@\lastskip \unskip
    \unpenalty
    \setbox\z@\lastbox
    \ifvoid\z@ \nointerlineskip \else {\sn@gify} \fi
    \hbox{\unhbox\z@}\nointerlineskip
    \vskip\skip\z@
    }

\newdimen\fbh \fbh=60pt % dimension for easy scaling:
\newdimen\fbw \fbw=60pt % height and width of character box

\newdimen\dh \newdimen\dw % height and width of current character box
\newdimen\lh % height of previous character box
\newdimen\lw \lw=.4pt % line weight, instead of default .4pt

\def\hdotfill{\noindent
    \leaders\hbox{\vrule width 1pt height\lw
                  \kern4pt
                  \vrule width.5pt height\lw}\hfill\hbox{}
    \par}
\def\hlinefill{\noindent
    \leaders\hbox{\vrule width 5.5pt height\lw         }\hfill\hbox{}
    \par}
\def\stippel{$\qquad\qquad\qquad\qquad$}
\makeatother
%%%%%%%%%%%%%%%%%%%

%\def\SansSerif{\Typeface:macHelvetica }
%\def\SerifFont{\Typeface:macTimes }
%\def\SansSerif{\Typeface:bsGillSans }
%\def\SerifFont{\Typeface:bsBaskerville }
\let\SansSerif\relax \def\italic{\it}
\let\SerifFont\relax \def\MainFont{\rm}
\let\SansSerif\relax
\let\SerifFont\relax
\let\PopIndentLevel\relax \let\PushIndentLevel\relax
\let\ToVerso\relax \let\ToRecto\relax

%\def\stop@command@suffix{stop}
%\let\PopListLevel\PopIndentLevel
%\let\FlushRight\relax
%\let\flushright\FlushRight
%\let\SetListIndent\LevelIndent
%\def\awp{\ifhmode\vadjust{\penalty-10000 }\else
%    \penalty-10000 \fi}
\let\awp\relax
\let\PopIndentLevel\relax \let\PopListLevel\relax

\showboxdepth=-1

%\input figs
\def\endofchapter{\vfill\noindent}


\begin{document}

%\chapter{The Structure of the \TeX\ Processor}
\chapter{\TeX\ 处理器的结构}

%This book treats the various aspects of \TeX\ in chapters
%that are concerned with relatively small, well-delineated,
%topics. In this chapter, therefore,
%a global picture of the way \TeX\ operates will be given.
%Of necessity, many details will be omitted here, but all of
%these are treated in later chapters. On the other hand,
%the few examples given in this chapter will be repeated
%in the appropriate places later on; they are included here
%to make this chapter self-contained.
本书涵盖了 \TeX\ 诸多方面的内容;其中每一章都讲述一个相对较小且易于描述的主题。
本章将简介 \TeX\ 处理文档的基本流程。
当然,本章会忽略很多细节,留待后续章节继续讨论。
相应的,为了说明问题,本章会给出一些例子;这些例子在后续的这些章节中会继续讨论。

%\section{Four \TeX\protect\ processors}
\section{\TeX\ 的 4 个处理器}

%The way \TeX\ processes its input can be viewed as
%happening on four levels. One might  say that
%the \TeX\ processor is split into four separate units,
%each one accepting the output of the previous stage, and
%delivering the input for the next stage. The input of
%the first stage is then the \n{.tex} input file; the output
%of the last stage is a \n{.dvi} file.
\TeX\ 处理输入的过程可分为 4 个层次。
你可以认为 \TeX\ 处理器(也成 \TeX\ 引擎)有 4 个独立的单元,
每个单元接收前序步骤的输出,并将输出作为后续单元的输入。
第一阶段的输入是 \n{.tex} 文件,最后一个阶段的输出是 \n{.dvi} 文件\footnote{译注(Liam0205):此处讨论的 \TeX\ 是 Knuth \TeX。后续扩展的 pdf\TeX 等不一定输出 \n{.tex} 文件。}。%

%For many purposes it is most convenient, and most insightful,
%to consider these four levels of processing as happening
%after one another, each one accepting the {\em completed\/}
%output of the previous level. In reality this is not true:
%all levels are simultaneously
%active, and there is interaction between them.
通常而言,这样理解上述 4 个层次方便且直观:
每个层次\emph{完整地}接收上一个层次的输出。
然而,这样并不完全正确:
4 个层次的处理是同时进行的,层次之间也有交互\footnote{译注(Liam0205):流式处理,上一层次处理完多少,下一个层次就接着继续处理这部分。}。

%The four levels are (corresponding roughly
%to the `eyes', `mouth', `stomach', and `bowels' respectively
%in Knuth's original terminology) as follows.
%\begin{enumerate}\item
%The input processor. This is the piece of \TeX\ that
%accepts input lines from the file system of whatever computer
%\TeX\ runs on, and turns them into tokens.
%Tokens are the internal objects of \TeX:
%there are character tokens that constitute the typeset
%text, and control sequence tokens that are commands
%to be processed by the next two levels.
%\item The expansion processor.
%Some but not all of the tokens generated in the first level
%\ldash macros, conditionals, and a number
%of primitive \TeX\ commands \rdash  are subject to expansion.
%Expansion is the process that replaces some (sequences of)
%tokens by other (or no) tokens.
%\item The execution processor.
%Control sequences that are not expandable are executable,
%and this execution takes place on the third level of the
%\TeX\ processor.
这 4 个层面(使用 Knuth 最初的术语,可分别大致理解为‘眼睛’、‘嘴巴’、
‘胃’和‘肠道’)如下:
\begin{enumerate}
\item 输入处理器:
这是 \TeX\ 从文件中接受文本行输入并将其处理为记号的部分。
记号是 \TeX\ 的内部对象:有构成所排版文本的字符记号,
有可被后续两个层面作为命令进行处理的控制序列记号。
\item 展开处理器:
在第一层面所产生的一些(并非全部)记号\ldash 宏、条件式以及
\TeX\ 的一些原始命令\rdash 都是要被展开的目标。
所谓展开,就是将一些记号序列替换为其它记号的过程。
\item 执行处理器:那些不能再被进一步展开的控制序列在这一层会被执行。

%One part of the activity here concerns changes to
%\TeX's internal state: assignments (including
%macro definitions) are typical activities in this
%category. The other major thing happening on this level
%is the construction of horizontal, vertical, and
%mathematical lists.
%\item The visual processor.
%In the final level of processing
%the visual part of \TeX\ processing is performed. Here
%horizontal lists are broken into paragraphs,
%vertical lists are broken into pages,
%and  formulas are built out of math lists.
%Also the output to the \n{dvi} file takes place on this level.
%The algorithms working here are not accessible to the user,
%but they can be influenced by a number of parameters.
%\end{enumerate}
这里需要关注的部分是 \TeX\ 内部状态的变化:
赋值(包括宏定义)是该层最典型的行为,
另外就是水平列、竖直列和数学列的构建。
\item 可视化处理器:
在这一层面完成 \TeX\ 排版内容的可视化。诸如水平列会被划分为段,
竖直列会被划分为页,数学列会被构建为公式。此外此层面还会输出 \n{dvi} 文件%
【译注:对于现代 \TeX\ 处理器而言,还可以是 \n{pdf} 和 \n{xml} 等文件】。
在这一层面工作的算法对于用户而言是不可见的,
但是用户可通过一部分参数去控制它们。
\end{enumerate}

%\section{The input processor}
\section{输入处理器}

%The input processor of \TeX\ is that part of \TeX\ that
%translates whatever characters it gets from the input file
%into tokens. The output of this processor is a stream
%of tokens: a token list. Most tokens fall into one of two categories:
%character tokens and control sequence tokens.
%The remaining category is that of the parameter tokens;
%these will not be treated in this chapter.
\TeX\ 的输入处理器是 \TeX\ 从输入文件所接受的任何字符翻译为记号的部分,
它输出记号流:记号列表。大部分记号可归为两类:字符记号与控制序列记号。
还有一类是参数记号,但本章不讨论它。

%\subsection{Character input}
\subsection{字符的输入}

%For simple input text, characters are made into
%character tokens. However, \TeX\ can ignore input characters:
%a row of spaces in the input is usually equivalent to just one
%space. Also, \TeX\ itself can insert tokens that do not correspond
%to any character in the input, for instance the space token
%at the end of the line, or the \cs{par} token after an empty line.
对于所输入的简单文本,字符会被直接转换为字符记号。
不过,\TeX\ 会忽略这样一些字符:将多个连续的空格符号仅视为一个空格。
还有,\TeX 自身会向记号列表中插入一些不太明显的记号,例如行尾的空格记号,
每个空行之后的 \cs{par} 记号。

%Not all character tokens signify characters to be typeset.
%\altt
%Characters fall into sixteen categories \ldash each one
%specifying a certain function that a character can have \rdash
%of which only two contain the characters that will be
%typeset. The other categories contain such characters
%as~\n{\char`\{}, \n{\char`\}},
%\n\&, and~\n\#. A~character token can be considered
%as a pair of numbers: the character code \ldash typically the \ascii\
%code \rdash  and the category code.
%It is possible to change
%the category code that is associated with a particular
%character code.
字符可被输入处理器转换为字符记号,
但是这并非意味着所有字符可被排印(排版及印刷)。
在 \TeX\ 中,字符被划分为 16 类——每一类都有特定的功能——%
其中仅有两类字符可被排印。其它字符类,像 \n{\char`\{}、\n{\char`\}}、\n\&
和 \n\#,都是不可被排印的。
一个字符记号可视为一对数字:字符码和类别码。字符码通常使用 \ascii\ 编码。
一个字符对应的类别码是可以修改的。

%When the escape character (by default~\cs{}$\,$) appears in the input,
%\TeX's behaviour in forming tokens is more complicated.
%Basically,
%\TeX\ builds a control sequence by taking a number of characters
%from the input and lumping them together into a single token.
有一种字符叫做转义字符,默认是~\cs{}$\,$。
当这种字符出现在输入的文本中时,为了生成它的记号,\TeX\ 输入处理器的行为会非常复杂。
基本上,输入处理器会将尾随于转义字符的字符序列处理为单个记号,从而构建一个控制序列记号。

%The behaviour with which \TeX's input processor
%reacts to category codes can be described
%as a machine that switches between three internal states:
%$N$,~new line; $M$,~middle of line; $S$,~skipping spaces.
%These states and the transitions between them are treated
%in Chapter~\ref{mouth}.
对于类别码的处理,\TeX\ 输入处理器的行为可想像为在三种内部状态($N$,新行;$M$,
行内;$S$,忽略空格)中切换的机器。在第~\ref{mouth}~章中我们会讨论这些状态及其切换。

%\subsection{Two-level input processing}
\subsection{输入处理器的两个层面}

%\TeX's input processor is in fact itself a two-level processor.
%Because of limitations of the terminal, the editor, or the operating
%system, the user may not be able to input certain desired characters.
%Therefore, \TeX\ provides a mechanism to access
%with two superscript characters all of the available character
%positions. This may be considered
%a separate stage of \TeX\ processing, taking place
%prior to the three-state machine mentioned above.
实际上 \TeX\ 的输入处理器自身又可划分为两个层面。
由于终端、编辑器或者操作系统的限制,用户可能无法输入一些所需的字符。
所以 \TeX\ 提供了使用两个上标字符来获取各个有效字符的编码的机制。
这一机制可视为 \TeX\ 输入处理过程中的一个独立的阶段,
在时序上要早于上一节中的三态状态机。

%For instance, the sequence \verb>^^+> is replaced by~\n{k} because
%the \ascii{} codes of \n k and \n + differ by~64.
%Since this replacement takes place before tokens are formed,
%writing \verb>\vs^^+ip 5cm> has the same effect as
%\verb>\vskip 5cm>. Examples more useful than this exist.
例如,\verb>^^+> 这个字串的输入会被处理为 \n{k},
这是因为 \n k 和 \n + 的 \ascii{} 码之差为 64,
由于这个替换过程发生在形成字符记号之前,所以输入 \verb>\vs^^+ip 5cm>
与输入 \verb>\vskip 5cm> 是等价的。还有一些例子比这个例子更有用。

%Note that this first stage is a transformation from
%characters to characters, without considering category
%codes. These come into play only in the second phase
%of input processing where characters are converted
%to character tokens by coupling the category code
%to the character code.
上述过程即为 \TeX\ 输入处理器的第一层面,所做的工作是将字符转换为字符,
并不考虑类别码的问题,后者是在第二层面(三态状态机)产生的,
它与字符码组合成字符记号。

%\section{The expansion processor}
\section{展开处理器}

%\TeX's  expansion processor accepts a stream of tokens
%and, if possible,
%expands the tokens in this stream one by one
%until only unexpandable tokens remain.
%Macro expansion is the clearest example of this:
%if a control sequence is a macro name, it is replaced
%(together possibly with parameter tokens) by
%the definition text of the macro.
\TeX\ 的展开处理器可以接受记号流并且对其中的记号逐一进行展开,
直至记号流中所有的记号都是不可展开的。最有代表性的例子是宏的展开:
如果一个控制序列记号是一个宏名,那么该记号(可能还包括它的参数记号)%
会被替换为这个宏的定义文本。

%Input for the expansion processor is provided mainly
%by the input processor. The stream of tokens coming
%from the first stage of \TeX\ processing is subject
%to the expansion process, and the result is a stream
%of unexpandable tokens which is fed to the execution processor.
展开处理器所接受的记号流主要来自输入处理器,
展开结果是一串不可展开的记号流,将会交给执行处理器。

%However, the expansion processor comes into play
%also when (among others) an \cs{edef} or \cs{write} is processed.
%The parameter token list of these commands is
%expanded very much as if the lists had been
%on the top level, instead of the argument to a command.
然而,在处理 \cs{edef} 或 \cs{write} 等时也会执行展开处理器。
在展开时,这些命令的参数记号列很像在顶层而非在命令参量中的。

%\subsection{The process of expansion}
\subsection{记号的展开过程}

%Expanding a token consists of the following steps:
%\begin{enumerate}
%\item See whether the token is expandable.
%\item If the token is unexpandable, pass it to the token
%      list currently being built, and take on the next token.
展开处理器展开一个记号的步骤如下:
\begin{enumerate}
\item 查看这个记号是否可展开。
\item 如果记号不可展开,那么就将它放入当前构建的记号列表中,然后读入下一个记号。

%\item If the token is expandable, replace it by its expansion.
%      For macros without parameters, and a few primitive commands
%      such as \cs{jobname}, this is indeed a simple replacement.
%      Usually, however, \TeX\ needs to absorb some argument tokens from
%      the stream in order to be able to form the replacement
%      of the current token.
%      For instance, if the token was a macro with parameters,
%      sufficiently many tokens need to be absorbed to form
%      the arguments corresponding to  these parameters.
\item 如果记号可展开,那么它可以展开成什么,就将其替换为什么。对于不带参数的宏以及像
\cs{jobname} 这样的一些原始命令,只需进行简单的记号序列替换即可。不过,通常 \TeX\
需要从记号流中吸收一些参量记号,以形成当前记号的替换文本。例如,对于一个带参数的宏,
那么就需要从记号流中析取足够的记号,以形成与它的参数对应的参量。

%\item Go on expanding, starting with the first token of the
%      expansion.
%\end{enumerate}
%%
%Deciding whether a token is expandable is
%a simple decision. Macros and active characters,
%conditionals, and a number of primitive \TeX\ commands
%(see the list on page~\pageref{expand:lijst})
%are expandable, other tokens are not.
%Thus the expansion processor replaces macros by their expansion,
%it evaluates conditionals and eliminates any irrelevant parts of
%these, but tokens such as \cs{vskip} and character tokens,
%including characters such as dollars and braces, are passed untouched.
\item 对于当前记号的展开结果中的第一个记号,返回至步骤1 继续进行展开。
\end{enumerate}
%
判断一个记号是否可展开很简单。宏和活动字符、条件式以及一部分 \TeX\ 原始命令%
(见第~\pageref{expand:lijst}~页的列表)都是可展开的,其他记号则都是不可展开的。
展开处理器根据这个判定规则将宏的记号替换为它的定义,对条件式进行计算以忽略无关的记号。
不过,对于像 \cs{vskip} 这样的记号和字符记号,包括像美元符和花括号这样的字符,
展开处理器会将它们原封不动地传送到执行处理器。

%\subsection{Special cases: \cs{expandafter}, \cs{noexpand}, and \cs{the}}
\subsection{几个特例}

%As stated above,
%after a token has been expanded, \TeX\ will start expanding
%the resulting tokens. At first sight the \cs{expandafter}
%command would seem to be an exception to this rule, because
%it expands only one step. What actually happens is that
%the sequence \begin{disp}\cs{expandafter}\gr{token$_1$}\gr{token$_2$}\end{disp}
%is replaced by
%\begin{disp}\gr{token$_1$}\gr{\italic expansion of token$_2$}\end{disp}
%and this replacement is in fact reexamined by the expansion
%processor.
如上文所述,在一个记号被展开后,\TeX\ 会对其展开结果中的记号继续进行展开。
但是 \cs{expandafter} 这个控制序列初看破坏了这个游戏规则,
因为它只做一步展开。实际发生的事情是:记号列
\begin{disp}\cs{expandafter}\gr{token$_1$}\gr{token$_2$}\end{disp}
会被替换为
\begin{disp}\gr{token$_1$}\gr{\italic expansion of token$_2$}\end{disp}
而这个替换结果还会被展开处理器再次处理。

%Real exceptions do exist, however. If the
%current token is the \cs{noexpand} command, the next
%token is considered for the moment to be unexpandable:
%it is handled as if it were \cs{relax}, and it is
%passed to the token list being built.
然而确实存在不遵守展开处理器游戏规则的情况。
如果当前处理的记号是 \cs{noexpand} 控制序列,
那么展开处理器把它的下一个记号视为不可展开的:
展开处理器如同 \cs{relax} 那样处理这个记号,
直接将其传送到所构建的记号列表中。

%For example,
%in the macro definition
%\begin{verbatim}
%\edef\a{\noexpand\b}
%\end{verbatim}
%the replacement text \verb>\noexpand\b> is expanded at definition
%time. The expansion of \cs{noexpand} is the next token, with
%a temporary meaning of \cs{relax}. Thus, when the expansion
%processor tackles the next token, the~\cs{b}, it will consider
%that to be unexpandable, and just pass it to the token list
%being built, which is the replacement text of the macro.
例如下面这个宏的定义:
\begin{verbatim}
\edef\a{\noexpand\b}
\end{verbatim}
替换文本 \verb>\noexpand\b> 会在宏定义时被展开。
\cs{noexpand} 的展开结果是其后的那个记号临时改变成 \cs{relax} 的含义。
因此,在展开处理器处理下个记号 \cs{b} 时,就将它视为不可展开的,
而直接将它扔到在建的记号列表中,从而 \cs{b} 就是这个宏的替换文本。

%Another exception is that the tokens
%resulting from \cs{the}\gr{token variable}
%are not expanded further if this statement occurs
%inside an \cs{edef} macro definition.
还有一种特例:在 \cs{edef} 语句里,
\cs{the}\gr{token variable} 的展开结果是不会被进一步展开的。

%\subsection{Braces in the expansion processor}
\subsection{展开处理器中的花括号}

%Above, it was said that braces are passed as unexpandable
%character tokens. In general this is true. For instance,
%the \cs{romannumeral} command is handled by the expansion
%processor; when confronted with
%\begin{verbatim}
%\romannumeral1\number\count2 3{4 ...
%\end{verbatim}
%\TeX\ will expand until the brace is encountered:
%if \cs{count2} has the value of zero, the result will be
%the roman numeral representation of~\n{103}.
上一节提到,花括号会被展开处理器作为不可展开的字符记号忽略。
通常而言这个说法是正确的。例如下面这个 \cs{romannumeral} 控制序列:
\begin{verbatim}
\romannumeral1\number\count2 3{4 ...
\end{verbatim}
它会被 \TeX\ 展开至花括号处停止:如果 \cs{count2} 的值是 0,
那么这个控制序列的展开结果是~\n{103}~的罗马数字表示。

%As another example, \begin{verbatim}
%\iftrue {\else }\fi
%\end{verbatim}
%is handled by the expansion processor
%completely analogous to
%\begin{disp}\cs{iftrue} {\italic a}\cs{else} {\italic b}\cs{fi}\end{disp}
%The result is a character token, independent of its category.
另外一个例子,对于
\begin{verbatim}
\iftrue {\else }\fi
\end{verbatim}
展开处理器将使用与
\begin{disp}\cs{iftrue} {\italic a}\cs{else} {\italic b}\cs{fi}\end{disp}
完全类似的方式进行处理,结果是 \verb|{| 字符记号,与它的类别码无关。

%However, in the context of macro expansion
%the expansion  processor will
%recognize braces.
%First of all, a balanced pair of braces marks off a group of tokens
%to be passed as one argument.
%If a macro has an argument \begin{verbatim}
%\def\macro#1{ ... }
%\end{verbatim}
%one can call it with a single token, as in
%\begin{verbatim}
%\macro 1 \macro \$
%\end{verbatim}
%or with a group of tokens, surrounded by braces
%\begin{verbatim}
%\macro {abc} \macro {d{ef}g}
%\end{verbatim}
但是在宏展开的环境中,展开处理器需要识别和处理花括号。
首先,配对的花括号可以标定一组记号用于形成一个参量,
例如下面这个带有 1 个参数的宏:
\begin{verbatim}
\def\macro#1{ ... }
\end{verbatim}
你可以使用单个记号作为参量调用这个宏,如下:
\begin{verbatim}
\macro 1 \macro \$
\end{verbatim}
也可以使用配对花括号包围的一组记号作为这个宏的参量,如下:
\begin{verbatim}
\macro {abc} \macro {d{ef}g}
\end{verbatim}

%Secondly, when the arguments for a macro with
%parameters are read, no expressions with unbalanced braces
%are accepted. In
%\begin{verbatim}
%\def\a#1\stop{ ... }
%\end{verbatim}
%the argument consists of all
%tokens up to the first occurrence of \cs{stop}
%that is not in braces: in
%\begin{verbatim}
%\a bc{d\stop}e\stop
%\end{verbatim}
%the argument of~\cs{a} is \verb>bc{d\stop}e>.
%Only balanced expressions
%are accepted here.
其次,对于带参数的宏,未配对花括号内的表达式不能形成宏参量。例如:
\begin{verbatim}
\def\a#1\stop{ ... }
\end{verbatim}
它的参量由第一次出现并且不在配对花括号内的 \cs{stop} 之前的记号组成:
对于
\begin{verbatim}
\a bc{d\stop}e\stop
\end{verbatim}
其中 \cs{a} 的参量是 \verb>bc{d\stop}e>。这里只接受平衡表达式作为参量。

%\section{The execution processor}
\section{执行处理器}

%The execution processor builds lists: horizontal, vertical,
%and math lists. Corresponding to these lists, it works
%in horizontal, vertical, or math mode. Of these three modes
%`internal' and `external' variants exist.
%In addition to building lists, this part of the \TeX\ processor
%also performs mode-independent processing, such as
%assignments.
执行处理器用于构建水平、竖直和数学列表。
与这些列表相应,执行处理器在水平、竖直和数学模式中运行。
这三种模式每种都有`内部的'和`外部的'两个类型。
执行处理器在构建列表的过程中,还需要进行一些与模式无关的操作,
例如赋值。

%Coming out of the expansion processor is a stream of
%unexpandable tokens to be processed by
%the execution processor.
%\relax From the point of view of the execution processor, this
%stream contains two types of tokens:
%\begin{itemize}
%\item Tokens signalling an assignment (this includes
%      macro definitions), and
%      other tokens signalling actions
%      that are independent of the mode, such
%      as \cs{show} and \cs{aftergroup}.
%\item Tokens that build lists:
%      characters, boxes, and glue. The way they are handled
%      depends on the current mode.
%\end{itemize}
执行处理器的输入来自展开处理器的输出,是一个不可展开的记号流。
从执行处理器的角度来看,这条记号流所包含的记号有两种类型:
\begin{itemize}
\item 用于赋值的记号(包括宏定义)以及像 \cs{show}、\cs{aftergroup}
  这样执行与模式无关操作的记号。
\item 用于构建列表的记号:字符、盒子和粘连。
  对这些记号的处理方式依赖执行处理器当前处于的模式。
\end{itemize}

%Some objects can be used in any mode; for instance boxes
%can appear in horizontal, vertical, and math lists.
%The effect of such an object will of course still depend on the mode.
%Other objects are  specific for one mode.
%For instance, characters (to be more precise:
%character tokens of categories 11 and~12),
%are intimately connected to horizontal mode:
%if the execution processor
%is in vertical mode when it encounters a character, it will
%switch to horizontal mode.
有些记号可以用于任何模式,例如盒子既可以出现在水平模式、竖直模式,
也可以出现在数学模式中,但是这些对象的作用与效果需要依赖于具体的模式。
其他记号是模式专用的,例如字符记号(确切的说是类别码为 11 和 12 的字符记号)%
只能用于水平模式,这意味着:当执行处理器在竖直模式中遇到字符记号时,
便会转入水平模式中工作。

%Not all character tokens signal characters to be typeset:
%the execution processor can also encounter math shift
%characters (by default~\n{\char`\$}) and beginning/end of group
%characters (by default \n{\char`\{} and~\n{\char`\}}).
%Math shift characters let \TeX\ enter or exit
%math mode, and braces let it enter or exit a~new level of
%grouping.
并非所有的字符记号都是可排印的,例如在 \TeX\ 的默认状态中,
执行处理器会将 \n{\char`\$} 作为数学模式的切换符,
并将\n{\char`\{} 和\n{\char`\}} 作为编组的起止符。
数学模式切换符用于告知执行处理器进入和退出数学模式,
而花括号让执行处理器进入和退出一个编组。

%One control sequence handled by the execution processor
%deserves special mention: \cs{relax}.
%This control sequence is not expandable, but the execution
%is to do nothing. Compare the effect of \cs{relax} in
%\begin{verbatim}
%\count0=1\relax 2
%\end{verbatim}
%with that of \cs{empty}
%defined by
%\begin{verbatim}
%\def\empty{}
%\end{verbatim}
%in
%\begin{verbatim}
%\count0=1\empty 2
%\end{verbatim}
%In the first case the expansion
%process that is forming the number stops at \cs{relax} and
%the number {\tt 1} is assigned; in the second case
%\cs{empty} expands to nothing, so {\tt 12} is assigned.
控制序列 \cs{relax} 需要在此关注一下,
它是横跨展开处理器与执行处理器两界的特殊公民,
在展开处理器中它是不可展开的,在执行处理器中它什么也不执行,
但是它并非一无是处,可以比较下面的两个示例的效果,
从中发现 \cs{relax} 的用途。\par 示例 1:
\begin{verbatim}
\count0=1\relax 2
\end{verbatim}
\par 示例 2:
\begin{verbatim}
\def\empty{}
\count0=1\empty 2
\end{verbatim}
这两个示例都是在为计数寄存器赋值,但是示例 1 赋的值为 {\tt 1},
而示例 2 赋的值为 {\tt 12}。这是因为在示例 1 中,\cs{relax}
在执行处理器获得数值 {\tt 1} 的时候阻断了它进一步获取数值 {\tt 2},
而在示例 2 中 \cs{empty} 的展开结果为空,执行处理器轻而易举地继 {\tt 1}
之后就拿到了{\tt 2},所以形成{\tt 12}。

%\section{The visual processor}
\section{可视化处理器}

%\TeX's output processor encompasses those algorithms that
%are outside direct user control: paragraph breaking,
%alignment, page breaking, math typesetting, and \n{dvi} file
%generation. Various parameters control the operation
%of these parts of \TeX.
\TeX\ 的可视化处理器包含了用户不可直接控制的一些算法,
用于处理断行、阵列、分页、数学排版以及 \n{dvi} 文件生成等。
用户可以通过一些参数间接控制 \TeX\ 的这部分操作。

%Some of these algorithms return their results in a form that
%can be handled by the execution processor. For instance,
%a paragraph that has been broken into lines is added to
%the main vertical list as a sequence of horizontal boxes
%with intermediate glue and penalties. Also, the page breaking
%algorithm stores its result in \cs{box255}, so output
%routines can dissect it. On the other hand, a math formula
%can not be broken into pieces, and, naturally,
%shipping a box to the \n{dvi} file is irreversible.
可视化处理器中有一部分算法返回的是可被执行处理器处理的结果。
例如,已完成断行的段落是一组带有行间粘连和惩罚的水平盒子,
并被添加到主竖直列中。再者,分页算法会将其处理结果存储在
\cs{box255} 中,以使输出例程能够产生页面。另一方面,
数学公式不可以被分解,而输送至 \n{dvi} 文件的盒子也是不可逆的。

%\section{Examples}
\section{示例}

%\subsection{Skipped spaces}
\subsection{被忽略的空格}

%Skipped spaces provide an illustration of the view that
%\TeX's levels of processing accept the completed input
%of the previous level. Consider the commands
%\begin{verbatim}
%\def\a{\penalty200}
%\a 0
%\end{verbatim}
%This is {\italic not\/} equivalent to
%\begin{verbatim}
%\penalty200 0
%\end{verbatim}
%which would place a penalty of \n{200}, and
%typeset the digit~\n0. Instead it expands to
%\begin{verbatim}
%\penalty2000
%\end{verbatim}
%because the space after \cs{a} is skipped in the
%input processor. Later stages of processing then receive
%the sequence \begin{verbatim}
%\a0
%\end{verbatim}
被忽略的空格可以反映数据在 \TeX\ 各层处理器之间的流动情况。例如:
\begin{verbatim}
\def\a{\penalty200}
\a 0
\end{verbatim}
展开的结果{\italic 并非}是(这将放置值为 \n{200} 的惩罚项,并排印数字 \n0)
\begin{verbatim}
\penalty200 0
\end{verbatim}
而是
\begin{verbatim}
\penalty2000
\end{verbatim}
这是由于 \cs{a} 后的空格会被输入处理器忽略,从而展开处理器所得到的控制序列是
\begin{verbatim}
\a0
\end{verbatim}

%\subsection{Internal quantities and their representations}
\subsection{内部量值及其表示}

%\TeX\ uses various sorts of internal quantities,
%such as integers and dimensions. These internal
%quantities have an external representation,
%which is a string of characters, such as
%\n{4711} or~\n{91.44cm}.
\TeX\ 拥有多种内部量值,诸如整数和尺寸。这些内部量值的外部表示方法只有一种,
那就是字符串表示,例如 \n{4711} 或者 \n{91.44cm}。

%Conversions between the internal value and the external
%representation take place on two different levels,
%depending on what direction the conversion goes.
%A~string of characters is converted to an internal
%value in assignments such as
%\begin{verbatim}
%\pageno=12 \baselineskip=13pt
%\end{verbatim}
%or statements such as
%\begin{verbatim}
%\vskip 5.71pt
%\end{verbatim}
%and all of these statements are handled by the execution
%processor.
内部量值与外部表示之间的转换分别发生在两个不同的层面,具体依赖于转换的方向。
对于字符串转换为内部量值,例如:
\begin{verbatim}
\pageno=12 \baselineskip=13pt
\end{verbatim}
或者
\begin{verbatim}
\vskip 5.71pt
\end{verbatim}
像这样的语句会在执行处理器中被处理。

%On the other hand, the conversion of the internal
%values into a representation as a string of
%characters is handled by the expansion processor.
%For instance, \begin{verbatim}
%\number\pageno \romannumeral\year
%\the\baselineskip
%\end{verbatim}
%are all processed by expansion.
另一方面,内部量值到外部表示的转换是由展开处理器完成的。例如:
\begin{verbatim}
\number\pageno \romannumeral\year
\the\baselineskip
\end{verbatim}
这些语句会被展开处理器处理为内部量值的字符串记号。

%As a final example, suppose \verb>\count2=45>, and
%consider the statement
%\begin{verbatim}
%\count0=1\number\count2 3
%\end{verbatim}
%The expansion processor tackles \verb>\number\count2>
%to give the characters \n{45}, and the space after
%the \n 2 does not end the number being assigned:
%it only serves as a delimiter
%of the number of the \cs{count} register.
%In the next stage of processing, the execution processor
%will then see the statement
%\begin{verbatim}
%\count0=1453
%\end{verbatim}
%and execute this.
最后一个例子,假设 \verb>\count2=45>,看下面的语句:
\begin{verbatim}
\count0=1\number\count2 3
\end{verbatim}
展开处理器可将 \verb>\number\count2> 展开为字符串 \n{45},
而 \n2 之后的空格并不会结束正在赋予的数值:
它只用于定界 \cs{count} 寄存器的数字。
从而下一层级的执行处理器看到的是:
\begin{verbatim}
\count0=1453
\end{verbatim}
于是它便奉命行事。

%%\endinput
%\endinput

%\endofchapter
%%%%% end of input file [bigpic]
\endofchapter
%%%% end of input file [bigpic]

\end{document}
