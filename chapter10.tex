% -*- coding: utf-8 -*-
\documentclass{book}

% -*- coding: utf-8 -*-

\usepackage[b5paper,text={5in,8in},centering]{geometry}

\usepackage[CJKchecksingle]{xeCJK}
\setmainfont[Mapping=tex-text]{TeX Gyre Schola}
%\setsansfont{URW Gothic L Book}
%\setmonofont{Nimbus Mono L}
\setCJKmainfont[BoldFont=FandolHei,ItalicFont=FandolKai]{FandolSong}
\setCJKsansfont{FandolHei}
\setCJKmonofont{FandolFang}
\xeCJKsetup{PunctStyle = kaiming}

\linespread{1.25}
\setlength{\parindent}{2em}

\usepackage{xcolor}
\definecolor{myblue}{rgb}{0,0.2,0.6}

\usepackage{titlesec}
\titleformat{\chapter}
    {\normalfont\Huge\sffamily\color{myblue}}
    {第\thechapter 章}
    {1em}
    {}
%\titlespacing{\chapter}{0pt}{50pt}{40pt}
\titleformat{\section}
    {\normalfont\Large\sffamily\color{myblue}}
    {\thesection}
    {1em}
    {}
%\titlespacing{\section}{0pt}{3.5ex plus 1ex minus .2ex}{2.3ex plus .2ex}
\titleformat{\subsection}
    {\normalfont\large\sffamily\color{myblue}}
    {\thesubsection}
    {1em}
    {}
%\titlespacing{\subsection}{0pt}{3.25ex plus 1ex minus .2ex}{1.5ex plus .2ex}
%
\newpagestyle{special}[\small\sffamily]{
  \headrule
  \sethead[\usepage][][\chaptertitle]
  {\chaptertitle}{}{\usepage}}
\newpagestyle{main}[\small\sffamily]{
  \headrule
  \sethead[\usepage][][第\thechapter 章\quad\chaptertitle]
  {\thesection\quad\sectiontitle}{}{\usepage}}

\usepackage{titletoc}
%\setcounter{tocdepth}{1}
%\titlecontents{标题层次}[左间距]{上间距和整体格式}{标题序号}{标题内容}{指引线和页码}[下间距]
\titlecontents{chapter}[1.5em]{\vspace{.5em}\bfseries\sffamily}{\color{myblue}\contentslabel{1.5em}}{}
    {\titlerule*[20pt]{$\cdot$}\contentspage}[]
\titlecontents{section}[4.5em]{\sffamily}{\color{myblue}\contentslabel{3em}}{}
    {\titlerule*[20pt]{$\cdot$}\contentspage}[]
%\titlecontents{subsection}[8.5em]{\sffamily}{\contentslabel{4em}}{}
%    {\titlerule*[20pt]{$\cdot$}\contentspage}

\usepackage{enumitem}
\setlist{topsep=2pt,itemsep=2pt,parsep=1pt,leftmargin=\parindent}

\usepackage{fancyvrb}
\DefineVerbatimEnvironment{verbatim}{Verbatim}
  {xleftmargin=2em,baselinestretch=1,formatcom=\color{teal}\upshape}

\usepackage{etoolbox}
\makeatletter
\preto{\FV@ListVSpace}{\topsep=2pt \partopsep=0pt }
\makeatother

\usepackage[colorlinks,plainpages,pagebackref]{hyperref}
\hypersetup{
   pdfstartview={FitH},
   citecolor=teal,
   linkcolor=myblue,
   urlcolor=black,
   bookmarksnumbered
}

\usepackage{comment,makeidx,multicol}

%\usepackage{german}
%% german
%\righthyphenmin=3
%\mdqoff
%\captionsenglish
\usepackage[english]{babel}
{\catcode`"=13 \gdef"#1{\ifx#1"\discretionary{}{}{}\fi\relax}}
\def\mdqon{\catcode`"=13\relax}
\def\mdqoff{\catcode`"=12\relax}
\makeindex
\hyphenation{ex-em-pli-fies}

\newdimen\tempdima \newdimen\tempdimb

% these are fine
\def\handbreak{\\ \message{^^JManual break!!!!^^J}}
\def\nl{\protect\\}\def\n#1{{\tt #1}}
\def\cs#1{\texorpdfstring{{\tt\char`\\#1}}{\textbackslash#1}} %\def\cs#1{{\tt\char`\\#1}}
\let\csc\cs
\def\lb{{\tt\char`\{}}\def\rb{{\tt\char`\}}}
\def\gr#1{\texorpdfstring{$\langle$#1$\rangle$}{<#1>}} %\def\gr#1{$\langle$#1$\rangle$}
\def\key#1{{\tt#1}}
\def\alt{}\def\altt{}%this way in manstijl
\def\ldash{\unskip\ --\nobreak\ \ignorespaces}
\def\rdash{\unskip\nobreak\ --\ \ignorespaces}
% check these
\def\hex{{\tt"}}
\def\ascii{{\sc ascii}}
\def\ebcdic{{\sc ebcdic}}
\def\IniTeX{Ini\TeX}\def\LamsTeX{LAMS\TeX}\def\VirTeX{Vir\TeX}
\def\AmsTeX{Ams\TeX}
\def\TeXbook{the \TeX\ book}\def\web{{\sc web}}
% needs major thinking
\newenvironment{myquote}{\list{}{%
    \topsep=2pt \partopsep=0pt%
    \leftmargin=\parindent \rightmargin=\parindent
    }\item[]}{\endlist}
\newenvironment{disp}{\begin{myquote}}{\end{myquote}}
\newenvironment{Disp}{\begin{myquote}}{\end{myquote}}
\newenvironment{tdisp}{\begin{myquote}}{\end{myquote}}
\newenvironment{example}{\begin{myquote}\noindent\itshape 例子:}{\end{myquote}}
\newenvironment{inventory}{\begin{description}\raggedright}{\end{description}}
\newenvironment{glossinventory}{\begin{description}}{\end{description}}
\def\gram#1{\gr{#1}}%???
%
% index
%
\def\indexterm#1{\emph{#1}\index{#1}}
\def\indextermsub#1#2{\emph{#1 #2}\index{#1!#2}}
\def\indextermbus#1#2{\emph{#1 #2}\index{#2!#1}}
\def\term#1\par{\index{#1}}
\def\howto#1\par{}
\def\cstoidx#1\par{\index{#1@\cs{#1}@}}
\def\thecstoidx#1\par{\index{#1@\protect\csname #1\endcsname}}
\def\thecstoidxsub#1#2{\index{#1, #2@\protect\csname #1\endcsname, #2}\ignorespaces}
\def\csterm#1\par{\cstoidx #1\par\cs{#1}}
\def\csidx#1{\cstoidx #1\par\cs{#1}}

\def\tmc{\tracingmacros=2 \tracingcommands\tracingmacros}

%%%%%%%%%%%%%%%%%%%
\makeatletter
\def\snugbox{\hbox\bgroup\setbox\z@\vbox\bgroup
    \leftskip\z@
    \bgroup\aftergroup\make@snug
    \let\next=}
\def\make@snug{\par\sn@gify\egroup \box\z@\egroup}
\def\sn@gify
   {\skip\z@=\lastskip \unskip
    \advance\skip\z@\lastskip \unskip
    \unpenalty
    \setbox\z@\lastbox
    \ifvoid\z@ \nointerlineskip \else {\sn@gify} \fi
    \hbox{\unhbox\z@}\nointerlineskip
    \vskip\skip\z@
    }

\newdimen\fbh \fbh=60pt % dimension for easy scaling:
\newdimen\fbw \fbw=60pt % height and width of character box

\newdimen\dh \newdimen\dw % height and width of current character box
\newdimen\lh % height of previous character box
\newdimen\lw \lw=.4pt % line weight, instead of default .4pt

\def\hdotfill{\noindent
    \leaders\hbox{\vrule width 1pt height\lw
                  \kern4pt
                  \vrule width.5pt height\lw}\hfill\hbox{}
    \par}
\def\hlinefill{\noindent
    \leaders\hbox{\vrule width 5.5pt height\lw         }\hfill\hbox{}
    \par}
\def\stippel{$\qquad\qquad\qquad\qquad$}
\makeatother
%%%%%%%%%%%%%%%%%%%

%\def\SansSerif{\Typeface:macHelvetica }
%\def\SerifFont{\Typeface:macTimes }
%\def\SansSerif{\Typeface:bsGillSans }
%\def\SerifFont{\Typeface:bsBaskerville }
\let\SansSerif\relax \def\italic{\it}
\let\SerifFont\relax \def\MainFont{\rm}
\let\SansSerif\relax
\let\SerifFont\relax
\let\PopIndentLevel\relax \let\PushIndentLevel\relax
\let\ToVerso\relax \let\ToRecto\relax

%\def\stop@command@suffix{stop}
%\let\PopListLevel\PopIndentLevel
%\let\FlushRight\relax
%\let\flushright\FlushRight
%\let\SetListIndent\LevelIndent
%\def\awp{\ifhmode\vadjust{\penalty-10000 }\else
%    \penalty-10000 \fi}
\let\awp\relax
\let\PopIndentLevel\relax \let\PopListLevel\relax

\showboxdepth=-1

%\input figs
\def\endofchapter{\vfill\noindent}

\setcounter{chapter}{9}

\begin{document}

%\chapter{Grouping}\label{group}
\chapter{编组}\label{group}

%\TeX\ has a grouping mechanism that is able to confine most
%changes to a~particular locality. This chapter explains
%what sort of actions can be local, and how groups are formed.
\TeX\ 提供的编组机制可以将大多数改变限制在局部范围内。
此章解释哪些操作是局部的,以及编组是如何形成的。

%\label{cschap:bgroup}\label{cschap:egroup}\label{cschap:begingroup}\label{cschap:endgroup}\label{cschap:aftergroup}\label{cschap:global}\label{cschap:globaldefs}
%\begin{inventory}
%\item [\cs{bgroup}] 
%Implicit beginning of group character.
%\item [\cs{egroup}] 
%Implicit end of group character.
%\item [\cs{begingroup}] 
%Open a group that must be closed with \cs{endgroup}.
%\item [\cs{endgroup}] 
%Close a group that was opened with \cs{begingroup}.
%\item [\cs{aftergroup}] 
%Save the next token for insertion after the current group ends.
%\item [\cs{global}] 
%Make assignments, macro definitions, and arithmetic global.
%\item [\cs{globaldefs}] 
%Parameter for overriding \cs{global} prefixes.
%\IniTeX\ default:~0.
%\end{inventory}
\label{cschap:bgroup}\label{cschap:egroup}\label{cschap:begingroup}\label{cschap:endgroup}%
\label{cschap:aftergroup}\label{cschap:global}\label{cschap:globaldefs}
\begin{inventory}
\item [\cs{bgroup}] 
隐式的编组开始符。
\item [\cs{egroup}] 
隐式的编组结束符。
\item [\cs{begingroup}] 
开始一个必须用 \cs{endgroup} 结束的编组。
\item [\cs{endgroup}] 
结束一个用\cs{begingroup}开始的编组。
\item [\cs{aftergroup}] 
保存下一个记号,并在当前编组结束后插入它。
\item [\cs{global}] 
使得赋值,宏定义和算术运算是全局的。
\item [\cs{globaldefs}] 
用于覆盖 \cs{global} 前缀的参数。
\IniTeX\ 默认值:0。
\end{inventory}

%%\point The grouping mechanism
%\section{The grouping mechanism}
%\point The grouping mechanism
\section{编组机制}

%A group is a sequence of tokens starting with a
%`beginning of group' token,
%and ending with an `end of group'
%token, and in which all such tokens are properly balanced. 
一个编组是一串记号,以`组开始'记号开头,以`组结束'记号结尾,
而且这两种记号在其中是配对的。

%The \indexterm{grouping} mechanism of \TeX\ is not the same as
%the block structured `scoping'
%of ordinary programming languages.
%Most languages with block structure are only able  to have
%local definitions. \TeX's grouping mechanism is stronger: 
%most assignments made inside a group
%are local to that group unless explicitly indicated otherwise,
%and outside the group old values are restored.
\TeX\ 的\indexterm{编组}机制与普通编程语言的块作用域不同。
大多数语言的块结构中只能有局部定义。
\TeX\ 的编组机制更加强大:编组中的大多数赋值是局部的,
在离开该编组后旧的值将被还原,但可以显式指明全局赋值。

%An example of local definitions
%\begin{verbatim}
%{\def\a{b}}\a
%\end{verbatim}
%gives an `undefined control sequence'
%message because \cs{a} is only defined inside the group.
%Similarly, the code
%\begin{verbatim}
%\count0=1 {\count0=2 } \showthe\count0
%\end{verbatim}
%will display the value~1; the assignment made inside the group
%is undone at the end of the group.
下面局部定义的例子
\begin{verbatim}
{\def\a{b}}\a
\end{verbatim}
将给出 `undefined control sequence' 的错误,
因为 \cs{a} 是在编组内部定义的,类似地,下列代码
\begin{verbatim}
\count0=1 {\count0=2 } \showthe\count0
\end{verbatim}
显示的值为~1;编组内的赋值在该编组结束后将被撤销。

%Bookkeeping of values that are to be restored outside the group
%is done through the mechanism
%of the \indexterm{save stack}. Overflow of the save stack is treated
%in Chapter~\ref{error}. The save stack is also used for
%a few other purposes: in calls such as \hbox{\verb>\hbox to 100pt{...}>}
%the specification \hbox{\n{to 100pt}} is put on the save
%stack before a new level of grouping is opened.
对编组结束时要还原的值的记录是通过\indexterm{保存堆栈}的结构来实现的。
保存堆栈的溢出问题在第~\ref{error}~章中介绍。
这个保存堆栈还有另外几个用途:比如在调用 \hbox{\verb>\hbox to 100pt{...}>} 时,
在开始新层级的编组之前,所指定的 \hbox{\n{to 100pt}} 将被放入保存堆栈中。

%In order to prevent a lot of trouble with the save stack,
%\IniTeX\ does not allow dumping a format inside a group.
%The \cs{end} command is allowed to occur inside a group,
%but \TeX\ will give a diagnostic message about this.
为避免给保存堆栈造成各种麻烦,
\IniTeX\ 不允许在编组内用 \cs{dump} 命令转储格式文件。
\cs{end} 命令可以出现在编组内,但 \TeX\ 将对此给出一个诊断信息。

%The \cs{aftergroup} control sequence saves a token for
%insertion after the current group. Several tokens can be
%set aside by this command, and they are inserted in the left-to-right
%order in which they were stated.
%This is treated in Chapter~\ref{expand}.
控制系列 \cs{aftergroup} 保存一个记号,并在当前编组结束后插入它。
此命令可以放置多个记号,这些记号将按出现的先后顺序被插入。
此命令将在第~\ref{expand}~章中讨论.


%%\point[global:assign] Local and global assignments
%\section{Local and global assignments}
%\index{assignment!global|(}
%\index{assignment!local|(}
%\label{global:assign}
%\point[global:assign] Local and global assignments
\section{局部和全局赋值}
\index{assignment!global|(}
\index{assignment!local|(}
\label{global:assign}

%An assignment or macro definition
%is usually made global by prefixing it with \csidx{global},
%but non-zero values of the \gr{integer parameter}
%\csidx{globaldefs} override \cs{global}
%specifications: if \cs{globaldefs} is positive every assignment
%is implicitly prefixed with \cs{global}, and if
%\cs{globaldefs} is negative, \cs{global} is
%ignored. Ordinarily this parameter is zero.
给赋值或宏定义加上 \csidx{global} 前缀,通常可让它们成为全局的,
但 \gr{integer parameter} \csidx{globaldefs} 的非零值将覆盖\cs{global} 指定:
如果 \cs{globaldefs} 为正数,每个赋值都被隐式地添加 \cs{global} 前缀,
而如果 \cs{globaldefs} 为负数,\cs{global} 将会被忽略。
通常这个参数的取值为零。

%Some assignment are always global: the \gr{global assignment}s are
%\begin{description}%\FlushRight:no
%\item [\gr{font assignment}]
%assignments involving \cs{fontdimen}, \cs{hyphenchar}, 
%and \cs{skew\-char}.
%\item [\gr{hyphenation assignment}]
%\cs{hyphenation} and \cs{patterns} commands
%(see Chapter~\ref{line:break}).
%\item [\gr{box size assignment}]
%altering box dimensions with \cs{ht}, \cs{dp}, and~\cs{wd}
%(see Chapter~\ref{boxes}).
%\item [\gr{interaction mode assignment}]
%run modes for a \TeX\ job (see Chapter~\ref{run}).
%\item [\gr{intimate assignment}]
%assignments to a \gr{special integer} or \gr{special dimen};
%see %Chapters \ref{number} and~\ref{glue}.
%pages \pageref{special:int:list} and~\pageref{special:dimen:list}.
%\end{description}
有些赋值总是全局的:所有的 \gr{global assignment} 如下:
\begin{description}%\FlushRight:no
\item [\gr{font assignment}]
\cs{fontdimen}、\cs{hyphenchar} 和 \cs{skew\-char} 的赋值。
\item [\gr{hyphenation assignment}]
\cs{hyphenation} 和 \cs{patterns} 命令(见第~\ref{line:break}~章)。
\item [\gr{box size assignment}]
用 \cs{ht}、\cs{dp} 和 \cs{wd} 修改盒子尺寸(见第~\ref{boxes}~章)。
\item [\gr{interaction mode assignment}]
\TeX\ 任务的运行模式(见第~\ref{run}~章)。
\item [\gr{intimate assignment}]
对 \gr{special integer} 或 \gr{special dimen} 的赋值;
见 %Chapters \ref{number} and~\ref{glue}.
第~\pageref{special:int:list} 和 \pageref{special:dimen:list} 页。
\end{description}

%\index{assignment!global|)}
%\index{assignment!local|)}
\index{assignment!global|)}
\index{assignment!local|)}

%\section{Group delimiters}
\section{编组定界符}

%The open and closing \indextermsub{group}{delimiters} can be character
%tokens of category code~1\index{category!1} for `beginning of group'
%and code~2\index{category!2} for `end of group'
%(\indextermbus{explicit}{braces}), or control sequence tokens that are
%\cs{let} to such characters (\indextermbus{implicit}{braces}), such as
%the \cs{bgroup} and \cs{egroup} in plain \TeX.  Implicit and explicit
%braces can match to delimit a group; see for instance the example in
%section~\ref{sec:aftergroup}.
\indextermsub{编组}{定界符}可以由类别码为 1 \index{category!1}的`组开始符'%
和类别码为 2 \index{category!2}的`组结束符' 组成(\indextermbus{显式}{花括号}),
或者用 \cs{let} 等价到这两类字符的控制序列组成(\indextermbus{隐式}{花括号}),
正如 plain \TeX\ 中的 \cs{bgroup} 和 \cs{egroup}。
隐式和显式花括号可以配对以定界一个编组;比如见第~\ref{sec:aftergroup}~节的例子。

%Groups can also be delimited by \csidx{begingroup} and
%\csidx{endgroup}. These two control sequences must
%be used together: they cannot be matched with implicit
%or explicit braces, nor can they function as the braces
%surrounding, for instance, boxed material.
编组也可以用 \csidx{begingroup} 和 \csidx{endgroup} 定界。
这两个控制序列必须一起使用,它们不能与隐式或显式花括号配对,
也不能像花括号那样用于围住盒内素材等。

%Delimiting with \cs{begingroup} and \cs{endgroup} can
%\label{begin:end:macros}%
%provide a limited form of run-time error checking. 
%In between these two group delimiters an excess
%open or close brace would result in
%\begin{verbatim}
%\begingroup ... } ... \endgroup
%\end{verbatim}
%or
%\begin{verbatim}
%\begingroup ... { ... \endgroup
%\end{verbatim}
%In both cases \TeX\ gives an error message about improper
%balancing. Using \cs{bgroup} and \cs{egroup} here would
%make an error much harder to find, because of the incorrect
%matching that would occur. This idea is used in the environment
%macros of several formats.
用 \cs{begingroup} 和 \cs{endgroup} 定界%
\label{begin:end:macros}%
可以提供一种有限的运行时错误检测。
在这两个编组定界符之间一个多余的开或闭花括号将有
\begin{verbatim}
\begingroup ... } ... \endgroup
\end{verbatim}
或者
\begin{verbatim}
\begingroup ... { ... \endgroup
\end{verbatim}
的形式。对这两种情形 \TeX\ 都将给出括号不配对的错误信息。
这里若改用 \cs{bgroup} 和 \cs{egroup} 将使得错误更加难以发现,
因为出现了不正确的配对。这个想法在多种环境宏中用到。

%The choice of the brace characters for the beginning and end of group
%characters is not hard-wired in \TeX. It is arranged
%\cstoidx bgroup\par\cstoidx egroup\par
%like this in the plain format:
%\begin{verbatim}
%\catcode`\{=1 % left brace is begin-group character
%\catcode`\}=2 % right brace is end-group character
%\end{verbatim}
%Implicit braces have also been defined in the plain format:
%\begin{verbatim}
%\let\bgroup={ \let\egroup=}
%\end{verbatim}
\TeX\ 并没有硬性规定花括号作为组开始和结束符。
在 plain 格式中,它是类似这样编写的:
\cstoidx bgroup\par\cstoidx egroup\par
\begin{verbatim}
\catcode`\{=1 % left brace is begin-group character
\catcode`\}=2 % right brace is end-group character
\end{verbatim}
隐式花括号也在 plain 格式中定义:
\begin{verbatim}
\let\bgroup={ \let\egroup=}
\end{verbatim}

%Special cases are the following:
%\begin{itemize} \item The replacement text of a macro must be enclosed
%in  explicit beginning and end of group character tokens.
%\item  The open and close braces for boxes, \cs{vadjust},
%and \cs{insert} can be implicit. This makes it possible
%to define, for instance
%\begin{verbatim}
%\def\openbox#1{\setbox#1=\hbox\bgroup}
%\def\closebox#1{\egroup\box#1}
%\openbox{15}Foo bar\closebox{15}
%\end{verbatim}
%\item The right-hand side of a token list assignment and the
%argument of the commands \cs{write}, \cs{message}, \cs{errmessage}, 
%\cs{uppercase}, \cs{lowercase}, 
%\cs{special}, and \cs{mark} is a \gr{general text}, defined
%as
%\begin{Disp} \gr{general text} $\longrightarrow$ \gr{filler}\lb
%      \gr{balanced text}\gr{right brace}\end{Disp}
%meaning that the left brace can be implicit, but the closing
%right brace must be an explicit character token with category
%code~2\index{category!2}.
%\end{itemize}
下面列出一些特殊情形:
\begin{itemize}
\item 宏的替换文本必须用显式组开始符和组结束符围住。
\item 盒子、\cs{vadjust} 和 \cs{insert} 的开始和结束花括号可以是隐式的。
这样就可能像下面这样定义
\begin{verbatim}
\def\openbox#1{\setbox#1=\hbox\bgroup}
\def\closebox#1{\egroup\box#1}
\openbox{15}Foo bar\closebox{15}
\end{verbatim}
\item 记号列赋值的右边,以及命令 \cs{write}、\cs{message}、\cs{errmessage}、 
\cs{uppercase}、\cs{lowercase}、\cs{special} 和 \cs{mark} 的参量是
\gr{general text},定义为
\begin{Disp} \gr{general text} $\longrightarrow$ \gr{filler}\lb
      \gr{balanced text}\gr{right brace}\end{Disp}
这意味着左花括号可以是隐式的,但右花括号必须是类别码 2 \index{category!2}
的显式字符记号。
\end{itemize}

%In cases where an implicit left brace suffices, and where
%expansion is not explicitly inhibited, \TeX\ will
%expand tokens until a left brace is encountered. This
%is the basis for such constructs as
%\verb=\uppercase\expandafter{\romannumeral80}=,
%which in this unexpanded form do not adhere to the
%syntax. If the first unexpandable token is not a left
%brace \TeX\ gives an error message.
在可以使用隐式花括号的情形中,如果展开没被显式禁止,
\TeX\ 将展开记号直到遇到一个左花括号。
这是 \verb=\uppercase\expandafter{\romannumeral80}= 这种写法的原理,
在未展开之前这样的写法并不符合语法。
如果第一个不可展开的记号不是一个左花括号,
\TeX\ 将给出一个错误信息。

%The grammar of \TeX\ (see Chapter~\ref{gramm})  uses
%\gr{left brace} and \gr{right brace} for explicit
%characters, that is, character tokens,
%and \n{\lb} and~\n{\rb} 
%for possibly implicit characters,
%\altt
%that is, control sequences that have been \cs{let} to such
%explicit characters.
在 \TeX\ 的文法(见第~\ref{gramm}~章)中,
用 \gr{left brace} 和 \gr{right brace} 表示显式字符,即字符记号;
而用 \n{\lb} 和 \n{\rb} 表示可能为隐式的字符,
即已经用 \cs{let} 等价到这些显式字符的控制序列。

%\section{More about braces}
%\index{braces|(}
\section{花括号进阶}
\index{braces|(}

%\subsection{Brace counters}
\subsection{花括号计数器}

%\TeX\ has two counters for keeping  track of grouping levels:
%the {\it master counter} and the {\it balance counter}.
%Both of these counters are syntactic counters: they count the
%explicit brace character tokens, but are not affected by implicit
%braces (such as \cs{bgroup}) that are semantically equivalent
%to an explicit brace.
\TeX\ 有两个计数器用于记录编组的层级:{\it 主计数器}和{\it 平衡计数器}。
%\footnote{译注:见 The \TeX Book 附录 D 第 5 节。}
这两个计数器都是语法计数器:它们计算显式花括号记号的个数,
但不受语义上与显式花括号等价的隐式花括号(比如 \cs{bgroup})的影响。

%The balance counter handles braces in all cases except in
%alignment. Its workings are intuitively clear: it goes up
%by one for every opening and down for every closing
%brace that is not being skipped. Thus
%\begin{verbatim}
%\iffalse{\fi
%\end{verbatim}
%increases the balance counter if
%this statement is merely scanned (for instance if it
%appears in a macro definition text); if this statement
%is executed the brace is skipped, so there is no effect on
%the balance counter.
平衡计数器处理除了阵列之外的所有情形。其工作原理直观而清晰:
对每个左花括号增加一,对每个右花括号减少一,
只要该花括号不被跳过。因此对于下面的语句
\begin{verbatim}
\iffalse{\fi
\end{verbatim}
在它仅仅被扫描时(比如它出现在宏定义文本中)将增加平衡计数器的值;
而在它被执行时花括号将被跳过,从而不会改变平衡计数器的值。

%The master counter is more tricky;
%it is used in alignments instead of the balance counter.
%This counter records all braces, even when they are skipped
%such as in \verb>\iffalse{\fi>.
%For this counter uncounted skipped braces are still possible:
%the alphabetic constants \n{`\lb} and \n{`\rb} have
%no effect on this counter when they are
%use by the execution processor as a~\gr{number};
%they do affect this counter when they are seen by the 
%input processor (which merely sees characters, and not
%the context).
主计数器更加复杂;在阵列中只使用主计数器,不使用平衡计数器。
这个计数器记录所有花括号,即使在被跳过时,像 \verb>\iffalse{\fi> 这种。
对这个计数器,未计数的已跳过花括号还是可能的:
当字母表常数 \n{`\lb} 和 \n{`\rb} 被执行处理器作为 \gr{number} 使用时,
不影响这个计数器;当它们被输入处理器查看(仅查看字符,不含上下文)时,
确实会影响这个计数器。

%%\spoint The brace as a token
%\subsection{The brace as a token}
%\spoint The brace as a token
\subsection{花括号作为记号}

%Explicit braces are character tokens, and as such they are
%unexpandable. This implies that they survive until the
%last stages of \TeX\ processing. For example,
%\begin{verbatim}
%\count255=1{2}
%\end{verbatim}
%will assign~1 to \cs{count255},
%and print~`2', because the
%opening brace functions as a delimiter for the number~1.
%Similarly
%\begin{verbatim}
%f{f}
%\end{verbatim}
%will prevent \TeX\ from forming
%an `\hbox{ff}' ligature.
显式花扩号是字符记号,而作为字符记号它们是不可展开的。
这表示它们直到 \TeX\ 处理的最后一个阶段还是存在的。例如,
\begin{verbatim}
\count255=1{2}
\end{verbatim}
将给 \cs{count255} 赋值~1,并排印~`2',
因为左花括号充当了数字 1 的定界符。类似地,
\begin{verbatim}
f{f}
\end{verbatim}
将阻止 \TeX\ 形成 `\hbox{ff}' 连写。

%From the fact that braces are unexpandable,
%it follows that their nesting is independent
%of the nesting of conditionals. For instance
%\begin{verbatim}
%\iftrue{\else}\fi
%\end{verbatim}
%will give an open brace,
%as conditionals are handled by expansion. The closing
%brace is simply skipped as part of the \gr{false text};
%any consequences it has for grouping only come into
%play in a later stage of \TeX\ processing.
从花括号不可展开的事实,可以得知它们的嵌套与条件语句的嵌套相互独立。
比如
\begin{verbatim}
\iftrue{\else}\fi
\end{verbatim}
将给出左花括号,因为条件语句是在展开时处理的。
右花括号作为 \gr{false text} 的一部分被直接跳过了;
这对编组的影响在 \TeX\ 处理的后续阶段才显现出来。

%Undelimited macro arguments are either single tokens
%or groups of tokens enclosed in explicit braces.
%Thus it is not possible for an explicit open or close brace
%to be a macro argument. However, braces can be assigned
%with \cs{let}, for instance as in
%\begin{verbatim}
%\let\bgroup={
%\end{verbatim}
%This is used in the plain \cs{footnote} macro
%(see page~\pageref{footnote:ex}).
非定界的宏参量或者是单个记号,或者是一个用显式花括号围住的编组。
因此,显式的左或右花括号不能作为宏参量。
然而,花括号可以用于 \cs{let} 赋值,就像这样
\begin{verbatim}
\let\bgroup={
\end{verbatim}
这在 plain \TeX\ 的 \cs{footnote} 宏(见第~\pageref{footnote:ex}~页)中用到。

%%\spoint \csc{\char 123} and \csc{\char 125}
%\subsection{Open and closing brace control symbols}
%% \csc{\char 123} and \csc{\char 125}}
%\spoint \csc{\char 123} and \csc{\char 125}
\subsection{花括号控制符号}
% \csc{\char 123} and \csc{\char 125}}

%The control sequences \verb-\{- and \verb-\}- do not really belong
%\cstoidx\char123\par\cstoidx\char125\par
%in this chapter,  not being concerned with grouping.
%They have been defined with \cs{let} as synonyms of
%\cs{lbrace} and \cs{rbrace} respectively,
%and these control sequences are \cs{delimiter} instructions
%(see Chapter~\ref{mathchar}).
控制序列 \verb-\{- 和 \verb-\}- 并不属于这一章,
\cstoidx\char123\par\cstoidx\char125\par
也与编组无关。它们被用 \cs{let} 分别定义为
\cs{lbrace} 和 \cs{rbrace} 的别名,
而这两个控制序列为 \cs{delimiter} 指令(见第~\ref{mathchar}~章)。

%The Computer Modern Roman font has no braces, but there are
%braces in the typewriter font, and for mathematics 
%there are braces of different sizes \ldash and extendable ones \rdash in
%the extension font.
计算机现代罗马字体中不包含花括号,但打字机字体中包含;
而对于数学公式,在扩展字体中有不同尺寸的\ldash 且可伸展的\rdash 花括号。

%\index{braces|)}
%\endofchapter
%%%%% end of input file [group]
\index{braces|)}
\endofchapter
%%%% end of input file [group]

\end{document}
