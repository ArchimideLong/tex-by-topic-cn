% -*- coding: utf-8 -*-
\documentclass[letterpaper]{book}

% -*- coding: utf-8 -*-

\usepackage[b5paper,text={5in,8in},centering]{geometry}

\usepackage[CJKchecksingle]{xeCJK}
\setmainfont[Mapping=tex-text]{TeX Gyre Schola}
%\setsansfont{URW Gothic L Book}
%\setmonofont{Nimbus Mono L}
\setCJKmainfont[BoldFont=FandolHei,ItalicFont=FandolKai]{FandolSong}
\setCJKsansfont{FandolHei}
\setCJKmonofont{FandolFang}
\xeCJKsetup{PunctStyle = kaiming}

\linespread{1.25}
\setlength{\parindent}{2em}

\usepackage{xcolor}
\definecolor{myblue}{rgb}{0,0.2,0.6}

\usepackage{titlesec}
\titleformat{\chapter}
    {\normalfont\Huge\sffamily\color{myblue}}
    {第\thechapter 章}
    {1em}
    {}
%\titlespacing{\chapter}{0pt}{50pt}{40pt}
\titleformat{\section}
    {\normalfont\Large\sffamily\color{myblue}}
    {\thesection}
    {1em}
    {}
%\titlespacing{\section}{0pt}{3.5ex plus 1ex minus .2ex}{2.3ex plus .2ex}
\titleformat{\subsection}
    {\normalfont\large\sffamily\color{myblue}}
    {\thesubsection}
    {1em}
    {}
%\titlespacing{\subsection}{0pt}{3.25ex plus 1ex minus .2ex}{1.5ex plus .2ex}
%
\newpagestyle{special}[\small\sffamily]{
  \headrule
  \sethead[\usepage][][\chaptertitle]
  {\chaptertitle}{}{\usepage}}
\newpagestyle{main}[\small\sffamily]{
  \headrule
  \sethead[\usepage][][第\thechapter 章\quad\chaptertitle]
  {\thesection\quad\sectiontitle}{}{\usepage}}

\usepackage{titletoc}
%\setcounter{tocdepth}{1}
%\titlecontents{标题层次}[左间距]{上间距和整体格式}{标题序号}{标题内容}{指引线和页码}[下间距]
\titlecontents{chapter}[1.5em]{\vspace{.5em}\bfseries\sffamily}{\color{myblue}\contentslabel{1.5em}}{}
    {\titlerule*[20pt]{$\cdot$}\contentspage}[]
\titlecontents{section}[4.5em]{\sffamily}{\color{myblue}\contentslabel{3em}}{}
    {\titlerule*[20pt]{$\cdot$}\contentspage}[]
%\titlecontents{subsection}[8.5em]{\sffamily}{\contentslabel{4em}}{}
%    {\titlerule*[20pt]{$\cdot$}\contentspage}

\usepackage{enumitem}
\setlist{topsep=2pt,itemsep=2pt,parsep=1pt,leftmargin=\parindent}

\usepackage{fancyvrb}
\DefineVerbatimEnvironment{verbatim}{Verbatim}
  {xleftmargin=2em,baselinestretch=1,formatcom=\color{teal}\upshape}

\usepackage{etoolbox}
\makeatletter
\preto{\FV@ListVSpace}{\topsep=2pt \partopsep=0pt }
\makeatother

\usepackage[colorlinks,plainpages,pagebackref]{hyperref}
\hypersetup{
   pdfstartview={FitH},
   citecolor=teal,
   linkcolor=myblue,
   urlcolor=black,
   bookmarksnumbered
}

\usepackage{comment,makeidx,multicol}

%\usepackage{german}
%% german
%\righthyphenmin=3
%\mdqoff
%\captionsenglish
\usepackage[english]{babel}
{\catcode`"=13 \gdef"#1{\ifx#1"\discretionary{}{}{}\fi\relax}}
\def\mdqon{\catcode`"=13\relax}
\def\mdqoff{\catcode`"=12\relax}
\makeindex
\hyphenation{ex-em-pli-fies}

\newdimen\tempdima \newdimen\tempdimb

% these are fine
\def\handbreak{\\ \message{^^JManual break!!!!^^J}}
\def\nl{\protect\\}\def\n#1{{\tt #1}}
\def\cs#1{\texorpdfstring{{\tt\char`\\#1}}{\textbackslash#1}} %\def\cs#1{{\tt\char`\\#1}}
\let\csc\cs
\def\lb{{\tt\char`\{}}\def\rb{{\tt\char`\}}}
\def\gr#1{\texorpdfstring{$\langle$#1$\rangle$}{<#1>}} %\def\gr#1{$\langle$#1$\rangle$}
\def\key#1{{\tt#1}}
\def\alt{}\def\altt{}%this way in manstijl
\def\ldash{\unskip\ --\nobreak\ \ignorespaces}
\def\rdash{\unskip\nobreak\ --\ \ignorespaces}
% check these
\def\hex{{\tt"}}
\def\ascii{{\sc ascii}}
\def\ebcdic{{\sc ebcdic}}
\def\IniTeX{Ini\TeX}\def\LamsTeX{LAMS\TeX}\def\VirTeX{Vir\TeX}
\def\AmsTeX{Ams\TeX}
\def\TeXbook{the \TeX\ book}\def\web{{\sc web}}
% needs major thinking
\newenvironment{myquote}{\list{}{%
    \topsep=2pt \partopsep=0pt%
    \leftmargin=\parindent \rightmargin=\parindent
    }\item[]}{\endlist}
\newenvironment{disp}{\begin{myquote}}{\end{myquote}}
\newenvironment{Disp}{\begin{myquote}}{\end{myquote}}
\newenvironment{tdisp}{\begin{myquote}}{\end{myquote}}
\newenvironment{example}{\begin{myquote}\noindent\itshape 例子:}{\end{myquote}}
\newenvironment{inventory}{\begin{description}\raggedright}{\end{description}}
\newenvironment{glossinventory}{\begin{description}}{\end{description}}
\def\gram#1{\gr{#1}}%???
%
% index
%
\def\indexterm#1{\emph{#1}\index{#1}}
\def\indextermsub#1#2{\emph{#1 #2}\index{#1!#2}}
\def\indextermbus#1#2{\emph{#1 #2}\index{#2!#1}}
\def\term#1\par{\index{#1}}
\def\howto#1\par{}
\def\cstoidx#1\par{\index{#1@\cs{#1}@}}
\def\thecstoidx#1\par{\index{#1@\protect\csname #1\endcsname}}
\def\thecstoidxsub#1#2{\index{#1, #2@\protect\csname #1\endcsname, #2}\ignorespaces}
\def\csterm#1\par{\cstoidx #1\par\cs{#1}}
\def\csidx#1{\cstoidx #1\par\cs{#1}}

\def\tmc{\tracingmacros=2 \tracingcommands\tracingmacros}

%%%%%%%%%%%%%%%%%%%
\makeatletter
\def\snugbox{\hbox\bgroup\setbox\z@\vbox\bgroup
    \leftskip\z@
    \bgroup\aftergroup\make@snug
    \let\next=}
\def\make@snug{\par\sn@gify\egroup \box\z@\egroup}
\def\sn@gify
   {\skip\z@=\lastskip \unskip
    \advance\skip\z@\lastskip \unskip
    \unpenalty
    \setbox\z@\lastbox
    \ifvoid\z@ \nointerlineskip \else {\sn@gify} \fi
    \hbox{\unhbox\z@}\nointerlineskip
    \vskip\skip\z@
    }

\newdimen\fbh \fbh=60pt % dimension for easy scaling:
\newdimen\fbw \fbw=60pt % height and width of character box

\newdimen\dh \newdimen\dw % height and width of current character box
\newdimen\lh % height of previous character box
\newdimen\lw \lw=.4pt % line weight, instead of default .4pt

\def\hdotfill{\noindent
    \leaders\hbox{\vrule width 1pt height\lw
                  \kern4pt
                  \vrule width.5pt height\lw}\hfill\hbox{}
    \par}
\def\hlinefill{\noindent
    \leaders\hbox{\vrule width 5.5pt height\lw         }\hfill\hbox{}
    \par}
\def\stippel{$\qquad\qquad\qquad\qquad$}
\makeatother
%%%%%%%%%%%%%%%%%%%

%\def\SansSerif{\Typeface:macHelvetica }
%\def\SerifFont{\Typeface:macTimes }
%\def\SansSerif{\Typeface:bsGillSans }
%\def\SerifFont{\Typeface:bsBaskerville }
\let\SansSerif\relax \def\italic{\it}
\let\SerifFont\relax \def\MainFont{\rm}
\let\SansSerif\relax
\let\SerifFont\relax
\let\PopIndentLevel\relax \let\PushIndentLevel\relax
\let\ToVerso\relax \let\ToRecto\relax

%\def\stop@command@suffix{stop}
%\let\PopListLevel\PopIndentLevel
%\let\FlushRight\relax
%\let\flushright\FlushRight
%\let\SetListIndent\LevelIndent
%\def\awp{\ifhmode\vadjust{\penalty-10000 }\else
%    \penalty-10000 \fi}
\let\awp\relax
\let\PopIndentLevel\relax \let\PopListLevel\relax

\showboxdepth=-1

%\input figs
\def\endofchapter{\vfill\noindent}

\setcounter{chapter}{9}

\begin{document}

%\chapter{Grouping}\label{group}
\chapter{Grouping}\label{group}

%\TeX\ has a grouping mechanism that is able to confine most
%changes to a~particular locality. This chapter explains
%what sort of actions can be local, and how groups are formed.
\TeX\ has a grouping mechanism that is able to confine most
changes to a~particular locality. This chapter explains
what sort of actions can be local, and how groups are formed.

%\label{cschap:bgroup}\label{cschap:egroup}\label{cschap:begingroup}\label{cschap:endgroup}\label{cschap:aftergroup}\label{cschap:global}\label{cschap:globaldefs}
%\begin{inventory}
%\item [\cs{bgroup}] 
%Implicit beginning of group character.
%\item [\cs{egroup}] 
%Implicit end of group character.
%\item [\cs{begingroup}] 
%Open a group that must be closed with \cs{endgroup}.
%\item [\cs{endgroup}] 
%Close a group that was opened with \cs{begingroup}.
%\item [\cs{aftergroup}] 
%Save the next token for insertion after the current group ends.
%\item [\cs{global}] 
%Make assignments, macro definitions, and arithmetic global.
%\item [\cs{globaldefs}] 
%Parameter for overriding \cs{global} prefixes.
%\IniTeX\ default:~0.
%\end{inventory}
\label{cschap:bgroup}\label{cschap:egroup}\label{cschap:begingroup}\label{cschap:endgroup}\label{cschap:aftergroup}\label{cschap:global}\label{cschap:globaldefs}
\begin{inventory}
\item [\cs{bgroup}] 
Implicit beginning of group character.
\item [\cs{egroup}] 
Implicit end of group character.
\item [\cs{begingroup}] 
Open a group that must be closed with \cs{endgroup}.
\item [\cs{endgroup}] 
Close a group that was opened with \cs{begingroup}.
\item [\cs{aftergroup}] 
Save the next token for insertion after the current group ends.
\item [\cs{global}] 
Make assignments, macro definitions, and arithmetic global.
\item [\cs{globaldefs}] 
Parameter for overriding \cs{global} prefixes.
\IniTeX\ default:~0.
\end{inventory}

%%\point The grouping mechanism
%\section{The grouping mechanism}
%\point The grouping mechanism
\section{The grouping mechanism}

%A group is a sequence of tokens starting with a
%`beginning of group' token,
%and ending with an `end of group'
%token, and in which all such tokens are properly balanced. 
A group is a sequence of tokens starting with a
`beginning of group' token,
and ending with an `end of group'
token, and in which all such tokens are properly balanced. 

%The \indexterm{grouping} mechanism of \TeX\ is not the same as
%the block structured `scoping'
%of ordinary programming languages.
%Most languages with block structure are only able  to have
%local definitions. \TeX's grouping mechanism is stronger: 
%most assignments made inside a group
%are local to that group unless explicitly indicated otherwise,
%and outside the group old values are restored.
The \indexterm{grouping} mechanism of \TeX\ is not the same as
the block structured `scoping'
of ordinary programming languages.
Most languages with block structure are only able  to have
local definitions. \TeX's grouping mechanism is stronger: 
most assignments made inside a group
are local to that group unless explicitly indicated otherwise,
and outside the group old values are restored.

%An example of local definitions
%\begin{verbatim}
%{\def\a{b}}\a
%\end{verbatim}
%gives an `undefined control sequence'
%message because \cs{a} is only defined inside the group.
%Similarly, the code
%\begin{verbatim}
%\count0=1 {\count0=2 } \showthe\count0
%\end{verbatim}
%will display the value~1; the assignment made inside the group
%is undone at the end of the group.
An example of local definitions
\begin{verbatim}
{\def\a{b}}\a
\end{verbatim}
gives an `undefined control sequence'
message because \cs{a} is only defined inside the group.
Similarly, the code
\begin{verbatim}
\count0=1 {\count0=2 } \showthe\count0
\end{verbatim}
will display the value~1; the assignment made inside the group
is undone at the end of the group.


%Bookkeeping of values that are to be restored outside the group
%is done through the mechanism
%of the \indexterm{save stack}. Overflow of the save stack is treated
%in Chapter~\ref{error}. The save stack is also used for
%a few other purposes: in calls such as \hbox{\verb>\hbox to 100pt{...}>}
%the specification \hbox{\n{to 100pt}} is put on the save
%stack before a new level of grouping is opened.
Bookkeeping of values that are to be restored outside the group
is done through the mechanism
of the \indexterm{save stack}. Overflow of the save stack is treated
in Chapter~\ref{error}. The save stack is also used for
a few other purposes: in calls such as \hbox{\verb>\hbox to 100pt{...}>}
the specification \hbox{\n{to 100pt}} is put on the save
stack before a new level of grouping is opened.

%In order to prevent a lot of trouble with the save stack,
%\IniTeX\ does not allow dumping a format inside a group.
%The \cs{end} command is allowed to occur inside a group,
%but \TeX\ will give a diagnostic message about this.
In order to prevent a lot of trouble with the save stack,
\IniTeX\ does not allow dumping a format inside a group.
The \cs{end} command is allowed to occur inside a group,
but \TeX\ will give a diagnostic message about this.

%The \cs{aftergroup} control sequence saves a token for
%insertion after the current group. Several tokens can be
%set aside by this command, and they are inserted in the left-to-right
%order in which they were stated.
%This is treated in Chapter~\ref{expand}.
The \cs{aftergroup} control sequence saves a token for
insertion after the current group. Several tokens can be
set aside by this command, and they are inserted in the left-to-right
order in which they were stated.
This is treated in Chapter~\ref{expand}.


%%\point[global:assign] Local and global assignments
%\section{Local and global assignments}
%\index{assignment!global|(}
%\index{assignment!local|(}
%\label{global:assign}
%\point[global:assign] Local and global assignments
\section{Local and global assignments}
\index{assignment!global|(}
\index{assignment!local|(}
\label{global:assign}

%An assignment or macro definition
%is usually made global by prefixing it with \csidx{global},
%but non-zero values of the \gr{integer parameter}
%\csidx{globaldefs} override \cs{global}
%specifications: if \cs{globaldefs} is positive every assignment
%is implicitly prefixed with \cs{global}, and if
%\cs{globaldefs} is negative, \cs{global} is
%ignored. Ordinarily this parameter is zero.
An assignment or macro definition
is usually made global by prefixing it with \csidx{global},
but non-zero values of the \gr{integer parameter}
\csidx{globaldefs} override \cs{global}
specifications: if \cs{globaldefs} is positive every assignment
is implicitly prefixed with \cs{global}, and if
\cs{globaldefs} is negative, \cs{global} is
ignored. Ordinarily this parameter is zero.

%Some assignment are always global: the \gr{global assignment}s are
%\begin{description}%\FlushRight:no
%\item [\gr{font assignment}]
%assignments involving \cs{fontdimen}, \cs{hyphenchar}, 
%and \cs{skew\-char}.
%\item [\gr{hyphenation assignment}]
%\cs{hyphenation} and \cs{patterns} commands
%(see Chapter~\ref{line:break}).
%\item [\gr{box size assignment}]
%altering box dimensions with \cs{ht}, \cs{dp}, and~\cs{wd}
%(see Chapter~\ref{boxes}).
%\item [\gr{interaction mode assignment}]
%run modes for a \TeX\ job (see Chapter~\ref{run}).
%\item [\gr{intimate assignment}]
%assignments to a \gr{special integer} or \gr{special dimen};
%see %Chapters \ref{number} and~\ref{glue}.
%pages \pageref{special:int:list} and~\pageref{special:dimen:list}.
%\end{description}
Some assignment are always global: the \gr{global assignment}s are
\begin{description}%\FlushRight:no
\item [\gr{font assignment}]
assignments involving \cs{fontdimen}, \cs{hyphenchar}, 
and \cs{skew\-char}.
\item [\gr{hyphenation assignment}]
\cs{hyphenation} and \cs{patterns} commands
(see Chapter~\ref{line:break}).
\item [\gr{box size assignment}]
altering box dimensions with \cs{ht}, \cs{dp}, and~\cs{wd}
(see Chapter~\ref{boxes}).
\item [\gr{interaction mode assignment}]
run modes for a \TeX\ job (see Chapter~\ref{run}).
\item [\gr{intimate assignment}]
assignments to a \gr{special integer} or \gr{special dimen};
see %Chapters \ref{number} and~\ref{glue}.
pages \pageref{special:int:list} and~\pageref{special:dimen:list}.
\end{description}

%\index{assignment!global|)}
%\index{assignment!local|)}
\index{assignment!global|)}
\index{assignment!local|)}

%\section{Group delimiters}
\section{Group delimiters}

%The open and closing \indextermsub{group}{delimiters} can be character
%tokens of category code~1\index{category!1} for `beginning of group'
%and code~2\index{category!2} for `end of group'
%(\indextermbus{explicit}{braces}), or control sequence tokens that are
%\cs{let} to such characters (\indextermbus{implicit}{braces}), such as
%the \cs{bgroup} and \cs{egroup} in plain \TeX.  Implicit and explicit
%braces can match to delimit a group; see for instance the example in
%section~\ref{sec:aftergroup}.
The open and closing \indextermsub{group}{delimiters} can be character
tokens of category code~1\index{category!1} for `beginning of group'
and code~2\index{category!2} for `end of group'
(\indextermbus{explicit}{braces}), or control sequence tokens that are
\cs{let} to such characters (\indextermbus{implicit}{braces}), such as
the \cs{bgroup} and \cs{egroup} in plain \TeX.  Implicit and explicit
braces can match to delimit a group; see for instance the example in
section~\ref{sec:aftergroup}.

%Groups can also be delimited by \csidx{begingroup} and
%\csidx{endgroup}. These two control sequences must
%be used together: they cannot be matched with implicit
%or explicit braces, nor can they function as the braces
%surrounding, for instance, boxed material.
Groups can also be delimited by \csidx{begingroup} and
\csidx{endgroup}. These two control sequences must
be used together: they cannot be matched with implicit
or explicit braces, nor can they function as the braces
surrounding, for instance, boxed material.

%Delimiting with \cs{begingroup} and \cs{endgroup} can
%\label{begin:end:macros}%
%provide a limited form of run-time error checking. 
%In between these two group delimiters an excess
%open or close brace would result in
%\begin{verbatim}
%\begingroup ... } ... \endgroup
%\end{verbatim}
%or
%\begin{verbatim}
%\begingroup ... { ... \endgroup
%\end{verbatim}
%In both cases \TeX\ gives an error message about improper
%balancing. Using \cs{bgroup} and \cs{egroup} here would
%make an error much harder to find, because of the incorrect
%matching that would occur. This idea is used in the environment
%macros of several formats.
Delimiting with \cs{begingroup} and \cs{endgroup} can
\label{begin:end:macros}%
provide a limited form of run-time error checking. 
In between these two group delimiters an excess
open or close brace would result in
\begin{verbatim}
\begingroup ... } ... \endgroup
\end{verbatim}
or
\begin{verbatim}
\begingroup ... { ... \endgroup
\end{verbatim}
In both cases \TeX\ gives an error message about improper
balancing. Using \cs{bgroup} and \cs{egroup} here would
make an error much harder to find, because of the incorrect
matching that would occur. This idea is used in the environment
macros of several formats.

%The choice of the brace characters for the beginning and end of group
%characters is not hard-wired in \TeX. It is arranged
%\cstoidx bgroup\par\cstoidx egroup\par
%like this in the plain format:
%\begin{verbatim}
%\catcode`\{=1 % left brace is begin-group character
%\catcode`\}=2 % right brace is end-group character
%\end{verbatim}
%Implicit braces have also been defined in the plain format:
%\begin{verbatim}
%\let\bgroup={ \let\egroup=}
%\end{verbatim}
The choice of the brace characters for the beginning and end of group
characters is not hard-wired in \TeX. It is arranged
\cstoidx bgroup\par\cstoidx egroup\par
like this in the plain format:
\begin{verbatim}
\catcode`\{=1 % left brace is begin-group character
\catcode`\}=2 % right brace is end-group character
\end{verbatim}
Implicit braces have also been defined in the plain format:
\begin{verbatim}
\let\bgroup={ \let\egroup=}
\end{verbatim}

%Special cases are the following:
%\begin{itemize} \item The replacement text of a macro must be enclosed
%in  explicit beginning and end of group character tokens.
%\item  The open and close braces for boxes, \cs{vadjust},
%and \cs{insert} can be implicit. This makes it possible
%to define, for instance
%\begin{verbatim}
%\def\openbox#1{\setbox#1=\hbox\bgroup}
%\def\closebox#1{\egroup\box#1}
%\openbox{15}Foo bar\closebox{15}
%\end{verbatim}
%\item The right-hand side of a token list assignment and the
%argument of the commands \cs{write}, \cs{message}, \cs{errmessage}, 
%\cs{uppercase}, \cs{lowercase}, 
%\cs{special}, and \cs{mark} is a \gr{general text}, defined
%as
%\begin{Disp} \gr{general text} $\longrightarrow$ \gr{filler}\lb
%      \gr{balanced text}\gr{right brace}\end{Disp}
%meaning that the left brace can be implicit, but the closing
%right brace must be an explicit character token with category
%code~2\index{category!2}.
%\end{itemize}
Special cases are the following:
\begin{itemize} \item The replacement text of a macro must be enclosed
in  explicit beginning and end of group character tokens.
\item  The open and close braces for boxes, \cs{vadjust},
and \cs{insert} can be implicit. This makes it possible
to define, for instance
\begin{verbatim}
\def\openbox#1{\setbox#1=\hbox\bgroup}
\def\closebox#1{\egroup\box#1}
\openbox{15}Foo bar\closebox{15}
\end{verbatim}
\item The right-hand side of a token list assignment and the
argument of the commands \cs{write}, \cs{message}, \cs{errmessage}, 
\cs{uppercase}, \cs{lowercase}, 
\cs{special}, and \cs{mark} is a \gr{general text}, defined
as
\begin{Disp} \gr{general text} $\longrightarrow$ \gr{filler}\lb
      \gr{balanced text}\gr{right brace}\end{Disp}
meaning that the left brace can be implicit, but the closing
right brace must be an explicit character token with category
code~2\index{category!2}.
\end{itemize}

%In cases where an implicit left brace suffices, and where
%expansion is not explicitly inhibited, \TeX\ will
%expand tokens until a left brace is encountered. This
%is the basis for such constructs as
%\verb=\uppercase\expandafter{\romannumeral80}=,
%which in this unexpanded form do not adhere to the
%syntax. If the first unexpandable token is not a left
%brace \TeX\ gives an error message.
In cases where an implicit left brace suffices, and where
expansion is not explicitly inhibited, \TeX\ will
expand tokens until a left brace is encountered. This
is the basis for such constructs as
\verb=\uppercase\expandafter{\romannumeral80}=,
which in this unexpanded form do not adhere to the
syntax. If the first unexpandable token is not a left
brace \TeX\ gives an error message.

%The grammar of \TeX\ (see Chapter~\ref{gramm})  uses
%\gr{left brace} and \gr{right brace} for explicit
%characters, that is, character tokens,
%and \n{\lb} and~\n{\rb} 
%for possibly implicit characters,
%\altt
%that is, control sequences that have been \cs{let} to such
%explicit characters.
The grammar of \TeX\ (see Chapter~\ref{gramm})  uses
\gr{left brace} and \gr{right brace} for explicit
characters, that is, character tokens,
and \n{\lb} and~\n{\rb} 
for possibly implicit characters,
\altt
that is, control sequences that have been \cs{let} to such
explicit characters.

%\section{More about braces}
%\index{braces|(}
\section{More about braces}
\index{braces|(}

%\subsection{Brace counters}
\subsection{Brace counters}

%\TeX\ has two counters for keeping  track of grouping levels:
%the {\it master counter} and the {\it balance counter}.
%Both of these counters are syntactic counters: they count the
%explicit brace character tokens, but are not affected by implicit
%braces (such as \cs{bgroup}) that are semantically equivalent
%to an explicit brace.
\TeX\ has two counters for keeping  track of grouping levels:
the {\it master counter} and the {\it balance counter}.
Both of these counters are syntactic counters: they count the
explicit brace character tokens, but are not affected by implicit
braces (such as \cs{bgroup}) that are semantically equivalent
to an explicit brace.

%The balance counter handles braces in all cases except in
%alignment. Its workings are intuitively clear: it goes up
%by one for every opening and down for every closing
%brace that is not being skipped. Thus
%\begin{verbatim}
%\iffalse{\fi
%\end{verbatim}
%increases the balance counter if
%this statement is merely scanned (for instance if it
%appears in a macro definition text); if this statement
%is executed the brace is skipped, so there is no effect on
%the balance counter.
The balance counter handles braces in all cases except in
alignment. Its workings are intuitively clear: it goes up
by one for every opening and down for every closing
brace that is not being skipped. Thus
\begin{verbatim}
\iffalse{\fi
\end{verbatim}
increases the balance counter if
this statement is merely scanned (for instance if it
appears in a macro definition text); if this statement
is executed the brace is skipped, so there is no effect on
the balance counter.

%The master counter is more tricky;
%it is used in alignments instead of the balance counter.
%This counter records all braces, even when they are skipped
%such as in \verb>\iffalse{\fi>.
%For this counter uncounted skipped braces are still possible:
%the alphabetic constants \n{`\lb} and \n{`\rb} have
%no effect on this counter when they are
%use by the execution processor as a~\gr{number};
%they do affect this counter when they are seen by the 
%input processor (which merely sees characters, and not
%the context).
The master counter is more tricky;
it is used in alignments instead of the balance counter.
This counter records all braces, even when they are skipped
such as in \verb>\iffalse{\fi>.
For this counter uncounted skipped braces are still possible:
the alphabetic constants \n{`\lb} and \n{`\rb} have
no effect on this counter when they are
use by the execution processor as a~\gr{number};
they do affect this counter when they are seen by the 
input processor (which merely sees characters, and not
the context).

%%\spoint The brace as a token
%\subsection{The brace as a token}
%\spoint The brace as a token
\subsection{The brace as a token}

%Explicit braces are character tokens, and as such they are
%unexpandable. This implies that they survive until the
%last stages of \TeX\ processing. For example,
%\begin{verbatim}
%\count255=1{2}
%\end{verbatim}
%will assign~1 to \cs{count255},
%and print~`2', because the
%opening brace functions as a delimiter for the number~1.
%Similarly
%\begin{verbatim}
%f{f}
%\end{verbatim}
%will prevent \TeX\ from forming
%an `\hbox{ff}' ligature.
Explicit braces are character tokens, and as such they are
unexpandable. This implies that they survive until the
last stages of \TeX\ processing. For example,
\begin{verbatim}
\count255=1{2}
\end{verbatim}
will assign~1 to \cs{count255},
and print~`2', because the
opening brace functions as a delimiter for the number~1.
Similarly
\begin{verbatim}
f{f}
\end{verbatim}
will prevent \TeX\ from forming
an `\hbox{ff}' ligature.

%From the fact that braces are unexpandable,
%it follows that their nesting is independent
%of the nesting of conditionals. For instance
%\begin{verbatim}
%\iftrue{\else}\fi
%\end{verbatim}
%will give an open brace,
%as conditionals are handled by expansion. The closing
%brace is simply skipped as part of the \gr{false text};
%any consequences it has for grouping only come into
%play in a later stage of \TeX\ processing.
From the fact that braces are unexpandable,
it follows that their nesting is independent
of the nesting of conditionals. For instance
\begin{verbatim}
\iftrue{\else}\fi
\end{verbatim}
will give an open brace,
as conditionals are handled by expansion. The closing
brace is simply skipped as part of the \gr{false text};
any consequences it has for grouping only come into
play in a later stage of \TeX\ processing.

%Undelimited macro arguments are either single tokens
%or groups of tokens enclosed in explicit braces.
%Thus it is not possible for an explicit open or close brace
%to be a macro argument. However, braces can be assigned
%with \cs{let}, for instance as in
%\begin{verbatim}
%\let\bgroup={
%\end{verbatim}
%This is used in the plain \cs{footnote} macro
%(see page~\pageref{footnote:ex}).
Undelimited macro arguments are either single tokens
or groups of tokens enclosed in explicit braces.
Thus it is not possible for an explicit open or close brace
to be a macro argument. However, braces can be assigned
with \cs{let}, for instance as in
\begin{verbatim}
\let\bgroup={
\end{verbatim}
This is used in the plain \cs{footnote} macro
(see page~\pageref{footnote:ex}).

%%\spoint \csc{\char 123} and \csc{\char 125}
%\subsection{Open and closing brace control symbols}
%% \csc{\char 123} and \csc{\char 125}}
%\spoint \csc{\char 123} and \csc{\char 125}
\subsection{Open and closing brace control symbols}
% \csc{\char 123} and \csc{\char 125}}

%The control sequences \verb-\{- and \verb-\}- do not really belong
%\cstoidx\char123\par\cstoidx\char125\par
%in this chapter,  not being concerned with grouping.
%They have been defined with \cs{let} as synonyms of
%\cs{lbrace} and \cs{rbrace} respectively,
%and these control sequences are \cs{delimiter} instructions
%(see Chapter~\ref{mathchar}).
The control sequences \verb-\{- and \verb-\}- do not really belong
\cstoidx\char123\par\cstoidx\char125\par
in this chapter,  not being concerned with grouping.
They have been defined with \cs{let} as synonyms of
\cs{lbrace} and \cs{rbrace} respectively,
and these control sequences are \cs{delimiter} instructions
(see Chapter~\ref{mathchar}).

%The Computer Modern Roman font has no braces, but there are
%braces in the typewriter font, and for mathematics 
%there are braces of different sizes \ldash and extendable ones \rdash in
%the extension font.
The Computer Modern Roman font has no braces, but there are
braces in the typewriter font, and for mathematics 
there are braces of different sizes \ldash and extendable ones \rdash in
the extension font.

%\index{braces|)}
%\endofchapter
%%%%% end of input file [group]
\index{braces|)}
\endofchapter
%%%% end of input file [group]

\end{document}
