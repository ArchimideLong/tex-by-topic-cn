
\chapter{Errors, Catastrophes,  and Help}\label{error}

When \TeX\ is running, various errors can occur.
This chapter treats how errors in the input are displayed,
and what sort of overflow of internal data structures
of \TeX\ can occur.

\label{cschap:errorcontextlines}\label{cschap:errmessage}\label{cschap:errhelp}
\begin{inventory}
\item [\cs{errorcontextlines}] 
      (\TeX3 only)
      Number of additional context lines shown in error messages.

\item [\cs{errmessage}] 
      Report an error, giving the parameter of this command as message.

\item [\cs{errhelp}] 
      Tokens that will be displayed if the user 
      asks further help after an \cs{errmessage}.

\end{inventory}

%\point Error messages
\section{Error messages}

When \TeX\ is running in \cs{errorstopmode} (which it usually is;
see Chapter~\ref{run} for the other running modes),
errors occurring are reported on the user terminal, and \TeX\
asks the user for further instructions.
Errors can occur either because of some internal condition
of \TeX, or because a macro has issued an \csidx{errmessage}
command.

If an error occurs \TeX\ shows the input 
line 
on which the error occurred. If the offending command was
not on that line but, for instance, in a macro that was
called \ldash possibly indirectly \rdash  from that line,
the line of that command is also shown.
If the offending command was indirectly called,
an additional \csidx{errorcontextlines} number of lines
is shown with the preceding macro calls.

A~value of \cs{errorcontextlines}${}=0$ causes \n{...}
to be printed as the sole indication that there is a context.
Negative values inhibit even this.

For each macro in the sequence that leads to the offending
command,
\TeX\ attempts to display some
preceding and some following tokens.
First one line is displayed ending with
the \ldash indirectly \rdash  offending command; then, one line lower
some following tokens are given. 

\begin{example}
\begin{verbatim}
This paragraph ends \vship1cm with a skip.
\end{verbatim}
gives
\begin{verbatim}
! Undefined control sequence.
l.1 This paragraph ends \vship
                              1cm with a skip.
\end{verbatim}
\end{example}

If \TeX\ is not running in some non-stop mode\label{interaction},
the user is given the chance to do some
\indexterm{error patching}, or to
ask for further information. In general the following
options are available:
\begin{description}\item [\gr{return}]
\TeX\ will continue processing. If the error was something
innocent that \TeX\ could either ignore or patch itself,
this is the easy way out.
\item [\n h]
Give further details about the error.
If the error was caused by an \cs{err\-message} command,
the \csidx{errhelp} tokens will be displayed here.
\item [\n i]
Insert. The user can insert some material. For example,
if a control sequence is misspelled, the correct command can
sometimes be inserted, as
\begin{verbatim}
i\vskip
\end{verbatim}
for the above
example. Also, this is an opportunity for inserting
\cs{show} commands to inspect \TeX's internal state.
However, if \TeX\ is in the middle of
scanning something complicated,
such commands will not be executed, or will even
add to the confusion.
\item [\n s]
    (\cs{scrollmode})
Scroll further errors, but display the messages. 
\TeX\ will patch any further errors.
This is a handy option, for instance if the error occurs
in an alignment, because the number of subsequent errors tends
to be rather large.
\item [\n r]
    (\cs{nonstopmode})
Run without stopping. \TeX\ will never stop for user interaction.
\item [\n q]
    (\cs{batchmode})
Quiet running. \TeX\ will never stop for user interaction,
and does not give any more terminal output.
\item [\n x]
Exit. Abort this run of \TeX.
\item [\n e]
Edit. This option is not available on all \TeX\ system.
If it is, the run of \TeX\ is aborted, and an editor is
started, opening with the input file, maybe even 
on the offending line.
\end{description}


%\point Overflow errors
\section{Overflow errors}
\index{overflow errors|(}

Harsh reality imposes some restrictions on how elaborate
\TeX's workings can get. Some restrictions are imposed by
compile-time constants, and are therefore fairly loose, but
some depend strongly on the actual computer implementation.

Here follows the list of all categories of overflow that
prompt \TeX\ to report `Capacity exceeded'.
Most bounds involved are (determined by) compile-time
constants; their values given here in parentheses are those
used in the source listing of \TeX\ in~\cite{Knuth:TeXbook}.
Actual values may differ, and probably will. Remember
that \TeX\ was developed in the good old days when even
big computers were fairly small.

%\spoint Buffer size {\rm(500)}
\subsection{Buffer size {\rm(500)}}

Current lines of all files that are open are kept in
\TeX's input buffer, as are control sequence names
that are being built with \verb-\csname...\endcsname-.

%\spoint Exception dictionary {\rm(307)}
\subsection{Exception dictionary {\rm(307)}}

The maximum number of hyphenation exceptions specified
by \cs{hyphenation} must be a prime number.
Two arrays with this many halfwords are allocated.

Changing this  number makes formats incompatible;
that is, \TeX\ can only use a format that was made by
an \IniTeX\ with the same value for this constant.

%\spoint Font memory (20$\,$000)
\subsection{Font memory (20$\,$000)}

Information about fonts is stored in an array of
memory words. This is easily overflowed by preloading too
many fonts in \IniTeX.

%\spoint Grouping levels
\subsection{Grouping levels}

The number of open groups  should be recordable 
in a quarter word. There is no compile-time constant corresponding
to this.

%\spoint Hash size {\rm(2100)}
\subsection{Hash size {\rm(2100)}}

Maximum number of control sequences. It is suggested that
this number should not exceed 10\% of the main memory size.
The values in \TeX\ and \IniTeX\ should agree; also the
\n{hash\_prime} values should agree.

This value is rather low; for macro packages that are more
elaborate than plain \TeX\ a value of about 3000 is more
realistic.

%\spoint Number of strings {\rm(3000)}
\subsection{Number of strings {\rm(3000)}}

The maximum number of strings must be recordable in a half word.

%\spoint Input stack size {\rm(200)}
\subsection{Input stack size {\rm(200)}}

For each input source an item is allocated on the input stack.
Typical input sources are input files (but their simultaneous
number is more limited; see below), and token lists
such as token variables, macro replacement texts, and
alignment templates. A~macro with `runaway recursion'
(for example, \verb>\def\mac{{\mac}}>)
will overflow this stack.

\TeX\ performs some optimization here: before the last call 
in a token list all token lists ending with this call are
cleared. This process is 
similar to `resolving tail recursion' (see Chapter~\ref{macro}).

%\spoint Main memory size (30$\,$000)
\subsection{Main memory size (30$\,$000)}

Almost all `dynamic' objects of \TeX, such as macro definition
texts and all material on the current page, 
are stored in the main memory array. 
Formats may already take $20\,000$ words of
main memory for macro definitions, and complicated pages containing
for instance the \LaTeX\ picture environment may easily
overflow this array. 

\TeX's main memory is divided in words, and a half word
is supposed to be able to address the whole of the memory.
Thus on current 32-bit computers the most common choice
is to let the main memory size be at most 64K bytes.
A~half word address can then be stored in 16 bits,
half a machine word.

However, so-called `Big \TeX' implementations exist
\thecstoidxsub{TeX}{big}
that have a main memory larger than 64K words.
Most compilers will then allocate  32-bit words for
addressing this memory, even if (say) 18 bits would
suffice. Big \TeX s therefore become immediately
a lot bigger when they cross the 64K threshold.
Thus they are usually not found on microcomputers,
although virtual memory schemes for these are possible;
see for instance~\cite{Thull}.

\TeX\ can have a bigger main memory than \IniTeX;
see Chapter~\ref{TeXcomm} for further details.

%\spoint Parameter stack size {\rm(60)}
\subsection{Parameter stack size {\rm(60)}}

Macro parameters may contain macro calls with
further parameters. The number of parameters that may occur
nested is bounded by the parameter stack size.

%\spoint Pattern memory {\rm(8000)}
\subsection{Pattern memory {\rm(8000)}}

Hyphenation patterns are stored in a trie array.
The default size of 8000 hyphenation patterns seems sufficient
for English or Italian,  for example, but it is not for
Dutch or German.

%\spoint Pattern memory ops per language
\subsection{Pattern memory ops per language}

The number of hyphenation ops (see the literature about
hyphenation: \cite{Liang} and appendix~H of~\cite{Knuth:TeXbook})
should be recordable 
in a quarter word. There is no compile-time constant corresponding
to this. \TeX\ version~2 had the same upper bound, but gave no
error message in case of overflow. Again, for languages such
as Dutch and German this bound is too low.
There are versions of \TeX\ that have a higher bound here.

%\spoint Pool size (32$\,$000)
\subsection{Pool size (32$\,$000)}

Strings are error messages and control sequence names.
They are stored using one byte per character.
\TeX\ has initially about $23\,000$ characters worth of
strings.

The pool will overflow if a user defines a large number of
control sequences on top of a substantial macro package.
However, even if the user does not define any new commands
\mdqon
overflow may occur: cross"-referencing schemes also
\mdqoff
work by defining control sequences. For large documents
a pool size of $40\,000$ or $60\,000$ is probably sufficient.

%\spoint Save size {\rm(600)}
\subsection{Save size {\rm(600)}}

Quantities that are assigned to inside a group must be
restored after the end of that group.
The save stack is where the values to be restored are kept;
the size of the
save stack limits the number of values that can be restored.

Alternating global and local assignments to a value
will lead to `save stack build-up': for each local
assignment following a global assignment the
previous value of the variable is saved. Thus an
alternation of such assignments will lead to
an unnecessary proliferation of items on the save stack.

%\spoint Semantic nest size {\rm(40)}
\subsection{Semantic nest size {\rm(40)}}

Each time \TeX\ switches to a mode nested inside another
mode (for instance when processing an \verb-\hbox- inside
a \verb-\vbox-) the current state is pushed on the
semantic nest stack. The semantic nest size is the maximum
number of levels that can be pushed.

%\spoint Text input levels {\rm(6)}
\subsection{Text input levels {\rm(6)}}

The number of nested \verb-\input- files 
has to be very limited,
as the current lines are all kept in the input buffer.

\index{overflow errors|)}

\endofchapter
