% -*- coding: utf-8 -*-
\documentclass{book}

% -*- coding: utf-8 -*-

\usepackage[b5paper,text={5in,8in},centering]{geometry}

\usepackage[CJKchecksingle]{xeCJK}
\setmainfont[Mapping=tex-text]{TeX Gyre Schola}
%\setsansfont{URW Gothic L Book}
%\setmonofont{Nimbus Mono L}
\setCJKmainfont[BoldFont=FandolHei,ItalicFont=FandolKai]{FandolSong}
\setCJKsansfont{FandolHei}
\setCJKmonofont{FandolFang}
\xeCJKsetup{PunctStyle = kaiming}

\linespread{1.25}
\setlength{\parindent}{2em}

\usepackage{xcolor}
\definecolor{myblue}{rgb}{0,0.2,0.6}

\usepackage{titlesec}
\titleformat{\chapter}
    {\normalfont\Huge\sffamily\color{myblue}}
    {第\thechapter 章}
    {1em}
    {}
%\titlespacing{\chapter}{0pt}{50pt}{40pt}
\titleformat{\section}
    {\normalfont\Large\sffamily\color{myblue}}
    {\thesection}
    {1em}
    {}
%\titlespacing{\section}{0pt}{3.5ex plus 1ex minus .2ex}{2.3ex plus .2ex}
\titleformat{\subsection}
    {\normalfont\large\sffamily\color{myblue}}
    {\thesubsection}
    {1em}
    {}
%\titlespacing{\subsection}{0pt}{3.25ex plus 1ex minus .2ex}{1.5ex plus .2ex}
%
\newpagestyle{special}[\small\sffamily]{
  \headrule
  \sethead[\usepage][][\chaptertitle]
  {\chaptertitle}{}{\usepage}}
\newpagestyle{main}[\small\sffamily]{
  \headrule
  \sethead[\usepage][][第\thechapter 章\quad\chaptertitle]
  {\thesection\quad\sectiontitle}{}{\usepage}}

\usepackage{titletoc}
%\setcounter{tocdepth}{1}
%\titlecontents{标题层次}[左间距]{上间距和整体格式}{标题序号}{标题内容}{指引线和页码}[下间距]
\titlecontents{chapter}[1.5em]{\vspace{.5em}\bfseries\sffamily}{\color{myblue}\contentslabel{1.5em}}{}
    {\titlerule*[20pt]{$\cdot$}\contentspage}[]
\titlecontents{section}[4.5em]{\sffamily}{\color{myblue}\contentslabel{3em}}{}
    {\titlerule*[20pt]{$\cdot$}\contentspage}[]
%\titlecontents{subsection}[8.5em]{\sffamily}{\contentslabel{4em}}{}
%    {\titlerule*[20pt]{$\cdot$}\contentspage}

\usepackage{enumitem}
\setlist{topsep=2pt,itemsep=2pt,parsep=1pt,leftmargin=\parindent}

\usepackage{fancyvrb}
\DefineVerbatimEnvironment{verbatim}{Verbatim}
  {xleftmargin=2em,baselinestretch=1,formatcom=\color{teal}\upshape}

\usepackage{etoolbox}
\makeatletter
\preto{\FV@ListVSpace}{\topsep=2pt \partopsep=0pt }
\makeatother

\usepackage[colorlinks,plainpages,pagebackref]{hyperref}
\hypersetup{
   pdfstartview={FitH},
   citecolor=teal,
   linkcolor=myblue,
   urlcolor=black,
   bookmarksnumbered
}

\usepackage{comment,makeidx,multicol}

%\usepackage{german}
%% german
%\righthyphenmin=3
%\mdqoff
%\captionsenglish
\usepackage[english]{babel}
{\catcode`"=13 \gdef"#1{\ifx#1"\discretionary{}{}{}\fi\relax}}
\def\mdqon{\catcode`"=13\relax}
\def\mdqoff{\catcode`"=12\relax}
\makeindex
\hyphenation{ex-em-pli-fies}

\newdimen\tempdima \newdimen\tempdimb

% these are fine
\def\handbreak{\\ \message{^^JManual break!!!!^^J}}
\def\nl{\protect\\}\def\n#1{{\tt #1}}
\def\cs#1{\texorpdfstring{{\tt\char`\\#1}}{\textbackslash#1}} %\def\cs#1{{\tt\char`\\#1}}
\let\csc\cs
\def\lb{{\tt\char`\{}}\def\rb{{\tt\char`\}}}
\def\gr#1{\texorpdfstring{$\langle$#1$\rangle$}{<#1>}} %\def\gr#1{$\langle$#1$\rangle$}
\def\key#1{{\tt#1}}
\def\alt{}\def\altt{}%this way in manstijl
\def\ldash{\unskip\ --\nobreak\ \ignorespaces}
\def\rdash{\unskip\nobreak\ --\ \ignorespaces}
% check these
\def\hex{{\tt"}}
\def\ascii{{\sc ascii}}
\def\ebcdic{{\sc ebcdic}}
\def\IniTeX{Ini\TeX}\def\LamsTeX{LAMS\TeX}\def\VirTeX{Vir\TeX}
\def\AmsTeX{Ams\TeX}
\def\TeXbook{the \TeX\ book}\def\web{{\sc web}}
% needs major thinking
\newenvironment{myquote}{\list{}{%
    \topsep=2pt \partopsep=0pt%
    \leftmargin=\parindent \rightmargin=\parindent
    }\item[]}{\endlist}
\newenvironment{disp}{\begin{myquote}}{\end{myquote}}
\newenvironment{Disp}{\begin{myquote}}{\end{myquote}}
\newenvironment{tdisp}{\begin{myquote}}{\end{myquote}}
\newenvironment{example}{\begin{myquote}\noindent\itshape 例子:}{\end{myquote}}
\newenvironment{inventory}{\begin{description}\raggedright}{\end{description}}
\newenvironment{glossinventory}{\begin{description}}{\end{description}}
\def\gram#1{\gr{#1}}%???
%
% index
%
\def\indexterm#1{\emph{#1}\index{#1}}
\def\indextermsub#1#2{\emph{#1 #2}\index{#1!#2}}
\def\indextermbus#1#2{\emph{#1 #2}\index{#2!#1}}
\def\term#1\par{\index{#1}}
\def\howto#1\par{}
\def\cstoidx#1\par{\index{#1@\cs{#1}@}}
\def\thecstoidx#1\par{\index{#1@\protect\csname #1\endcsname}}
\def\thecstoidxsub#1#2{\index{#1, #2@\protect\csname #1\endcsname, #2}\ignorespaces}
\def\csterm#1\par{\cstoidx #1\par\cs{#1}}
\def\csidx#1{\cstoidx #1\par\cs{#1}}

\def\tmc{\tracingmacros=2 \tracingcommands\tracingmacros}

%%%%%%%%%%%%%%%%%%%
\makeatletter
\def\snugbox{\hbox\bgroup\setbox\z@\vbox\bgroup
    \leftskip\z@
    \bgroup\aftergroup\make@snug
    \let\next=}
\def\make@snug{\par\sn@gify\egroup \box\z@\egroup}
\def\sn@gify
   {\skip\z@=\lastskip \unskip
    \advance\skip\z@\lastskip \unskip
    \unpenalty
    \setbox\z@\lastbox
    \ifvoid\z@ \nointerlineskip \else {\sn@gify} \fi
    \hbox{\unhbox\z@}\nointerlineskip
    \vskip\skip\z@
    }

\newdimen\fbh \fbh=60pt % dimension for easy scaling:
\newdimen\fbw \fbw=60pt % height and width of character box

\newdimen\dh \newdimen\dw % height and width of current character box
\newdimen\lh % height of previous character box
\newdimen\lw \lw=.4pt % line weight, instead of default .4pt

\def\hdotfill{\noindent
    \leaders\hbox{\vrule width 1pt height\lw
                  \kern4pt
                  \vrule width.5pt height\lw}\hfill\hbox{}
    \par}
\def\hlinefill{\noindent
    \leaders\hbox{\vrule width 5.5pt height\lw         }\hfill\hbox{}
    \par}
\def\stippel{$\qquad\qquad\qquad\qquad$}
\makeatother
%%%%%%%%%%%%%%%%%%%

%\def\SansSerif{\Typeface:macHelvetica }
%\def\SerifFont{\Typeface:macTimes }
%\def\SansSerif{\Typeface:bsGillSans }
%\def\SerifFont{\Typeface:bsBaskerville }
\let\SansSerif\relax \def\italic{\it}
\let\SerifFont\relax \def\MainFont{\rm}
\let\SansSerif\relax
\let\SerifFont\relax
\let\PopIndentLevel\relax \let\PushIndentLevel\relax
\let\ToVerso\relax \let\ToRecto\relax

%\def\stop@command@suffix{stop}
%\let\PopListLevel\PopIndentLevel
%\let\FlushRight\relax
%\let\flushright\FlushRight
%\let\SetListIndent\LevelIndent
%\def\awp{\ifhmode\vadjust{\penalty-10000 }\else
%    \penalty-10000 \fi}
\let\awp\relax
\let\PopIndentLevel\relax \let\PopListLevel\relax

\showboxdepth=-1

%\input figs
\def\endofchapter{\vfill\noindent}

\setcounter{chapter}{16}

\begin{document}

%\chapter{Paragraph End}\label{par:end}
%\index{paragraph!end|(}
\chapter{Paragraph End}\label{par:end}
\index{paragraph!end|(}

%\TeX's mechanism for ending a paragraph is ingenious and effective.
%This chapter explains the mechanism, the role of \cs{par} in it,
%and it gives a number of practical remarks.
\TeX's mechanism for ending a paragraph is ingenious and effective.
This chapter explains the mechanism, the role of \cs{par} in it,
and it gives a number of practical remarks.

%\label{cschap:par}\label{cschap:endgraf}\label{cschap:parfillskip}
%\begin{inventory}
%\item [\cs{par}] 
%      Finish off a paragraph and go into vertical mode.
\label{cschap:par}\label{cschap:endgraf}\label{cschap:parfillskip}
\begin{inventory}
\item [\cs{par}] 
      Finish off a paragraph and go into vertical mode.

%\item [\cs{endgraf}]
%      Synonym for \cs{par}: \verb>\let\endgraf=\par>
\item [\cs{endgraf}]
      Synonym for \cs{par}: \verb>\let\endgraf=\par>

%\item [\cs{parfillskip}] 
%      Glue that is placed between the last          
%      element of the paragraph and the line end.
%      Plain \TeX\ default:~\n{0pt plus 1fil}.
%\end{inventory}
\item [\cs{parfillskip}] 
      Glue that is placed between the last          
      element of the paragraph and the line end.
      Plain \TeX\ default:~\n{0pt plus 1fil}.
\end{inventory}

%\section{The way paragraphs end}
\section{The way paragraphs end}

%A paragraph is terminated by the primitive \cs{par} command, 
%which can
%be explicitly typed by the user (or inserted by
%a macro expansion):
%\begin{verbatim}
%... last words.\par
%A new paragraph ...
%\end{verbatim}
%It can be implicitly generated in the input processor of \TeX\
%by an empty line (see Chapter~\ref{mouth}):
%\begin{verbatim}
%... last words.

%A new paragraph ...
%\end{verbatim}
A paragraph is terminated by the primitive \cs{par} command, 
which can
be explicitly typed by the user (or inserted by
a macro expansion):
\begin{verbatim}
... last words.\par
A new paragraph ...
\end{verbatim}
It can be implicitly generated in the input processor of \TeX\
by an empty line (see Chapter~\ref{mouth}):
\begin{verbatim}
... last words.

A new paragraph ...
\end{verbatim} 
%The \cs{par} can be inserted because a \gr{vertical command}
%occurred in unrestricted horizontal mode:
%\begin{verbatim}
%... last words.\vskip6pt
%A new paragraph ...
%\end{verbatim}
%Also, a paragraph ends if a closing brace is found
%in horizontal mode inside \cs{vbox}, \cs{insert}, or \cs{output}.
The \cs{par} can be inserted because a \gr{vertical command}
occurred in unrestricted horizontal mode:
\begin{verbatim}
... last words.\vskip6pt
A new paragraph ...
\end{verbatim}
Also, a paragraph ends if a closing brace is found
in horizontal mode inside \cs{vbox}, \cs{insert}, or \cs{output}.

%After the \cs{par} command \TeX\ goes into vertical mode
%and exercises the page builder (see page~\pageref{par:page:build}).
%If the \cs{par} was inserted because a vertical command occurred in
%horizontal mode, the vertical command is then examined anew.
%The \cs{par} does not insert any vertical
%glue or penalties itself. A~\cs{par} command also clears
%the paragraph shape parameters (see Chapter~\ref{par:shape}).
After the \cs{par} command \TeX\ goes into vertical mode
and exercises the page builder (see page~\pageref{par:page:build}).
If the \cs{par} was inserted because a vertical command occurred in
horizontal mode, the vertical command is then examined anew.
The \cs{par} does not insert any vertical
glue or penalties itself. A~\cs{par} command also clears
the paragraph shape parameters (see Chapter~\ref{par:shape}).

%%\spoint The \cs{par} command and the \cs{par} token
%\subsection{The \cs{par} command and the \cs{par} token}
%\spoint The \cs{par} command and the \cs{par} token
\subsection{The \cs{par} command and the \cs{par} token}

%It is important to distinguish between the \cs{par} token
%and the primitive \cs{par} command that is the initial meaning of
%that token. The \cs{par} token is inserted when the input
%processor sees an empty
%line, or when the execution processor finds a \gram{vertical command}
%in horizontal mode;
%the \cs{par} command is what actually closes off a paragraph.
%Decoupling the token and the command is an important tool
%for special effects in paragraphs (see some examples in
%Chapters \ref{boxes} and~\ref{rules}).
It is important to distinguish between the \cs{par} token
and the primitive \cs{par} command that is the initial meaning of
that token. The \cs{par} token is inserted when the input
processor sees an empty
line, or when the execution processor finds a \gram{vertical command}
in horizontal mode;
the \cs{par} command is what actually closes off a paragraph.
Decoupling the token and the command is an important tool
for special effects in paragraphs (see some examples in
Chapters \ref{boxes} and~\ref{rules}).


%%\spoint Paragraph filling: \cs{parfillskip}
%\subsection{Paragraph filling: \cs{parfillskip}}
%\spoint Paragraph filling: \cs{parfillskip}
\subsection{Paragraph filling: \cs{parfillskip}}

%After the last element of the paragraph \TeX\ implicitly inserts
%the equivalent of
%\cstoidx parfillskip\par
%\begin{verbatim}
%\unskip \penalty10000 \hskip\parfillskip
%\end{verbatim}
%The \cs{unskip} serves to remove any spurious glue at the
%paragraph end, such as the space generated by the
%line end if the \cs{par} was inserted by the input processor.
%For example:\message{check unsplit paragraph example}
%\begin{verbatim}
%end.

%\noindent Begin
%\end{verbatim}
After the last element of the paragraph \TeX\ implicitly inserts
the equivalent of
\cstoidx parfillskip\par
\begin{verbatim}
\unskip \penalty10000 \hskip\parfillskip
\end{verbatim}
The \cs{unskip} serves to remove any spurious glue at the
paragraph end, such as the space generated by the
line end if the \cs{par} was inserted by the input processor.
For example:\message{check unsplit paragraph example}
\begin{verbatim}
end.

\noindent Begin
\end{verbatim}
%results in the tokens
%\begin{disp}\n{end.\char32}\cs{par} \n{Begin}\end{disp}
%With the sequence inserted by the \cs{par} this becomes
%\begin{disp}\n{end.\char32}\verb>\unskip\penalty10000\hskip ...>\end{disp}
%which in turn gives
%\begin{disp}\verb>end.\penalty ...>\end{disp}
results in the tokens
\begin{disp}\n{end.\char32}\cs{par} \n{Begin}\end{disp}
With the sequence inserted by the \cs{par} this becomes
\begin{disp}\n{end.\char32}\verb>\unskip\penalty10000\hskip ...>\end{disp}
which in turn gives
\begin{disp}\verb>end.\penalty ...>\end{disp}

%The \cs{parfillskip} is in plain \TeX\ first-order infinite
%(\n{0pt plus 1fil}),
%so ending a paragraph with \verb.\hfil$\bullet$\par.
%will give a bullet halfway between the last word and the
%line end; with \verb.\hfill$\bullet$\par. it will be
%flush right.
The \cs{parfillskip} is in plain \TeX\ first-order infinite
(\n{0pt plus 1fil}),
so ending a paragraph with \verb.\hfil$\bullet$\par.
will give a bullet halfway between the last word and the
line end; with \verb.\hfill$\bullet$\par. it will be
flush right.


%%\point Assorted remarks
%\section{Assorted remarks}
%\point Assorted remarks
\section{Assorted remarks}

%%\spoint Ending a paragraph and a group at the same time
%\subsection{Ending a paragraph and a group at the same time}
%\spoint Ending a paragraph and a group at the same time
\subsection{Ending a paragraph and a group at the same time}

%If a paragraph is set in a group, 
%it may be necessary to ensure that the \cs{par} ending
%the paragraph occurs inside the group.
%The parameters influencing the typesetting of the paragraph,
%such as the \cs{leftskip} and the \cs{baselineskip},
%are only looked at when the paragraph is finished.
%Thus finishing off a paragraph with 
%\begin{verbatim}
%... last words.}\par
%\end{verbatim}
%causes the values to be used
%that prevail outside the group, instead of those inside.
If a paragraph is set in a group, 
it may be necessary to ensure that the \cs{par} ending
the paragraph occurs inside the group.
The parameters influencing the typesetting of the paragraph,
such as the \cs{leftskip} and the \cs{baselineskip},
are only looked at when the paragraph is finished.
Thus finishing off a paragraph with 
\begin{verbatim}
... last words.}\par
\end{verbatim}
causes the values to be used
that prevail outside the group, instead of those inside.

%Better ways to end the paragraph are
%\begin{verbatim}
%... last words.\par}
%\end{verbatim}
%or
%\begin{verbatim}
%... last words.\medskip}
%\end{verbatim}
%In the second example the vertical command \cs{medskip}
%causes the \cs{par} token to be inserted.
Better ways to end the paragraph are
\begin{verbatim}
... last words.\par}
\end{verbatim}
or
\begin{verbatim}
... last words.\medskip}
\end{verbatim}
In the second example the vertical command \cs{medskip}
causes the \cs{par} token to be inserted.

%%\spoint Ending a paragraph with \cs{hfill}\cs{break}
%\subsection{Ending a paragraph with \cs{hfill}\cs{break}}
%\spoint Ending a paragraph with \cs{hfill}\cs{break}
\subsection{Ending a paragraph with \cs{hfill}\cs{break}}

%The sequence \verb.\hfill\break. is a way to force
%a `newline' inside a paragraph. If you end a paragraph
%with this, however, you will probably
%get an \verb-Underfull \hbox- error.
%Surprisingly, the underfull box is not the broken line
%\ldash after all, that one was filled \rdash 
%but a completely empty box following it (actually, it
%does contain the \cs{leftskip} and \cs{rightskip}). 
The sequence \verb.\hfill\break. is a way to force
a `newline' inside a paragraph. If you end a paragraph
with this, however, you will probably
get an \verb-Underfull \hbox- error.
Surprisingly, the underfull box is not the broken line
\ldash after all, that one was filled \rdash 
but a completely empty box following it (actually, it
does contain the \cs{leftskip} and \cs{rightskip}). 

%What happens?
%The paragraph ends with
%\begin{verbatim}
%\hfill\break\par
%\end{verbatim}
%which turns into 
%\begin{verbatim}
%\hfill\break\unskip\nobreak\hskip\parfillskip
%\end{verbatim}
%The \cs{unskip} finds no preceding glue, so the \cs{break}
%is followed by a penalty item and a glue item, both of
%which disappear after the line break has been chosen at the
%\cs{break}.
%However, \TeX\ has already decided that there should be an extra
%line, that is, an \verb.\hbox to \hsize.. And there is nothing
%\alt
%to fill it with, so an underfull box results.
What happens?
The paragraph ends with
\begin{verbatim}
\hfill\break\par
\end{verbatim}
which turns into 
\begin{verbatim}
\hfill\break\unskip\nobreak\hskip\parfillskip
\end{verbatim}
The \cs{unskip} finds no preceding glue, so the \cs{break}
is followed by a penalty item and a glue item, both of
which disappear after the line break has been chosen at the
\cs{break}.
However, \TeX\ has already decided that there should be an extra
line, that is, an \verb.\hbox to \hsize.. And there is nothing
\alt
to fill it with, so an underfull box results.

%%\spoint Ending a paragraph with a rule
%\subsection{Ending a paragraph with a rule}
%\spoint Ending a paragraph with a rule
\subsection{Ending a paragraph with a rule}

%See page~\pageref{par:leaders:end} for paragraphs ending with 
%rule leaders instead of the default \cs{parfillskip}
%white space.
See page~\pageref{par:leaders:end} for paragraphs ending with 
rule leaders instead of the default \cs{parfillskip}
white space.

%%\spoint No page breaks in between paragraphs
%\subsection{No page breaks in between paragraphs}
%\spoint No page breaks in between paragraphs
\subsection{No page breaks in between paragraphs}

%The \cs{par} command does not insert any glue in the
%\howto Prevent page breaks in between paragraphs\par
%vertical list, so
%in the sequence
%\begin{verbatim}
% ... last words.\par \nobreak \medskip 
%\noindent First words ...
%\end{verbatim}
%no page breaks will occur  between the paragraphs.
%The vertical list generated is
%\begin{verbatim}
%\hbox(6.94444+0.0)x ...       % last line of paragraph
%\penalty 10000                % \nobreak
%\glue 6.0 plus 2.0 minus 2.0  % \medskip
%\glue(\parskip) 0.0 plus 1.0  % \parskip
%\glue(\baselineskip) 5.05556  % interline glue
%\hbox(6.94444+0.0)x ...       % first line of paragraph
%\end{verbatim}
%\TeX\ will not break this vertical list above the \cs{medskip},
%because the penalty value prohibits it; it will not break
%at any other place, because it can only break at glue if
%that glue is preceded by a  non-discardable item.
The \cs{par} command does not insert any glue in the
\howto Prevent page breaks in between paragraphs\par
vertical list, so
in the sequence
\begin{verbatim}
 ... last words.\par \nobreak \medskip 
\noindent First words ...
\end{verbatim}
no page breaks will occur  between the paragraphs.
The vertical list generated is
\begin{verbatim}
\hbox(6.94444+0.0)x ...       % last line of paragraph
\penalty 10000                % \nobreak
\glue 6.0 plus 2.0 minus 2.0  % \medskip
\glue(\parskip) 0.0 plus 1.0  % \parskip
\glue(\baselineskip) 5.05556  % interline glue
\hbox(6.94444+0.0)x ...       % first line of paragraph
\end{verbatim}
\TeX\ will not break this vertical list above the \cs{medskip},
because the penalty value prohibits it; it will not break
at any other place, because it can only break at glue if
that glue is preceded by a  non-discardable item.

%%\spoint Finite \cs{parfillskip}
%\subsection{Finite \cs{parfillskip}}
%\spoint Finite \cs{parfillskip}
\subsection{Finite \cs{parfillskip}}

%In plain \TeX, \cs{parfillskip} has a (first-order) infinite
%stretch component. All other glue in the last line of a 
%paragraph will then be set at natural width.
%If the \cs{parfillskip} has only finite (or possibly zero)
%stretch, other glue will be stretched or shrunk.
%A display formula in a paragraph with such a last line
%will be surrounded by \cs{abovedisplayskip} and \cs{belowdisplayskip},
%even if \cs{abovedisplayshortskip} glue would be in order.
In plain \TeX, \cs{parfillskip} has a (first-order) infinite
stretch component. All other glue in the last line of a 
paragraph will then be set at natural width.
If the \cs{parfillskip} has only finite (or possibly zero)
stretch, other glue will be stretched or shrunk.
A display formula in a paragraph with such a last line
will be surrounded by \cs{abovedisplayskip} and \cs{belowdisplayskip},
even if \cs{abovedisplayshortskip} glue would be in order.

%The reason for this is that glue setting is slightly
%machine-dependent, and any such processes should be kept
%out of \TeX's global decisions.
The reason for this is that glue setting is slightly
machine-dependent, and any such processes should be kept
out of \TeX's global decisions.

%%\spoint A precaution for paragraphs that do not indent
%\subsection{A precaution for paragraphs that do not indent}
%\spoint A precaution for paragraphs that do not indent
\subsection{A precaution for paragraphs that do not indent}

%If you are setting a text with both the paragraph indentation
%and the white space between paragraphs zero, you run the risk
%that the start of a new paragraph may be indiscernible when
%the last line of the previous paragraph ends almost
%or completely flush right.
%A~sensible precaution for this is to set the \cs{parfillskip}
%to, for instance
%\begin{verbatim}
% \parfillskip=1cm plus 1fil
%\end{verbatim}
%instead of the usual \n{0cm~plus~1fil}.
If you are setting a text with both the paragraph indentation
and the white space between paragraphs zero, you run the risk
that the start of a new paragraph may be indiscernible when
the last line of the previous paragraph ends almost
or completely flush right.
A~sensible precaution for this is to set the \cs{parfillskip}
to, for instance
\begin{verbatim}
 \parfillskip=1cm plus 1fil
\end{verbatim}
instead of the usual \n{0cm~plus~1fil}.

%On the other hand, you may let yourself be convinced by
%\cite{Tsch} that paragraphs should always indent.
On the other hand, you may let yourself be convinced by
\cite{Tsch} that paragraphs should always indent.


%\index{paragraph!end|)}
%\endofchapter
\index{paragraph!end|)}
\endofchapter

\end{document}
