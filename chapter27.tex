% -*- coding: utf-8 -*-
\documentclass[letterpaper]{book}

% -*- coding: utf-8 -*-

\usepackage[b5paper,text={5in,8in},centering]{geometry}

\usepackage[CJKchecksingle]{xeCJK}
\setmainfont[Mapping=tex-text]{TeX Gyre Schola}
%\setsansfont{URW Gothic L Book}
%\setmonofont{Nimbus Mono L}
\setCJKmainfont[BoldFont=FandolHei,ItalicFont=FandolKai]{FandolSong}
\setCJKsansfont{FandolHei}
\setCJKmonofont{FandolFang}
\xeCJKsetup{PunctStyle = kaiming}

\linespread{1.25}
\setlength{\parindent}{2em}

\usepackage{xcolor}
\definecolor{myblue}{rgb}{0,0.2,0.6}

\usepackage{titlesec}
\titleformat{\chapter}
    {\normalfont\Huge\sffamily\color{myblue}}
    {第\thechapter 章}
    {1em}
    {}
%\titlespacing{\chapter}{0pt}{50pt}{40pt}
\titleformat{\section}
    {\normalfont\Large\sffamily\color{myblue}}
    {\thesection}
    {1em}
    {}
%\titlespacing{\section}{0pt}{3.5ex plus 1ex minus .2ex}{2.3ex plus .2ex}
\titleformat{\subsection}
    {\normalfont\large\sffamily\color{myblue}}
    {\thesubsection}
    {1em}
    {}
%\titlespacing{\subsection}{0pt}{3.25ex plus 1ex minus .2ex}{1.5ex plus .2ex}
%
\newpagestyle{special}[\small\sffamily]{
  \headrule
  \sethead[\usepage][][\chaptertitle]
  {\chaptertitle}{}{\usepage}}
\newpagestyle{main}[\small\sffamily]{
  \headrule
  \sethead[\usepage][][第\thechapter 章\quad\chaptertitle]
  {\thesection\quad\sectiontitle}{}{\usepage}}

\usepackage{titletoc}
%\setcounter{tocdepth}{1}
%\titlecontents{标题层次}[左间距]{上间距和整体格式}{标题序号}{标题内容}{指引线和页码}[下间距]
\titlecontents{chapter}[1.5em]{\vspace{.5em}\bfseries\sffamily}{\color{myblue}\contentslabel{1.5em}}{}
    {\titlerule*[20pt]{$\cdot$}\contentspage}[]
\titlecontents{section}[4.5em]{\sffamily}{\color{myblue}\contentslabel{3em}}{}
    {\titlerule*[20pt]{$\cdot$}\contentspage}[]
%\titlecontents{subsection}[8.5em]{\sffamily}{\contentslabel{4em}}{}
%    {\titlerule*[20pt]{$\cdot$}\contentspage}

\usepackage{enumitem}
\setlist{topsep=2pt,itemsep=2pt,parsep=1pt,leftmargin=\parindent}

\usepackage{fancyvrb}
\DefineVerbatimEnvironment{verbatim}{Verbatim}
  {xleftmargin=2em,baselinestretch=1,formatcom=\color{teal}\upshape}

\usepackage{etoolbox}
\makeatletter
\preto{\FV@ListVSpace}{\topsep=2pt \partopsep=0pt }
\makeatother

\usepackage[colorlinks,plainpages,pagebackref]{hyperref}
\hypersetup{
   pdfstartview={FitH},
   citecolor=teal,
   linkcolor=myblue,
   urlcolor=black,
   bookmarksnumbered
}

\usepackage{comment,makeidx,multicol}

%\usepackage{german}
%% german
%\righthyphenmin=3
%\mdqoff
%\captionsenglish
\usepackage[english]{babel}
{\catcode`"=13 \gdef"#1{\ifx#1"\discretionary{}{}{}\fi\relax}}
\def\mdqon{\catcode`"=13\relax}
\def\mdqoff{\catcode`"=12\relax}
\makeindex
\hyphenation{ex-em-pli-fies}

\newdimen\tempdima \newdimen\tempdimb

% these are fine
\def\handbreak{\\ \message{^^JManual break!!!!^^J}}
\def\nl{\protect\\}\def\n#1{{\tt #1}}
\def\cs#1{\texorpdfstring{{\tt\char`\\#1}}{\textbackslash#1}} %\def\cs#1{{\tt\char`\\#1}}
\let\csc\cs
\def\lb{{\tt\char`\{}}\def\rb{{\tt\char`\}}}
\def\gr#1{\texorpdfstring{$\langle$#1$\rangle$}{<#1>}} %\def\gr#1{$\langle$#1$\rangle$}
\def\key#1{{\tt#1}}
\def\alt{}\def\altt{}%this way in manstijl
\def\ldash{\unskip\ --\nobreak\ \ignorespaces}
\def\rdash{\unskip\nobreak\ --\ \ignorespaces}
% check these
\def\hex{{\tt"}}
\def\ascii{{\sc ascii}}
\def\ebcdic{{\sc ebcdic}}
\def\IniTeX{Ini\TeX}\def\LamsTeX{LAMS\TeX}\def\VirTeX{Vir\TeX}
\def\AmsTeX{Ams\TeX}
\def\TeXbook{the \TeX\ book}\def\web{{\sc web}}
% needs major thinking
\newenvironment{myquote}{\list{}{%
    \topsep=2pt \partopsep=0pt%
    \leftmargin=\parindent \rightmargin=\parindent
    }\item[]}{\endlist}
\newenvironment{disp}{\begin{myquote}}{\end{myquote}}
\newenvironment{Disp}{\begin{myquote}}{\end{myquote}}
\newenvironment{tdisp}{\begin{myquote}}{\end{myquote}}
\newenvironment{example}{\begin{myquote}\noindent\itshape 例子:}{\end{myquote}}
\newenvironment{inventory}{\begin{description}\raggedright}{\end{description}}
\newenvironment{glossinventory}{\begin{description}}{\end{description}}
\def\gram#1{\gr{#1}}%???
%
% index
%
\def\indexterm#1{\emph{#1}\index{#1}}
\def\indextermsub#1#2{\emph{#1 #2}\index{#1!#2}}
\def\indextermbus#1#2{\emph{#1 #2}\index{#2!#1}}
\def\term#1\par{\index{#1}}
\def\howto#1\par{}
\def\cstoidx#1\par{\index{#1@\cs{#1}@}}
\def\thecstoidx#1\par{\index{#1@\protect\csname #1\endcsname}}
\def\thecstoidxsub#1#2{\index{#1, #2@\protect\csname #1\endcsname, #2}\ignorespaces}
\def\csterm#1\par{\cstoidx #1\par\cs{#1}}
\def\csidx#1{\cstoidx #1\par\cs{#1}}

\def\tmc{\tracingmacros=2 \tracingcommands\tracingmacros}

%%%%%%%%%%%%%%%%%%%
\makeatletter
\def\snugbox{\hbox\bgroup\setbox\z@\vbox\bgroup
    \leftskip\z@
    \bgroup\aftergroup\make@snug
    \let\next=}
\def\make@snug{\par\sn@gify\egroup \box\z@\egroup}
\def\sn@gify
   {\skip\z@=\lastskip \unskip
    \advance\skip\z@\lastskip \unskip
    \unpenalty
    \setbox\z@\lastbox
    \ifvoid\z@ \nointerlineskip \else {\sn@gify} \fi
    \hbox{\unhbox\z@}\nointerlineskip
    \vskip\skip\z@
    }

\newdimen\fbh \fbh=60pt % dimension for easy scaling:
\newdimen\fbw \fbw=60pt % height and width of character box

\newdimen\dh \newdimen\dw % height and width of current character box
\newdimen\lh % height of previous character box
\newdimen\lw \lw=.4pt % line weight, instead of default .4pt

\def\hdotfill{\noindent
    \leaders\hbox{\vrule width 1pt height\lw
                  \kern4pt
                  \vrule width.5pt height\lw}\hfill\hbox{}
    \par}
\def\hlinefill{\noindent
    \leaders\hbox{\vrule width 5.5pt height\lw         }\hfill\hbox{}
    \par}
\def\stippel{$\qquad\qquad\qquad\qquad$}
\makeatother
%%%%%%%%%%%%%%%%%%%

%\def\SansSerif{\Typeface:macHelvetica }
%\def\SerifFont{\Typeface:macTimes }
%\def\SansSerif{\Typeface:bsGillSans }
%\def\SerifFont{\Typeface:bsBaskerville }
\let\SansSerif\relax \def\italic{\it}
\let\SerifFont\relax \def\MainFont{\rm}
\let\SansSerif\relax
\let\SerifFont\relax
\let\PopIndentLevel\relax \let\PushIndentLevel\relax
\let\ToVerso\relax \let\ToRecto\relax

%\def\stop@command@suffix{stop}
%\let\PopListLevel\PopIndentLevel
%\let\FlushRight\relax
%\let\flushright\FlushRight
%\let\SetListIndent\LevelIndent
%\def\awp{\ifhmode\vadjust{\penalty-10000 }\else
%    \penalty-10000 \fi}
\let\awp\relax
\let\PopIndentLevel\relax \let\PopListLevel\relax

\showboxdepth=-1

%\input figs
\def\endofchapter{\vfill\noindent}

\setcounter{chapter}{26}

\begin{document}

%\chapter{Page Breaking}\label{page:break}
\chapter{Page Breaking}\label{page:break}

%\index{page!breaking|(}
\index{page!breaking|(}

%This chapter treats the `page builder': the part of \TeX\
%that decides where to break the main vertical list into pages.
%The page builder operates before the output routine,
%and it hands its result in \cs{box255} to the output routine.
This chapter treats the `page builder': the part of \TeX\
that decides where to break the main vertical list into pages.
The page builder operates before the output routine,
and it hands its result in \cs{box255} to the output routine.

%\label{cschap:vsplit2}\label{cschap:splittopskip}\label{cschap:pagegoal}\label{cschap:pagetotal}\label{cschap:pagedepth}\label{cschap:pagestretch}\label{cschap:pagefilstretch}\label{cschap:pagefillstretch}\label{cschap:pagefilllstretch}\label{cschap:pageshrink}\label{cschap:outputpenalty}\label{cschap:interlinepenalty}\label{cschap:clubpenalty}\label{cschap:widowpenalty}\label{cschap:displaywidowpenalty}\label{cschap:brokenpenalty}\label{cschap:penalty2}\label{cschap:lastpenalty}\label{cschap:unpenalty}
%\begin{inventory}
%\item [\cs{vsplit}]
%      Split of a top part of a box. This is comparable
%      with page breaking.
\label{cschap:vsplit2}\label{cschap:splittopskip}\label{cschap:pagegoal}\label{cschap:pagetotal}\label{cschap:pagedepth}\label{cschap:pagestretch}\label{cschap:pagefilstretch}\label{cschap:pagefillstretch}\label{cschap:pagefilllstretch}\label{cschap:pageshrink}\label{cschap:outputpenalty}\label{cschap:interlinepenalty}\label{cschap:clubpenalty}\label{cschap:widowpenalty}\label{cschap:displaywidowpenalty}\label{cschap:brokenpenalty}\label{cschap:penalty2}\label{cschap:lastpenalty}\label{cschap:unpenalty}
\begin{inventory}
\item [\cs{vsplit}]
      Split of a top part of a box. This is comparable
      with page breaking.

%\item [\cs{splittopskip}] 
%      Minimum distance between the top of what remains after a
%      \cs{vsplit} operation, and the first item in that box.
%      Plain \TeX\ default:~\n{10pt}
\item [\cs{splittopskip}] 
      Minimum distance between the top of what remains after a
      \cs{vsplit} operation, and the first item in that box.
      Plain \TeX\ default:~\n{10pt}

%\item [\cs{pagegoal}] 
%      Goal height of the page box. This starts at \cs{vsize},
%      and is diminished by heights of insertion items.
\item [\cs{pagegoal}] 
      Goal height of the page box. This starts at \cs{vsize},
      and is diminished by heights of insertion items.

%\item [\cs{pagetotal}] 
%      Accumulated natural height of the current page.
\item [\cs{pagetotal}] 
      Accumulated natural height of the current page.

%\item [\cs{pagedepth}] 
%      Depth of the current page.
\item [\cs{pagedepth}] 
      Depth of the current page.

%\item [\cs{pagestretch}] 
%      Accumulated zeroth-order stretch of the current page.
\item [\cs{pagestretch}] 
      Accumulated zeroth-order stretch of the current page.

%\item [\cs{pagefilstretch}] 
%      Accumulated first-order stretch of the current page.
\item [\cs{pagefilstretch}] 
      Accumulated first-order stretch of the current page.

%\item [\cs{pagefillstretch}] 
%      Accumulated second-order stretch of the current page.
\item [\cs{pagefillstretch}] 
      Accumulated second-order stretch of the current page.

%\item [\cs{pagefilllstretch}] 
%      Accumulated third-order stretch of the current page.
\item [\cs{pagefilllstretch}] 
      Accumulated third-order stretch of the current page.

%\item [\cs{pageshrink}] 
%      Accumulated shrink of the current page.
\item [\cs{pageshrink}] 
      Accumulated shrink of the current page.

%\item [\cs{outputpenalty}]   
%      Value of the penalty at the current page break,
%      or $10\,000$ if the break was not at a penalty.
\item [\cs{outputpenalty}]   
      Value of the penalty at the current page break,
      or $10\,000$ if the break was not at a penalty.

%\item [\cs{interlinepenalty}] 
%      Penalty for breaking a page between lines of a paragraph. 
%      Plain \TeX\ default:~\n{0}
\item [\cs{interlinepenalty}] 
      Penalty for breaking a page between lines of a paragraph. 
      Plain \TeX\ default:~\n{0}

%\item [\cs{clubpenalty}] 
%      Additional penalty for breaking a page after 
%      the first line of a paragraph. 
%      Plain \TeX\ default:~\n{150}
\item [\cs{clubpenalty}] 
      Additional penalty for breaking a page after 
      the first line of a paragraph. 
      Plain \TeX\ default:~\n{150}

%\item [\cs{widowpenalty}] 
%      Additional penalty for breaking a page before 
%      the last line of a paragraph. 
%      Plain \TeX\ default:~\n{150}
\item [\cs{widowpenalty}] 
      Additional penalty for breaking a page before 
      the last line of a paragraph. 
      Plain \TeX\ default:~\n{150}

%\item [\cs{displaywidowpenalty}] 
%      Additional penalty for breaking a page before the last line 
%      above a display formula. 
%      Plain \TeX\ default:~\n{50}
\item [\cs{displaywidowpenalty}] 
      Additional penalty for breaking a page before the last line 
      above a display formula. 
      Plain \TeX\ default:~\n{50}

%\item [\cs{brokenpenalty}] 
%      Additional penalty for breaking a page after a hyphenated line. 
%      Plain \TeX\ default:~\n{100}
\item [\cs{brokenpenalty}] 
      Additional penalty for breaking a page after a hyphenated line. 
      Plain \TeX\ default:~\n{100}

%\item [\cs{penalty}]
%      Place a penalty on the current list.
%\item [\cs{lastpenalty}]
%      If the last item on the list was a penalty, the value of this.
%\item [\cs{unpenalty}]
%      Remove the last item of the current list if this
%      was a penalty.
\item [\cs{penalty}]
      Place a penalty on the current list.
\item [\cs{lastpenalty}]
      If the last item on the list was a penalty, the value of this.
\item [\cs{unpenalty}]
      Remove the last item of the current list if this
      was a penalty.

%\end{inventory}
\end{inventory}

%%\point The current page and the recent contributions
%\section{The current page and the recent contributions}
%\point The current page and the recent contributions
\section{The current page and the recent contributions}

%The main vertical list of \TeX\ is divided in two parts:
%the \indexterm{current page} and the list of \indexterm{recent contributions}.
%Any material that is added to the main vertical list is
%appended to the recent contributions; the act of moving
%the recent contributions to the current page is known
%as exercising the \indextermsub{page}{builder}.
The main vertical list of \TeX\ is divided in two parts:
the \indexterm{current page} and the list of \indexterm{recent contributions}.
Any material that is added to the main vertical list is
appended to the recent contributions; the act of moving
the recent contributions to the current page is known
as exercising the \indextermsub{page}{builder}.

%Every time something is moved to the current page, \TeX\
%computes the cost of breaking the page at that point.
%If it decides that it is past the optimal point,
%the current page up to 
%\altt 
%the best break so far
%is put in \cs{box255} and the remainder of
%the current page is moved back on top of the recent contributions.
%If the page is broken at a penalty,
%\label{break:penalty}%
%that value is recorded in \cs{outputpenalty}, and 
%a penalty of size $10\,000$ is placed on top of the
%recent contributions; otherwise, \csidx{outputpenalty}
%is set to~$10\,000$.
Every time something is moved to the current page, \TeX\
computes the cost of breaking the page at that point.
If it decides that it is past the optimal point,
the current page up to 
\altt 
the best break so far
is put in \cs{box255} and the remainder of
the current page is moved back on top of the recent contributions.
If the page is broken at a penalty,
\label{break:penalty}%
that value is recorded in \cs{outputpenalty}, and 
a penalty of size $10\,000$ is placed on top of the
recent contributions; otherwise, \csidx{outputpenalty}
is set to~$10\,000$.

%If the current page is empty, discardable items that are moved
%from the recent contributions are discarded. This is the mechanism
%that lets glue disappear after a page break and at the top of
%the first page. When the first non-discardable item is moved
%to the current page, the \cs{topskip} glue is inserted;
%see the previous chapter.
If the current page is empty, discardable items that are moved
from the recent contributions are discarded. This is the mechanism
that lets glue disappear after a page break and at the top of
the first page. When the first non-discardable item is moved
to the current page, the \cs{topskip} glue is inserted;
see the previous chapter.

%The workings of the page builder can be made visible by
%setting \cs{tracingpages} to some positive value
%(see Chapter~\ref{trace}).
The workings of the page builder can be made visible by
setting \cs{tracingpages} to some positive value
(see Chapter~\ref{trace}).

%%\point Activating the page builder
%\section{Activating the page builder}
%\point Activating the page builder
\section{Activating the page builder}

%The page builder comes into play  in the
%following circumstances.
%\begin{itemize}\item Around paragraphs: after the \cs{everypar}
%    tokens have been inserted, and after the paragraph has been
%    added to the vertical list. See the end of this chapter for 
%    an example.
%\item Around display formulas: after the \cs{everydisplay}
%    tokens have been inserted, and after the display has been
%    added to the list.
%\item After \cs{par} commands, boxes, insertions, 
%    and explicit penalties in vertical mode. 
%\item After an output routine has ended. \end{itemize}
%In these places the page builder moves the recent
%contributions to the current page. Note that \TeX\ need not be 
%in vertical mode when the page builder is exercised.
%In horizontal mode, activating the page builder
%serves to move preceding vertical glue (for example, \cs{parskip},
%\cs{abovedisplayskip}) to the page.
The page builder comes into play  in the
following circumstances.
\begin{itemize}\item Around paragraphs: after the \cs{everypar}
    tokens have been inserted, and after the paragraph has been
    added to the vertical list. See the end of this chapter for 
    an example.
\item Around display formulas: after the \cs{everydisplay}
    tokens have been inserted, and after the display has been
    added to the list.
\item After \cs{par} commands, boxes, insertions, 
    and explicit penalties in vertical mode. 
\item After an output routine has ended. \end{itemize}
In these places the page builder moves the recent
contributions to the current page. Note that \TeX\ need not be 
in vertical mode when the page builder is exercised.
In horizontal mode, activating the page builder
serves to move preceding vertical glue (for example, \cs{parskip},
\cs{abovedisplayskip}) to the page.

%The \cs{end} command \ldash which is only allowed in
%external vertical mode \rdash  terminates a \TeX\ job, but only if the
%main vertical list is empty and \cs{deadcycles}${}=0$.
%If this is not the case the combination
%\label{end:play}%
%\begin{disp}\verb>\hbox{}\vfill\penalty>$-2^{30}$\end{disp}
%is appended, which forces the output routine to act.
The \cs{end} command \ldash which is only allowed in
external vertical mode \rdash  terminates a \TeX\ job, but only if the
main vertical list is empty and \cs{deadcycles}${}=0$.
If this is not the case the combination
\label{end:play}%
\begin{disp}\verb>\hbox{}\vfill\penalty>$-2^{30}$\end{disp}
is appended, which forces the output routine to act.

%%\point Page length bookkeeping
%\section{Page length bookkeeping}
%\point Page length bookkeeping
\section{Page length bookkeeping}

%\index{page!length|(}
\index{page!length|(}

%The height and depth of the page box that reaches the output
%routine are determined by \cs{vsize}, \cs{topskip},
%and~\cs{maxdepth} as described in the previous chapter.
%\TeX\ places the \cs{topskip} glue
%when the first box is placed on the current page; the
%\cs{vsize} and \cs{maxdepth} are read when the first
%box or insertion occurs on the page. Any subsequent changes to these
%parameters will not be noticeable until the next page or,
%more strictly, until after the output routine has been called.
The height and depth of the page box that reaches the output
routine are determined by \cs{vsize}, \cs{topskip},
and~\cs{maxdepth} as described in the previous chapter.
\TeX\ places the \cs{topskip} glue
when the first box is placed on the current page; the
\cs{vsize} and \cs{maxdepth} are read when the first
box or insertion occurs on the page. Any subsequent changes to these
parameters will not be noticeable until the next page or,
more strictly, until after the output routine has been called.

%After the first box, rule, or insertion on the current page
%the \cs{vsize} is recorded in \cs{pagegoal},
%and its value is not looked at  until \cs{output}
%has been active.
%Changing \cs{pagegoal} does have an effect on the current
%page.
%When the page is empty,
%the pagegoal is \cs{maxdimen}, and \cs{pagetotal} is zero.
After the first box, rule, or insertion on the current page
the \cs{vsize} is recorded in \cs{pagegoal},
and its value is not looked at  until \cs{output}
has been active.
Changing \cs{pagegoal} does have an effect on the current
page.
When the page is empty,
the pagegoal is \cs{maxdimen}, and \cs{pagetotal} is zero.

%Accumulated dimensions and stretch are available in
%the parameters \cs{pagetotal}, \cs{pagedepth}, 
%\cs{pagestretch}, \cs{pagefilstretch}, \cs{pagefillstretch}, 
%\cs{pageshrink},
%and \cs{pagefilllstretch}.
%\cstoidx pagetotal\par\cstoidx pagedepth\par
%\cstoidx pagestretch\par\cstoidx pagefilstretch\par
%\cstoidx pagefillstretch\par
%\cstoidx pageshrink\par\cstoidx pagefilllstretch\par
%They are set by the page builder. The stretch and
%shrink parameters are updated every time glue is added
%to the page. The depth parameter becomes zero
%if the last item was kern or glue.
Accumulated dimensions and stretch are available in
the parameters \cs{pagetotal}, \cs{pagedepth}, 
\cs{pagestretch}, \cs{pagefilstretch}, \cs{pagefillstretch}, 
\cs{pageshrink},
and \cs{pagefilllstretch}.
\cstoidx pagetotal\par\cstoidx pagedepth\par
\cstoidx pagestretch\par\cstoidx pagefilstretch\par
\cstoidx pagefillstretch\par
\cstoidx pageshrink\par\cstoidx pagefilllstretch\par
They are set by the page builder. The stretch and
shrink parameters are updated every time glue is added
to the page. The depth parameter becomes zero
if the last item was kern or glue.

%These parameters are \gr{special dimen}s; an assignment
%to any of them is an \gr{intimate assignment},
%and it is automatically global.
These parameters are \gr{special dimen}s; an assignment
to any of them is an \gr{intimate assignment},
and it is automatically global.

%\index{page!length|)}
\index{page!length|)}

%\section{Breakpoints}
\section{Breakpoints}

%%\spoint Possible breakpoints
%\subsection{Possible breakpoints}
%\spoint Possible breakpoints
\subsection{Possible breakpoints}

%Page breaks can occur at the same kind of locations where
%line breaks can occur. The \indexterm{breakpoints in vertical lists}
%are:
%\begin{itemize}\item at glue that is preceded by a non-discardable
%item;\item at a kern that is immediately followed by glue;
%\item at a penalty.\end{itemize}
%\TeX\ inserts interline glue and various sorts of
%interline penalties when the lines of a paragraph are
%added to the vertical list, so there will usually be 
%sufficient breakpoints on the page.
Page breaks can occur at the same kind of locations where
line breaks can occur. The \indexterm{breakpoints in vertical lists}
are:
\begin{itemize}\item at glue that is preceded by a non-discardable
item;\item at a kern that is immediately followed by glue;
\item at a penalty.\end{itemize}
\TeX\ inserts interline glue and various sorts of
interline penalties when the lines of a paragraph are
added to the vertical list, so there will usually be 
sufficient breakpoints on the page.

%%\spoint Breakpoint penalties
%\subsection{Breakpoint penalties}
%\spoint Breakpoint penalties
\subsection{Breakpoint penalties}

%If \TeX\ decides to break a page at a penalty item, this
%penalty will, most of the time, be one that
%has been inserted automatically
%between the lines of a paragraph.
If \TeX\ decides to break a page at a penalty item, this
penalty will, most of the time, be one that
has been inserted automatically
between the lines of a paragraph.

%If the last item on a list (not necessarily a vertical list)
%\alt
%is a penalty, the value of this is recorded
%in the parameter \csidx{lastpenalty}. If the item is other than
%a penalty, this parameter has the value zero.
%The last penalty of a list can be removed with the command
%\csidx{unpenalty}. See Section~\ref{varioset} for an example.
%\message{Spoint ref varioset}
If the last item on a list (not necessarily a vertical list)
\alt
is a penalty, the value of this is recorded
in the parameter \csidx{lastpenalty}. If the item is other than
a penalty, this parameter has the value zero.
The last penalty of a list can be removed with the command
\csidx{unpenalty}. See Section~\ref{varioset} for an example.
\message{Spoint ref varioset}

%Here is a list of the \indextermsub{penalties}{in vertical mode}:
%\begin{inventory}
%\item [\cs{interlinepenalty}] 
%      Penalty for breaking a page between lines of a paragraph. 
%      In plain \TeX\ this is zero, so no penalty is added in
%      between lines. \TeX\ can then find a valid breakpoint at the
%      \cs{baselineskip} glue.
Here is a list of the \indextermsub{penalties}{in vertical mode}:
\begin{inventory}
\item [\cs{interlinepenalty}] 
      Penalty for breaking a page between lines of a paragraph. 
      In plain \TeX\ this is zero, so no penalty is added in
      between lines. \TeX\ can then find a valid breakpoint at the
      \cs{baselineskip} glue.

%\item [\cs{clubpenalty}] 
%      Extra penalty for breaking a page after the first line of a paragraph. 
%      In plain \TeX\ this is~\n{150}.
%      This amount, and the following penalties, are 
%      added to the \cs{interlinepenalty}, and
%      a penalty of the resulting size is inserted after the
%      \cs{hbox} containing the first line of a paragraph
%      instead of the \cs{interlinepenalty}.
\item [\cs{clubpenalty}] 
      Extra penalty for breaking a page after the first line of a paragraph. 
      In plain \TeX\ this is~\n{150}.
      This amount, and the following penalties, are 
      added to the \cs{interlinepenalty}, and
      a penalty of the resulting size is inserted after the
      \cs{hbox} containing the first line of a paragraph
      instead of the \cs{interlinepenalty}.

%\item [\cs{widowpenalty}] 
%      Extra penalty for breaking a page before the last line of a paragraph. 
%      In plain \TeX\ this is~\n{150}.
\item [\cs{widowpenalty}] 
      Extra penalty for breaking a page before the last line of a paragraph. 
      In plain \TeX\ this is~\n{150}.

%\item [\cs{displaywidowpenalty}] 
%      Extra penalty for breaking a page before the last line 
%      above a display formula. The default value in plain \TeX\
%      is~\n{50}.
\item [\cs{displaywidowpenalty}] 
      Extra penalty for breaking a page before the last line 
      above a display formula. The default value in plain \TeX\
      is~\n{50}.

%\item [\cs{brokenpenalty}] 
%      Extra penalty for breaking a page after a hyphenated line. 
%      The default value in plain \TeX\ is~\n{100}.
%\end{inventory}
%If the resulting penalty is zero, it is not placed.
\item [\cs{brokenpenalty}] 
      Extra penalty for breaking a page after a hyphenated line. 
      The default value in plain \TeX\ is~\n{100}.
\end{inventory}
If the resulting penalty is zero, it is not placed.

%Penalties can also be inserted by the user. For instance,
%the plain format has macros to encourage (possibly, force)
%or prohibit page breaks\cstoidx penalty\par:
%\begin{verbatim}
%\def\break{\penalty-10000 }        % force break
%\def\nobreak{\penalty10000 }       % prohibit break
%\def\goodbreak{\par\penalty-500 }  % encourage page break
%\end{verbatim}
%Also, \verb>\vadjust{\penalty ... }> is a way of getting
%penalties in the vertical list. This can be used to
%discourage or encourage page breaking after a certain
%line of a paragraph.
Penalties can also be inserted by the user. For instance,
the plain format has macros to encourage (possibly, force)
or prohibit page breaks\cstoidx penalty\par:
\begin{verbatim}
\def\break{\penalty-10000 }        % force break
\def\nobreak{\penalty10000 }       % prohibit break
\def\goodbreak{\par\penalty-500 }  % encourage page break
\end{verbatim}
Also, \verb>\vadjust{\penalty ... }> is a way of getting
penalties in the vertical list. This can be used to
discourage or encourage page breaking after a certain
line of a paragraph.

%%\spoint Breakpoint computation
%\subsection{Breakpoint computation}
%\spoint Breakpoint computation
\subsection{Breakpoint computation}

%\index{breakpoints!computation of|(}
%\advance\rightskip by 5.5cm
\index{breakpoints!computation of|(}
\advance\rightskip by 5.5cm

%Whenever an item is moved to the current page, \TeX\
%\vadjust{\advance\hsize by -5.5cm
%  \hbox to \hsize{\hfil\rlap{\hskip.4cm\vtop to 0pt
%  {\kern-2\baselineskip
%    \SansSerif %\pointSize:8 \Style:roman
%   \parindent0pt \offinterlineskip
%   \def\tbox#1{\hbox{\quad\quad #1%
%               \vrule height 10pt depth3pt width0cm }}
%   \hbox
%   {\vrule width\lw \kern-\lw
%    \vbox{\hsize=5cm
%          \hrule height\lw \ \vskip0cm
%           \kern40pt
%           \tbox{underfull page}
%           \tbox{$b=10\,000$}
%           \kern40pt
%          \hrule height\lw
%           \kern8pt
%           \tbox{feasible breakpoints}
%           \tbox{$b<10\,000$}
%           \kern8pt
%          \hrule height\lw
%           \kern8pt
%           \tbox{overfull page}
%           \tbox{$b=\infty$}
%           \kern3pt
%           \tbox{.\vrule height3.5pt depth1pt width0cm}
%           \tbox{.\vrule height3.5pt depth1pt width0cm}
%           \tbox{.\vrule height3.5pt depth1pt width0cm}
%           \kern8pt
%           }%
%    \kern-\lw \vrule width\lw}%
%    \vss}}}}
%computes the penalty $p$ and the badness $b$ associated with
%breaking the page at that place. From the penalty and
%the badness the cost $c$ of breaking is computed.
Whenever an item is moved to the current page, \TeX\
\vadjust{\advance\hsize by -5.5cm
  \hbox to \hsize{\hfil\rlap{\hskip.4cm\vtop to 0pt
  {\kern-2\baselineskip
    \SansSerif %\pointSize:8 \Style:roman
   \parindent0pt \offinterlineskip
   \def\tbox#1{\hbox{\quad\quad #1%
               \vrule height 10pt depth3pt width0cm }}
   \hbox
   {\vrule width\lw \kern-\lw
    \vbox{\hsize=5cm
          \hrule height\lw \ \vskip0cm
           \kern40pt
           \tbox{underfull page}
           \tbox{$b=10\,000$}
           \kern40pt
          \hrule height\lw
           \kern8pt
           \tbox{feasible breakpoints}
           \tbox{$b<10\,000$}
           \kern8pt
          \hrule height\lw
           \kern8pt
           \tbox{overfull page}
           \tbox{$b=\infty$}
           \kern3pt
           \tbox{.\vrule height3.5pt depth1pt width0cm}
           \tbox{.\vrule height3.5pt depth1pt width0cm}
           \tbox{.\vrule height3.5pt depth1pt width0cm}
           \kern8pt
           }%
    \kern-\lw \vrule width\lw}%
    \vss}}}}
computes the penalty $p$ and the badness $b$ associated with
breaking the page at that place. From the penalty and
the badness the cost $c$ of breaking is computed.

%The place of least cost is remembered, and when
%the cost is infinite, that is, the page is overfull, or
%when the penalty is $p\leq-10\,000$, the current page is broken
%at  the (last remembered) place of least cost. 
%The broken-off piece is then
%put in \cs{box255} and the output routine token list
%is inserted. Box 255 is always given a height of \cs{vsize},
%regardless of how much material it has.
The place of least cost is remembered, and when
the cost is infinite, that is, the page is overfull, or
when the penalty is $p\leq-10\,000$, the current page is broken
at  the (last remembered) place of least cost. 
The broken-off piece is then
put in \cs{box255} and the output routine token list
is inserted. Box 255 is always given a height of \cs{vsize},
regardless of how much material it has.

%The badness calculation is based on the amount of stretching
%or shrinking that is necessary to fit the page in
%a box with height \cs{vsize} 
%and maximum depth \cs{maxdepth}. This calculation is
%the same as for line breaking (see Chapter~\ref{glue}).
%Badness is a value $0\leq b\leq 10\,000$, except when
%pages are overfull; then~$b=\infty$.
The badness calculation is based on the amount of stretching
or shrinking that is necessary to fit the page in
a box with height \cs{vsize} 
and maximum depth \cs{maxdepth}. This calculation is
the same as for line breaking (see Chapter~\ref{glue}).
Badness is a value $0\leq b\leq 10\,000$, except when
pages are overfull; then~$b=\infty$.

%\advance\rightskip by -5.5cm
\advance\rightskip by -5.5cm

%Some penalties are implicitly inserted by \TeX, 
%for instance the \cs{interlinepenalty}
%which is put in between every pair of lines of a paragraph.
%Other penalties can
%be explicitly inserted by the user or a user macro. 
%A~penalty
%value $p\geq10\,000$ inhibits breaking; a penalty
%$p\leq-10\,000$ (in external vertical mode)
%\alt
%forces a page break, and immediately
%activates the output routine. 
Some penalties are implicitly inserted by \TeX, 
for instance the \cs{interlinepenalty}
which is put in between every pair of lines of a paragraph.
Other penalties can
be explicitly inserted by the user or a user macro. 
A~penalty
value $p\geq10\,000$ inhibits breaking; a penalty
$p\leq-10\,000$ (in external vertical mode)
\alt
forces a page break, and immediately
activates the output routine. 

%Cost calculation proceeds as follows:
%\begin{enumerate} \item When a penalty is so low that it forces
%a page break and immediate invocation of the output routine, 
%but the page is not overfull, that is
%\begin{disp}$b<\infty\quad\hbox{and}\quad p\leq-10\,000$\end{disp}
%the cost is equal to the penalty:~$c=p$.
Cost calculation proceeds as follows:
\begin{enumerate} \item When a penalty is so low that it forces
a page break and immediate invocation of the output routine, 
but the page is not overfull, that is
\begin{disp}$b<\infty\quad\hbox{and}\quad p\leq-10\,000$\end{disp}
the cost is equal to the penalty:~$c=p$.

%\item When penalties do not force anything, and the page is not
%overfull, that is
%\begin{disp}$b<\infty\quad\hbox{and}\quad |p|<10\,000$\end{disp}
%the cost is~$c=b+p$.
\item When penalties do not force anything, and the page is not
overfull, that is
\begin{disp}$b<\infty\quad\hbox{and}\quad |p|<10\,000$\end{disp}
the cost is~$c=b+p$.

%\item For pages that are very bad, that is
%\begin{disp}$b=10\,000\quad\hbox{and}\quad |p|<10\,000$\end{disp}
%the cost is~$c=10\,000$.
\item For pages that are very bad, that is
\begin{disp}$b=10\,000\quad\hbox{and}\quad |p|<10\,000$\end{disp}
the cost is~$c=10\,000$.

%\item An overfull page, that is
%\begin{disp}$b=\infty\quad\hbox{and}\quad p<10\,000$\end{disp}
%gives infinite cost:~$c=\infty$.
%In this case \TeX\ decides that the optimal break point
%must have occurred earlier, and it invokes the output routine.
%Values of \cs{insertpenalties} (see Chapter~\ref{insert})
%that exceed $10\,000$
%also give infinite cost.
%\end{enumerate}
\item An overfull page, that is
\begin{disp}$b=\infty\quad\hbox{and}\quad p<10\,000$\end{disp}
gives infinite cost:~$c=\infty$.
In this case \TeX\ decides that the optimal break point
must have occurred earlier, and it invokes the output routine.
Values of \cs{insertpenalties} (see Chapter~\ref{insert})
that exceed $10\,000$
also give infinite cost.
\end{enumerate}

%The fact that a penalty $p\leq-10\,000$ activates
%the output routine is used extensively
%in the \LaTeX\ output routine: 
%the excess $\mathopen|p\mathclose|-10\,000$ is
%a code indicating the reason for calling the output routine;
%see also the second example in the next chapter.
The fact that a penalty $p\leq-10\,000$ activates
the output routine is used extensively
in the \LaTeX\ output routine: 
the excess $\mathopen|p\mathclose|-10\,000$ is
a code indicating the reason for calling the output routine;
see also the second example in the next chapter.

%\index{breakpoints!computation of|)}
\index{breakpoints!computation of|)}

%\section{\protect\cs{vsplit}}
%\label{vsplit}
\section{\protect\cs{vsplit}}
\label{vsplit}

%The page-breaking operation is available to the user
%through the \csidx{vsplit} operation.
The page-breaking operation is available to the user
through the \csidx{vsplit} operation.

%\begin{example}
%\begin{verbatim}
%\setbox1 = \vsplit2 to \dimen3
%\end{verbatim}
%assigns to box~1 the top part of size \cs{dimen3}
%of box~2. This material is actually removed from box~2.
%Compare this with splitting off a chunk of size \cs{vsize}
%from the current page.
%\end{example}
\begin{example}
\begin{verbatim}
\setbox1 = \vsplit2 to \dimen3
\end{verbatim}
assigns to box~1 the top part of size \cs{dimen3}
of box~2. This material is actually removed from box~2.
Compare this with splitting off a chunk of size \cs{vsize}
from the current page.
\end{example}

%The extracted
%result of 
%\begin{disp}\cs{vsplit}\gr{8-bit number}\n{to}\gr{dimen}
%\end{disp} is a box with the following properties.
%\begin{itemize} \item Height equal to the specified \gr{dimen}; \TeX\ will
%      go through the original box register (which must contain
%      a vertical box) to find the best breakpoint. This may
%      result in an underfull box.
%\item Depth at most \csidx{splitmaxdepth}; this is analogous to
%            the \cs{maxdepth} for the page box, rather than the \cs{boxmaxdepth}
%      that holds for any box.
%\item A first and last mark in the \cs{splitfirstmark} and 
%      \cs{splitbotmark} registers.
%\end{itemize}
The extracted
result of 
\begin{disp}\cs{vsplit}\gr{8-bit number}\n{to}\gr{dimen}
\end{disp} is a box with the following properties.
\begin{itemize} \item Height equal to the specified \gr{dimen}; \TeX\ will
      go through the original box register (which must contain
      a vertical box) to find the best breakpoint. This may
      result in an underfull box.
\item Depth at most \csidx{splitmaxdepth}; this is analogous to
            the \cs{maxdepth} for the page box, rather than the \cs{boxmaxdepth}
      that holds for any box.
\item A first and last mark in the \cs{splitfirstmark} and 
      \cs{splitbotmark} registers.
\end{itemize}

%The remainder of the \cs{vsplit} operation is a box where
%\begin{itemize} \item all discardables have been removed
%      from the top;
%\item glue of size \csidx{splittopskip} has been inserted on top;
%      if the box being split was box~255, it
%      already had \cs{topskip} glue on top;
%\item its depth has been forced to be at most \cs{splitmaxdepth}.
%\end{itemize}
The remainder of the \cs{vsplit} operation is a box where
\begin{itemize} \item all discardables have been removed
      from the top;
\item glue of size \csidx{splittopskip} has been inserted on top;
      if the box being split was box~255, it
      already had \cs{topskip} glue on top;
\item its depth has been forced to be at most \cs{splitmaxdepth}.
\end{itemize}

%The bottom of the original box is always a valid breakpoint
%for the \cs{vsplit} operation. If this breakpoint is taken,
%the remainder box register is void. The extracted box
%can be empty; it is only void if the original box
%was void, or not a vertical box.
The bottom of the original box is always a valid breakpoint
for the \cs{vsplit} operation. If this breakpoint is taken,
the remainder box register is void. The extracted box
can be empty; it is only void if the original box
was void, or not a vertical box.

%Typically, the \cs{vsplit} operation is used to split off part
%of \cs{box255}. By setting \cs{splitmaxdepth} equal to \cs{boxmaxdepth}
%the result is something that could have been made by \TeX's page
%builder. After pruning the top of \cs{box255}, the 
%mark registers \cs{firstmark} and \cs{botmark} contain the first
%and last marks on the remainder of box~255.
%See the next chapter for more information on marks.
Typically, the \cs{vsplit} operation is used to split off part
of \cs{box255}. By setting \cs{splitmaxdepth} equal to \cs{boxmaxdepth}
the result is something that could have been made by \TeX's page
builder. After pruning the top of \cs{box255}, the 
mark registers \cs{firstmark} and \cs{botmark} contain the first
and last marks on the remainder of box~255.
See the next chapter for more information on marks.

%%\point Examples of page breaking
%\section{Examples of page breaking}
%\point Examples of page breaking
\section{Examples of page breaking}

%%\spoint Filling up a page
%\subsection{Filling up a page}
%\spoint Filling up a page
\subsection{Filling up a page}

%Suppose a certain vertical box is too large
%to fit on the remainder of the page.
%Then
%\begin{verbatim}
%\vfil\vbox{ ... }
%\end{verbatim}
%is the wrong way
%to fill up the page and push the box to the next.
%\TeX\ can only break at the start of the glue, and
%the \cs{vfil} is discarded after the break: the result
%is an underfull, or at least horribly stretched, page.
%On the other hand,
%\begin{verbatim}
%\vfil\penalty0 % or any other value
%\vbox{ ... }
%\end{verbatim}
%is the correct way: \TeX\ will break
%at the penalty, and the page will be filled.
Suppose a certain vertical box is too large
to fit on the remainder of the page.
Then
\begin{verbatim}
\vfil\vbox{ ... }
\end{verbatim}
is the wrong way
to fill up the page and push the box to the next.
\TeX\ can only break at the start of the glue, and
the \cs{vfil} is discarded after the break: the result
is an underfull, or at least horribly stretched, page.
On the other hand,
\begin{verbatim}
\vfil\penalty0 % or any other value
\vbox{ ... }
\end{verbatim}
is the correct way: \TeX\ will break
at the penalty, and the page will be filled.

%%\spoint Determining the breakpoint
%\subsection{Determining the breakpoint}
%\spoint Determining the breakpoint
\subsection{Determining the breakpoint}

%In the following examples the \cs{vsplit} operation is
%used, which has the same
%mechanism as page breaking.
In the following examples the \cs{vsplit} operation is
used, which has the same
mechanism as page breaking.

%Let the macros and
%parameter settings
%\begin{verbatim}
%\offinterlineskip \showboxdepth=1
%\def\High{\hbox{\vrule height5pt}}
%\def\HighAndDeep{\hbox{\vrule height2.5pt depth2.5pt}}
%\end{verbatim}
%be given.
Let the macros and
parameter settings
\begin{verbatim}
\offinterlineskip \showboxdepth=1
\def\High{\hbox{\vrule height5pt}}
\def\HighAndDeep{\hbox{\vrule height2.5pt depth2.5pt}}
\end{verbatim}
be given.

%First let us consider
%an example where a vertical list is simply stretched
%in order to reach a break point.
%\begin{verbatim}
%\splitmaxdepth=4pt 
%\setbox1=\vbox{\High \vfil \HighAndDeep}
%\setbox2=\vsplit1 to 9pt
%\end{verbatim}
%gives
%\begin{verbatim}
%> \box2=
%\vbox(9.0+2.5)x0.4, glue set 1.5fil
%.\hbox(5.0+0.0)x0.4 []
%.\glue 0.0 plus 1.0fil
%.\glue(\lineskip) 0.0
%.\hbox(2.5+2.5)x0.4 []
%\end{verbatim}
%The two boxes together have a height of \n{7.5pt},
%so the glue has to stretch~\n{1.5pt}.
First let us consider
an example where a vertical list is simply stretched
in order to reach a break point.
\begin{verbatim}
\splitmaxdepth=4pt 
\setbox1=\vbox{\High \vfil \HighAndDeep}
\setbox2=\vsplit1 to 9pt
\end{verbatim}
gives
\begin{verbatim}
> \box2=
\vbox(9.0+2.5)x0.4, glue set 1.5fil
.\hbox(5.0+0.0)x0.4 []
.\glue 0.0 plus 1.0fil
.\glue(\lineskip) 0.0
.\hbox(2.5+2.5)x0.4 []
\end{verbatim}
The two boxes together have a height of \n{7.5pt},
so the glue has to stretch~\n{1.5pt}.

%Next, we decrease the allowed depth of the resulting list.
%\begin{verbatim}
%\splitmaxdepth=2pt 
%\setbox1=\vbox{\High \vfil \HighAndDeep}
%\setbox2=\vsplit1 to 9pt
%\end{verbatim}
%gives
%\begin{verbatim}
%> \box2=
%\vbox(9.0+2.0)x0.4, glue set 1.0fil
%.\hbox(5.0+0.0)x0.4 []
%.\glue 0.0 plus 1.0fil
%.\glue(\lineskip) 0.0
%.\hbox(2.5+2.5)x0.4 []
%\end{verbatim}
%The reference point is moved down half a point,
%and the stretch is correspondingly diminished,
%\alt
%but this motion cannot lead to a larger dimension
%than was specified.
Next, we decrease the allowed depth of the resulting list.
\begin{verbatim}
\splitmaxdepth=2pt 
\setbox1=\vbox{\High \vfil \HighAndDeep}
\setbox2=\vsplit1 to 9pt
\end{verbatim}
gives
\begin{verbatim}
> \box2=
\vbox(9.0+2.0)x0.4, glue set 1.0fil
.\hbox(5.0+0.0)x0.4 []
.\glue 0.0 plus 1.0fil
.\glue(\lineskip) 0.0
.\hbox(2.5+2.5)x0.4 []
\end{verbatim}
The reference point is moved down half a point,
and the stretch is correspondingly diminished,
\alt
but this motion cannot lead to a larger dimension
than was specified.

%As an example of this,
%\alt
%consider the sequence
%\begin{verbatim}
%\splitmaxdepth=3pt 
%\setbox1=\vbox{\High \kern1.5pt \HighAndDeep}
%\setbox2=\vsplit1 to 9pt
%\end{verbatim}
%This gives a box exactly 9 points high and 2.5 points deep.
%Setting \verb>\splitmaxdepth=2pt> does not increase
%the height by half a point; instead, an underfull box
%results because an earlier break is taken.
As an example of this,
\alt
consider the sequence
\begin{verbatim}
\splitmaxdepth=3pt 
\setbox1=\vbox{\High \kern1.5pt \HighAndDeep}
\setbox2=\vsplit1 to 9pt
\end{verbatim}
This gives a box exactly 9 points high and 2.5 points deep.
Setting \verb>\splitmaxdepth=2pt> does not increase
the height by half a point; instead, an underfull box
results because an earlier break is taken.

%Sometimes the timing of actions is important.
%\TeX\ first locates a breakpoint that will lead
%to the requested height, then checks whether accommodating
%the \cs{maxdepth} or \cs{splitmaxdepth} will not
%violate that height.
Sometimes the timing of actions is important.
\TeX\ first locates a breakpoint that will lead
to the requested height, then checks whether accommodating
the \cs{maxdepth} or \cs{splitmaxdepth} will not
violate that height.

%Consider an example of this timing:
%\alt
%in
%\begin{verbatim}
%\splitmaxdepth=4pt 
%\setbox1=\vbox{\High \vfil \HighAndDeep}
%\setbox2=\vsplit1 to 7pt
%\end{verbatim}
%the result is {\italic not\/} a box
%of 7 points high and 3 points deep. Instead,
%\begin{verbatim}
%> \box2=
%\vbox(7.0+0.0)x0.4
%.\hbox(5.0+0.0)x0.4 []
%\end{verbatim}
%which is an underfull box.
Consider an example of this timing:
\alt
in
\begin{verbatim}
\splitmaxdepth=4pt 
\setbox1=\vbox{\High \vfil \HighAndDeep}
\setbox2=\vsplit1 to 7pt
\end{verbatim}
the result is {\italic not\/} a box
of 7 points high and 3 points deep. Instead,
\begin{verbatim}
> \box2=
\vbox(7.0+0.0)x0.4
.\hbox(5.0+0.0)x0.4 []
\end{verbatim}
which is an underfull box.

%%\spoint[par:page:build] The page builder after a paragraph
%\subsection{The page builder after a paragraph}
%\label{par:page:build}
%\spoint[par:page:build] The page builder after a paragraph
\subsection{The page builder after a paragraph}
\label{par:page:build}

%After a paragraph, the page builder moves material
%to the current page, but it does not decide whether a breakpoint
%has been found yet.
After a paragraph, the page builder moves material
to the current page, but it does not decide whether a breakpoint
has been found yet.

%\begin{example}
%\begin{verbatim}
%\output{\interrupt \plainoutput}% show when you're active
%\def\nl{\hfil\break}\vsize=22pt % make pages of two lines
%a\nl b\nl c\par \showlists      % make a 3-line paragraph
%\end{verbatim}
%will report
%\begin{verbatim}
%### current page:
%[...]
%total height 34.0
% goal height 22.0
%prevdepth 0.0, prevgraf 3 lines
%\end{verbatim}
%Even though more than enough
%material has been gathered, \cs{output} is only invoked
%when the next paragraph starts: typing a \n d gives
%\begin{verbatim}
%! Undefined control sequence.
%<output> {\interrupt 
%                     \plainoutput }
%<to be read again> 
%                   d
%\end{verbatim}
%when \cs{output} is inserted after \cs{everypar}.
%\end{example}
\begin{example}
\begin{verbatim}
\output{\interrupt \plainoutput}% show when you're active
\def\nl{\hfil\break}\vsize=22pt % make pages of two lines
a\nl b\nl c\par \showlists      % make a 3-line paragraph
\end{verbatim}
will report
\begin{verbatim}
### current page:
[...]
total height 34.0
 goal height 22.0
prevdepth 0.0, prevgraf 3 lines
\end{verbatim}
Even though more than enough
material has been gathered, \cs{output} is only invoked
when the next paragraph starts: typing a \n d gives
\begin{verbatim}
! Undefined control sequence.
<output> {\interrupt 
                     \plainoutput }
<to be read again> 
                   d
\end{verbatim}
when \cs{output} is inserted after \cs{everypar}.
\end{example}

%\index{page!breaking|)}
\index{page!breaking|)}

%\endofchapter
\endofchapter

\end{document}
