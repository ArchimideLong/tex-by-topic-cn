% -*- coding: utf-8 -*-
\documentclass[twoside,letterpaper,openright]{rapport3}

% -*- coding: utf-8 -*-

\usepackage[b5paper,text={5in,8in},centering]{geometry}

\usepackage[CJKchecksingle]{xeCJK}
\setmainfont[Mapping=tex-text]{TeX Gyre Schola}
%\setsansfont{URW Gothic L Book}
%\setmonofont{Nimbus Mono L}
\setCJKmainfont[BoldFont=FandolHei,ItalicFont=FandolKai]{FandolSong}
\setCJKsansfont{FandolHei}
\setCJKmonofont{FandolFang}
\xeCJKsetup{PunctStyle = kaiming}

\linespread{1.25}
\setlength{\parindent}{2em}

\usepackage{xcolor}
\definecolor{myblue}{rgb}{0,0.2,0.6}

\usepackage{titlesec}
\titleformat{\chapter}
    {\normalfont\Huge\sffamily\color{myblue}}
    {第\thechapter 章}
    {1em}
    {}
%\titlespacing{\chapter}{0pt}{50pt}{40pt}
\titleformat{\section}
    {\normalfont\Large\sffamily\color{myblue}}
    {\thesection}
    {1em}
    {}
%\titlespacing{\section}{0pt}{3.5ex plus 1ex minus .2ex}{2.3ex plus .2ex}
\titleformat{\subsection}
    {\normalfont\large\sffamily\color{myblue}}
    {\thesubsection}
    {1em}
    {}
%\titlespacing{\subsection}{0pt}{3.25ex plus 1ex minus .2ex}{1.5ex plus .2ex}
%
\newpagestyle{special}[\small\sffamily]{
  \headrule
  \sethead[\usepage][][\chaptertitle]
  {\chaptertitle}{}{\usepage}}
\newpagestyle{main}[\small\sffamily]{
  \headrule
  \sethead[\usepage][][第\thechapter 章\quad\chaptertitle]
  {\thesection\quad\sectiontitle}{}{\usepage}}

\usepackage{titletoc}
%\setcounter{tocdepth}{1}
%\titlecontents{标题层次}[左间距]{上间距和整体格式}{标题序号}{标题内容}{指引线和页码}[下间距]
\titlecontents{chapter}[1.5em]{\vspace{.5em}\bfseries\sffamily}{\color{myblue}\contentslabel{1.5em}}{}
    {\titlerule*[20pt]{$\cdot$}\contentspage}[]
\titlecontents{section}[4.5em]{\sffamily}{\color{myblue}\contentslabel{3em}}{}
    {\titlerule*[20pt]{$\cdot$}\contentspage}[]
%\titlecontents{subsection}[8.5em]{\sffamily}{\contentslabel{4em}}{}
%    {\titlerule*[20pt]{$\cdot$}\contentspage}

\usepackage{enumitem}
\setlist{topsep=2pt,itemsep=2pt,parsep=1pt,leftmargin=\parindent}

\usepackage{fancyvrb}
\DefineVerbatimEnvironment{verbatim}{Verbatim}
  {xleftmargin=2em,baselinestretch=1,formatcom=\color{teal}\upshape}

\usepackage{etoolbox}
\makeatletter
\preto{\FV@ListVSpace}{\topsep=2pt \partopsep=0pt }
\makeatother

\usepackage[colorlinks,plainpages,pagebackref]{hyperref}
\hypersetup{
   pdfstartview={FitH},
   citecolor=teal,
   linkcolor=myblue,
   urlcolor=black,
   bookmarksnumbered
}

\usepackage{comment,makeidx,multicol}

%\usepackage{german}
%% german
%\righthyphenmin=3
%\mdqoff
%\captionsenglish
\usepackage[english]{babel}
{\catcode`"=13 \gdef"#1{\ifx#1"\discretionary{}{}{}\fi\relax}}
\def\mdqon{\catcode`"=13\relax}
\def\mdqoff{\catcode`"=12\relax}
\makeindex
\hyphenation{ex-em-pli-fies}

\newdimen\tempdima \newdimen\tempdimb

% these are fine
\def\handbreak{\\ \message{^^JManual break!!!!^^J}}
\def\nl{\protect\\}\def\n#1{{\tt #1}}
\def\cs#1{\texorpdfstring{{\tt\char`\\#1}}{\textbackslash#1}} %\def\cs#1{{\tt\char`\\#1}}
\let\csc\cs
\def\lb{{\tt\char`\{}}\def\rb{{\tt\char`\}}}
\def\gr#1{\texorpdfstring{$\langle$#1$\rangle$}{<#1>}} %\def\gr#1{$\langle$#1$\rangle$}
\def\key#1{{\tt#1}}
\def\alt{}\def\altt{}%this way in manstijl
\def\ldash{\unskip\ --\nobreak\ \ignorespaces}
\def\rdash{\unskip\nobreak\ --\ \ignorespaces}
% check these
\def\hex{{\tt"}}
\def\ascii{{\sc ascii}}
\def\ebcdic{{\sc ebcdic}}
\def\IniTeX{Ini\TeX}\def\LamsTeX{LAMS\TeX}\def\VirTeX{Vir\TeX}
\def\AmsTeX{Ams\TeX}
\def\TeXbook{the \TeX\ book}\def\web{{\sc web}}
% needs major thinking
\newenvironment{myquote}{\list{}{%
    \topsep=2pt \partopsep=0pt%
    \leftmargin=\parindent \rightmargin=\parindent
    }\item[]}{\endlist}
\newenvironment{disp}{\begin{myquote}}{\end{myquote}}
\newenvironment{Disp}{\begin{myquote}}{\end{myquote}}
\newenvironment{tdisp}{\begin{myquote}}{\end{myquote}}
\newenvironment{example}{\begin{myquote}\noindent\itshape 例子:}{\end{myquote}}
\newenvironment{inventory}{\begin{description}\raggedright}{\end{description}}
\newenvironment{glossinventory}{\begin{description}}{\end{description}}
\def\gram#1{\gr{#1}}%???
%
% index
%
\def\indexterm#1{\emph{#1}\index{#1}}
\def\indextermsub#1#2{\emph{#1 #2}\index{#1!#2}}
\def\indextermbus#1#2{\emph{#1 #2}\index{#2!#1}}
\def\term#1\par{\index{#1}}
\def\howto#1\par{}
\def\cstoidx#1\par{\index{#1@\cs{#1}@}}
\def\thecstoidx#1\par{\index{#1@\protect\csname #1\endcsname}}
\def\thecstoidxsub#1#2{\index{#1, #2@\protect\csname #1\endcsname, #2}\ignorespaces}
\def\csterm#1\par{\cstoidx #1\par\cs{#1}}
\def\csidx#1{\cstoidx #1\par\cs{#1}}

\def\tmc{\tracingmacros=2 \tracingcommands\tracingmacros}

%%%%%%%%%%%%%%%%%%%
\makeatletter
\def\snugbox{\hbox\bgroup\setbox\z@\vbox\bgroup
    \leftskip\z@
    \bgroup\aftergroup\make@snug
    \let\next=}
\def\make@snug{\par\sn@gify\egroup \box\z@\egroup}
\def\sn@gify
   {\skip\z@=\lastskip \unskip
    \advance\skip\z@\lastskip \unskip
    \unpenalty
    \setbox\z@\lastbox
    \ifvoid\z@ \nointerlineskip \else {\sn@gify} \fi
    \hbox{\unhbox\z@}\nointerlineskip
    \vskip\skip\z@
    }

\newdimen\fbh \fbh=60pt % dimension for easy scaling:
\newdimen\fbw \fbw=60pt % height and width of character box

\newdimen\dh \newdimen\dw % height and width of current character box
\newdimen\lh % height of previous character box
\newdimen\lw \lw=.4pt % line weight, instead of default .4pt

\def\hdotfill{\noindent
    \leaders\hbox{\vrule width 1pt height\lw
                  \kern4pt
                  \vrule width.5pt height\lw}\hfill\hbox{}
    \par}
\def\hlinefill{\noindent
    \leaders\hbox{\vrule width 5.5pt height\lw         }\hfill\hbox{}
    \par}
\def\stippel{$\qquad\qquad\qquad\qquad$}
\makeatother
%%%%%%%%%%%%%%%%%%%

%\def\SansSerif{\Typeface:macHelvetica }
%\def\SerifFont{\Typeface:macTimes }
%\def\SansSerif{\Typeface:bsGillSans }
%\def\SerifFont{\Typeface:bsBaskerville }
\let\SansSerif\relax \def\italic{\it}
\let\SerifFont\relax \def\MainFont{\rm}
\let\SansSerif\relax
\let\SerifFont\relax
\let\PopIndentLevel\relax \let\PushIndentLevel\relax
\let\ToVerso\relax \let\ToRecto\relax

%\def\stop@command@suffix{stop}
%\let\PopListLevel\PopIndentLevel
%\let\FlushRight\relax
%\let\flushright\FlushRight
%\let\SetListIndent\LevelIndent
%\def\awp{\ifhmode\vadjust{\penalty-10000 }\else
%    \penalty-10000 \fi}
\let\awp\relax
\let\PopIndentLevel\relax \let\PopListLevel\relax

\showboxdepth=-1

%\input figs
\def\endofchapter{\vfill\noindent}

\setcounter{chapter}{19}

\begin{document}

%\chapter{Spacing}\label{space}
%\index{spacing|(}
\chapter{Spacing}\label{space}
\index{spacing|(}

%The usual interword space in \TeX\ is specified in the
%font information, but the user can override this.
%This chapter explains the rules by which
%\TeX\ calculates interword space.
The usual interword space in \TeX\ is specified in the
font information, but the user can override this.
This chapter explains the rules by which
\TeX\ calculates interword space.

%\label{cschap:char322}\label{cschap:spaceskip}\label{cschap:xspaceskip}\label{cschap:spacefactor}\label{cschap:sfcode}\label{cschap:frenchspacing}\label{cschap:nonfrenchspacing}
%\begin{inventory}
\label{cschap:char322}\label{cschap:spaceskip}\label{cschap:xspaceskip}\label{cschap:spacefactor}\label{cschap:sfcode}\label{cschap:frenchspacing}\label{cschap:nonfrenchspacing}
\begin{inventory}

%\item [\cs{\char32}]
%      Control space.
%      Insert the same amount of space as a space token would
%      if \cs{spacefactor}${}=1000$.
\item [\cs{\char32}]
      Control space.
      Insert the same amount of space as a space token would
      if \cs{spacefactor}${}=1000$.

%\item [\cs{spaceskip}] 
%      Interword glue if non-zero.
\item [\cs{spaceskip}] 
      Interword glue if non-zero.

%\item [\cs{xspaceskip}] 
%      Interword glue if non-zero and \cs{spacefactor}${}\geq2000$.
\item [\cs{xspaceskip}] 
      Interword glue if non-zero and \cs{spacefactor}${}\geq2000$.

%\item [\cs{spacefactor}] 
%      1000 times the ratio by which the stretch (shrink) component of the
%      interword glue should be multiplied (divided).
\item [\cs{spacefactor}] 
      1000 times the ratio by which the stretch (shrink) component of the
      interword glue should be multiplied (divided).

%\item [\cs{sfcode}] 
%      Value for \cs{spacefactor} associated with a character.
\item [\cs{sfcode}] 
      Value for \cs{spacefactor} associated with a character.

%\item [\cs{frenchspacing}]
%      Macro to switch off extra space after punctuation.
\item [\cs{frenchspacing}]
      Macro to switch off extra space after punctuation.

%\item [\cs{nonfrenchspacing}]
%      Macro to switch on extra space after punctuation.
\item [\cs{nonfrenchspacing}]
      Macro to switch on extra space after punctuation.

%\end{inventory}
\end{inventory}


%\section{Introduction}
\section{Introduction}

%In between words in a text, \TeX\ inserts space. This space has a
%natural component, plus stretch and shrink to make justified
%(right-aligned) text possible. Now, in certain styles of typesetting,
%there is more space after punctuation. This chapter discusses the
%mechanism that \TeX\ uses to realize such effect.
In between words in a text, \TeX\ inserts space. This space has a
natural component, plus stretch and shrink to make justified
(right-aligned) text possible. Now, in certain styles of typesetting,
there is more space after punctuation. This chapter discusses the
mechanism that \TeX\ uses to realize such effect.

%Here is the general idea:
%\begin{itemize}
%\item After every character token, the \cs{spacefactor} quantity is
%  updated with the space factor code of that character.
%\item When space is inserted, its natural size can be augmented
%  (if \cs{spacefactor}${}\geq2000$), and in general its stretch is
%  multiplied, and its shrink divided, by \cs{spacefactor}${}/1000$.
%\item There are further rules, for instance so that in \n{...word.)
%  And...} the space is modified according to the period, not the
%  closing parenthesis.
%\end{itemize}
Here is the general idea:
\begin{itemize}
\item After every character token, the \cs{spacefactor} quantity is
  updated with the space factor code of that character.
\item When space is inserted, its natural size can be augmented
  (if \cs{spacefactor}${}\geq2000$), and in general its stretch is
  multiplied, and its shrink divided, by \cs{spacefactor}${}/1000$.
\item There are further rules, for instance so that in \n{...word.)
  And...} the space is modified according to the period, not the
  closing parenthesis.
\end{itemize}

%%\point Automatic interword space
%\section{Automatic interword space}
%\point Automatic interword space
\section{Automatic interword space}


%For every space token in horizontal mode the interword glue
%of the current font
%is inserted, with stretch and shrink components, all
%determined by \cs{fontdimen} parameters.
%To be specific, font dimension~2 is the normal interword space,
%dimension~3 is the amount of stretch of the interword
%space, and 4~is the amount of shrink. Font dimension
%7 is called the `extra space'; see below (the list
%of all the font dimensions appears on page~\pageref{font:dims}). 
For every space token in horizontal mode the interword glue
of the current font
is inserted, with stretch and shrink components, all
determined by \cs{fontdimen} parameters.
To be specific, font dimension~2 is the normal interword space,
dimension~3 is the amount of stretch of the interword
space, and 4~is the amount of shrink. Font dimension
7 is called the `extra space'; see below (the list
of all the font dimensions appears on page~\pageref{font:dims}). 

%Ordinarily all spaces between words (in one font) would be treated the
%same. To allow for differently sized spaces \ldash for instance a
%typeset equivalent of the double spacing after punctuation in
%typewritten documents \rdash \TeX\ associates with each character a
%so-called \indextermsub{space}{factor}.
Ordinarily all spaces between words (in one font) would be treated the
same. To allow for differently sized spaces \ldash for instance a
typeset equivalent of the double spacing after punctuation in
typewritten documents \rdash \TeX\ associates with each character a
so-called \indextermsub{space}{factor}.

%When a character is added to the current horizontal list,
%the space factor code (\csidx{sfcode})
%of that character
%is assigned to the space factor \csidx{spacefactor}.
%There are two exceptions to this rule:
%\begin{itemize}
%\item When the space factor code is zero, the \cs{spacefactor} does
%  not change. This mechanism allows space factors to persist through
%  parentheses and such; see section~\ref{sec:sf-through-paren}.
%\item When the space factor code of the last character is ${>}1000$
%  and the current space factor is ${<}1000$, the space factor
%  becomes~1000. This mechanism prevents elongated spaces after
%  initials; see section~\ref{sec:sf-punct}.
%\end{itemize}
%The maximum space factor  is~$32\,767$.
When a character is added to the current horizontal list,
the space factor code (\csidx{sfcode})
of that character
is assigned to the space factor \csidx{spacefactor}.
There are two exceptions to this rule:
\begin{itemize}
\item When the space factor code is zero, the \cs{spacefactor} does
  not change. This mechanism allows space factors to persist through
  parentheses and such; see section~\ref{sec:sf-through-paren}.
\item When the space factor code of the last character is ${>}1000$
  and the current space factor is ${<}1000$, the space factor
  becomes~1000. This mechanism prevents elongated spaces after
  initials; see section~\ref{sec:sf-punct}.
\end{itemize}
The maximum space factor  is~$32\,767$.

%The stretch component of the interword space is
%multiplied by the space factor divided by 1000;
%the shrink component is divided by this factor. 
%The extra space (font dimension~7) is
%added to the natural component of the
%interword space when the space factor is~${}\geq2000$.
The stretch component of the interword space is
multiplied by the space factor divided by 1000;
the shrink component is divided by this factor. 
The extra space (font dimension~7) is
added to the natural component of the
interword space when the space factor is~${}\geq2000$.

%%\point User interword space
%\section{User interword space}
%\point User interword space
\section{User interword space}

%The user can override the interword space contained in
%the \cs{fontdimen} parameters
%by setting the
%\csidx{spaceskip} and the \csidx{xspaceskip} to non-zero values.
%If \cs{spaceskip} is non-zero, it is taken instead
%of the normal interword space
%(\cs{fontdimen2} plus \cs{fontdimen3} minus \cs{fontdimen4}), but
%a non-zero \cs{xspaceskip} is used as interword space if
%the space factor is~${}\geq2000$.
The user can override the interword space contained in
the \cs{fontdimen} parameters
by setting the
\csidx{spaceskip} and the \csidx{xspaceskip} to non-zero values.
If \cs{spaceskip} is non-zero, it is taken instead
of the normal interword space
(\cs{fontdimen2} plus \cs{fontdimen3} minus \cs{fontdimen4}), but
a non-zero \cs{xspaceskip} is used as interword space if
the space factor is~${}\geq2000$.

%If the \cs{spaceskip} is used,
%its stretch and shrink components are
%multiplied and divided respectively by \cs{spacefactor}$/1000$.
If the \cs{spaceskip} is used,
its stretch and shrink components are
multiplied and divided respectively by \cs{spacefactor}$/1000$.

%Note that, if \cs{spaceskip} and \cs{xspaceskip} are
%defined in terms of \n{em}, they change with the font.
Note that, if \cs{spaceskip} and \cs{xspaceskip} are
defined in terms of \n{em}, they change with the font.

%\begin{example} Let the following macros be given:
%\begin{verbatim}
%\def\a.{\vrule height10pt width4pt\spacefactor=1000\relax}
%\def\b.{\vrule height10pt width4pt\spacefactor=3000\relax}
%\def\c{\vrule height10pt width4pt\relax}
%\end{verbatim} 
%    then
\begin{example} Let the following macros be given:
\begin{verbatim}
\def\a.{\vrule height10pt width4pt\spacefactor=1000\relax}
\def\b.{\vrule height10pt width4pt\spacefactor=3000\relax}
\def\c{\vrule height10pt width4pt\relax}
\end{verbatim} 
    then

%%\begin{disp}\leavevmode\PopIndentLevel
%\begin{disp}\leavevmode\PopIndentLevel

%\hbox{%
%$\vcenter{\hsize=3in \snugbox{%
%\begin{verbatim}
%\vbox{
%\fontdimen2\font=4pt % normal space 
%\fontdimen7\font=3pt % extra space
%\a. \b. \c\par
%% zero extra space
%\fontdimen7\font=0pt
%\a. \b. \c\par
%% set \spaceskip for normal space
%\spaceskip=2\fontdimen2\font
%\a. \b. \c\par
%% set \xspaceskip
%\xspaceskip=2pt
%\a. \b. \c\par
%}
%\end{verbatim}
%}}$%
%%
%\quad\quad gives\quad\quad
%%
%\message{Check snug and drop!}%
%$\vcenter{\snugbox{\parindent0pt\parskip=0pt
%\def\a.{\vrule height10pt width4pt\spacefactor=1000\relax}
%\def\b.{\vrule height10pt width4pt\spacefactor=3000\relax}
%\def\c{\vrule height10pt width4pt\relax}
%\leavevmode\strut\par\hbox{}\hbox{}
%% set the normal space and extra space
%\fontdimen2\font=4pt \fontdimen7\font=3pt
%\a. \b. \c\par \vskip2\baselineskip
%% zero extra space
%\fontdimen7\font=0pt
%\a. \b. \c\par \vskip2\baselineskip
%% set \spaceskip for normal space
%\spaceskip=2\fontdimen2\font
%\a. \b. \c\par \vskip2\baselineskip
%% set \xspaceskip
%\xspaceskip=2pt
%\a. \b. \c\par \leavevmode\strut
%}}$%
%%
%}
%%\end{disp}
\hbox{%
$\vcenter{\hsize=3in \snugbox{%
\begin{verbatim}
\vbox{
\fontdimen2\font=4pt % normal space 
\fontdimen7\font=3pt % extra space
\a. \b. \c\par
% zero extra space
\fontdimen7\font=0pt
\a. \b. \c\par
% set \spaceskip for normal space
\spaceskip=2\fontdimen2\font
\a. \b. \c\par
% set \xspaceskip
\xspaceskip=2pt
\a. \b. \c\par
}
\end{verbatim}
}}$%
%
\quad\quad gives\quad\quad
%
\message{Check snug and drop!}%
$\vcenter{\snugbox{\parindent0pt\parskip=0pt
\def\a.{\vrule height10pt width4pt\spacefactor=1000\relax}
\def\b.{\vrule height10pt width4pt\spacefactor=3000\relax}
\def\c{\vrule height10pt width4pt\relax}
\leavevmode\strut\par\hbox{}\hbox{}
% set the normal space and extra space
\fontdimen2\font=4pt \fontdimen7\font=3pt
\a. \b. \c\par \vskip2\baselineskip
% zero extra space
\fontdimen7\font=0pt
\a. \b. \c\par \vskip2\baselineskip
% set \spaceskip for normal space
\spaceskip=2\fontdimen2\font
\a. \b. \c\par \vskip2\baselineskip
% set \xspaceskip
\xspaceskip=2pt
\a. \b. \c\par \leavevmode\strut
}}$%
%
}
%\end{disp}

%In all of these lines the glue is set at natural width. In the first
%line the high space factor value after \cs{b} causes the extra
%space \cs{fontdimen7} to be added. If this is zero (second line), the
%only difference between space factor values is the stretch/shrink
%ratio. In the third line the \cs{spaceskip} is taken
%for all space factor values. If the \cs{xspaceskip} is nonzero,
%it is taken (fourth line) instead of the \cs{spaceskip}
%for the high value of the space factor.
%\end{example}
In all of these lines the glue is set at natural width. In the first
line the high space factor value after \cs{b} causes the extra
space \cs{fontdimen7} to be added. If this is zero (second line), the
only difference between space factor values is the stretch/shrink
ratio. In the third line the \cs{spaceskip} is taken
for all space factor values. If the \cs{xspaceskip} is nonzero,
it is taken (fourth line) instead of the \cs{spaceskip}
for the high value of the space factor.
\end{example}

%%\point[tie] Control space and tie
%\section{Control space and tie}
%\label{tie}
%\point[tie] Control space and tie
\section{Control space and tie}
\label{tie}

%The control character \csc{\char32}, \indextermbus{control}{space}
%is a horizontal command
%which inserts a space,  \csidx{\char32}
%acting as if the current space factor is~1000.
%However, it does not affect the value of \cs{spacefactor}.
The control character \csc{\char32}, \indextermbus{control}{space}
is a horizontal command
which inserts a space,  \csidx{\char32}
acting as if the current space factor is~1000.
However, it does not affect the value of \cs{spacefactor}.

%Control space has two main uses. First, it is convenient to use after
%a control sequence: \verb+\TeX\ is fun!+
%Secondly, it can be used after abbreviations when \cs{nonfrenchspacing}
%(see below) is in effect. For example:
%\begin{verbatim}
%\hbox spread 9pt{\nonfrenchspacing
%      The Reverend Dr. Drofnats}
%\end{verbatim}
%gives
%\begin{disp} \hbadness=10000 \leavevmode
%\hbox spread 9pt{\nonfrenchspacing
%      The Reverend Dr. Drofnats}\end{disp}
%while
%\begin{verbatim}
%\hbox spread 9pt{\nonfrenchspacing
%      The Reverend Dr.\ Drofnats}
%\end{verbatim}
%gives
%\begin{disp} \hbadness=10000 \leavevmode
%\hbox spread 9pt{\nonfrenchspacing
%      The Reverend Dr.\ Drofnats}\end{disp}
%(The \n{spread 9pt} is used to make the effect more visible.)
Control space has two main uses. First, it is convenient to use after
a control sequence: \verb+\TeX\ is fun!+
Secondly, it can be used after abbreviations when \cs{nonfrenchspacing}
(see below) is in effect. For example:
\begin{verbatim}
\hbox spread 9pt{\nonfrenchspacing
      The Reverend Dr. Drofnats}
\end{verbatim}
gives
\begin{disp} \hbadness=10000 \leavevmode
\hbox spread 9pt{\nonfrenchspacing
      The Reverend Dr. Drofnats}\end{disp}
while
\begin{verbatim}
\hbox spread 9pt{\nonfrenchspacing
      The Reverend Dr.\ Drofnats}
\end{verbatim}
gives
\begin{disp} \hbadness=10000 \leavevmode
\hbox spread 9pt{\nonfrenchspacing
      The Reverend Dr.\ Drofnats}\end{disp}
(The \n{spread 9pt} is used to make the effect more visible.)

%The active character (in the plain format) tilde or
%\indexterm{tie},~\n{\char126},
%\term ~@\char126\par
%uses control space: it is defined as
%\begin{verbatim}
%\catcode`\~=\active
%\def~{\penalty10000\ }
%\end{verbatim}
%Such an active tilde is called a `tie'; it inserts an ordinary 
%amount of space, and prohibits breaking at this space.
The active character (in the plain format) tilde or
\indexterm{tie},~\n{\char126},
\term ~@\char126\par
uses control space: it is defined as
\begin{verbatim}
\catcode`\~=\active
\def~{\penalty10000\ }
\end{verbatim}
Such an active tilde is called a `tie'; it inserts an ordinary 
amount of space, and prohibits breaking at this space.


%%\point More on the space factor
%\section{More on the space factor}
%\point More on the space factor
\section{More on the space factor}

%%\spoint Space factor assignments
%\subsection{Space factor assignments}
%\spoint Space factor assignments
\subsection{Space factor assignments}

%The space factor of a particular character 
%is contained in its \indexterm{spacefactor code}
%and can be assigned as
%\cstoidx sfcode\par
%\begin{disp}\cs{sfcode}\gr{8-bit number}\gr{equals}\gr{number}\end{disp}
The space factor of a particular character 
is contained in its \indexterm{spacefactor code}
and can be assigned as
\cstoidx sfcode\par
\begin{disp}\cs{sfcode}\gr{8-bit number}\gr{equals}\gr{number}\end{disp}

%\IniTeX\ assigns a space factor code of 1000 to all characters
%\label{ini:sf}%
%except uppercase characters; they get a space factor code of~999.
%The plain format then assigns space factor codes greater than
%1000 to various punctuation symbols, for instance
%\verb-\sfcode`\.=3000-, which triples the stretch and shrink
%after a full stop. Also, for all space factor values $\geq2000$
%the extra space is added; see above.
\IniTeX\ assigns a space factor code of 1000 to all characters
\label{ini:sf}%
except uppercase characters; they get a space factor code of~999.
The plain format then assigns space factor codes greater than
1000 to various punctuation symbols, for instance
\verb-\sfcode`\.=3000-, which triples the stretch and shrink
after a full stop. Also, for all space factor values $\geq2000$
the extra space is added; see above.

%%\spoint Punctuation
%\subsection{Punctuation}
%\label{sec:sf-punct}
%\spoint Punctuation
\subsection{Punctuation}
\label{sec:sf-punct}

%Because the space factor cannot jump from a value below 1000
%to one above, a punctuation symbol after an uppercase
%character will not have the effect on the interword space
%that punctuation after a lowercase character has.
Because the space factor cannot jump from a value below 1000
to one above, a punctuation symbol after an uppercase
character will not have the effect on the interword space
that punctuation after a lowercase character has.

%\begin{example}
%\begin{verbatim}
%a% \sfcode`a=1000, space factor becomes 1000
%.% \sfcode`.=3000, spacefactor becomes 3000
% % subsequent spaces will be increased.

%A% \sfcode`A=999,  space factor becomes 999
%.% \sfcode`.=3000, space factor becomes 1000
% % subsequent spaces will not be increased.
%\end{verbatim}
%\end{example}
\begin{example}
\begin{verbatim}
a% \sfcode`a=1000, space factor becomes 1000
.% \sfcode`.=3000, spacefactor becomes 3000
 % subsequent spaces will be increased.

A% \sfcode`A=999,  space factor becomes 999
.% \sfcode`.=3000, space factor becomes 1000
 % subsequent spaces will not be increased.
\end{verbatim}
\end{example}

%Thus, initials
%are not mistaken for sentence ends.
%If an uppercase character does end a sentence, for instance
%\begin{verbatim}
%... and NASA.
%\end{verbatim}
%there are several solutions:
%\begin{verbatim}
%... NASA\spacefactor=1000.
%\end{verbatim}
%or
%\begin{verbatim}
%... NASA\hbox{}.
%\end{verbatim}
%which abuses the fact that after
%a box the space factor is set to~1000.
%The \LaTeX\ macro \cs{@} is equivalent to the first
%possibility.
Thus, initials
are not mistaken for sentence ends.
If an uppercase character does end a sentence, for instance
\begin{verbatim}
... and NASA.
\end{verbatim}
there are several solutions:
\begin{verbatim}
... NASA\spacefactor=1000.
\end{verbatim}
or
\begin{verbatim}
... NASA\hbox{}.
\end{verbatim}
which abuses the fact that after
a box the space factor is set to~1000.
The \LaTeX\ macro \cs{@} is equivalent to the first
possibility.

%In the plain format two macros are defined that switch between
%\cstoidx frenchspacing\par
%\cstoidx nonfrenchspacing\par
%uniform interword spacing, \indexterm{frenchspacing},
%and extra space after punctuation, which is more an American custom.
%The macro \cs{frenchspacing} sets the space factor code
%of all punctuation to~1000; the macro \cs{nonfrenchspacing}
%sets it to values greater than~1000.
In the plain format two macros are defined that switch between
\cstoidx frenchspacing\par
\cstoidx nonfrenchspacing\par
uniform interword spacing, \indexterm{frenchspacing},
and extra space after punctuation, which is more an American custom.
The macro \cs{frenchspacing} sets the space factor code
of all punctuation to~1000; the macro \cs{nonfrenchspacing}
sets it to values greater than~1000.

%Here are the actual definitions from \n{plain.tex}:
%\begin{verbatim}
%\def\frenchspacing{\sfcode`\.\@m \sfcode`\?\@m 
%  \sfcode`\!\@m \sfcode`\:\@m 
%  \sfcode`\;\@m \sfcode`\,\@m}
%\def\nonfrenchspacing{\sfcode`\.3000 \sfcode`\?3000
%  \sfcode`\!3000 \sfcode`\:2000
%  \sfcode`\;1500 \sfcode`\,1250 }
%\end{verbatim}
%where
%\begin{verbatim}
%\mathchardef\@m=1000
%\end{verbatim}
%is given in the plain format.
Here are the actual definitions from \n{plain.tex}:
\begin{verbatim}
\def\frenchspacing{\sfcode`\.\@m \sfcode`\?\@m 
  \sfcode`\!\@m \sfcode`\:\@m 
  \sfcode`\;\@m \sfcode`\,\@m}
\def\nonfrenchspacing{\sfcode`\.3000 \sfcode`\?3000
  \sfcode`\!3000 \sfcode`\:2000
  \sfcode`\;1500 \sfcode`\,1250 }
\end{verbatim}
where
\begin{verbatim}
\mathchardef\@m=1000
\end{verbatim}
is given in the plain format.

%French spacing is a somewhat controversial issue:
%\TeXbook\ acts as if non-French spacing
%is standard practice in printing, but for instance in~\cite{Hart}
%one finds `The space of the line should be used after
%all points in normal text'.
%Extra space after punctuation 
%may be considered a `typewriter habit', but this is
%not entirely true. It used to be a lot more common
%than it is nowadays, and there are rational arguments
%against it: the full stop (point, period) at the end of a
%sentence, where extra punctuation is most visible, 
%is rather small, so it carries some extra visual space 
% of its own above it. This book does not use extra space
%after punctuation.
French spacing is a somewhat controversial issue:
\TeXbook\ acts as if non-French spacing
is standard practice in printing, but for instance in~\cite{Hart}
one finds `The space of the line should be used after
all points in normal text'.
Extra space after punctuation 
may be considered a `typewriter habit', but this is
not entirely true. It used to be a lot more common
than it is nowadays, and there are rational arguments
against it: the full stop (point, period) at the end of a
sentence, where extra punctuation is most visible, 
is rather small, so it carries some extra visual space 
 of its own above it. This book does not use extra space
after punctuation.

%%\spoint Other non-letters
%\subsection{Other non-letters}
%\label{sec:sf-through-paren}
%\spoint Other non-letters
\subsection{Other non-letters}
\label{sec:sf-through-paren}

%The zero value of the space factor code makes
%characters that are not a letter and not punctuation
%`transparent' for the space factor.
The zero value of the space factor code makes
characters that are not a letter and not punctuation
`transparent' for the space factor.

%\message{check break after Example}
%\begin{example}
%\begin{verbatim}
%a% \sfcode`a=1000, space factor becomes 1000
%.% \sfcode`.=3000, spacefactor becomes 3000
% % subsequent spaces will be increased.

%a% \sfcode`a=1000, space factor becomes 1000
%.% \sfcode`.=3000, space factor becomes 3000
%)% \sfcode`)=0,    space factor stays 3000
% % subsequent spaces will be increased.
%\end{verbatim}
%\end{example}
\message{check break after Example}
\begin{example}
\begin{verbatim}
a% \sfcode`a=1000, space factor becomes 1000
.% \sfcode`.=3000, spacefactor becomes 3000
 % subsequent spaces will be increased.

a% \sfcode`a=1000, space factor becomes 1000
.% \sfcode`.=3000, space factor becomes 3000
)% \sfcode`)=0,    space factor stays 3000
 % subsequent spaces will be increased.
\end{verbatim}
\end{example}

%%\spoint Other influences on the space factor
%\subsection{Other influences on the space factor}
%\spoint Other influences on the space factor
\subsection{Other influences on the space factor}

%The space factor is 1000 when \TeX\ starts forming a
%horizontal list, in particular after \cs{indent}, \cs{noindent},
%and directly after a display. It is also 1000 after
%a \cs{vrule}, an accent, or a \gr{box} (in horizontal mode), but
%it is not influenced by \cs{unhbox} or \cs{unhcopy}
%commands.
The space factor is 1000 when \TeX\ starts forming a
horizontal list, in particular after \cs{indent}, \cs{noindent},
and directly after a display. It is also 1000 after
a \cs{vrule}, an accent, or a \gr{box} (in horizontal mode), but
it is not influenced by \cs{unhbox} or \cs{unhcopy}
commands.

%In the first column of a \cs{valign} the space factor of
%the surrounding horizontal list is carried over; similarly,
%after a vertical alignment the space factor is set to the
%value reached in the last column.
In the first column of a \cs{valign} the space factor of
the surrounding horizontal list is carried over; similarly,
after a vertical alignment the space factor is set to the
value reached in the last column.

%\endofchapter
\endofchapter

\end{document}

