
\chapter{Fonts in Formulas}\label{mathfont}

For math typesetting a single current font is not sufficient, as it
is for text typesetting. Instead \TeX\ uses several font families,
and each family can contain three fonts. This chapter
explains how font families are organized, and how \TeX\ determines
from what families characters should be taken.

\label{cschap:fam}\label{cschap:newfam}\label{cschap:textfont}\label{cschap:scriptfont}\label{cschap:scriptscriptfont}
\begin{inventory}

\item [\cs{fam}]  
      The number of the current font family.

\item [\cs{newfam}]  
      Allocate a new math font family.

\item [\cs{textfont}]   
      Access the textstyle font of a family.\alt

\item [\cs{scriptfont}]  
      Access the scriptstyle font of a family.\alt

\item [\cs{scriptscriptfont}]  
      Access the scriptscriptstyle font of a family.\alt

\end{inventory}

%\point Determining the font of a character in math mode
\section{Determining the font of a character in math mode}

The characters in math formulas can be taken from several
different fonts (or better, \indexterm{font families}) without any user
commands. For instance, in plain \TeX\ math formulas use
the roman font, the math italic font, 
the symbol font and the math extension font.

In order to determine from which font a character is to be
taken, \TeX\ considers for each character in a formula its
\cs{mathcode} (this is treated in Chapter~\ref{mathchar}).
A~\cs{mathcode} is a 15-bit number of the form
\verb."xyzz., where the hex digits
have the following meaning:
\begin{disp}
\n x:~class,\nl
\n y:~family,\nl
\n{zz}:~position in font.\end{disp}

In general only the family determines from what font 
a character is to be taken.
The class of a math character is mostly used to
control spacing and other aspects of  typesetting.
Typical classes include `relation', `operator', `delimiter';
see section~\ref{math:class} for details.

Class~7 is special in this respect: 
it is called `variable family'. 
If a character has a \cs{mathcode} of the form \verb."7yzz.
it is taken from family \n{y},
unless the parameter \cs{fam} has a value in the range 0--15;
then it is taken from family~\cs{fam}.


%\point Initial family settings
\section{Initial family settings}

Both lowercase and uppercase letters
are defined by \IniTeX\ to have math codes \verb>"71zz>,
\label{ini:fam}%
which means that they are of variable family, initially from
family~1.
As \TeX\ sets \verb.fam=-1., that is,
an invalid value, when a formula starts, 
characters are indeed taken from
family~1, which in plain \TeX\ is math italic.

Digits have math code \verb>"70zz> so they are initially from
family~0, in plain \TeX\ the roman font. 
All other character codes have a mathcode
assigned by \IniTeX\ as
\begin{disp}\cs{mathcode}$\,x=x$\end{disp} which puts them in class~0,
ordinary, and family~0, roman in plain \TeX.

In plain \TeX, commands such as \cs{sl} then set both a font and
a family:
\begin{verbatim}
\def\sl{\fam\slfam\tensl}
\end{verbatim}
so putting \cs{sl} in a formula will cause all letters, digits,
and  uppercase Greek characters, to change to
slanted style.

In most cases, any font can be assigned to any family, but
two families in \TeX\ have a special meaning: these are
families 2 and~3.
For instance, their number of \cs{fontdimen} parameters
is different from the usual~7. Family~2 needs 22 parameters,
and family~3 needs~13. These parameters have all a very
specialized meaning for positioning in math typesetting. 
Their meaning is explained below, but for the full story
the reader is referred to appendix~G of \TeXbook.


%\point Family definition
\section{Family definition}

\TeX\ can access 16 families of fonts in math mode;
font families have numbers 0--15. 
The number of the
current family is recorded in the parameter~\csidx{fam}.

The macro \csidx{newfam} gives the number of an unused family.
This number is assigned using \cs{chardef} to the control sequence.


Each font family can have a font meant for text style, script style,
and scriptscript style. Below it is explained how \TeX\
determines in what style a (sub-) formula is to be typeset.

Fonts are assigned to a family
\cstoidx textfont\par\cstoidx scriptfont\par\cstoidx scriptscriptfont\par
as follows:
\begin{verbatim}
\newfam\MyFam
\textfont\MyFam=\tfont \scriptfont\MyFam=\sfont
\scriptscriptfont\MyFam=\ssfont
\end{verbatim}
for the text, script, and scriptscript fonts of a family.
In general it is not necessary to fill all three members
of a family (but it is for family~3). 
If \TeX\ needs a character from a family member
that has not been filled,
it uses the \cs{nullfont} instead,
a~primitive font that has no characters (nor a \n{.tfm} file).


%\point Some specific font changes
\section{Some specific font changes}

%\spoint Change the font of ordinary characters and uppercase Greek
\subsection{Change the font of ordinary characters and uppercase Greek}

All letters and the uppercase Greek characters are
by default in plain \TeX\ of class~7,
variable family, so changing \cs{fam} will change the font
from which they are taken.
For example
\begin{verbatim}
{\fam=9 x}
\end{verbatim}
 gives an \n{x} from family~9.

Uppercase Greek characters are defined by
\cs{mathchardef} statements in the plain format as \verb>"70zz>,
that is, variable family, initially roman.
Therefore, uppercase Greek character also change with the family.

%\spoint Change uppercase Greek independent of text font
\subsection{Change uppercase Greek independent of text font}

In the Computer Modern font layout, uppercase Greek letters
are part of the roman font; see page~\pageref{cmr:table}.
\alt
Therefore, introducing another
text font (with another layout)
will change the uppercase Greek characters
(or even make them disappear).
One way of remedying this is by introducing a new family in
which the \n{cmr} font, which contains the uppercase Greek,
resides.
The control sequences accessing these characters then have
to be redefined:
\begin{verbatim}
\newfam\Kgreek 
\textfont\Kgreek=cmr10 ...
\def\hex#1{\ifcase#10\or 1\or 2\or 3\or 4\or 5\or 6\or
    7\or 8\or 9\or A\or B\or C\or D\or E\or F\fi}
\mathchardef\Gamma="0\hex\Kgreek00 % was: "0100
\mathchardef\Beta ="0\hex\Kgreek01 % was: "0101
\mathchardef\Gamma ...
\end{verbatim}
Note, by the way,
the absence of a either a space or a \cs{relax} token after
\n{\#1} in the definition of \cs{hex}. This implies that this
macro can only be called with an argument that is a 
control sequence.

%\spoint Change the font of lowercase Greek 
\subsection{Change the font of lowercase Greek
       and mathematical symbols}

Lowercase Greek characters have math code
\verb>"01zz>, meaning they are always from the math italic family. 
In order to change this one might redefine them,
\mdqon
for instance \verb.\mathchardef.""\verb.\alpha="710B., 
\mdqoff
to make them variable family.
This is not done in plain \TeX, because the Computer Modern
roman font does not
have Greek lowercase, although it does have the uppercase characters.

Another way is to redefine them like \verb.\mathchardef\alpha="0n0B.
where \n{n} is the (hexadecimal) number of a family
compatible with math italic, containing for instance a bold
math italic font.


%\point Assorted remarks
\section{Assorted remarks}

%\spoint New fonts in formulas
\subsection{New fonts in formulas}

There are two ways to access a font inside mathematics.
\howto Change fonts in a math formula\par
\mdqon
After \cs{font}""\cs{newfont=....} it is not possible to get
\mdqoff
the `a' of the new font by \verb-$...{\newfont a}...$-
because \TeX\ does not look at the current font in math mode.
What does work is
\begin{verbatim}
$ ... \hbox{\newfont a} ...$
\end{verbatim}
but this precludes the use of the new font in script and 
scriptscript styles.

The proper solution  takes a bit more work:
\begin{verbatim}
\font\newtextfont=... 
\font\newscriptfont=... \font\newsscriptfont=...
\newfam\newfontfam
\textfont\newfontfam=\newtextfont
\scriptfont\newfontfam=\newscriptfont
\scriptscriptfont\newfontfam=\newsscriptfont
\def\newfont{\newtextfont \fam=\newfontfam}
\end{verbatim}
after which the font can be used as
\begin{verbatim}
$... {\newfont a_{b_c}} ...$
\end{verbatim}
in all three styles.

%\spoint Evaluating the families
\subsection{Evaluating the families}

\TeX\ will only look at what is actually in the \cs{textfont}
et cetera of the various families at the end of the whole
formula. Switching fonts in the families is thus not possible
inside a single formula.
The number of 16 families may therefore turn out to be restrictive
for some applications.


\endofchapter
