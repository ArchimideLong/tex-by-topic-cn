
\chapter{Macros}\label{macro}

Macros are \TeX's abbreviation mechanism for sequences of commands
that are needed more than once,
somewhat like procedures in ordinary programming languages.
\TeX's parameter mechanism, however, is quite unusual.
This chapter explains how \TeX\ macros work. It also
treats the commands \cs{let} and~\cs{futurelet}.

\label{cschap:def}\label{cschap:gdef}\label{cschap:edef}\label{cschap:xdef}\label{cschap:csname}\label{cschap:endcsname}\label{cschap:global2}\label{cschap:outer}\label{cschap:long}\label{cschap:let}\label{cschap:futurelet}
\begin{inventory}
\item [\cs{def}] 
      Start a macro definition.

\item [\cs{gdef}] 
      Synonym for \verb-\global\def-.

\item [\cs{edef}] 
      Start a macro definition; 
      the replacement text is expanded at definition time.
      This command is treated also in the next chapter.

\item [\cs{xdef}] 
      Synonym for \verb-\global\edef-.

\item [\cs{csname}] 
      Start forming the name of a control sequence.

\item [\cs{endcsname}] 
      Stop forming the name of a control sequence.

\item [\cs{global}] 
      Make the next definition, arithmetic statement,
      or assignment global.

\item [\cs{outer}] 
      Prefix indicating that the macro being defined 
      can be used on the `outer' level only.

\item [\cs{long}] 
      Prefix indicating that the arguments of the macro being defined
      may contain \cs{par} tokens.

\item [\cs{let}] 
      Define a control sequence to be equivalent to the next token.

\item [\cs{futurelet}] 
      Define a control sequence to be equivalent to
      the token after the next token.

\end{inventory}

\section{Introduction}

A \indexterm{macro}
is basically a sequence of tokens that has
been abbreviated into a control sequence.
Statements starting with (among others) \cs{def}
are called {\italic macro definitions}\alt, and
writing
\begin{verbatim}
\def\abc{\de f\g}
\end{verbatim}
defines the macro \cs{abc},
with the {\italic replacement text\/} \verb>\de f\g>.
Macros can be used in this way to abbreviate
pieces of text or sequences of commands
that have to be given more than once.
Any time that \TeX's expansion processor
encounters the control sequence \cs{abc},
it replaces it by the replacement text.

If a macro should be sensitive to the context
where it is used, it can be defined with parameters.
A~macro \verb+\PickTwo+ defined as
\begin{verbatim}
\def\PickTwo#1#2{(#1,#2)}
\end{verbatim}
has two \emph{parameters}. When it is used, it will scoop up two pieces of
text, the corresponding \emph{arguments},
and reproduces them in parentheses.
For example:

\begin{tabular}{|l|llllll|}
\hline
          &macro              &argument1& argument2& 
    &expansion&\\ \hline
definition&\verb+\def\PickTwo+&\verb+#1+& \verb+#2+& \verb+{+& 
    \verb+(#1,#2)+& \verb+}+\\
use&       \verb+\PickTwo+&1&2&&(1,2)&\\
use&       \verb+\PickTwo+&\verb+{ab}+&\verb+{cd}+&&(ab,cd)&\\
\hline
\end{tabular}

The activity of substituting the replacement text
for a macro and its arguments is called {\italic macro expansion}.

%\point Layout of a macro definition
\section{Layout of a macro definition}

A \indextermsub{macro}{definition} consists of, in sequence,
\begin{enumerate} \item any number of \cs{global},
\cs{long}, and \cs{outer} prefixes,
\item a \gr{def} control sequence, or anything
that has been \cs{let} to one,
\item a control sequence or active character to be defined, 
\item possibly a \gr{parameter text} specifying among other things
how many parameters the macro has, and
\item a replacement text enclosed in explicit character tokens
with category codes 1\index{category!1} and~2\index{category!2},
by default \verb-{- and~\verb-}-
in plain \TeX.
\end{enumerate}
These elements will all be explained in subsequent sections.

After a macro definition is completed, any saved \cs{afterassignment}
token (see section~\ref{sec:afterassignment}) is inserted.

The `expanding' definitions \cs{edef} and \cs{xdef}
are treated in Chapter~\ref{expand}.

%\point Prefixes
\section{Prefixes}

There are three \emph{prefixes}\index{prefixes !macro}
that alter the status of the
macro definition:
\begin{description}
\item [\csidx{global}]
If the definition occurs inside a  group, this prefix
makes the definition global.
This prefix can also be used for assignments other than
macro definitions; in fact,
for macro definitions abbreviations exist obviating the
use of \cs{global}:
\begin{disp}\verb>\gdef\foo...>\quad is equivalent to\quad \verb>\global\def\foo...>
\end{disp} and
\begin{disp}\verb>\xdef\foo...>\quad is equivalent to\quad \verb>\global\edef\foo...>
\end{disp}

If the parameter \cs{globaldefs}
is positive, all assignments are
implicitly global;
if\handbreak \cs{globaldefs} is negative any \cs{global} prefixes are
ignored,
and \cs{gdef} and \cs{xdef} make local definitions
(see Chapter~\ref{group}).

\item [\cs{outer}]
The mechanism of defining an \indextermbus{outer}{macro} is supposed to facilitate
\cstoidx outer\par
locating (among other errors) unbalanced braces: an \cs{outer}
macro is supposed
to  appear only in non-embedded contexts.
To be precise, it is not allowed to occur 
\begin{itemize}
\item in macro replacement texts (but it can appear in
    for instance \cs{edef} after 
    \cs{noexpand}, and after \cs{meaning}),
\item in parameter texts,
\item in skipped conditional text,
\item in alignment preambles, and
\item in the \gram{balanced text} of a \cs{message}, \cs{write},
et cetera. \end{itemize}
For certain applications, however, it is inconvenient
that some of the plain macros  are outer, 
in particular macros such as \cs{newskip}. One remedy is to
redefine them, without the `outer' option, which
is done for instance in \LaTeX, but  cleverer tricks are possible.

\item [\cs{long}]
Ordinarily, macro parameters are not supposed to contain
\cstoidx long\par
\cs{par} tokens. This restriction is useful (much more so
than the \cs{outer} definitions) in locating
forgotten closing braces. 
For example, \TeX\ will complain about a `runaway argument'
\message{Example on}
in the following sequence:
\begin{verbatim}
\def\a#1{ ... #1 ... }
\a {This sentence should be in braces.

And this is not supposed to be part of the argument
\end{verbatim}
\message{one page}
The empty line generates a \cs{par}, which most of the times
means that a closing brace has been forgotten.

If arguments to a particular macro should be allowed
to contain \cs{par} tokens,  then the macro must be declared
to be \cs{long}. \end{description}

The \cs{ifx} test for equality of tokens 
(see Chapter~\ref{if}) takes prefixes into
account when testing whether two tokens have the same definition.

\begin{comment}
With a little ingenuity it is possible 
for \cs{par} tokens to sneak into macro arguments anyway.
Consider the example
\begin{verbatim}
\def\a#1\par!{ ... }
\a bc\par ef\par!
\end{verbatim}
Here the macro \cs{a} is not \cs{long}, but the argument
is \verb>bc\par ef>, which contains a \cs{par} token.
However,
this is of no importance in general.
\end{comment}

%\point The definition type
\section{The definition type}

There are four \gr{def} control sequences in \TeX:
\csidx{def}, \csidx{gdef}, \csidx{edef}, and \csidx{xdef}.
The control sequence 
\alt
\cs{gdef} is a synonym for \verb>\global\def> and
\cs{xdef} is a synonym for \verb>\global\edef>.
The `expanding definition' \cs{edef} is treated in 
Chapter~\ref{expand}.

The difference between the various types of macro definitions
is only relevant at the time of the definition.
When a macro is called there is no way of telling how
it was defined.

%\point[param:text] The parameter text
\section{The parameter text}
\label{param:text}

Between the control sequence or active character to be defined
and the opening brace of the replacement text, a \gr{parameter
text} can occur, somewhat corresponding to \indexterm{arguments}
in regular programming languages. This specifies whether the macro has 
parameters\index{parameter},
how many, and how they are delimited. 
The \gr{parameter text} cannot contain
explicit braces.

A macro can have at most nine parameters. 
A~parameter is indicated by a parameter token,
consisting of a macro parameter character
(that is, a character of category code~6\index{category!6},
in plain \TeX~\verb=#=) 
followed by a digit~\n1--\n9. 
For instance, \verb>#6>~denotes the sixth parameter of a macro.
Parameter tokens cannot appear outside the context
of a macro definition.

In the parameter text,
parameters must be numbered consecutively, starting at~1.
A~space after a parameter token is significant,
both in the parameter text and the replacement text.

Parameters can be delimited or undelimited; this determines what the
extent of the macro arguments will be. A~parameter
is called undelimited if it is followed immediately
by another parameter in the \gr{parameter text}, so in
\verb+\def\foo#1#2+ the first parameter is undelimited.
A~parameter is also undelimited if it is immediately followed
by the opening brace of the replacement text, as in \verb+\def\foo#1{...}+.
A~parameter is called delimited if it is followed by any other token; 
in \verb+\def\foo#1!#2{...}+ the first parameter is delimited by the
exclamation sign.

The tokens (zero or more) that are substituted for
a parameter when a macro is expanded (or `called')
are called
the `argument' corresponding to that parameter.

%\spoint Undelimited parameters
\subsection{Undelimited parameters}

When a macro with an \indextermbus{undelimited}{parameter}, for instance
a macro \cs{foo} with one parameter
\begin{verbatim}
\def\foo#1{ ... #1 ...}
\end{verbatim}
is expanded, \TeX\ scans ahead (without expanding)
until a non-blank token is found.
If this token is not an explicit \gr{left brace}, 
it is taken to be the argument
corresponding to the parameter. Otherwise a \gr{balanced text}
is absorbed by scanning until the matching explicit
\gr{right brace} has been found.
This balanced text then
constitutes the argument.

An example with three undelimited parameters follows: with
\begin{verbatim}
\def\foo#1#2#3{#1(#2)#3}
\end{verbatim}
the macro call \cs{foo123} gives `\hbox{1(2)3}';
but \hbox{\verb-\foo 1 2 3-} also gives the same result.
In the call
\begin{disp}\cs{foo}\n{\char32 1\char32 2\char 32 3}\end{disp}
the first space is skipped in the input processor of \TeX.
The argument corresponding to the first parameter is then
the~\n1. In order to find the second parameter \TeX\ then
skips all blanks, in this case exactly one. As second
parameter \TeX\ finds then the~\n2. Similarly the third
parameter is~\n3.


In order to pass several tokens as one undelimited argument
one can use braces. With the above definition of \cs{foo}
the call \verb>\foo a{bc}d> gives `\hbox{a(bc)d}'.
When the argument of a macro is a balanced text instead of
a single token, the delimiting braces are not inserted when 
the argument is
inserted in the replacement text.
For example:
\begin{verbatim}
\def\foo#1{\count0=1#1\relax}
\foo{23}
\end{verbatim}
will expand to \verb>\count0=123\relax>,
which assigns the value of 123 to the counter.
On the other hand,  the statement
\begin{verbatim}
\count0=1{23}
\end{verbatim}
would
assign~1 and print~23.

%\spoint Delimited parameters
\subsection{Delimited parameters}

Apart from enclosing it in braces there is another way
to pass a sequence of tokens as a single argument to a macro,
namely by using a \indextermbus{delimited}{parameter}.

Any non-parameter tokens in the \gr{parameter text} occurring
after a macro parameter (that is, after the parameter number
following the parameter character)
act as a delimiter for that parameter. This includes space tokens:
a space after a parameter number is significant.
Delimiting tokens can also occur between the control
sequence being defined and the first parameter token~\verb>#1>.

Character tokens acting as delimiters in the parameter text
have both their character code and
category code stored; the delimiting character tokens of the
actual arguments have to match both.
Category codes of such characters may include some that
can normally only appear in special contexts; for instance, after
the definition
\begin{verbatim}
\def\foo#1_#2^{...}
\end{verbatim}
the macro \cs{foo}
can be used outside math mode.

When looking for the argument corresponding to
a delimited parameter, \TeX\ absorbs all tokens without expansion (but
balancing braces) until the 
(exact sequence of) delimiting tokens is encountered.
The delimiting tokens are not part of the argument;
they are removed from the input stream during the macro call.

%\spoint Examples with delimited arguments
\subsection{Examples with delimited arguments}

As a simple example,
\begin{verbatim}
\def\DoASentence#1#2.{{#1#2.}}
\end{verbatim}
defines a macro with an undelimited first parameter,
and a second parameter delimited by a period.
In the call
\begin{verbatim}
\DoASentence \bf This sentence is the argument.
\end{verbatim}
the arguments are:
\begin{verbatim}
#1<-\bf
#2<-This sentence is the argument
\end{verbatim}
Note that the closing period is not in the argument, but it has
been absorbed; it is no longer in the input stream.

A~commonly used delimiter is \cs{par}:
\begin{verbatim}
\def\section#1. #2\par{\medskip\noindent {\bf#1. #2\par}}
\end{verbatim}
This macro has a first parameter that is delimited by~`\n{.\char32}',
and a second parameter that is delimited by \cs{par}.
The call\message{example on one page}
\begin{verbatim}
\section 2.5. Some title

The text of the section...
\end{verbatim}
will give
\begin{disp}\verb>#1<-2.5>\nl
\verb>#2<-Some title>\n{\char32}\end{disp}
Note that there is a space at the end of the second argument
generated by the line end. If this space is unwanted one might
define
\begin{verbatim}
\def\section#1. #2 \par{...}
\end{verbatim}
with \n{\char32}\cs{par} delimiting the second
argument. This approach, however,
precludes  the user's writing the \cs{par} explicitly:
\begin{verbatim}
\section 2.5 Some title\par
\end{verbatim}
One way out of this dilemma is to write
\verb>#2\unskip> on all places in the definition text
where the trailing space would be unwanted.

Control sequences acting as delimiters need not be defined,
as they are absorbed without expansion. Thus
\begin{verbatim}
\def\control#1\sequence{...}
\end{verbatim}
is a useful
definition, even if \cs{sequence} is undefined.

The importance of category codes in delimited arguments
is shown by the following example:
\begin{verbatim}
\def\a#1 #2.{ ... }
\catcode`\ =12
\a b c
d.
\end{verbatim}
which gives
\begin{verbatim}
\a #1 #2.-> ...
#1<- b c
#2<-d
\end{verbatim}
Explanation: the delimiter between parameters 1 and~2 is a space
of category~10\index{category!10}.
In between \n{a} and \n{b} there is a space
of category~12\index{category!12};
the first space of  category~10
is the space that is generated by the line end.

For a `real-life' application of matching of category codes,
see the explanation of \cs{newif} in Chapter~\ref{if},
and the example on page~\pageref{ex:jobnumber}.


%\spoint Empty arguments
\subsection{Empty arguments}

If the user specifies a \gr{balanced text} in braces
when \TeX\ expects a macro
argument, that text is used as the argument.
Thus, specifying \verb-{}- will give an argument that is
an empty list of tokens; this is called an `empty argument'.

Empty arguments can also arise from the use of delimited
parameters. For example, after the definition
\begin{verbatim}
\def\mac#1\ro{ ... }
\end{verbatim}
the call
\begin{verbatim}
\mac\ro
\end{verbatim}
will give an empty argument. 

\begin{comment}
However, only
one empty argument can be created this way: 
if the macro had been defined as
\begin{verbatim}
\def\mac#1#2\ro{ ... }
\end{verbatim}
the same call
\begin{verbatim}
\mac\ro \othermacro \stillothermacro
\end{verbatim}
will probably cause a `\n{Runaway argument?}' error message.
Explanation: the first parameter is undelimited, so the corresponding
argument is `\cs{ro}'; after that \TeX\ starts looking for a list
of tokens delimited by~\cs{ro}.
\end{comment}

\subsection{The macro parameter character}

When \TeX's input processor scans a macro definition text, it inserts
a parameter token for any occurrence of a macro
\indextermsub{parameter}{character}\index{character!parameter}
followed by a digit.  In effect, a
parameter token in the replacement text states `insert parameter
number such and such here'.  Two parameter characters in a row are
replaced by a single one.

The latter fact can be used for nested macro definitions.
\label{nest:def}\howto Nested macro definitions\par
Thus
\begin{verbatim}
\def\a{\def\b#1{...}}
\end{verbatim}
gives an error message
because \cs{a} was defined without parameters, and
yet there is a parameter token in its replacement text.

The following
\begin{verbatim}
\def\a#1{\def\b#1{...}}
\end{verbatim}
defines a macro \cs{a} that
defines a macro \cs{b}. However, \cs{b} still does not
have any parameters: the call
\begin{verbatim}
\a z
\end{verbatim}
defines a macro \cs{b} without parameters,
that has to be followed by a~\n z.
Note that this
does not attempt to define a macro \cs{bz}, because the
control sequence \cs{b} has already been formed in \TeX's
input processor when that input line was read.

Finally,
\begin{verbatim}
\def\a{\def\b##1{...}}
\end{verbatim}
defines a macro \cs{b} 
with one parameter.

Let us examine the handling of the parameter character
in some detail.
Consider
\begin{verbatim}
\def\a#1{ .. #1 .. \def\b##1{ ... }}
\end{verbatim}
When this is read as input, the input processor
\begin{itemize}
\item replaces the characters \verb>#1> by \gr{parameter token$_1$}, and
\item replaces the characters \verb>##> by \verb>#>\end{itemize}
A macro call of \cs{a} will then let the input processor scan
\begin{verbatim}
\def\b#1{ ... }
\end{verbatim}
in which the two characters \verb>#1> are
\alt
replaced by a parameter token.

%\spoint Brace delimiting
\subsection{Brace delimiting}

Ordinarily, it is not possible to have left or right
braces in the \gr{parameter text} of a definition.
There is a special mechanism, however, that can make
the last parameter of a macro act as if it is delimited
by an opening brace. 

If the last parameter token
is followed by a parameter character (\verb>#>),
which in turn is followed by the opening brace of the
replacement text, \TeX\ makes the last parameter
be delimited by a beginning-of-group character.
Furthermore, unlike other delimiting tokens in
parameter texts, this opening brace is not
removed from the input stream.

Consider an example.
Suppose we want to have a macro
\cs{every} that can fill token lists as follows:
\begin{verbatim}
\every par{abc} \every display{def}
\end{verbatim}
This macro can be defined as
\begin{verbatim}
\def\every#1#{\csname every#1\endcsname}
\end{verbatim}
In the first call above, the argument corresponding to
the parameter is \n{abc}, so the call 
expands to
\begin{verbatim}
\csname everypar\endcsname{abc}
\end{verbatim}
which gives the desired result.


\section{Construction of control sequences}
\label{cs:name}

The commands \csidx{csname} and \csidx{endcsname} can be used
to construct a control sequence. 
For instance
\begin{verbatim}
\csname hskip\endcsname 5pt
\end{verbatim}
is equivalent to \verb=\hskip5pt=.

During this construction process
all macros and other expandable control sequences
between \cs{csname} and \cs{endcsname}
are expanded as usual, until only unexpandable
character tokens remain. A~variation of the above example,
\begin{verbatim}
\csname \ifhmode h\else v\fi skip\endcsname 5pt
\end{verbatim}
performs an \cs{hskip} or \cs{vskip} depending on the mode.
The final result of the expansion should 
consist of only character tokens, but
their category codes do not matter.
An unexpandable control sequence gives an error here:
\TeX\ will insert an \cs{endcsname} right before it
as an attempt at error recovery.

With \cs{csname} it is possible to construct
control sequences that cannot ordinarily be written,
because the constituent character tokens may have another category
\alt
than~11, letter. This principle can be used to hide
\howto Hide counters from the user\par
inner control sequences of a macro package from the user.
\begin{example}
\begin{verbatim}
\def\newcounter#1{\expandafter\newcount
    \csname #1:counter\endcsname}
\def\stepcounter#1{\expandafter\advance
    \csname #1:counter\endcsname 1\relax}
\end{verbatim}
In the second definition the \cs{expandafter} is superfluous,
but it does no harm, and it is conceptually clearer.
\end{example}

The name of the actual counter created by \cs{newcounter}
contains a colon, so that it takes some effort to write this
control sequence. In effect, the counter
is now hidden from the user, who can only
access it through control sequences such as \cs{stepcounter}.
By the way, the macro \cs{newcount} is defined \cs{outer} in
the plain format, so the above definition of \cs{newcounter}
can only be written after \cs{newcount} has been redefined.

If a control sequence formed with \verb>\csname...\endcsname>
has not been defined
before, its meaning is set to \cs{relax}.
Thus if \verb=\xx= is an undefined control sequence, the
command
\begin{verbatim}
\csname xx\endcsname
\end{verbatim}
will {\em not\/}
give an error message, as it is equivalent to \verb=\relax=.
Moreover, after this execution of the
\verb-\csname...\endcsname- statement, the control sequence
\verb=\xx= is itself equivalent to \cs{relax}, so it
will no longer give an `undefined control sequence' error
(see also page~\pageref{relax:cs}).


%\point Token assignments by \cs{let} and \cs{futurelet}
\section{Token assignments by \protect\cs{let} and \protect\cs{futurelet}}

There are two \gr{let assignment}s in \TeX.
Their syntax is
\begin{disp}\cs{let}\gr{control sequence}\gr{equals}%
     \gr{one optional space}\gr{token}\nl
     \cs{futurelet}\gr{control sequence}\gr{token}\gr{token}
     \end{disp}
In the syntax of a \cs{futurelet} assignment
no optional equals sign appears.

%\spoint[let] \cs{let}
\subsection{\protect\cs{let}}
\label{let}

The primitive command \csidx{let} assigns the current meaning
of a~token to a control sequence or active character.

For instance, in the plain format \cs{endgraf} is defined
as
\begin{verbatim}
\let\endgraf=\par
\end{verbatim}
This enables macro writers to redefine \cs{par}, while
still having the functionality of the primitive \cs{par}
command available. For example,
\begin{verbatim}
\everypar={\bgroup\it\def\par{\endgraf\egroup}}
\end{verbatim}

The case where the \gr{token} to be assigned is not a control
sequence but a character token instead has been treated 
in Chapter~\ref{char}.

%\spoint \cs{futurelet}
\subsection{\protect\cs{futurelet}}

As was explained above, the sequence with \cs{let}
\begin{disp}\cs{let}\gr{control sequence}\gr{token$_1$}\gr{token$_2$}%
       \gr{token$_3$}\gr{token$\cdots$}\end{disp}
assigns (the meaning of) \gr{token$_1$} to the control sequence, 
and the remaining input stream looks like
\begin{disp}\gr{token$_2$}\gr{token$_3$}\gr{token$\cdots$}\end{disp}
That is, the \gr{token$_1$} has disappeared from the stream.

The command \csidx{futurelet} works slightly differently:
given the input stream
\begin{disp}\cs{futurelet}\gr{control sequence}\gr{token$_1$}\gr{token$_2$}%
       \gr{token$_3$}\gr{token$\cdots$}\end{disp}
it assigns (the meaning of) \gr{token$_2$} to the control sequence, 
and the remaining stream looks like
\begin{disp}\gr{token$_1$}\gr{token$_2$}\gr{token$_3$}\gr{token$\cdots$}\end{disp}
That is, neither \gr{token$_1$} nor \gr{token$_2$} has
been lifted from the stream.
However, now \gr{token$_1$}
`knows' what \gr{token$_2$} is, without having had to absorb it
as a macro parameter. See an example below.

If a character token has been \cs{futurelet} to a control
sequence, its category code is fixed.
The subsequent \gr{token$_1$} cannot change
it anymore.

%\point Assorted remarks
\section{Assorted remarks}

%\spoint Active characters
\subsection{Active characters}

A character token of category~13\index{category!13} is called an
\indexterm{active character}, and it
can be defined just like a control sequence.
If the definition of the character appears inside a macro,
the character has to be active at the time of the definition
of that macro.

Consider for example the following definition
(taken from Chapter~\ref{mouth}):
\begin{verbatim}
{\catcode`\^^M=13 %
 \gdef\obeylines{\catcode`\^^M=13 \def^^M{\par}}%
}
\end{verbatim}
The unusual category of the \verb>^^M> character
has to be set during the definition of \cs{obeylines},
otherwise \TeX\ would think that the line ended
after \cs{def}.

\subsection{Macros versus primitives}

The distinction between \indexterm{primitive
  commands}\indexterm{command !primitive} and user macros is not
nearly as important in \TeX\ as it is in other programming
languages.
\begin{itemize}
\item The user can use primitive commands under different names:
     
\begin{verbatim}
\let\StopThisParagraph=\par
\end{verbatim}
\item Names of primitive commands can be used for
      user macros:
\begin{verbatim}
\def\par{\hfill$\bullet$\endgraf}
\end{verbatim}
\item Both user macros and a number of \TeX\ primitives
      are subject to expansion, for instance all conditionals,
      and commands such as \cs{number} and~\cs{jobname}.
\end{itemize}

%\spoint Tail recursion
\subsection{Tail recursion}

Macros in \TeX, like procedures in most modern programming
languages, are allowed to be \emph{recursive}\index{recursion}: that is, the 
definition of a macro can contain a call to this same macro,
or to another macro that will call this macro.
Recursive macros tend to clutter up \TeX's memory
if too many `incarnations' of such a macro are active
at the same time. However, \TeX\ is able to prevent this
in one frequently occurring case of recursion: tail recursion.

In order to  appreciate what goes on here, some background
knowledge is needed. When \TeX\ starts executing a macro
it absorbs the parameters, and places an item pointing to
the replacement text on the \indextermsub{input}{stack},
so that the scanner will next be directed to
this replacement. Once it has been processed, the item on the 
input stack can be removed.
However, if the definition text
of a macro contains further macros, this process will be
repeated for them: new items may be placed on the input stack
directing the scanner to other macros
even before the first one has been completed.

In general this `stack build-up' is a necessary evil, but
it can be prevented if the nested macro call is the
{\em last\/} token in the replacement text of the original
macro. After the last token no further tokens need to be
considered, so one might as well clear the top item
from the input stack
before a new one is put there.
This is what \TeX\ does.

The \csidx{loop} macro of plain \TeX\ provides a good illustration
\label{loop:ex}
of this principle. The definition is
\begin{verbatim}
\def\loop#1\repeat{\def\body{#1}\iterate}
\def\iterate{\body \let\next=\iterate
    \else \let\next=\relax\fi \next}
\end{verbatim}
and this macro can be called for example as follows:
\begin{verbatim}
\loop \message{\number\MyCount}
    \advance\MyCount by 1
    \ifnum\MyCount<100 \repeat
\end{verbatim}
The macro \cs{iterate} can call itself and, when it does so,
the recursive call is performed by the last token in the list.
It would have been possible to define \cs{iterate}
as
\begin{verbatim}
\def\iterate{\body \iterate\fi}
\end{verbatim}
but then \TeX\ would not have been able to resolve the recursion
as the call \cs{iterate} is not the last token in the replacement
text of \cs{iterate}. Assigning \verb>\let\next=\iterate>
is here a way to let
the recursive call be the last token in the list.

Another way of resolving tail recursion is to use
\cs{expandafter} (see page~\pageref{after:cond}): in
\begin{verbatim}
\def\iterate{\body \expandafter\iterate\fi}
\end{verbatim}
it removes the \cs{fi} token.
Tail recursion would also be resolved if the last
tokens in the list were arguments for the
recursive macro.

An aside: by defining \cs{iterate} as
\begin{verbatim}
\def\iterate{\let\next\relax 
    \body \let\next\iterate \fi \next}
\end{verbatim}
it becomes possible to write
\begin{verbatim}
\loop ... \if... ... \else ... \repeat
\end{verbatim}

%\point Macro techniques
\section{Macro techniques}

%\spoint Unknown number of arguments
\subsection{Unknown number of arguments}

In some applications,
\howto  Macros with an undetermined number
of arguments\par
a macro is needed that can have a
number of arguments that is not specified in advance.

Consider the problem of translating a position on a chess board
(for full macros and fonts, see~\cite{chess} and~\cite{Tut}),
given like
\begin{verbatim}
\White(Ke1,Qd1,Na1,e2,f4)
\end{verbatim} 
to a sequence of typesetting instructions
\begin{verbatim}
\WhitePiece{K}{e1} \WhitePiece{Q}{d1} \WhitePiece{N}{a1} 
\WhitePiece{P}{e2} \WhitePiece{P}{f4}
\end{verbatim}
Note that for pawns the `P' is omitted in the list of positions.

The first problem is that the list of pieces 
is of variable length, so we append a terminator piece:
\begin{verbatim}
\def\White(#1){\xWhite#1,xxx,}
\def\endpiece{xxx}
\end{verbatim}
for which we can test.
Next, the macro \cs{xWhite} takes one position from the list,
tests whether it is the terminator, and if not,
subjects it to a test to see whether it is a pawn.
\begin{verbatim}
\def\xWhite#1,{\def\temp{#1}%
    \ifx\temp\endpiece 
    \else \WhitePieceOrPawn#1XY%
          \expandafter\xWhite 
    \fi}
\end{verbatim}
An \cs{expandafter} command is necessary to remove the
\cs{fi} (see page~\pageref{after:cond}), so that 
\cs{xWhite} will get the next position as argument
instead of \cs{fi}.

Positions  are either two or three characters long.
The call to \cs{White\-Piece\-OrPawn}, a four-parameter macro,
appended a terminator string \n{XY}. 
In the case of a pawn, therefore, argument~3 is the character~\n X
and argument~4 is empty; for all other pieces argument~1
is the piece, 2~and~3 are the position, and argument~4 is~\n X.
\begin{verbatim}
\def\WhitePieceOrPawn#1#2#3#4Y{
    \if#3X \WhitePiece{P}{#1#2}%
    \else  \WhitePiece{#1}{#2#3}\fi}
\end{verbatim}

%\spoint Examining the argument
\subsection{Examining the argument}

It may be necessary in some cases to test whether a macro
\howto Examine a macro argument for the presence of some element\par
\howto Apply \cs{uppercase} when the argument has a \cs{footnote}\par
argument contains some element. For a real-life example,
consider the following (see also the \cs{DisplayEquation}
\alt
example on page~\pageref{left:display}).

Suppose the title and author of an article are given as
\begin{verbatim}
\title{An angle trisector}
\author{A.B. Cee\footnote*{Research supported by the
Very Big Company of America}}
\end{verbatim}
with multiple authors
given as
\begin{verbatim}
\author{A.B. Cee\footnote*{Supported by NSF grant 1}
        \and
        X.Y. Zee\footnote{**}{Supported by NATO grant 2}}
\end{verbatim}
Suppose further that the \cs{title} and \cs{author} macros
are defined as
\begin{verbatim}
\def\title#1{\def\TheTitle{#1}}  \def\author#1{\def\TheAuthor{#1}}
\end{verbatim}
which will be used as
\begin{verbatim}
\def\ArticleHeading{ ... \TheTitle ... \TheAuthor ... }
\end{verbatim}

For some journals it is required to
have the authorship and the title of the article in all capitals.
The implementation of this could be
\begin{verbatim}
\def\ArticleCapitalHeading
   { ...
    \uppercase\expandafter{\TheTitle}
     ...
    \uppercase\expandafter{\TheAuthor}
     ...
   }
\end{verbatim}
Now the \cs{expandafter} commands will expand the title and
author into the actual texts, and the \cs{uppercase} commands
will capitalize them. However, for the authors this is wrong,
since the \cs{uppercase} command will also capitalize the
footnote texts.
The problem is then to uppercase only the parts
of the title in between the footnotes.

As a first attempt, let us take the case of one author, and
let the basic call be
\begin{verbatim}
\expandafter\UCnoFootnote\TheAuthor
\end{verbatim}
This expands into
\begin{verbatim}
\UCnoFootnote A.B. Cee\footnote*{Supported ... }
\end{verbatim}
The macro
\begin{verbatim}
\def\UCnoFootnote#1\footnote#2#3{\uppercase{#1}\footnote{#2}{#3}}
\end{verbatim}
will analyse this correctly:
\begin{verbatim}
#1<-A.B. Cee
#2<-*
#3<-Supported ...
\end{verbatim}
However, if there is no footnote, this macro is completely wrong.

As a first refinement we add a footnote ourselves, just to make
sure that one is present:
\begin{verbatim}
\expandafter\UCnoFootnote\TheAuthor\footnote 00
\end{verbatim}
Now we have to test what kind of footnote we find:
\begin{verbatim}
\def\stopper{0}
\def\UCnoFootnote#1\footnote#2#3{\uppercase{#1}\def\tester{#2}%
    \ifx\stopper\tester
    \else\footnote{#2}{#3}\fi}
\end{verbatim}
With \cs{ifx} we test the delimiter footnote sign against the
actual sign encountered. Note that a solution with
\begin{verbatim}
\ifx0#2
\end{verbatim}
would be wrong if the footnote sign consists
of more than one token, for instance~\verb>{**}>.

The macro so far is correct if there was no footnote,
but if there was one it is wrong:
the terminating tokens remain to be disposed of.
They are taken care of in the following version:
\begin{verbatim}
\def\stopper{0}
\def\UCnoFootnote#1\footnote#2#3{\uppercase{#1}\def\tester{#2}%
    \ifx\stopper\tester
    \else\footnote{#2}{#3}\expandafter\UCnoFootnote
    \fi}
\end{verbatim}
A repeated call to \cs{UCnoFootnote} removes the delimiter tokens
(the \cs{expandafter} first removes the \cs{fi}),
and as an added bonus, this macro is also correct for multiple
authors.


%\spoint Optional macro parameters with \cs{futurelet}
\subsection{Optional macro parameters with \protect\cs{futurelet}}

One standard application of \cs{futurelet} is implementing
\howto Macros with optional parameters\par
optional parameters of macros. The general course of action
is as follows:
\begin{verbatim}
\def\Com{\futurelet\testchar\MaybeOptArgCom}
\def\MaybeOptArgCom{\ifx[\testchar \let\next\OptArgCom 
                 \else \let\next\NoOptArgCom \fi \next}
\def\OptArgCom[#1]#2{ ... }\def\NoOptArgCom#1{ ... }
\end{verbatim}
Note that \cs{ifx} is used even though it tests
for a character. The reason is of course that,
if the optional argument is omitted, there might be an
expandable control sequence behind the~\cs{Com}.

The macro \cs{Com} now has one optional and one regular
argument; it can be called as 
\begin{verbatim}
\Com{argument}
\end{verbatim}
or as
\begin{verbatim}
\Com[optional]{argument}
\end{verbatim}
Often the call without the optional argument will insert some
default value:
\begin{disp}\verb>\def\NoOptArgCom#1{\OptArgCom[>%
{\italic default\/}\verb>]{#1}}>\end{disp}
This mechanism is widely used in formats such as \LaTeX\ and
\LamsTeX; see also~\cite{svb:future}.



%\spoint Two-step macros
\subsection{Two-step macros}

Often what looks to the user like one macro is in reality
a two-step process, where one macro will set up conditions,
and a second macro will do the work.

As an example, here is
a macro \cs{PickToEol}\label{pick:eol}
\howto Take an input line as macro argument\par
with an argument that is delimited by the line end.
First we write a macro without arguments that 
changes the category code of the line end, and then
calls the second macro.
\begin{verbatim}
\def\PickToEol{\begingroup\catcode`\^^M=12 \xPickToEol}
\end{verbatim}
The second macro can then take as an argument everything
up to the end of the line:
\begin{verbatim}
\def\xPickToEol#1^^M{ ... #1 ... \endgroup}
\end{verbatim}
There is one problem with this definition: the \verb>^^M> character
should have category~12. We arrive at the following:
\begin{verbatim}
\def\PickToEol{\begingroup\catcode`\^^M=12 \xPickToEol}
{\catcode`\^^M=12 %
 \gdef\xPickToEol#1^^M{ ... #1 ... \endgroup}%
}
\end{verbatim}
where the category code of \verb>^^M> is changed for the
sake of the definition of \cs{xPickToEol}. Note that
the \verb>^^M> in \cs{PickToEol} occurs in a control symbol,
so there the category code  is irrelevant. Therefore that
definition can be outside the group where the category code 
of \verb>^^M> is redefined.


\subsection{A comment environment}

As an application of the above idea of two-step macros,
\howto Comment environment\par
and in order to illustrate tail recursion, here are 
macros for a `comment' environment.

Often it is necessary to remove a part of \TeX\
input temporarily. For this one would like to
write
\begin{verbatim}
\comment
...
\endcomment
\end{verbatim}
The simplest implementation of this, 
\begin{verbatim}
\def\comment#1\endcomment{}
\end{verbatim} 
has a number of weaknesses. For instance,
it cannot cope with outer macros or input that 
does not have balanced braces. Its worst
shortcoming, however, is that it reads the complete
comment text as a macro argument. This limits the size
of the comment to that of \TeX's input buffer.

It would be a better idea to take on the out-commented
text one line at a time. For this we want to write
a recursive macro with a basic structure
\begin{verbatim}
\def\comment#1^^M{ ... \comment }
\end{verbatim}
In order to be able to write this definition at all,
the category code of the line end must be changed; as above
\altt
we will have
\begin{verbatim}
\def\comment{\begingroup \catcode`\^^M=12 \xcomment}
{\catcode`\^^M=12 \endlinechar=-1 %
 \gdef\xcomment#1^^M{ ... \xcomment}
}
\end{verbatim}
Changing the \cs{endlinechar} is merely to 
prevent having to put comment characters at the end
of every line of the definition.

Of course, the process must stop at a certain time.
To this purpose we investigate the line that was
scooped up as macro argument:
\begin{verbatim}
{\catcode`\^^M=12 \endlinechar=-1 %
 \gdef\xcomment#1^^M{\def\test{#1}
    \ifx\test\endcomment \let\next=\endgroup
    \else \let\next=\xcomment \fi
    \next}
}
\end{verbatim}
and we have to define \cs{endcomment}:
\begin{verbatim}
\def\endcomment{\endcomment}
\end{verbatim}
This command will never be executed: it is merely for purposes
of testing whether the end of the environment has been reached.

We may want to comment out text that is not syntactically
correct. Therefore we switch to a
\indexterm{verbatim mode}
when commenting. The following macro is given 
in plain \TeX:
\begin{verbatim}
\def\dospecials{\do\ \do\\\do\{\do\}\do\$\do\&%
  \do\#\do\^\do\^^K\do\_\do\^^A\do\%\do\~}
\end{verbatim}
We use it to define \cs{comment} as follows:
\begin{verbatim}
\def\makeinnocent#1{\catcode`#1=12 }
\def\comment{\begingroup
    \let\do=\makeinnocent \dospecials
    \endlinechar`\^^M \catcode`\^^M=12 \xcomment}
\end{verbatim}
Apart from the possibility mentioned above of commenting
out text that is not syntactically correct, for instance
because of unmatched braces, this solution can handle
outer macros. The former implementation of \cs{xcomment}
would cause a \TeX\ error if one occurred in the comment text.

However, using verbatim mode poses the problem of concluding the 
environment.
\altt
The final line of the comment is now not the control sequence
\cs{endcomment}, but the characters constituting it. We have
to test for these then:
\begin{verbatim}
{\escapechar=-1
 \xdef\endcomment{\string\\endcomment}
}
\end{verbatim}
The sequence \verb>\string\\> gives a backslash.
We could not have used
\begin{verbatim}
\edef\endcomment{\string\endcomment}
\end{verbatim}
because
the letters of the word \n{endcomment} would then have
category code~12, instead of the 11 that the ones on the
last line of the comment will have.

\endofchapter
%%%% end of input file [macro]
