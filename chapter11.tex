% -*- coding: utf-8 -*-
\documentclass{book}

% -*- coding: utf-8 -*-

\usepackage[b5paper,text={5in,8in},centering]{geometry}

\usepackage[CJKchecksingle]{xeCJK}
\setmainfont[Mapping=tex-text]{TeX Gyre Schola}
%\setsansfont{URW Gothic L Book}
%\setmonofont{Nimbus Mono L}
\setCJKmainfont[BoldFont=FandolHei,ItalicFont=FandolKai]{FandolSong}
\setCJKsansfont{FandolHei}
\setCJKmonofont{FandolFang}
\xeCJKsetup{PunctStyle = kaiming}

\linespread{1.25}
\setlength{\parindent}{2em}

\usepackage{xcolor}
\definecolor{myblue}{rgb}{0,0.2,0.6}

\usepackage{titlesec}
\titleformat{\chapter}
    {\normalfont\Huge\sffamily\color{myblue}}
    {第\thechapter 章}
    {1em}
    {}
%\titlespacing{\chapter}{0pt}{50pt}{40pt}
\titleformat{\section}
    {\normalfont\Large\sffamily\color{myblue}}
    {\thesection}
    {1em}
    {}
%\titlespacing{\section}{0pt}{3.5ex plus 1ex minus .2ex}{2.3ex plus .2ex}
\titleformat{\subsection}
    {\normalfont\large\sffamily\color{myblue}}
    {\thesubsection}
    {1em}
    {}
%\titlespacing{\subsection}{0pt}{3.25ex plus 1ex minus .2ex}{1.5ex plus .2ex}
%
\newpagestyle{special}[\small\sffamily]{
  \headrule
  \sethead[\usepage][][\chaptertitle]
  {\chaptertitle}{}{\usepage}}
\newpagestyle{main}[\small\sffamily]{
  \headrule
  \sethead[\usepage][][第\thechapter 章\quad\chaptertitle]
  {\thesection\quad\sectiontitle}{}{\usepage}}

\usepackage{titletoc}
%\setcounter{tocdepth}{1}
%\titlecontents{标题层次}[左间距]{上间距和整体格式}{标题序号}{标题内容}{指引线和页码}[下间距]
\titlecontents{chapter}[1.5em]{\vspace{.5em}\bfseries\sffamily}{\color{myblue}\contentslabel{1.5em}}{}
    {\titlerule*[20pt]{$\cdot$}\contentspage}[]
\titlecontents{section}[4.5em]{\sffamily}{\color{myblue}\contentslabel{3em}}{}
    {\titlerule*[20pt]{$\cdot$}\contentspage}[]
%\titlecontents{subsection}[8.5em]{\sffamily}{\contentslabel{4em}}{}
%    {\titlerule*[20pt]{$\cdot$}\contentspage}

\usepackage{enumitem}
\setlist{topsep=2pt,itemsep=2pt,parsep=1pt,leftmargin=\parindent}

\usepackage{fancyvrb}
\DefineVerbatimEnvironment{verbatim}{Verbatim}
  {xleftmargin=2em,baselinestretch=1,formatcom=\color{teal}\upshape}

\usepackage{etoolbox}
\makeatletter
\preto{\FV@ListVSpace}{\topsep=2pt \partopsep=0pt }
\makeatother

\usepackage[colorlinks,plainpages,pagebackref]{hyperref}
\hypersetup{
   pdfstartview={FitH},
   citecolor=teal,
   linkcolor=myblue,
   urlcolor=black,
   bookmarksnumbered
}

\usepackage{comment,makeidx,multicol}

%\usepackage{german}
%% german
%\righthyphenmin=3
%\mdqoff
%\captionsenglish
\usepackage[english]{babel}
{\catcode`"=13 \gdef"#1{\ifx#1"\discretionary{}{}{}\fi\relax}}
\def\mdqon{\catcode`"=13\relax}
\def\mdqoff{\catcode`"=12\relax}
\makeindex
\hyphenation{ex-em-pli-fies}

\newdimen\tempdima \newdimen\tempdimb

% these are fine
\def\handbreak{\\ \message{^^JManual break!!!!^^J}}
\def\nl{\protect\\}\def\n#1{{\tt #1}}
\def\cs#1{\texorpdfstring{{\tt\char`\\#1}}{\textbackslash#1}} %\def\cs#1{{\tt\char`\\#1}}
\let\csc\cs
\def\lb{{\tt\char`\{}}\def\rb{{\tt\char`\}}}
\def\gr#1{\texorpdfstring{$\langle$#1$\rangle$}{<#1>}} %\def\gr#1{$\langle$#1$\rangle$}
\def\key#1{{\tt#1}}
\def\alt{}\def\altt{}%this way in manstijl
\def\ldash{\unskip\ --\nobreak\ \ignorespaces}
\def\rdash{\unskip\nobreak\ --\ \ignorespaces}
% check these
\def\hex{{\tt"}}
\def\ascii{{\sc ascii}}
\def\ebcdic{{\sc ebcdic}}
\def\IniTeX{Ini\TeX}\def\LamsTeX{LAMS\TeX}\def\VirTeX{Vir\TeX}
\def\AmsTeX{Ams\TeX}
\def\TeXbook{the \TeX\ book}\def\web{{\sc web}}
% needs major thinking
\newenvironment{myquote}{\list{}{%
    \topsep=2pt \partopsep=0pt%
    \leftmargin=\parindent \rightmargin=\parindent
    }\item[]}{\endlist}
\newenvironment{disp}{\begin{myquote}}{\end{myquote}}
\newenvironment{Disp}{\begin{myquote}}{\end{myquote}}
\newenvironment{tdisp}{\begin{myquote}}{\end{myquote}}
\newenvironment{example}{\begin{myquote}\noindent\itshape 例子:}{\end{myquote}}
\newenvironment{inventory}{\begin{description}\raggedright}{\end{description}}
\newenvironment{glossinventory}{\begin{description}}{\end{description}}
\def\gram#1{\gr{#1}}%???
%
% index
%
\def\indexterm#1{\emph{#1}\index{#1}}
\def\indextermsub#1#2{\emph{#1 #2}\index{#1!#2}}
\def\indextermbus#1#2{\emph{#1 #2}\index{#2!#1}}
\def\term#1\par{\index{#1}}
\def\howto#1\par{}
\def\cstoidx#1\par{\index{#1@\cs{#1}@}}
\def\thecstoidx#1\par{\index{#1@\protect\csname #1\endcsname}}
\def\thecstoidxsub#1#2{\index{#1, #2@\protect\csname #1\endcsname, #2}\ignorespaces}
\def\csterm#1\par{\cstoidx #1\par\cs{#1}}
\def\csidx#1{\cstoidx #1\par\cs{#1}}

\def\tmc{\tracingmacros=2 \tracingcommands\tracingmacros}

%%%%%%%%%%%%%%%%%%%
\makeatletter
\def\snugbox{\hbox\bgroup\setbox\z@\vbox\bgroup
    \leftskip\z@
    \bgroup\aftergroup\make@snug
    \let\next=}
\def\make@snug{\par\sn@gify\egroup \box\z@\egroup}
\def\sn@gify
   {\skip\z@=\lastskip \unskip
    \advance\skip\z@\lastskip \unskip
    \unpenalty
    \setbox\z@\lastbox
    \ifvoid\z@ \nointerlineskip \else {\sn@gify} \fi
    \hbox{\unhbox\z@}\nointerlineskip
    \vskip\skip\z@
    }

\newdimen\fbh \fbh=60pt % dimension for easy scaling:
\newdimen\fbw \fbw=60pt % height and width of character box

\newdimen\dh \newdimen\dw % height and width of current character box
\newdimen\lh % height of previous character box
\newdimen\lw \lw=.4pt % line weight, instead of default .4pt

\def\hdotfill{\noindent
    \leaders\hbox{\vrule width 1pt height\lw
                  \kern4pt
                  \vrule width.5pt height\lw}\hfill\hbox{}
    \par}
\def\hlinefill{\noindent
    \leaders\hbox{\vrule width 5.5pt height\lw         }\hfill\hbox{}
    \par}
\def\stippel{$\qquad\qquad\qquad\qquad$}
\makeatother
%%%%%%%%%%%%%%%%%%%

%\def\SansSerif{\Typeface:macHelvetica }
%\def\SerifFont{\Typeface:macTimes }
%\def\SansSerif{\Typeface:bsGillSans }
%\def\SerifFont{\Typeface:bsBaskerville }
\let\SansSerif\relax \def\italic{\it}
\let\SerifFont\relax \def\MainFont{\rm}
\let\SansSerif\relax
\let\SerifFont\relax
\let\PopIndentLevel\relax \let\PushIndentLevel\relax
\let\ToVerso\relax \let\ToRecto\relax

%\def\stop@command@suffix{stop}
%\let\PopListLevel\PopIndentLevel
%\let\FlushRight\relax
%\let\flushright\FlushRight
%\let\SetListIndent\LevelIndent
%\def\awp{\ifhmode\vadjust{\penalty-10000 }\else
%    \penalty-10000 \fi}
\let\awp\relax
\let\PopIndentLevel\relax \let\PopListLevel\relax

\showboxdepth=-1

%\input figs
\def\endofchapter{\vfill\noindent}

\setcounter{chapter}{10}

\begin{document}

%\chapter{Macros}\label{macro}
\chapter{宏定义}\label{macro}

%Macros are \TeX's abbreviation mechanism for sequences of commands
%that are needed more than once,
%somewhat like procedures in ordinary programming languages.
%\TeX's parameter mechanism, however, is quite unusual.
%This chapter explains how \TeX\ macros work. It also
%treats the commands \cs{let} and~\cs{futurelet}.
在 \TeX\ 中,宏是对一串需要多次用到的命令的缩写机制,
它有点像普通编程语言的过程。然而 \TeX\ 的参数机制是与众不同的。
这一章解释 \TeX\ 的宏如何运作,并介绍命令\cs{let} 和 \cs{futurelet}。

%\label{cschap:def}\label{cschap:gdef}\label{cschap:edef}\label{cschap:xdef}
%\label{cschap:csname}\label{cschap:endcsname}\label{cschap:global2}
%\label{cschap:outer}\label{cschap:long}\label{cschap:let}\label{cschap:futurelet}
%\begin{inventory}
%\item [\cs{def}] 
%      Start a macro definition.
\label{cschap:def}\label{cschap:gdef}\label{cschap:edef}\label{cschap:xdef}
\label{cschap:csname}\label{cschap:endcsname}\label{cschap:global2}
\label{cschap:outer}\label{cschap:long}\label{cschap:let}\label{cschap:futurelet}
\begin{inventory}
\item [\cs{def}] 
      开始一个宏定义。

%\item [\cs{gdef}] 
%      Synonym for \verb-\global\def-.
\item [\cs{gdef}] 
      等同于 \verb-\global\def-。

%\item [\cs{edef}] 
%      Start a macro definition; 
%      the replacement text is expanded at definition time.
%      This command is treated also in the next chapter.
\item [\cs{edef}] 
      开始一个宏定义;在定义时替换文本被展开。此命令在下一章也会介绍。 

%\item [\cs{xdef}] 
%      Synonym for \verb-\global\edef-.
\item [\cs{xdef}] 
      等同于 \verb-\global\edef-。

%\item [\cs{csname}] 
%      Start forming the name of a control sequence.
\item [\cs{csname}] 
      开始生成一个控制序列的名称。

%\item [\cs{endcsname}] 
%      Stop forming the name of a control sequence.
\item [\cs{endcsname}] 
      结束生成一个控制序列的名称。

%\item [\cs{global}] 
%      Make the next definition, arithmetic statement,
%      or assignment global.
\item [\cs{global}] 
      使得后面的定义、算术语句或赋值是全局的。

%\item [\cs{outer}] 
%      Prefix indicating that the macro being defined 
%      can be used on the `outer' level only.
\item [\cs{outer}] 
      此前缀表示正在定义的宏只能用在`外部'层级中。

%\item [\cs{long}] 
%      Prefix indicating that the arguments of the macro being defined
%      may contain \cs{par} tokens.
\item [\cs{long}] 
      此前缀表示正在定义的宏的参量可以包含 \cs{par} 记号。

%\item [\cs{let}] 
%      Define a control sequence to be equivalent to the next token.
\item [\cs{let}] 
      将一个控制序列定义为等价于下一个记号。

%\item [\cs{futurelet}] 
%      Define a control sequence to be equivalent to
%      the token after the next token.
\item [\cs{futurelet}] 
      将一个控制序列定义为等价于下一个记号之后的记号。

%\end{inventory}
\end{inventory}

%\section{Introduction}
\section{介绍}

%A \indexterm{macro}
%is basically a sequence of tokens that has
%been abbreviated into a control sequence.
%Statements starting with (among others) \cs{def}
%are called {\italic macro definitions}\alt, and
%writing
%\begin{verbatim}
%\def\abc{\de f\g}
%\end{verbatim}
%defines the macro \cs{abc},
%with the {\italic replacement text\/} \verb>\de f\g>.
%Macros can be used in this way to abbreviate
%pieces of text or sequences of commands
%that have to be given more than once.
%Any time that \TeX's expansion processor
%encounters the control sequence \cs{abc},
%it replaces it by the replacement text.
基本上\indexterm{宏}是缩写为一个控制序列的一串记号。
以 \cs{def} 等开始的语句称为{\italic 宏定义},而语句
\begin{verbatim}
\def\abc{\de f\g}
\end{verbatim}
就定义了以 \verb>\de f\g> 为{\italic 替换文本\/}的宏 \cs{abc}。
以这种方式,宏可以用于缩写需要多次用到的一段文本或一串命令。
在任何时候,只要 \TeX\ 的展开处理器遇到控制序列 \cs{abc},
它就将这个宏用其替换文本代替。

%If a macro should be sensitive to the context
%where it is used, it can be defined with parameters.
%A~macro \verb+\PickTwo+ defined as
%\begin{verbatim}
%\def\PickTwo#1#2{(#1,#2)}
%\end{verbatim}
%has two \emph{parameters}. When it is used, it will scoop up two pieces of
%text, the corresponding \emph{arguments},
%and reproduces them in parentheses.
%For example:
如果想让宏依赖它被用到时的上下文,就可以将它定义为带参数的宏。
像下面这样定义的 \verb+\PickTwo+
\begin{verbatim}
\def\PickTwo#1#2{(#1,#2)}
\end{verbatim}
有两个\emph{参数}。这个宏被用到时将取出两段文本,即对应的\emph{参量},
并将它们排印在圆括号中。例如:

%\begin{tabular}{|l|llllll|}
%\hline
%          &macro              &argument1& argument2& 
%    &expansion&\\ \hline
%definition&\verb+\def\PickTwo+&\verb+#1+& \verb+#2+& \verb+{+& 
%    \verb+(#1,#2)+& \verb+}+\\
%use&       \verb+\PickTwo+&1&2&&(1,2)&\\
%use&       \verb+\PickTwo+&\verb+{ab}+&\verb+{cd}+&&(ab,cd)&\\
%\hline
%\end{tabular}
\begin{tabular}{|l|llllll|}
\hline
          &macro              &argument1& argument2& 
    &expansion&\\ \hline
definition&\verb+\def\PickTwo+&\verb+#1+& \verb+#2+& \verb+{+& 
    \verb+(#1,#2)+& \verb+}+\\
use&       \verb+\PickTwo+&1&2&&(1,2)&\\
use&       \verb+\PickTwo+&\verb+{ab}+&\verb+{cd}+&&(ab,cd)&\\
\hline
\end{tabular}

%The activity of substituting the replacement text
%for a macro and its arguments is called {\italic macro expansion}.
将宏及其参量用替换文本代替的活动称为{\italic 宏展开}。

%%\point Layout of a macro definition
%\section{Layout of a macro definition}
%\point Layout of a macro definition
\section{宏定义的结构}

%A \indextermsub{macro}{definition} consists of, in sequence,
%\begin{enumerate} \item any number of \cs{global},
%\cs{long}, and \cs{outer} prefixes,
%\item a \gr{def} control sequence, or anything
%that has been \cs{let} to one,
%\item a control sequence or active character to be defined, 
%\item possibly a \gr{parameter text} specifying among other things
%how many parameters the macro has, and
%\item a replacement text enclosed in explicit character tokens
%with category codes 1\index{category!1} and~2\index{category!2},
%by default \verb-{- and~\verb-}-
%in plain \TeX.
%\end{enumerate}
%These elements will all be explained in subsequent sections.
一个\indextermsub{宏}{定义}按照顺序包含下列各部分:
\begin{enumerate}
\item 任意多个 \cs{global}、\cs{long} 和 \cs{outer} 前缀,
\item 一个 \gr{def} 控制序列,或者任何用 \cs{let} 等价到此种控制序列的东西,
\item 一个将要定义的控制序列或活动字符,
\item 一个可能存在的 \gr{parameter text} 用于指定宏参数的个数及其他东西,以及
\item 一个用类别码为 1\index{category!1} 和\index{category!2} 2
的显式字符记号包围的替换文本,在 plain \TeX\ 中这两类字符默认为
\verb-{- 和 \verb-}-。
\end{enumerate}
这些元素将在接下来各节中解释。

%After a macro definition is completed, any saved \cs{afterassignment}
%token (see section~\ref{sec:afterassignment}) is inserted.
在宏定义完成之后,任何已保存的 \cs{afterassignment} 记号(见
第~\ref{sec:afterassignment}~节)将被插入进来。

%The `expanding' definitions \cs{edef} and \cs{xdef}
%are treated in Chapter~\ref{expand}.
`展开的'定义 \cs{edef} 和 \cs{xdef} 将在第~\ref{expand}~章讨论。

%%\point Prefixes
%\section{Prefixes}
%\point Prefixes
\section{前缀}

%There are three \emph{prefixes}\index{prefixes !macro}
%that alter the status of the
%macro definition:
%\begin{description}
%\item [\csidx{global}]
%If the definition occurs inside a  group, this prefix
%makes the definition global.
%This prefix can also be used for assignments other than
%macro definitions; in fact,
%for macro definitions abbreviations exist obviating the
%use of \cs{global}:
%\begin{disp}\verb>\gdef\foo...>\quad is equivalent to\quad \verb>\global\def\foo...>
%\end{disp} and
%\begin{disp}\verb>\xdef\foo...>\quad is equivalent to\quad \verb>\global\edef\foo...>
%\end{disp}
有三种\emph{前缀}\index{prefixes !macro}用于改变宏定义的状态:
\begin{description}
\item [\csidx{global}]
如果定义出现在编组内,此前缀将使得该定义成为全局的。
除了宏定义之外,此前缀还能用于赋值;实际上,对于宏定义,
可以用缩写形式而不用 \cs{global}:
\begin{disp}\verb>\gdef\foo...>\quad 等价于\quad \verb>\global\def\foo...>
\end{disp} 而
\begin{disp}\verb>\xdef\foo...>\quad 等价于\quad \verb>\global\edef\foo...>
\end{disp}

%If the parameter \cs{globaldefs}
%is positive, all assignments are
%implicitly global;
%if\handbreak \cs{globaldefs} is negative any \cs{global} prefixes are
%ignored,
%and \cs{gdef} and \cs{xdef} make local definitions
%(see Chapter~\ref{group}).
如果参数 \cs{globaldefs} 为正数,所有赋值都被视为全局的;
而如果 \cs{globaldefs} 为负数,所有 \cs{global} 前缀都被忽略,
而且 \cs{gdef} 和 \cs{xdef} 也生成局部的定义(见第~\ref{group}~章)。

%\item [\cs{outer}]
%The mechanism of defining an \indextermbus{outer}{macro} is supposed to facilitate
%\cstoidx outer\par
%locating (among other errors) unbalanced braces: an \cs{outer}
%macro is supposed
%to  appear only in non-embedded contexts.
%To be precise, it is not allowed to occur 
%\begin{itemize}
%\item in macro replacement texts (but it can appear in
%    for instance \cs{edef} after 
%    \cs{noexpand}, and after \cs{meaning}),
%\item in parameter texts,
%\item in skipped conditional text,
%\item in alignment preambles, and
%\item in the \gram{balanced text} of a \cs{message}, \cs{write},
%et cetera. \end{itemize}
%For certain applications, however, it is inconvenient
%that some of the plain macros  are outer, 
%in particular macros such as \cs{newskip}. One remedy is to
%redefine them, without the `outer' option, which
%is done for instance in \LaTeX, but  cleverer tricks are possible.
\item [\cs{outer}]
\emph{外部宏}\index{宏!外部的}的定义机制用于帮助%
\cstoidx outer\par
定位未配对花括号(及其他错误):\cs{outer} 宏只能出现在非嵌入语境中。
更准确地说,它不可以出现
\begin{itemize}
\item 在宏的替换文本中(但如果将它放在 \cs{noexpand} 或 \cs{meaning} 之后,
就可以出现在 \cs{edef} 的替换文本中),
\item 在参数文本中,
\item 在跳过的条件文本中,
\item 在阵列的导言中,以及
\item 在 \cs{message}、\cs{write} 等的 \gram{balanced text} 中。
\end{itemize}
然而,在特定应用中某些 plain 宏是外部的很不方便,
特别是像 \cs{newskip} 这样的宏。有个补救方法是将它们重新定义为非外部的宏,
这正是 \LaTeX\ 所做的,但还有更巧妙的做法。

%\item [\cs{long}]
%Ordinarily, macro parameters are not supposed to contain
%\cstoidx long\par
%\cs{par} tokens. This restriction is useful (much more so
%than the \cs{outer} definitions) in locating
%forgotten closing braces. 
%For example, \TeX\ will complain about a `runaway argument'
%\message{Example on}
%in the following sequence:
%\begin{verbatim}
%\def\a#1{ ... #1 ... }
%\a {This sentence should be in braces.

%And this is not supposed to be part of the argument
%\end{verbatim}
%\message{one page}
%The empty line generates a \cs{par}, which most of the times
%means that a closing brace has been forgotten.
\item [\cs{long}]
宏的参量通常不允许包含 \cstoidx long\par\cs{par} 记号。
这个限制在定位遗漏的右括号时很有用(比 \cs{outer} 定义有用得多)。
例如,对于下面的输入 \TeX\ 将抱怨`runaway argument':
\message{Example on}
\begin{verbatim}
\def\a#1{ ... #1 ... }
\a {This sentence should be in braces.

And this is not supposed to be part of the argument
\end{verbatim}
\message{one page}
其中的空行生成一个 \cs{par},这在多数时候意味着有个右花括号漏掉了。

%If arguments to a particular macro should be allowed
%to contain \cs{par} tokens,  then the macro must be declared
%to be \cs{long}. \end{description}
如果某个宏允许其参量包含 \cs{par} 记号,则这个宏应当定义为 \cs{long} 宏。
\end{description}

%The \cs{ifx} test for equality of tokens 
%(see Chapter~\ref{if}) takes prefixes into
%account when testing whether two tokens have the same definition.
在测试两个记号是否相等时(见第~\ref{if}~章),\cs{ifx} 也会将前缀考虑在内。

%\begin{comment}
%With a little ingenuity it is possible 
%for \cs{par} tokens to sneak into macro arguments anyway.
%Consider the example
%\begin{verbatim}
%\def\a#1\par!{ ... }
%\a bc\par ef\par!
%\end{verbatim}
%Here the macro \cs{a} is not \cs{long}, but the argument
%is \verb>bc\par ef>, which contains a \cs{par} token.
%However,
%this is of no importance in general.
%\end{comment}
\begin{comment}
With a little ingenuity it is possible 
for \cs{par} tokens to sneak into macro arguments anyway.
Consider the example
\begin{verbatim}
\def\a#1\par!{ ... }
\a bc\par ef\par!
\end{verbatim}
Here the macro \cs{a} is not \cs{long}, but the argument
is \verb>bc\par ef>, which contains a \cs{par} token.
However,
this is of no importance in general.
\end{comment}

%%\point The definition type
%\section{The definition type}
%\point The definition type
\section{定义的类型}

%There are four \gr{def} control sequences in \TeX:
%\csidx{def}, \csidx{gdef}, \csidx{edef}, and \csidx{xdef}.
%The control sequence 
%\alt
%\cs{gdef} is a synonym for \verb>\global\def> and
%\cs{xdef} is a synonym for \verb>\global\edef>.
%The `expanding definition' \cs{edef} is treated in 
%Chapter~\ref{expand}.
在 \TeX\ 中有四种 \gr{def} 控制序列:
\csidx{def}、\csidx{gdef}、\csidx{edef} 和 \csidx{xdef}。
其中
\cs{gdef} 是 \verb>\global\def> 的同义词,而
\cs{xdef} 是 \verb>\global\edef> 的同义词。
而`展开的定义' \cs{edef} 在第~\ref{expand}~章中介绍。

%The difference between the various types of macro definitions
%is only relevant at the time of the definition.
%When a macro is called there is no way of telling how
%it was defined.
各种宏定义仅在定义时有区别。在宏被调用时是无法知道它们是如何定义的。

%%\point[param:text] The parameter text
%\section{The parameter text}
%\label{param:text}
%\point[param:text] The parameter text
\section{参数文本}
\label{param:text}

%Between the control sequence or active character to be defined
%and the opening brace of the replacement text, a \gr{parameter
%text} can occur, somewhat corresponding to \indexterm{arguments}
%in regular programming languages. This specifies whether the macro has 
%parameters\index{parameter},
%how many, and how they are delimited. 
%The \gr{parameter text} cannot contain
%explicit braces.
在所定义的控制序列或活动字符以及替换文本的左花括号之间,
可以有\gr{parameter text},参数文本有点像普通编程语言的\indexterm{参量}。
它指定这个宏是否有\indexterm{参数},有多少个参数,以及各参数之间如何定界。
\gr{parameter text} 不能包含显式花括号。

%A macro can have at most nine parameters. 
%A~parameter is indicated by a parameter token,
%consisting of a macro parameter character
%(that is, a character of category code~6\index{category!6},
%in plain \TeX~\verb=#=) 
%followed by a digit~\n1--\n9. 
%For instance, \verb>#6>~denotes the sixth parameter of a macro.
%Parameter tokens cannot appear outside the context
%of a macro definition.
一个宏最多可以有九个参数。参数用参数记号表示,参数记号由宏参数字符%
(即类别码为~6\index{category!6} 的字符,在 plain \TeX\ 中为 \verb=#=)% 
后跟一个 \n1--\n9 的数字组成。举个例子,\verb>#6> 表示宏的第六个参数。
参数记号不能出现在宏定义之外的其他地方。

%In the parameter text,
%parameters must be numbered consecutively, starting at~1.
%A~space after a parameter token is significant,
%both in the parameter text and the replacement text.
在参数文本中,参数必须从~1~开始顺序编号。
参数记号后的空格是有意义的,不管是在参数文本还是在替换文本中。

%Parameters can be delimited or undelimited; this determines what the
%extent of the macro arguments will be. A~parameter
%is called undelimited if it is followed immediately
%by another parameter in the \gr{parameter text}, so in
%\verb+\def\foo#1#2+ the first parameter is undelimited.
%A~parameter is also undelimited if it is immediately followed
%by the opening brace of the replacement text, as in \verb+\def\foo#1{...}+.
%A~parameter is called delimited if it is followed by any other token; 
%in \verb+\def\foo#1!#2{...}+ the first parameter is delimited by the
%exclamation sign.
参数分为定界参数和非定界参数;它决定宏参量的范围。
如果在 \gr{parameter text} 中一个参数后面紧跟着另一个参数,就像
\verb+\def\foo#1#2+ 这样,则前面的参数就是非定界的。
如果一个参数后面紧跟着替换文本的左花括号,就像
\verb+\def\foo#1{...}+ 这样,则它也是非定界的。
一个参数称为定界的,如果它后面紧跟着其他记号;
比如 \verb+\def\foo#1!#2{...}+ 的第一个参数就是被感叹号定界的参数。

%The tokens (zero or more) that are substituted for
%a parameter when a macro is expanded (or `called')
%are called
%the `argument' corresponding to that parameter.
在宏展开(或`调用')时,
用于替换参数的(零个或多个)记号称为该参数对应的`参量'。

%%\spoint Undelimited parameters
%\subsection{Undelimited parameters}
%\spoint Undelimited parameters
\subsection{非定界参数}

%When a macro with an \indextermbus{undelimited}{parameter}, for instance
%a macro \cs{foo} with one parameter
%\begin{verbatim}
%\def\foo#1{ ... #1 ...}
%\end{verbatim}
%is expanded, \TeX\ scans ahead (without expanding)
%until a non-blank token is found.
%If this token is not an explicit \gr{left brace}, 
%it is taken to be the argument
%corresponding to the parameter. Otherwise a \gr{balanced text}
%is absorbed by scanning until the matching explicit
%\gr{right brace} has been found.
%This balanced text then
%constitutes the argument.
在带有\indextermbus{非定界}{参数}的宏,比如单参数宏 \cs{foo}
\begin{verbatim}
\def\foo#1{ ... #1 ...}
\end{verbatim}
被展开时,\TeX\ 往前扫描(但不展开)直到遇到一个非空格记号。
如果这个记号不是显式 \gr{left brace},它就被取为对应于该参数的参量。
否则,通过扫描直到找到匹配的显式 \gr{right brace},
\TeX\ 得到一个 \gr{balanced text}。这个平衡文本就要找的参量。

%An example with three undelimited parameters follows: with
%\begin{verbatim}
%\def\foo#1#2#3{#1(#2)#3}
%\end{verbatim}
%the macro call \cs{foo123} gives `\hbox{1(2)3}';
%but \hbox{\verb-\foo 1 2 3-} also gives the same result.
%In the call
%\begin{disp}\cs{foo}\n{\char32 1\char32 2\char 32 3}\end{disp}
%the first space is skipped in the input processor of \TeX.
%The argument corresponding to the first parameter is then
%the~\n1. In order to find the second parameter \TeX\ then
%skips all blanks, in this case exactly one. As second
%parameter \TeX\ finds then the~\n2. Similarly the third
%parameter is~\n3.
这里是个带有三个非定界参数的宏的例子:对于
\begin{verbatim}
\def\foo#1#2#3{#1(#2)#3}
\end{verbatim}
宏调用 \cs{foo123} 给出 `\hbox{1(2)3}';
而 \hbox{\verb-\foo 1 2 3-} 给出同样的结果。在调用
\begin{disp}\cs{foo}\n{\char32 1\char32 2\char 32 3}\end{disp}
中,第一个空格被 \TeX\ 的输入处理器跳过。
从而对应于第一个参数的参量是~\n1。为找到第二个参量,
\TeX\ 跳过所有空格(在此例子中跳过一个空格),
最后找到的第二个参量是~\n2。类似的第三个参量是~\n3。


%In order to pass several tokens as one undelimited argument
%one can use braces. With the above definition of \cs{foo}
%the call \verb>\foo a{bc}d> gives `\hbox{a(bc)d}'.
%When the argument of a macro is a balanced text instead of
%a single token, the delimiting braces are not inserted when 
%the argument is
%inserted in the replacement text.
%For example:
%\begin{verbatim}
%\def\foo#1{\count0=1#1\relax}
%\foo{23}
%\end{verbatim}
%will expand to \verb>\count0=123\relax>,
%which assigns the value of 123 to the counter.
%On the other hand,  the statement
%\begin{verbatim}
%\count0=1{23}
%\end{verbatim}
%would
%assign~1 and print~23.
为了将多个记号作为一个非定界参量,你可以使用花括号。
对于上述的 \cs{foo} 定义,调用 \verb>\foo a{bc}d> 将给出 `\hbox{a(bc)d}'。
当宏的参量是平衡文本而非单个记号时,
在将参量插入替换文本时定界花括号会被去掉。例如:
\begin{verbatim}
\def\foo#1{\count0=1#1\relax}
\foo{23}
\end{verbatim}
将展开为 \verb>\count0=123\relax>,这将对计数器赋值 123。
另一方面,下面语句
\begin{verbatim}
\count0=1{23}
\end{verbatim}
将赋值 1 并排印~23。

%%\spoint Delimited parameters
%\subsection{Delimited parameters}
%\spoint Delimited parameters
\subsection{定界参数}

%Apart from enclosing it in braces there is another way
%to pass a sequence of tokens as a single argument to a macro,
%namely by using a \indextermbus{delimited}{parameter}.
除了将它们括在花括号里面,还有另一种方式可将一串记号作为宏的一个参量,
即使用\indextermbus{定界的}{参数}。

%Any non-parameter tokens in the \gr{parameter text} occurring
%after a macro parameter (that is, after the parameter number
%following the parameter character)
%act as a delimiter for that parameter. This includes space tokens:
%a space after a parameter number is significant.
%Delimiting tokens can also occur between the control
%sequence being defined and the first parameter token~\verb>#1>.
在 \gr{parameter text} 中,出现在宏参数之后%
(即在紧跟参数字符的参数编号之后)的非参数记号被当作该参数的定界子。
定界子可以包含空格记号:参数编号之后的空格是有意义的。
定界记号同样出现在所定义的控制序列和它的第一个参数记号 \verb>#1> 之间。

%Character tokens acting as delimiters in the parameter text
%have both their character code and
%category code stored; the delimiting character tokens of the
%actual arguments have to match both.
%Category codes of such characters may include some that
%can normally only appear in special contexts; for instance, after
%the definition
%\begin{verbatim}
%\def\foo#1_#2^{...}
%\end{verbatim}
%the macro \cs{foo}
%can be used outside math mode.
对于在参数文本中充当定界子的字符记号,它们的字符码和类别码都被存储下来;
实际参量的定界字符记号必须和两者都匹配。
这些字符可以拥有一些通常只出现在特殊环境中的类别码;比如,在定义
\begin{verbatim}
\def\foo#1_#2^{...}
\end{verbatim}
之后,宏 \cs{foo} 可以在数学模式之外使用。

%When looking for the argument corresponding to
%a delimited parameter, \TeX\ absorbs all tokens without expansion (but
%balancing braces) until the 
%(exact sequence of) delimiting tokens is encountered.
%The delimiting tokens are not part of the argument;
%they are removed from the input stream during the macro call.
在寻找对应于一个定界参数的参量时,
\TeX\ 吸收所有记号而不展开它们(但保持花括号配对),
直到遇到(完全一致的)定界记号串。
定界记号串不是参量的一部分;在宏调用时它们被从输入流中移除。

%%\spoint Examples with delimited arguments
%\subsection{Examples with delimited arguments}
%\spoint Examples with delimited arguments
\subsection{定界参量举例}

%As a simple example,
%\begin{verbatim}
%\def\DoASentence#1#2.{{#1#2.}}
%\end{verbatim}
%defines a macro with an undelimited first parameter,
%and a second parameter delimited by a period.
%In the call
%\begin{verbatim}
%\DoASentence \bf This sentence is the argument.
%\end{verbatim}
%the arguments are:
%\begin{verbatim}
%#1<-\bf
%#2<-This sentence is the argument
%\end{verbatim}
%Note that the closing period is not in the argument, but it has
%been absorbed; it is no longer in the input stream.
作为一个简单例子,我们定义一个宏
\begin{verbatim}
\def\DoASentence#1#2.{{#1#2.}}
\end{verbatim}
它的第一个参数是非定界的,而第二个参数是用句号定界的。
像下面这样调用时
\begin{verbatim}
\DoASentence \bf This sentence is the argument.
\end{verbatim}
它的两个参量分别是:
\begin{verbatim}
#1<-\bf
#2<-This sentence is the argument
\end{verbatim}
注意结尾的句号不在参量中,但它已经被吸收了,不会再出现在输入流中。

%A~commonly used delimiter is \cs{par}:
%\begin{verbatim}
%\def\section#1. #2\par{\medskip\noindent {\bf#1. #2\par}}
%\end{verbatim}
%This macro has a first parameter that is delimited by~`\n{.\char32}',
%and a second parameter that is delimited by \cs{par}.
%The call\message{example on one page}
%\begin{verbatim}
%\section 2.5. Some title

%The text of the section...
%\end{verbatim}
常用的定界子是 \cs{par},例如:
\begin{verbatim}
\def\section#1. #2\par{\medskip\noindent {\bf#1. #2\par}}
\end{verbatim}
这个宏第一个参数用 `\n{.\char32}' 定界,而第二个参数用 \cs{par} 定界。
像下面调用此宏\message{example on one page}
\begin{verbatim}
\section 2.5. Some title

The text of the section...
\end{verbatim}
%will give
%\begin{disp}\verb>#1<-2.5>\nl
%\verb>#2<-Some title>\n{\char32}\end{disp}
%Note that there is a space at the end of the second argument
%generated by the line end. If this space is unwanted one might
%define
%\begin{verbatim}
%\def\section#1. #2 \par{...}
%\end{verbatim}
%with \n{\char32}\cs{par} delimiting the second
%argument. This approach, however,
%precludes  the user's writing the \cs{par} explicitly:
%\begin{verbatim}
%\section 2.5 Some title\par
%\end{verbatim}
%One way out of this dilemma is to write
%\verb>#2\unskip> on all places in the definition text
%where the trailing space would be unwanted.
将给出
\begin{disp}\verb>#1<-2.5>\nl
\verb>#2<-Some title>\n{\char32}\end{disp}
注意在第二个由行尾符生成的参量的末尾有个空格。
如果这个空格是多余的,你可以定义
\begin{verbatim}
\def\section#1. #2 \par{...}
\end{verbatim}
从而用 \n{\char32}\cs{par} 定界第二个参量。
然而这种方法导致用户不可以显式写上 \cs{par}:
\begin{verbatim}
\section 2.5 Some title\par
\end{verbatim}
解决这种两难选择的其中一种方法是,
在需要去掉结尾空格时,在定义文本中写上 \verb>#2\unskip> 。

%Control sequences acting as delimiters need not be defined,
%as they are absorbed without expansion. Thus
%\begin{verbatim}
%\def\control#1\sequence{...}
%\end{verbatim}
%is a useful
%definition, even if \cs{sequence} is undefined.
充当定界子的控制序列不需要已经定义,因为它们只被吸收不会被展开。因此
\begin{verbatim}
\def\control#1\sequence{...}
\end{verbatim}
是有效的定义,即使在 \cs{sequence} 未定义时。

%The importance of category codes in delimited arguments
%is shown by the following example:
%\begin{verbatim}
%\def\a#1 #2.{ ... }
%\catcode`\ =12
%\a b c
%d.
%\end{verbatim}
%which gives
%\begin{verbatim}
%\a #1 #2.-> ...
%#1<- b c
%#2<-d
%\end{verbatim}
%Explanation: the delimiter between parameters 1 and~2 is a space
%of category~10\index{category!10}.
%In between \n{a} and \n{b} there is a space
%of category~12\index{category!12};
%the first space of  category~10
%is the space that is generated by the line end.
下面的例子展示了定界参量的类别码是至关重要的:
\begin{verbatim}
\def\a#1 #2.{ ... }
\catcode`\ =12
\a b c
d.
\end{verbatim}
将给出
\begin{verbatim}
\a #1 #2.-> ...
#1<- b c
#2<-d
\end{verbatim}
解释:参数~1 和参数~2 之间的定界子是一个第~10~类\index{category!10}的空格。
在 \n{a} 和 \n{b} 之间有一个第~12~类\index{category!12}的空格;
第一个第~10~类的空格是由行尾生成的空格。

%For a `real-life' application of matching of category codes,
%see the explanation of \cs{newif} in Chapter~\ref{if},
%and the example on page~\pageref{ex:jobnumber}.
在第~\ref{if}~章对 \cs{newif} 的解释,以及
在第~\pageref{ex:jobnumber}~页的例子中,有类别码匹配的实际例子。


%%\spoint Empty arguments
%\subsection{Empty arguments}
%\spoint Empty arguments
\subsection{空参量}

%If the user specifies a \gr{balanced text} in braces
%when \TeX\ expects a macro
%argument, that text is used as the argument.
%Thus, specifying \verb-{}- will give an argument that is
%an empty list of tokens; this is called an `empty argument'.
在 \TeX\ 需要一个宏参量时,如果用户指定一个围在花括号中的
\gr{balanced text},则该文本就被用作参量。
因此,指定 \verb-{}- 给出的参量是一个空记号列;它称为`空参量'。

%Empty arguments can also arise from the use of delimited
%parameters. For example, after the definition
%\begin{verbatim}
%\def\mac#1\ro{ ... }
%\end{verbatim}
%the call
%\begin{verbatim}
%\mac\ro
%\end{verbatim}
%will give an empty argument. 
在使用定界参数时也可能得到空参量。例如,在下述定义
\begin{verbatim}
\def\mac#1\ro{ ... }
\end{verbatim}
之后,这样调用
\begin{verbatim}
\mac\ro
\end{verbatim}
将给出一个空参量。

%\begin{comment}
%However, only
%one empty argument can be created this way: 
%if the macro had been defined as
%\begin{verbatim}
%\def\mac#1#2\ro{ ... }
%\end{verbatim}
%the same call
%\begin{verbatim}
%\mac\ro \othermacro \stillothermacro
%\end{verbatim}
%will probably cause a `\n{Runaway argument?}' error message.
%Explanation: the first parameter is undelimited, so the corresponding
%argument is `\cs{ro}'; after that \TeX\ starts looking for a list
%of tokens delimited by~\cs{ro}.
%\end{comment}
\begin{comment}
However, only
one empty argument can be created this way: 
if the macro had been defined as
\begin{verbatim}
\def\mac#1#2\ro{ ... }
\end{verbatim}
the same call
\begin{verbatim}
\mac\ro \othermacro \stillothermacro
\end{verbatim}
will probably cause a `\n{Runaway argument?}' error message.
Explanation: the first parameter is undelimited, so the corresponding
argument is `\cs{ro}'; after that \TeX\ starts looking for a list
of tokens delimited by~\cs{ro}.
\end{comment}

%\subsection{The macro parameter character}
\subsection{宏参数字符}

%When \TeX's input processor scans a macro definition text, it inserts
%a parameter token for any occurrence of a macro
%\indextermsub{parameter}{character}\index{character!parameter}
%followed by a digit.  In effect, a
%parameter token in the replacement text states `insert parameter
%number such and such here'.  Two parameter characters in a row are
%replaced by a single one.
在 \TeX\ 的输入处理器扫描宏定义文本时,它对后面跟着数字的每个宏%
\indextermsub{参数}{字符}\index{character!parameter}插入一个参数记号。
实际上,在替换文本中,参数记号表示`在这里插入某某编号的参量'。
连续两个参数字符被替换为一个。

%The latter fact can be used for nested macro definitions.
%\label{nest:def}\howto Nested macro definitions\par
%Thus
%\begin{verbatim}
%\def\a{\def\b#1{...}}
%\end{verbatim}
%gives an error message
%because \cs{a} was defined without parameters, and
%yet there is a parameter token in its replacement text.
后一个的事实可以用于嵌套的宏定义。
\label{nest:def}\howto Nested macro definitions\par
下面的定义
\begin{verbatim}
\def\a{\def\b#1{...}}
\end{verbatim}
给出一个错误信息,因为 \cs{a} 被定义为不带参数的宏,
而在它的替换文本中有一个参数记号。

%The following
%\begin{verbatim}
%\def\a#1{\def\b#1{...}}
%\end{verbatim}
%defines a macro \cs{a} that
%defines a macro \cs{b}. However, \cs{b} still does not
%have any parameters: the call
%\begin{verbatim}
%\a z
%\end{verbatim}
%defines a macro \cs{b} without parameters,
%that has to be followed by a~\n z.
%Note that this
%does not attempt to define a macro \cs{bz}, because the
%control sequence \cs{b} has already been formed in \TeX's
%input processor when that input line was read.
下面的语句
\begin{verbatim}
\def\a#1{\def\b#1{...}}
\end{verbatim}
定义了宏 \cs{a},这个宏又定义了另一个宏 \cs{b}。然而,
\cs{b} 任然不带任何参数:这样调用
\begin{verbatim}
\a z
\end{verbatim}
就定义了一个不带参数的宏 \cs{b},其后必须跟着 \n z。
注意这并不是在定义宏 \cs{bz},
因为 \TeX\ 的输入处理器读取输入行时就已经形成了控制序列 \cs{b}。

%Finally,
%\begin{verbatim}
%\def\a{\def\b##1{...}}
%\end{verbatim}
%defines a macro \cs{b} 
%with one parameter.
最后,
\begin{verbatim}
\def\a{\def\b##1{...}}
\end{verbatim}
定义了带一个参数的宏 \cs{b}。

%Let us examine the handling of the parameter character
%in some detail.
%Consider
%\begin{verbatim}
%\def\a#1{ .. #1 .. \def\b##1{ ... }}
%\end{verbatim}
%When this is read as input, the input processor
%\begin{itemize}
%\item replaces the characters \verb>#1> by \gr{parameter token$_1$}, and
%\item replaces the characters \verb>##> by \verb>#>\end{itemize}
%A macro call of \cs{a} will then let the input processor scan
%\begin{verbatim}
%\def\b#1{ ... }
%\end{verbatim}
%in which the two characters \verb>#1> are
%\alt
%replaced by a parameter token.
我们来仔细检查一下参数字符的处理过程。考虑下面定义
\begin{verbatim}
\def\a#1{ .. #1 .. \def\b##1{ ... }}
\end{verbatim}
当这个语句作为输入被读取时,输入处理器
\begin{itemize}
\item 将字符串 \verb>#1> 替换为 \gr{parameter token$_1$},并且
\item 将字符串 \verb>##> 替换为 \verb>#>
\end{itemize}
在调用宏 \cs{a} 时,输入处理器将扫描
\begin{verbatim}
\def\b#1{ ... }
\end{verbatim}
并将其中两个字符的 \verb>#1> 替换为一个参数记号。

%%\spoint Brace delimiting
%\subsection{Brace delimiting}
%\spoint Brace delimiting
\subsection{花括号定界}

%Ordinarily, it is not possible to have left or right
%braces in the \gr{parameter text} of a definition.
%There is a special mechanism, however, that can make
%the last parameter of a macro act as if it is delimited
%by an opening brace. 
在定义的 \gr{parameter text} 中通常是不可能有左或右花括号的。
然而,有一种特殊规定可以让宏的最后一个参数看似用左花括号定界的。

%If the last parameter token
%is followed by a parameter character (\verb>#>),
%which in turn is followed by the opening brace of the
%replacement text, \TeX\ makes the last parameter
%be delimited by a beginning-of-group character.
%Furthermore, unlike other delimiting tokens in
%parameter texts, this opening brace is not
%removed from the input stream.
如果最后一个参数记号后面是一个参数字符(\verb>#>),
接着是替换文本的左花括号,\TeX\ 将让最后一个参数以组开始符定界。
此外,与参数文本的其他定界记号不同,这个左花括号不会被从输入流中移除。

%Consider an example.
%Suppose we want to have a macro
%\cs{every} that can fill token lists as follows:
%\begin{verbatim}
%\every par{abc} \every display{def}
%\end{verbatim}
%This macro can be defined as
%\begin{verbatim}
%\def\every#1#{\csname every#1\endcsname}
%\end{verbatim}
%In the first call above, the argument corresponding to
%the parameter is \n{abc}, so the call 
%expands to
%\begin{verbatim}
%\csname everypar\endcsname{abc}
%\end{verbatim}
%which gives the desired result.
考虑一个例子。假设你想有个宏 \cs{every} 宏用于像下面这样填充记号列:
\begin{verbatim}
\every par{abc} \every display{def}
\end{verbatim}
这个宏可以定义为
\begin{verbatim}
\def\every#1#{\csname every#1\endcsname}
\end{verbatim}
在上面的第一个调用中,对应于参数的参量是 \n{abc},因此该调用展开为
\begin{verbatim}
\csname everypar\endcsname{abc}
\end{verbatim}
这就给出所要的结果。


%\section{Construction of control sequences}
%\label{cs:name}
\section{构造控制序列}
\label{cs:name}

%The commands \csidx{csname} and \csidx{endcsname} can be used
%to construct a control sequence. 
%For instance
%\begin{verbatim}
%\csname hskip\endcsname 5pt
%\end{verbatim}
%is equivalent to \verb=\hskip5pt=.
命令 \csidx{csname} 和 \csidx{endcsname} 可用于构造控制序列。例如
\begin{verbatim}
\csname hskip\endcsname 5pt
\end{verbatim}
等价于 \verb=\hskip5pt=。

%During this construction process
%all macros and other expandable control sequences
%between \cs{csname} and \cs{endcsname}
%are expanded as usual, until only unexpandable
%character tokens remain. A~variation of the above example,
%\begin{verbatim}
%\csname \ifhmode h\else v\fi skip\endcsname 5pt
%\end{verbatim}
%performs an \cs{hskip} or \cs{vskip} depending on the mode.
%The final result of the expansion should 
%consist of only character tokens, but
%their category codes do not matter.
%An unexpandable control sequence gives an error here:
%\TeX\ will insert an \cs{endcsname} right before it
%as an attempt at error recovery.
在构造的过程中,介于 \cs{csname} 和 \cs{endcsname}
之间的所有宏和其他可展开控制序列都如常展开,
直到仅留下不可展开的字符记号。若将上面例子改写为
\begin{verbatim}
\csname \ifhmode h\else v\fi skip\endcsname 5pt
\end{verbatim}
则它将依据当前模式执行 \cs{hskip} 或 \cs{vskip}。
展开的最后结果必须只包含字符记号,但对它们的类别码却无限制。
在碰到不可展开的控制序列时,\TeX\ 将抛出一个错误,
并在错误恢复时在该处之前插入一个 \cs{endcsname}。

%With \cs{csname} it is possible to construct
%control sequences that cannot ordinarily be written,
%because the constituent character tokens may have another category
%\alt
%than~11, letter. This principle can be used to hide
%\howto Hide counters from the user\par
%inner control sequences of a macro package from the user.
%\begin{example}
%\begin{verbatim}
%\def\newcounter#1{\expandafter\newcount
%    \csname #1:counter\endcsname}
%\def\stepcounter#1{\expandafter\advance
%    \csname #1:counter\endcsname 1\relax}
%\end{verbatim}
%In the second definition the \cs{expandafter} is superfluous,
%but it does no harm, and it is conceptually clearer.
%\end{example}
利用 \cs{csname},我们可以构造通常无法写出的,
包含类别码不为~11(字母)的字符的控制序列。
这个原理可用于对用户隐藏宏包内部的控制序列。
\howto Hide counters from the user\par
\begin{example}
\begin{verbatim}
\def\newcounter#1{\expandafter\newcount
    \csname #1:counter\endcsname}
\def\stepcounter#1{\expandafter\advance
    \csname #1:counter\endcsname 1\relax}
\end{verbatim}
第二个定义中的 \cs{expandafter} 是多余的,但它没有坏处,
且可以让代码更清楚。
\end{example}

%The name of the actual counter created by \cs{newcounter}
%contains a colon, so that it takes some effort to write this
%control sequence. In effect, the counter
%is now hidden from the user, who can only
%access it through control sequences such as \cs{stepcounter}.
%By the way, the macro \cs{newcount} is defined \cs{outer} in
%the plain format, so the above definition of \cs{newcounter}
%can only be written after \cs{newcount} has been redefined.
\cs{newcounter} 创建的计数器的名称中包含一个冒号,因此写起来有点麻烦。
好处是现在这个计数器对用户是隐藏的,
它只能通过类似 \cs{stepcounter} 的控制序列来访问。
顺便说一下,在 plain 格式中 \cs{newcount} 被定义为 \cs{outer} 宏,
因此在重新定义 \cs{newcount} 后才能写出上述定义。

%If a control sequence formed with \verb>\csname...\endcsname>
%has not been defined
%before, its meaning is set to \cs{relax}.
%Thus if \verb=\xx= is an undefined control sequence, the
%command
%\begin{verbatim}
%\csname xx\endcsname
%\end{verbatim}
%will {\em not\/}
%give an error message, as it is equivalent to \verb=\relax=.
%Moreover, after this execution of the
%\verb-\csname...\endcsname- statement, the control sequence
%\verb=\xx= is itself equivalent to \cs{relax}, so it
%will no longer give an `undefined control sequence' error
%(see also page~\pageref{relax:cs}).
如果用 \verb>\csname...\endcsname> 生成的控制序列之前尚未定义,
它就被定义为 \cs{relax}。因此如果 \verb=\xx= 是一个未定义的控制序列,
命令
\begin{verbatim}
\csname xx\endcsname
\end{verbatim}
将{\em 不会}导致错误信息,因为它等价于\verb=\relax=。
此外,在执行 \verb-\csname...\endcsname- 语句后,控制序列
\verb=\xx= 本身等价于 \cs{relax},因此它也不再导致
`undefined control sequence' 的错误(另见第~\pageref{relax:cs}~页)。


%%\point Token assignments by \cs{let} and \cs{futurelet}
%\section{Token assignments by \protect\cs{let} and \protect\cs{futurelet}}
%\point Token assignments by \cs{let} and \cs{futurelet}
\section{用 \protect\cs{let} 和 \protect\cs{futurelet} 给出记号赋值}

%There are two \gr{let assignment}s in \TeX.
%Their syntax is
%\begin{disp}\cs{let}\gr{control sequence}\gr{equals}%
%     \gr{one optional space}\gr{token}\nl
%     \cs{futurelet}\gr{control sequence}\gr{token}\gr{token}
%     \end{disp}
%In the syntax of a \cs{futurelet} assignment
%no optional equals sign appears.
在 \TeX\ 中有两个 \gr{let assignment}。它们的语法为
\begin{disp}\cs{let}\gr{control sequence}\gr{equals}%
     \gr{one optional space}\gr{token}\nl
     \cs{futurelet}\gr{control sequence}\gr{token}\gr{token}
     \end{disp}
在 \cs{futurelet} 赋值的语法中,不可以出现可选的等号。

%%\spoint[let] \cs{let}
%\subsection{\protect\cs{let}}
%\label{let}
%\spoint[let] \cs{let}
\subsection{\protect\cs{let}}
\label{let}

%The primitive command \csidx{let} assigns the current meaning
%of a~token to a control sequence or active character.
原始命令 \csidx{let} 将某个记号的当前含义赋予一个控制序列或活动字符。

%For instance, in the plain format \cs{endgraf} is defined
%as
%\begin{verbatim}
%\let\endgraf=\par
%\end{verbatim}
%This enables macro writers to redefine \cs{par}, while
%still having the functionality of the primitive \cs{par}
%command available. For example,
%\begin{verbatim}
%\everypar={\bgroup\it\def\par{\endgraf\egroup}}
%\end{verbatim}
例如,在 plain 格式中,\cs{endgraf} 定义为
\begin{verbatim}
\let\endgraf=\par
\end{verbatim}
这使得宏的编写者可以重新定义 \cs{par},
而原始的 \cs{par} 命令的功能仍然可以使用。例如,
\begin{verbatim}
\everypar={\bgroup\it\def\par{\endgraf\egroup}}
\end{verbatim}

%The case where the \gr{token} to be assigned is not a control
%sequence but a character token instead has been treated 
%in Chapter~\ref{char}.
在第~\ref{char}~章中已经讨论了 \gr{token} 不是控制序列而是字符记号的情形。

%%\spoint \cs{futurelet}
%\subsection{\protect\cs{futurelet}}
%\spoint \cs{futurelet}
\subsection{\protect\cs{futurelet}}

%As was explained above, the sequence with \cs{let}
%\begin{disp}\cs{let}\gr{control sequence}\gr{token$_1$}\gr{token$_2$}%
%       \gr{token$_3$}\gr{token$\cdots$}\end{disp}
%assigns (the meaning of) \gr{token$_1$} to the control sequence, 
%and the remaining input stream looks like
%\begin{disp}\gr{token$_2$}\gr{token$_3$}\gr{token$\cdots$}\end{disp}
%That is, the \gr{token$_1$} has disappeared from the stream.
如上所述,\cs{let} 系列
\begin{disp}\cs{let}\gr{control sequence}\gr{token$_1$}\gr{token$_2$}%
       \gr{token$_3$}\gr{token$\cdots$}\end{disp}
将 \gr{token$_1$}(的含义)赋予该控制序列,而剩下的输入流如下 
\begin{disp}\gr{token$_2$}\gr{token$_3$}\gr{token$\cdots$}\end{disp}
即 \gr{token$_1$} 从输入流中消失了。

%The command \csidx{futurelet} works slightly differently:
%given the input stream
%\begin{disp}\cs{futurelet}\gr{control sequence}\gr{token$_1$}\gr{token$_2$}%
%       \gr{token$_3$}\gr{token$\cdots$}\end{disp}
%it assigns (the meaning of) \gr{token$_2$} to the control sequence, 
%and the remaining stream looks like
%\begin{disp}\gr{token$_1$}\gr{token$_2$}\gr{token$_3$}\gr{token$\cdots$}\end{disp}
%That is, neither \gr{token$_1$} nor \gr{token$_2$} has
%been lifted from the stream.
%However, now \gr{token$_1$}
%`knows' what \gr{token$_2$} is, without having had to absorb it
%as a macro parameter. See an example below.
命令 \csidx{futurelet} 的运作稍有不同:对于下面的输入流
\begin{disp}\cs{futurelet}\gr{control sequence}\gr{token$_1$}\gr{token$_2$}%
       \gr{token$_3$}\gr{token$\cdots$}\end{disp}
它将 \gr{token$_2$}(的含义)赋予该控制序列,而剩下的输入流如下
\begin{disp}\gr{token$_1$}\gr{token$_2$}\gr{token$_3$}\gr{token$\cdots$}\end{disp}
也就是说,\gr{token$_1$} 和 \gr{token$_2$} 都不会被从输入流中去掉。
然而,现在 \gr{token$_1$} 无需将 \gr{token$_2$} 取为宏参数就`知道'它是什么。
见后面给出的例子。

%If a character token has been \cs{futurelet} to a control
%sequence, its category code is fixed.
%The subsequent \gr{token$_1$} cannot change
%it anymore.
在字符记号已经被 \cs{futurelet} 到一个控制序列时,它的类别码就已经确定了。
随后的 \gr{token$_1$} 无法再改变它。

%%\point Assorted remarks
%\section{Assorted remarks}
%\point Assorted remarks
\section{杂项注记}

%%\spoint Active characters
%\subsection{Active characters}
%\spoint Active characters
\subsection{活动字符}

%A character token of category~13\index{category!13} is called an
%\indexterm{active character}, and it
%can be defined just like a control sequence.
%If the definition of the character appears inside a macro,
%the character has to be active at the time of the definition
%of that macro.
类别码~13\index{category!13}~的字符记号称为\indexterm{活动字符},
我们可以像控制序列那样定义它。如果这种字符的定义出现在宏里面,
在宏定义时这个字符必须是活动的。

%Consider for example the following definition
%(taken from Chapter~\ref{mouth}):
%\begin{verbatim}
%{\catcode`\^^M=13 %
% \gdef\obeylines{\catcode`\^^M=13 \def^^M{\par}}%
%}
%\end{verbatim}
%The unusual category of the \verb>^^M> character
%has to be set during the definition of \cs{obeylines},
%otherwise \TeX\ would think that the line ended
%after \cs{def}.
以下面的定义为例(取自第~\ref{mouth}~章):
\begin{verbatim}
{\catcode`\^^M=13 %
 \gdef\obeylines{\catcode`\^^M=13 \def^^M{\par}}%
}
\end{verbatim}
在定义 \cs{obeylines} 前必须先设定好 \verb>^^M> 字符的类别码,
否则 \TeX\ 会认为该行在 \cs{def} 之后结束。

%\subsection{Macros versus primitives}
\subsection{宏与原始命令}

%The distinction between \indexterm{primitive
%  commands}\indexterm{command !primitive} and user macros is not
%nearly as important in \TeX\ as it is in other programming
%languages.
%\begin{itemize}
%\item The user can use primitive commands under different names:
%     
%\begin{verbatim}
%\let\StopThisParagraph=\par
%\end{verbatim}
%\item Names of primitive commands can be used for
%      user macros:
%\begin{verbatim}
%\def\par{\hfill$\bullet$\endgraf}
%\end{verbatim}
%\item Both user macros and a number of \TeX\ primitives
%      are subject to expansion, for instance all conditionals,
%      and commands such as \cs{number} and~\cs{jobname}.
%\end{itemize}
相比其他编程语言,在 \TeX\ 中\indexterm{原始命令}\index{命令!原始的}%
与用户宏的区分并不大重要。 
\begin{itemize}
\item 用户可以用别的名称使用原始命令: 
\begin{verbatim}
\let\StopThisParagraph=\par
\end{verbatim}
\item 原始命令的名称可以给用户宏使用:
\begin{verbatim}
\def\par{\hfill$\bullet$\endgraf}
\end{verbatim}
\item 很多原始命令和用户宏一样都可以展开,比如所有条件句,
以及类似 \cs{number} 和 \cs{jobname} 的命令。
\end{itemize}

%%\spoint Tail recursion
%\subsection{Tail recursion}
%\spoint Tail recursion
\subsection{尾递归}

%Macros in \TeX, like procedures in most modern programming
%languages, are allowed to be \emph{recursive}\index{recursion}: that is, the 
%definition of a macro can contain a call to this same macro,
%or to another macro that will call this macro.
%Recursive macros tend to clutter up \TeX's memory
%if too many `incarnations' of such a macro are active
%at the same time. However, \TeX\ is able to prevent this
%in one frequently occurring case of recursion: tail recursion.
\TeX 中的宏,与大多数现代编程语言一样,可以是\emph{递归的}\index{递归}:
即在宏定义中可以调用这个宏本身,或者另一个将调用这个宏的宏。
如果同一个递归宏的`化身'在同一时间出现太多次,它容易弄乱 \TeX\ 的内存。
然而,对于尾递归这种常见的递归情形,\TeX\ 能够避免出现混乱。

%In order to  appreciate what goes on here, some background
%knowledge is needed. When \TeX\ starts executing a macro
%it absorbs the parameters, and places an item pointing to
%the replacement text on the \indextermsub{input}{stack},
%so that the scanner will next be directed to
%this replacement. Once it has been processed, the item on the 
%input stack can be removed.
%However, if the definition text
%of a macro contains further macros, this process will be
%repeated for them: new items may be placed on the input stack
%directing the scanner to other macros
%even before the first one has been completed.
要理解这里发生的事情,需要一些背景知识。
在开始执行一个宏时,\TeX\ 抓取宏的参量,
然后在\indextermsub{输入}{栈}上放置一个指向替换文本的项目,
这样扫描器接着将转到替换文本中。一旦它处理完毕,
输入栈中的这个项目将被移除。然而,如果宏的定义文本中包含另外的宏,
这个过程将对它们重复进行:新的项目将被放在输入栈上,
引导扫描器到其他宏中,甚至在第一个宏还未完成时,

%In general this `stack build-up' is a necessary evil, but
%it can be prevented if the nested macro call is the
%{\em last\/} token in the replacement text of the original
%macro. After the last token no further tokens need to be
%considered, so one might as well clear the top item
%from the input stack
%before a new one is put there.
%This is what \TeX\ does.
一般来说,这种`栈构建'不是很好但却是无可避免的;
然而如果嵌套宏调用出现在宏的替换文本{\em 末尾的}记号处,
这是可以避免的。在末尾的记号后没有其他记号需要处理,
因此我们可以在新的宏放进来之前清除输入栈顶部的项目。
这正是 \TeX\ 所做的事情。

%The \csidx{loop} macro of plain \TeX\ provides a good illustration
%\label{loop:ex}
%of this principle. The definition is
%\begin{verbatim}
%\def\loop#1\repeat{\def\body{#1}\iterate}
%\def\iterate{\body \let\next=\iterate
%    \else \let\next=\relax\fi \next}
%\end{verbatim}
%and this macro can be called for example as follows:
%\begin{verbatim}
%\loop \message{\number\MyCount}
%    \advance\MyCount by 1
%    \ifnum\MyCount<100 \repeat
%\end{verbatim}
%The macro \cs{iterate} can call itself and, when it does so,
%the recursive call is performed by the last token in the list.
%It would have been possible to define \cs{iterate}
%as
%\begin{verbatim}
%\def\iterate{\body \iterate\fi}
%\end{verbatim}
%but then \TeX\ would not have been able to resolve the recursion
%as the call \cs{iterate} is not the last token in the replacement
%text of \cs{iterate}. Assigning \verb>\let\next=\iterate>
%is here a way to let
%the recursive call be the last token in the list.
Plain \TeX\ 的 \csidx{loop} 对这个原理给出很好的说明。
\label{loop:ex}它的定义是
\begin{verbatim}
\def\loop#1\repeat{\def\body{#1}\iterate}
\def\iterate{\body \let\next=\iterate
    \else \let\next=\relax\fi \next}
\end{verbatim}
而这个宏可以用类似下面的例子来调用:
\begin{verbatim}
\loop \message{\number\MyCount}
    \advance\MyCount by 1
    \ifnum\MyCount<100 \repeat
\end{verbatim}
宏 \cs{iterate} 可以调用它自己,而且当它这样做时,递归调用出现在列表的末尾记号。
也许可以将 \cs{iterate} 定义为
\begin{verbatim}
\def\iterate{\body \iterate\fi}
\end{verbatim}
但这样 \TeX\ 将无法消除这个递归,因为对 \cs{iterate} 的调用并不出现在
\cs{iterate} 的替换文本的末尾记号. 这里的赋值 \verb>\let\next=\iterate>
就是一种让递归调用出现在列表的末尾记号的方法。

%Another way of resolving tail recursion is to use
%\cs{expandafter} (see page~\pageref{after:cond}): in
%\begin{verbatim}
%\def\iterate{\body \expandafter\iterate\fi}
%\end{verbatim}
%it removes the \cs{fi} token.
%Tail recursion would also be resolved if the last
%tokens in the list were arguments for the
%recursive macro.
另一种消除尾递归的方法是使用 \cs{expandafter}(见第~\pageref{after:cond}~页):
在下面
\begin{verbatim}
\def\iterate{\body \expandafter\iterate\fi}
\end{verbatim}
这种写法中,它去掉了 \cs{fi} 记号。如果列表的末尾记号是递归宏的参量,尾递归也将被消除。

%An aside: by defining \cs{iterate} as
%\begin{verbatim}
%\def\iterate{\let\next\relax 
%    \body \let\next\iterate \fi \next}
%\end{verbatim}
%it becomes possible to write
%\begin{verbatim}
%\loop ... \if... ... \else ... \repeat
%\end{verbatim}
顺便说一句:如果将 \cs{iterate} 定义为
\begin{verbatim}
\def\iterate{\let\next\relax 
    \body \let\next\iterate \fi \next}
\end{verbatim}
就可以这些写
\begin{verbatim}
\loop ... \if... ... \else ... \repeat
\end{verbatim}

%%\point Macro techniques
%\section{Macro techniques}
%\point Macro techniques
\section{宏的技巧}

%%\spoint Unknown number of arguments
%\subsection{Unknown number of arguments}
%\spoint Unknown number of arguments
\subsection{不确定个数的参量}

%In some applications,
%\howto  Macros with an undetermined number
%of arguments\par
%a macro is needed that can have a
%number of arguments that is not specified in advance.
在某些应用中,我们希望宏的参量个数不用先规定好。
\howto  Macros with an undetermined number
of arguments\par

%Consider the problem of translating a position on a chess board
%(for full macros and fonts, see~\cite{chess} and~\cite{Tut}),
%given like
%\begin{verbatim}
%\White(Ke1,Qd1,Na1,e2,f4)
%\end{verbatim} 
%to a sequence of typesetting instructions
%\begin{verbatim}
%\WhitePiece{K}{e1} \WhitePiece{Q}{d1} \WhitePiece{N}{a1} 
%\WhitePiece{P}{e2} \WhitePiece{P}{f4}
%\end{verbatim}
%Note that for pawns the `P' is omitted in the list of positions.
考虑在国际象棋棋盘上转换位置的问题%
(全部宏和字体可以在 \cite{chess} 和 \cite{Tut} 中找到),
比如要将棋谱记法
\begin{verbatim}
\White(Ke1,Qd1,Na1,e2,f4)
\end{verbatim} 
转换为一串排版好的走法
\begin{verbatim}
\WhitePiece{K}{e1} \WhitePiece{Q}{d1} \WhitePiece{N}{a1} 
\WhitePiece{P}{e2} \WhitePiece{P}{f4}
\end{verbatim}
注意在位置列表中兵前面的 `P' 被省略掉了。

%The first problem is that the list of pieces 
%is of variable length, so we append a terminator piece:
%\begin{verbatim}
%\def\White(#1){\xWhite#1,xxx,}
%\def\endpiece{xxx}
%\end{verbatim}
%for which we can test.
%Next, the macro \cs{xWhite} takes one position from the list,
%tests whether it is the terminator, and if not,
%subjects it to a test to see whether it is a pawn.
%\begin{verbatim}
%\def\xWhite#1,{\def\temp{#1}%
%    \ifx\temp\endpiece 
%    \else \WhitePieceOrPawn#1XY%
%          \expandafter\xWhite 
%    \fi}
%\end{verbatim}
%An \cs{expandafter} command is necessary to remove the
%\cs{fi} (see page~\pageref{after:cond}), so that 
%\cs{xWhite} will get the next position as argument
%instead of \cs{fi}.
首要问题在于棋子列表是变长的,因此我们添加一个终止棋子:
\begin{verbatim}
\def\White(#1){\xWhite#1,xxx,}
\def\endpiece{xxx}
\end{verbatim}
这样我们就可以检测到。接下来,\cs{xWhite} 从列表中读取一个位置,
检测它是否为终止棋子;如果不是,依据它是否为一个兵分别处理。
\begin{verbatim}
\def\xWhite#1,{\def\temp{#1}%
    \ifx\temp\endpiece 
    \else \WhitePieceOrPawn#1XY%
          \expandafter\xWhite 
    \fi}
\end{verbatim}
其中必须用 \cs{expandafter} 命令去掉 \cs{fi}(见第~\pageref{after:cond}~页),
使得 \cs{xWhite} 以下一个位置而不是 \cs{fi} 为参量。

%Positions  are either two or three characters long.
%The call to \cs{White\-Piece\-OrPawn}, a four-parameter macro,
%appended a terminator string \n{XY}. 
%In the case of a pawn, therefore, argument~3 is the character~\n X
%and argument~4 is empty; for all other pieces argument~1
%is the piece, 2~and~3 are the position, and argument~4 is~\n X.
%\begin{verbatim}
%\def\WhitePieceOrPawn#1#2#3#4Y{
%    \if#3X \WhitePiece{P}{#1#2}%
%    \else  \WhitePiece{#1}{#2#3}\fi}
%\end{verbatim}
位置的长度为两个或三个字符长。在调用四参数宏 \cs{White\-Piece\-OrPawn}
时添加了一个终止字符串 \n{XY}。因此,对棋子为兵的情形,
第~3~个参量为字符 \n X 而第~4~个参量为空;
对其他棋子的情形,第~1~个参量是该棋子,
而第~2~和~3~个参量为具体的位置,第~4~个参量为~\n X。
\begin{verbatim}
\def\WhitePieceOrPawn#1#2#3#4Y{
    \if#3X \WhitePiece{P}{#1#2}%
    \else  \WhitePiece{#1}{#2#3}\fi}
\end{verbatim}

%%\spoint Examining the argument
%\subsection{Examining the argument}
%\spoint Examining the argument
\subsection{检查参量}

%It may be necessary in some cases to test whether a macro
%\howto Examine a macro argument for the presence of some element\par
%\howto Apply \cs{uppercase} when the argument has a \cs{footnote}\par
%argument contains some element. For a real-life example,
%consider the following (see also the \cs{DisplayEquation}
%\alt
%example on page~\pageref{left:display}).
在有些时候。我们需要检查宏参量中是否包含某个元素。
\howto Examine a macro argument for the presence of some element\par
\howto Apply \cs{uppercase} when the argument has a \cs{footnote}\par
考虑下面的实际例子(另见第~\pageref{left:display}~页的
\cs{DisplayEquation} 的例子)。

%Suppose the title and author of an article are given as
%\begin{verbatim}
%\title{An angle trisector}
%\author{A.B. Cee\footnote*{Research supported by the
%Very Big Company of America}}
%\end{verbatim}
%with multiple authors
%given as
%\begin{verbatim}
%\author{A.B. Cee\footnote*{Supported by NSF grant 1}
%        \and
%        X.Y. Zee\footnote{**}{Supported by NATO grant 2}}
%\end{verbatim}
%Suppose further that the \cs{title} and \cs{author} macros
%are defined as
%\begin{verbatim}
%\def\title#1{\def\TheTitle{#1}}  \def\author#1{\def\TheAuthor{#1}}
%\end{verbatim}
%which will be used as
%\begin{verbatim}
%\def\ArticleHeading{ ... \TheTitle ... \TheAuthor ... }
%\end{verbatim}
假设文章的标题和作者为
\begin{verbatim}
\title{An angle trisector}
\author{A.B. Cee\footnote*{Research supported by the
Very Big Company of America}}
\end{verbatim}
而有多个作者时为
\begin{verbatim}
\author{A.B. Cee\footnote*{Supported by NSF grant 1}
        \and
        X.Y. Zee\footnote{**}{Supported by NATO grant 2}}
\end{verbatim}
另外假设 \cs{title} 和 \cs{author} 宏定义为
\begin{verbatim}
\def\title#1{\def\TheTitle{#1}}  \def\author#1{\def\TheAuthor{#1}}
\end{verbatim}
这将会被这样使用
\begin{verbatim}
\def\ArticleHeading{ ... \TheTitle ... \TheAuthor ... }
\end{verbatim}

%For some journals it is required to
%have the authorship and the title of the article in all capitals.
%The implementation of this could be
%\begin{verbatim}
%\def\ArticleCapitalHeading
%   { ...
%    \uppercase\expandafter{\TheTitle}
%     ...
%    \uppercase\expandafter{\TheAuthor}
%     ...
%   }
%\end{verbatim}
%Now the \cs{expandafter} commands will expand the title and
%author into the actual texts, and the \cs{uppercase} commands
%will capitalize them. However, for the authors this is wrong,
%since the \cs{uppercase} command will also capitalize the
%footnote texts.
%The problem is then to uppercase only the parts
%of the title in between the footnotes.
有些期刊要求文章的作者和标题全部大写。这种要求的实现方式可能是
\begin{verbatim}
\def\ArticleCapitalHeading
   { ...
    \uppercase\expandafter{\TheTitle}
     ...
    \uppercase\expandafter{\TheAuthor}
     ...
   }
\end{verbatim}
现在 \cs{expandafter} 命令将展开标题和作者为实际文本,
接着 \cs{uppercase} 命令将把它们变为大写。然而,
对于这样写对作者名来说是错误的,因为 \cs{uppercase} 命令将脚注文本也大写了。
现在的问题在于如何仅将在脚注之间的文本变为大写。

%As a first attempt, let us take the case of one author, and
%let the basic call be
%\begin{verbatim}
%\expandafter\UCnoFootnote\TheAuthor
%\end{verbatim}
%This expands into
%\begin{verbatim}
%\UCnoFootnote A.B. Cee\footnote*{Supported ... }
%\end{verbatim}
%The macro
%\begin{verbatim}
%\def\UCnoFootnote#1\footnote#2#3{\uppercase{#1}\footnote{#2}{#3}}
%\end{verbatim}
%will analyse this correctly:
%\begin{verbatim}
%#1<-A.B. Cee
%#2<-*
%#3<-Supported ...
%\end{verbatim}
%However, if there is no footnote, this macro is completely wrong.
作为第一次尝试,让我们先考虑只有一个作者的情形,设基本调用为
\begin{verbatim}
\expandafter\UCnoFootnote\TheAuthor
\end{verbatim}
这将展开为
\begin{verbatim}
\UCnoFootnote A.B. Cee\footnote*{Supported ... }
\end{verbatim}
这个宏
\begin{verbatim}
\def\UCnoFootnote#1\footnote#2#3{\uppercase{#1}\footnote{#2}{#3}}
\end{verbatim}
将正确地分析它:
\begin{verbatim}
#1<-A.B. Cee
#2<-*
#3<-Supported ...
\end{verbatim}
然而,如果脚注不存在时,这个宏将是完全错误的。

%As a first refinement we add a footnote ourselves, just to make
%sure that one is present:
%\begin{verbatim}
%\expandafter\UCnoFootnote\TheAuthor\footnote 00
%\end{verbatim}
%Now we have to test what kind of footnote we find:
%\begin{verbatim}
%\def\stopper{0}
%\def\UCnoFootnote#1\footnote#2#3{\uppercase{#1}\def\tester{#2}%
%    \ifx\stopper\tester
%    \else\footnote{#2}{#3}\fi}
%\end{verbatim}
%With \cs{ifx} we test the delimiter footnote sign against the
%actual sign encountered. Note that a solution with
%\begin{verbatim}
%\ifx0#2
%\end{verbatim}
%would be wrong if the footnote sign consists
%of more than one token, for instance~\verb>{**}>.
作为首个改进,我们自己添加脚注,以让它始终存在: 
\begin{verbatim}
\expandafter\UCnoFootnote\TheAuthor\footnote 00
\end{verbatim}
现在我们得检测所找到的脚注的类型:
\begin{verbatim}
\def\stopper{0}
\def\UCnoFootnote#1\footnote#2#3{\uppercase{#1}\def\tester{#2}%
    \ifx\stopper\tester
    \else\footnote{#2}{#3}\fi}
\end{verbatim}
我们用 \cs{ifx} 区分出定界脚注号和实际脚注号。注意如果改用
\begin{verbatim}
\ifx0#2
\end{verbatim}
将是错误的,因为脚注号可能包含多个记号,比如~\verb>{**}>。

%The macro so far is correct if there was no footnote,
%but if there was one it is wrong:
%the terminating tokens remain to be disposed of.
%They are taken care of in the following version:
%\begin{verbatim}
%\def\stopper{0}
%\def\UCnoFootnote#1\footnote#2#3{\uppercase{#1}\def\tester{#2}%
%    \ifx\stopper\tester
%    \else\footnote{#2}{#3}\expandafter\UCnoFootnote
%    \fi}
%\end{verbatim}
%A repeated call to \cs{UCnoFootnote} removes the delimiter tokens
%(the \cs{expandafter} first removes the \cs{fi}),
%and as an added bonus, this macro is also correct for multiple
%authors.
到目前为止,如果没有脚注这个宏是正确的,但如果有个脚注它是错误的:
这几个终结记号还有待我们处理掉。在下面这个版本中它们会被小心处理好:
\begin{verbatim}
\def\stopper{0}
\def\UCnoFootnote#1\footnote#2#3{\uppercase{#1}\def\tester{#2}%
    \ifx\stopper\tester
    \else\footnote{#2}{#3}\expandafter\UCnoFootnote
    \fi}
\end{verbatim}
重复调用 \cs{UCnoFootnote} 将去掉定界记号(\cs{expandafter} 先去掉了 \cs{fi}),
而且作为额外收获,这个宏对于多个作者的情形也是正确的。


%%\spoint Optional macro parameters with \cs{futurelet}
%\subsection{Optional macro parameters with \protect\cs{futurelet}}
%\spoint Optional macro parameters with \cs{futurelet}
\subsection{用 \protect\cs{futurelet} 实现可选的宏参数}

%One standard application of \cs{futurelet} is implementing
%\howto Macros with optional parameters\par
%optional parameters of macros. The general course of action
%is as follows:
%\begin{verbatim}
%\def\Com{\futurelet\testchar\MaybeOptArgCom}
%\def\MaybeOptArgCom{\ifx[\testchar \let\next\OptArgCom 
%                 \else \let\next\NoOptArgCom \fi \next}
%\def\OptArgCom[#1]#2{ ... }\def\NoOptArgCom#1{ ... }
%\end{verbatim}
%Note that \cs{ifx} is used even though it tests
%for a character. The reason is of course that,
%if the optional argument is omitted, there might be an
%expandable control sequence behind the~\cs{Com}.
\cs{futurelet} 的其中一个标准用法就是实现宏的可选参数。
\howto Macros with optional parameters\par
一般做法如下:
\begin{verbatim}
\def\Com{\futurelet\testchar\MaybeOptArgCom}
\def\MaybeOptArgCom{\ifx[\testchar \let\next\OptArgCom 
                 \else \let\next\NoOptArgCom \fi \next}
\def\OptArgCom[#1]#2{ ... }\def\NoOptArgCom#1{ ... }
\end{verbatim}
注意即便它测试的是个字符还是使用了 \cs{ifx}。原因当然是,
如果可选参量不存在,在 \cs{Com} 之后可能会是一个可展开的控制序列。

%The macro \cs{Com} now has one optional and one regular
%argument; it can be called as 
%\begin{verbatim}
%\Com{argument}
%\end{verbatim}
%or as
%\begin{verbatim}
%\Com[optional]{argument}
%\end{verbatim}
%Often the call without the optional argument will insert some
%default value:
%\begin{disp}\verb>\def\NoOptArgCom#1{\OptArgCom[>%
%{\italic default\/}\verb>]{#1}}>\end{disp}
%This mechanism is widely used in formats such as \LaTeX\ and
%\LamsTeX; see also~\cite{svb:future}.
现在宏 \cs{Com} 有一个可选参量以及一个常规参量;我们可以这样
\begin{verbatim}
\Com{argument}
\end{verbatim}
或者这样
\begin{verbatim}
\Com[optional]{argument}
\end{verbatim}
调用它。通常不带可选参量的调用将插入默认值:
\begin{disp}\verb>\def\NoOptArgCom#1{\OptArgCom[>%
{\italic default\/}\verb>]{#1}}>\end{disp}
这种机制在类似 \LaTeX\ 和 \LamsTeX\ 的格式中广泛用到;另外可以参考~\cite{svb:future}。



%%\spoint Two-step macros
%\subsection{Two-step macros}
%\spoint Two-step macros
\subsection{两步宏}

%Often what looks to the user like one macro is in reality
%a two-step process, where one macro will set up conditions,
%and a second macro will do the work.
用户经常会觉得某个宏事实上是一个两步的过程,其中第一个宏设定好条件,
而第二个宏将执行操作。

%As an example, here is
%a macro \cs{PickToEol}\label{pick:eol}
%\howto Take an input line as macro argument\par
%with an argument that is delimited by the line end.
%First we write a macro without arguments that 
%changes the category code of the line end, and then
%calls the second macro.
%\begin{verbatim}
%\def\PickToEol{\begingroup\catcode`\^^M=12 \xPickToEol}
%\end{verbatim}
%The second macro can then take as an argument everything
%up to the end of the line:
%\begin{verbatim}
%\def\xPickToEol#1^^M{ ... #1 ... \endgroup}
%\end{verbatim}
%There is one problem with this definition: the \verb>^^M> character
%should have category~12. We arrive at the following:
%\begin{verbatim}
%\def\PickToEol{\begingroup\catcode`\^^M=12 \xPickToEol}
%{\catcode`\^^M=12 %
% \gdef\xPickToEol#1^^M{ ... #1 ... \endgroup}%
%}
%\end{verbatim}
%where the category code of \verb>^^M> is changed for the
%sake of the definition of \cs{xPickToEol}. Note that
%the \verb>^^M> in \cs{PickToEol} occurs in a control symbol,
%so there the category code  is irrelevant. Therefore that
%definition can be outside the group where the category code 
%of \verb>^^M> is redefined.
作为例子,这里有个 \cs{PickToEol}\label{pick:eol} 宏,
\howto Take an input line as macro argument\par
它的参量以行尾符定界。首先我们写出第一个不带参量的宏,
它改变行尾符的类别码,然后调用第二个宏。
\begin{verbatim}
\def\PickToEol{\begingroup\catcode`\^^M=12 \xPickToEol}
\end{verbatim}
现在第二个宏将以直到行尾的所有内容作为它的参量:
\begin{verbatim}
\def\xPickToEol#1^^M{ ... #1 ... \endgroup}
\end{verbatim}
这个定义有个问题:\verb>^^M> 符的类别码必须为~12。我们改进一下:
\begin{verbatim}
\def\PickToEol{\begingroup\catcode`\^^M=12 \xPickToEol}
{\catcode`\^^M=12 %
 \gdef\xPickToEol#1^^M{ ... #1 ... \endgroup}%
}
\end{verbatim}
其中为 \cs{xPickToEol} 的定义改变了 \verb>^^M> 的类别码。
注意 \cs{PickToEol} 中的 \verb>^^M> 作为控制符号出现,
此时它的类别码是无关紧要的。因此那个定义可以出现在
重新定义 \verb>^^M> 的类别码的编组的外边。


%\subsection{A comment environment}
\subsection{注释环境}

%As an application of the above idea of two-step macros,
%\howto Comment environment\par
%and in order to illustrate tail recursion, here are 
%macros for a `comment' environment.
作为上述两步宏的想法的应用,
\howto Comment environment\par
也为了演示尾递归,这里是一个`注释'环境的宏。

%Often it is necessary to remove a part of \TeX\
%input temporarily. For this one would like to
%write
%\begin{verbatim}
%\comment
%...
%\endcomment
%\end{verbatim}
%The simplest implementation of this, 
%\begin{verbatim}
%\def\comment#1\endcomment{}
%\end{verbatim} 
%has a number of weaknesses. For instance,
%it cannot cope with outer macros or input that 
%does not have balanced braces. Its worst
%shortcoming, however, is that it reads the complete
%comment text as a macro argument. This limits the size
%of the comment to that of \TeX's input buffer.
我们经常需要临时删除部分 \TeX\ 输入。为此我们想这样写
\begin{verbatim}
\comment
...
\endcomment
\end{verbatim}
最简单的实现方式,
\begin{verbatim}
\def\comment#1\endcomment{}
\end{verbatim} 
有许多不足。比如,它无法妥善处理外部宏或者花括号不匹配的输入。
然而,最大的不足是它将整个注释文本作为宏参量读取。
这使得注释文本的长度不能超过 \TeX\ 的输入缓冲区的大小。

%It would be a better idea to take on the out-commented
%text one line at a time. For this we want to write
%a recursive macro with a basic structure
%\begin{verbatim}
%\def\comment#1^^M{ ... \comment }
%\end{verbatim}
%In order to be able to write this definition at all,
%the category code of the line end must be changed; as above
%\altt
%we will have
%\begin{verbatim}
%\def\comment{\begingroup \catcode`\^^M=12 \xcomment}
%{\catcode`\^^M=12 \endlinechar=-1 %
% \gdef\xcomment#1^^M{ ... \xcomment}
%}
%\end{verbatim}
%Changing the \cs{endlinechar} is merely to 
%prevent having to put comment characters at the end
%of every line of the definition.
将注释文本逐行取出会更好些。为此我们需要编写一个递归宏
\begin{verbatim}
\def\comment#1^^M{ ... \comment }
\end{verbatim}
为使这个定义能写出来,行尾符的类别码必须改变。和上面一样我们将有
\begin{verbatim}
\def\comment{\begingroup \catcode`\^^M=12 \xcomment}
{\catcode`\^^M=12 \endlinechar=-1 %
 \gdef\xcomment#1^^M{ ... \xcomment}
}
\end{verbatim}
改变 \cs{endlinechar} 只不过是为了避免在这个定义的每行末尾加上注释符。

%Of course, the process must stop at a certain time.
%To this purpose we investigate the line that was
%scooped up as macro argument:
%\begin{verbatim}
%{\catcode`\^^M=12 \endlinechar=-1 %
% \gdef\xcomment#1^^M{\def\test{#1}
%    \ifx\test\endcomment \let\next=\endgroup
%    \else \let\next=\xcomment \fi
%    \next}
%}
%\end{verbatim}
%and we have to define \cs{endcomment}:
%\begin{verbatim}
%\def\endcomment{\endcomment}
%\end{verbatim}
%This command will never be executed: it is merely for purposes
%of testing whether the end of the environment has been reached.
当然,这个过程必须在某个地方停止。为此我们考察已被取为宏参量的文本行:
\begin{verbatim}
{\catcode`\^^M=12 \endlinechar=-1 %
 \gdef\xcomment#1^^M{\def\test{#1}
    \ifx\test\endcomment \let\next=\endgroup
    \else \let\next=\xcomment \fi
    \next}
}
\end{verbatim}
而我们必须定义 \cs{endcomment} 如下:
\begin{verbatim}
\def\endcomment{\endcomment}
\end{verbatim}
这个命令将永远不会被执行;它只是用于测试是否已经到达环境的结尾处。

%We may want to comment out text that is not syntactically
%correct. Therefore we switch to a
%\indexterm{verbatim mode}
%when commenting. The following macro is given 
%in plain \TeX:
%\begin{verbatim}
%\def\dospecials{\do\ \do\\\do\{\do\}\do\$\do\&%
%  \do\#\do\^\do\^^K\do\_\do\^^A\do\%\do\~}
%\end{verbatim}
%We use it to define \cs{comment} as follows:
%\begin{verbatim}
%\def\makeinnocent#1{\catcode`#1=12 }
%\def\comment{\begingroup
%    \let\do=\makeinnocent \dospecials
%    \endlinechar`\^^M \catcode`\^^M=12 \xcomment}
%\end{verbatim}
%Apart from the possibility mentioned above of commenting
%out text that is not syntactically correct, for instance
%because of unmatched braces, this solution can handle
%outer macros. The former implementation of \cs{xcomment}
%would cause a \TeX\ error if one occurred in the comment text.
我们也许想注释掉语法不正确的文本。因此在注释时我们切换到\indexterm{抄录模式}。
下面的宏是 plain \TeX\ 给出的:
\begin{verbatim}
\def\dospecials{\do\ \do\\\do\{\do\}\do\$\do\&%
  \do\#\do\^\do\^^K\do\_\do\^^A\do\%\do\~}
\end{verbatim}
我们把它用到 \cs{comment} 的定义中:
\begin{verbatim}
\def\makeinnocent#1{\catcode`#1=12 }
\def\comment{\begingroup
    \let\do=\makeinnocent \dospecials
    \endlinechar`\^^M \catcode`\^^M=12 \xcomment}
\end{verbatim}
这种方法除了能注释掉语法错误比如花括号不匹配的文本外,还能处理外部宏。
之前的 \cs{xcomment} 实现在注释文本包含外部宏时将会产生 \TeX\ 错误。

%However, using verbatim mode poses the problem of concluding the 
%environment.
%\altt
%The final line of the comment is now not the control sequence
%\cs{endcomment}, but the characters constituting it. We have
%to test for these then:
%\begin{verbatim}
%{\escapechar=-1
% \xdef\endcomment{\string\\endcomment}
%}
%\end{verbatim}
%The sequence \verb>\string\\> gives a backslash.
%We could not have used
%\begin{verbatim}
%\edef\endcomment{\string\endcomment}
%\end{verbatim}
%because
%the letters of the word \n{endcomment} would then have
%category code~12, instead of the 11 that the ones on the
%last line of the comment will have.
然而,使用抄录模式给结束环境造成了问题。注释的最后一行现在不是控制序列
\cs{endcomment},而是由一串字符组成。这样我们就必须测试这个字符串:
\begin{verbatim}
{\escapechar=-1
 \xdef\endcomment{\string\\endcomment}
}
\end{verbatim}
代码 \verb>\string\\> 给出一个反斜线。我们不能这样写
\begin{verbatim}
\edef\endcomment{\string\endcomment}
\end{verbatim}
因为这样写单词 \n{endcomment} 各字母的类别码将为~12,
而与它们在注释最后一行中的类别码~11~不相等。

%\endofchapter
%%%%% end of input file [macro]
\endofchapter
%%%% end of input file [macro]

\end{document}
