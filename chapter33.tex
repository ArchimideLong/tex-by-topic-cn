% -*- coding: utf-8 -*-

%\chapter{\TeX\ and the Outside World}\label{TeXcomm}
\chapter{\TeX\ and the Outside World}\label{TeXcomm}

%This
%chapter treats those commands that bear relevance to
%\n{dvi} files and formats. It gives some global information
%about \IniTeX, font and format files,
%Computer Modern typefaces, and \web.
This
chapter treats those commands that bear relevance to
\n{dvi} files and formats. It gives some global information
about \IniTeX, font and format files,
Computer Modern typefaces, and \web.

%\label{cschap:dump}\label{cschap:special}\label{cschap:mag}\label{cschap:year}\label{cschap:month}\label{cschap:day}\label{cschap:time}\label{cschap:fmtname}\label{cschap:fmtversion}
%\begin{inventory}
%\item [\cs{dump}] 
%      Dump a format file; possible only in \IniTeX, 
%      not allowed inside a group.
\label{cschap:dump}\label{cschap:special}\label{cschap:mag}\label{cschap:year}\label{cschap:month}\label{cschap:day}\label{cschap:time}\label{cschap:fmtname}\label{cschap:fmtversion}
\begin{inventory}
\item [\cs{dump}] 
      Dump a format file; possible only in \IniTeX, 
      not allowed inside a group.

%\item [\cs{special}] 
%      Write a \gr{balanced text} to the \n{dvi} file.
\item [\cs{special}] 
      Write a \gr{balanced text} to the \n{dvi} file.

%\item [\cs{mag}] 
%      1000 times the magnification of the document.
\item [\cs{mag}] 
      1000 times the magnification of the document.

%\item [\cs{year}] 
%      The year of the current job.
\item [\cs{year}] 
      The year of the current job.

%\item [\cs{month}] 
%      The month of the current job.
\item [\cs{month}] 
      The month of the current job.

%\item [\cs{day}] 
%      The day of the current job.
\item [\cs{day}] 
      The day of the current job.

%\item [\cs{time}]
%      Number of minutes after midnight that the current job started.
\item [\cs{time}]
      Number of minutes after midnight that the current job started.

%\item [\cs{fmtname}] 
%      Macro containing the name of the format dumped.
\item [\cs{fmtname}] 
      Macro containing the name of the format dumped.

%\item [\cs{fmtversion}] 
%      Macro containing the version of the format dumped.
\item [\cs{fmtversion}] 
      Macro containing the version of the format dumped.

%\end{inventory}
\end{inventory}


%%\point \TeX, \IniTeX, \VirTeX
%\section{\TeX, \IniTeX, \VirTeX}
%\point \TeX, \IniTeX, \VirTeX
\section{\TeX, \IniTeX, \VirTeX}

%In the terminology established in {\italic \TeX: the Program},
%\cite{Knuth:TeXprogram},
%\thecstoidx TeX\par\thecstoidx IniTeX\par\thecstoidx VirTeX\par
%\TeX\ programs come in three flavours.
%\IniTeX\ is a version of \TeX\ that can generate formats;
%\VirTeX\ is a production version without preloaded format,
%and \TeX\ is a production version with 
%preloaded (plain) format. Unfortunately, this terminology is
%not adhered to in general. A~lot of systems do not use preloaded
%formats (the procedure for making them may be impossible on
%some operating systems),
%and call the `virgin \TeX' simply \TeX.
%This manual also follows that convention.
In the terminology established in {\italic \TeX: the Program},
\cite{Knuth:TeXprogram},
\thecstoidx TeX\par\thecstoidx IniTeX\par\thecstoidx VirTeX\par
\TeX\ programs come in three flavours.
\IniTeX\ is a version of \TeX\ that can generate formats;
\VirTeX\ is a production version without preloaded format,
and \TeX\ is a production version with 
preloaded (plain) format. Unfortunately, this terminology is
not adhered to in general. A~lot of systems do not use preloaded
formats (the procedure for making them may be impossible on
some operating systems),
and call the `virgin \TeX' simply \TeX.
This manual also follows that convention.

%%\spoint Formats: loading
%\subsection{Formats: loading}
%\spoint Formats: loading
\subsection{Formats: loading}

%A \indexterm{format file}
%(usually with extension~\n{.fmt})
%is a compact dump of \TeX's internal structures.
%Loading a format file takes a considerably shorter time than
%would be needed for
%loading the font information and the macros that
%constitute the format.
A \indexterm{format file}
(usually with extension~\n{.fmt})
is a compact dump of \TeX's internal structures.
Loading a format file takes a considerably shorter time than
would be needed for
loading the font information and the macros that
constitute the format.

%Both \TeX\ and \IniTeX\ can load a format; the user specifies
%this by putting the name on the command line 
%\begin{verbatim}
%% tex &plain 
%\end{verbatim}
%or at the \n{**} prompt
%\begin{verbatim}
%% tex
%This is TeX. Version ....
%** &plain 
%\end{verbatim}
%preceded by an ampersand (for UNIX, this should be \verb>\&> on
%the command line). An input file name can follow the
%format name in both places.
Both \TeX\ and \IniTeX\ can load a format; the user specifies
this by putting the name on the command line 
\begin{verbatim}
% tex &plain 
\end{verbatim}
or at the \n{**} prompt
\begin{verbatim}
% tex
This is TeX. Version ....
** &plain 
\end{verbatim}
preceded by an ampersand (for UNIX, this should be \verb>\&> on
the command line). An input file name can follow the
format name in both places.

%\IniTeX\ does not need a format,
%but if no format is specified for (Vir)\TeX, it will try to
%load the plain format, and halt if that cannot be found.
\IniTeX\ does not need a format,
but if no format is specified for (Vir)\TeX, it will try to
load the plain format, and halt if that cannot be found.

%%\spoint Formats: dumping
%\subsection{Formats: dumping}
%\spoint Formats: dumping
\subsection{Formats: dumping}

%\IniTeX\ is the only version of \TeX\ 
%that can dump a format, since it is
%the only version of \TeX\ that has
%the command~\csidx{dump},
%which causes the internal structures
%to be dumped as a format.
%It is also the only version of \TeX\ that has the command
%\cs{patterns}, which
%is needed to specify a list of hyphenation
%patterns.
\IniTeX\ is the only version of \TeX\ 
that can dump a format, since it is
the only version of \TeX\ that has
the command~\csidx{dump},
which causes the internal structures
to be dumped as a format.
It is also the only version of \TeX\ that has the command
\cs{patterns}, which
is needed to specify a list of hyphenation
patterns.

%Dumping is not allowed inside a group, that is
%\begin{verbatim}
%{ ... \dump }
%\end{verbatim}
%is not allowed. This restriction
%prevents difficulties with \TeX's save stack.
%After the \cs{dump} command \TeX\ gives an elaborate listing of
%its internal state, and of the font names associated with
%fonts that have been loaded and ends the job.
Dumping is not allowed inside a group, that is
\begin{verbatim}
{ ... \dump }
\end{verbatim}
is not allowed. This restriction
prevents difficulties with \TeX's save stack.
After the \cs{dump} command \TeX\ gives an elaborate listing of
its internal state, and of the font names associated with
fonts that have been loaded and ends the job.

%An interesting possibility arises from the fact that
%\IniTeX\ can both load and dump a format.
%Suppose you have written a set of macros that build
%on top of plain \TeX, \n{superplain.tex}.
%You could then call
%\begin{verbatim}
%% initex &plain superplain
%*\dump
%\end{verbatim}
%and get a format file \n{superplain.fmt} that
%has all of plain, and all of your macros.
An interesting possibility arises from the fact that
\IniTeX\ can both load and dump a format.
Suppose you have written a set of macros that build
on top of plain \TeX, \n{superplain.tex}.
You could then call
\begin{verbatim}
% initex &plain superplain
*\dump
\end{verbatim}
and get a format file \n{superplain.fmt} that
has all of plain, and all of your macros.

%%\spoint Formats: preloading
%\subsection{Formats: preloading}
%\spoint Formats: preloading
\subsection{Formats: preloading}

%On some systems it is possible to interrupt a running program,
%and save its `core image' such that this can be started as
%an independent program. 
%The executable made from the
%core image of a \TeX\ program interrupted after it has loaded 
%a format is called a \TeX\ program with preloaded format.
%The idea behind preloaded formats is
%that interrupting \TeX\ after it has loaded a format, and making
%this program available to the user, 
%saves in each run the time for loading the format.
%In the good old days when computers were quite a bit slower
%this procedure made sense.
%Nowadays, it does not seem so necessary.
%Besides, dumping a core image may not always be possible.
On some systems it is possible to interrupt a running program,
and save its `core image' such that this can be started as
an independent program. 
The executable made from the
core image of a \TeX\ program interrupted after it has loaded 
a format is called a \TeX\ program with preloaded format.
The idea behind preloaded formats is
that interrupting \TeX\ after it has loaded a format, and making
this program available to the user, 
saves in each run the time for loading the format.
In the good old days when computers were quite a bit slower
this procedure made sense.
Nowadays, it does not seem so necessary.
Besides, dumping a core image may not always be possible.

%%\spoint The knowledge of \IniTeX
%\subsection{The knowledge of \IniTeX}
%\spoint The knowledge of \IniTeX
\subsection{The knowledge of \IniTeX}

%If no format has been loaded, \IniTeX\ knows very little.
%For instance, it has no open/close group characters.
%However, it can not be completely devoid of knowledge
%lest there be no way to define anything.
If no format has been loaded, \IniTeX\ knows very little.
For instance, it has no open/close group characters.
However, it can not be completely devoid of knowledge
lest there be no way to define anything.

%Here is the extent of its knowledge.
%\begin{itemize} \mathsurround=1.5pt
%%\flushright:no
%\item \verb>\catcode`\\=0>, \verb>\escapechar=`\\>
%      (see page~\pageref{ini:esc}).
%\item \verb>\catcode`\^^M=5>, \verb>\endlinechar=`\^^M>
%      (see page~\pageref{ini:eol}).
%\item \verb>\catcode`\ =10>
%      (see page~\pageref{ini:sp}).
%\item \verb>\catcode`\%=14>
%      (see page~\pageref{ini:comm}).
%\item \verb>\catcode`\^^?=15>
%      (see page~\pageref{ini:invalid}).
%\item \cs{catcode}$x$\n{=11} for $x={}$\n{`a..`z,`A..`Z}
%      (see page~\pageref{ini:let}).
%\item \cs{catcode}$x$\n{=12} for all other character codes\nl
%      (see page~\pageref{ini:other}).
%\item \cs{sfcode}$x$=\n{999} for $x={}$\n{`A..`Z},
%      \cs{sfcode}$x$\n{=1000} for all other characters
%      (see page~\pageref{ini:sf}).
%\item \verb>\lccode`a..`z,`A..`Z=`a..`z>, \verb>\uccode`a..`z,`A..`Z=`A..`Z>,
%      \cs{lccode}$x$\n{=0}, \cs{uccode}$x$\n{=0} for all other characters
%      (see page~\pageref{ini:uclc}).
%\item \verb>\delcode`.=0>, \cs{delcode}$x$\n{=-1} for all other characters
%      (see page~\pageref{ini:del}).
%\item \cs{mathcode}$x$\n{="!7100}${}+x$ for all lowercase and uppercase
%      letters, \cs{mathcode}$x$\n{="!7000}${}+x$ for all digits,
%      \cs{mathcode}$x$\n=$x$ for all other characters
%      (see page~\pageref{ini:fam}).
%\item \cs{tolerance=10000}, \cs{mag=1000},
%      \cs{maxdeadcycles=25}.
%\end{itemize}
Here is the extent of its knowledge.
\begin{itemize} \mathsurround=1.5pt
%\flushright:no
\item \verb>\catcode`\\=0>, \verb>\escapechar=`\\>
      (see page~\pageref{ini:esc}).
\item \verb>\catcode`\^^M=5>, \verb>\endlinechar=`\^^M>
      (see page~\pageref{ini:eol}).
\item \verb>\catcode`\ =10>
      (see page~\pageref{ini:sp}).
\item \verb>\catcode`\%=14>
      (see page~\pageref{ini:comm}).
\item \verb>\catcode`\^^?=15>
      (see page~\pageref{ini:invalid}).
\item \cs{catcode}$x$\n{=11} for $x={}$\n{`a..`z,`A..`Z}
      (see page~\pageref{ini:let}).
\item \cs{catcode}$x$\n{=12} for all other character codes\nl
      (see page~\pageref{ini:other}).
\item \cs{sfcode}$x$=\n{999} for $x={}$\n{`A..`Z},
      \cs{sfcode}$x$\n{=1000} for all other characters
      (see page~\pageref{ini:sf}).
\item \verb>\lccode`a..`z,`A..`Z=`a..`z>, \verb>\uccode`a..`z,`A..`Z=`A..`Z>,
      \cs{lccode}$x$\n{=0}, \cs{uccode}$x$\n{=0} for all other characters
      (see page~\pageref{ini:uclc}).
\item \verb>\delcode`.=0>, \cs{delcode}$x$\n{=-1} for all other characters
      (see page~\pageref{ini:del}).
\item \cs{mathcode}$x$\n{="!7100}${}+x$ for all lowercase and uppercase
      letters, \cs{mathcode}$x$\n{="!7000}${}+x$ for all digits,
      \cs{mathcode}$x$\n=$x$ for all other characters
      (see page~\pageref{ini:fam}).
\item \cs{tolerance=10000}, \cs{mag=1000},
      \cs{maxdeadcycles=25}.
\end{itemize}

%%\spoint Memory sizes of \TeX\ and \IniTeX
%\subsection{Memory sizes of \TeX\ and \IniTeX}
%\spoint Memory sizes of \TeX\ and \IniTeX
\subsection{Memory sizes of \TeX\ and \IniTeX}

%The main memory size of \TeX\ and \IniTeX\ is controlled by
%four constants in the source code:
%\n{mem\_bot}, \n{mem\_top}, \n{mem\_min}, and~\n{mem\_max}.
%For Ini\TeX's memory \n{mem\_bot${}={}$mem\_min} 
%and \n{mem\_top${}={}$mem\_max};
%for \TeX\ \n{mem\_bot} and \n{mem\_top} record the main memory
%size of the Ini\TeX\ used to dump the format.
%Thus versions of \TeX\ and \IniTeX\ have to be adapted
%to each other in this    respect.
The main memory size of \TeX\ and \IniTeX\ is controlled by
four constants in the source code:
\n{mem\_bot}, \n{mem\_top}, \n{mem\_min}, and~\n{mem\_max}.
For Ini\TeX's memory \n{mem\_bot${}={}$mem\_min} 
and \n{mem\_top${}={}$mem\_max};
for \TeX\ \n{mem\_bot} and \n{mem\_top} record the main memory
size of the Ini\TeX\ used to dump the format.
Thus versions of \TeX\ and \IniTeX\ have to be adapted
to each other in this    respect.

%\TeX's own main memory can be bigger than that of the
%corresponding \IniTeX: in general
%\n{mem\_min${}\leq{}$mem\_bot} and \n{mem\_top${}\leq{}$mem\_max}.
\TeX's own main memory can be bigger than that of the
corresponding \IniTeX: in general
\n{mem\_min${}\leq{}$mem\_bot} and \n{mem\_top${}\leq{}$mem\_max}.

%For \IniTeX\ a smaller main memory can suffice, 
%as this program is typically
%not meant to do real typesetting. 
%There may even be a real need for the main memory
%to be smaller, because \IniTeX\ needs a lot of auxiliary
%storage for initialization and for building the
%hyphenation table.
For \IniTeX\ a smaller main memory can suffice, 
as this program is typically
not meant to do real typesetting. 
There may even be a real need for the main memory
to be smaller, because \IniTeX\ needs a lot of auxiliary
storage for initialization and for building the
hyphenation table.


%%\point More about formats
%\section{More about formats}
%\point More about formats
\section{More about formats}

%%\spoint Compatibility
%\subsection{Compatibility}
%\spoint Compatibility
\subsection{Compatibility}

%\TeX\ has a curious error message: `Fatal format error: I'm stymied',
%which is given if \TeX\ tries to load a format that was made
%with an incompatible version of \IniTeX. See the point
%above about memory sizes, and Chapter~\ref{error} for
%the hash size (parameters \n{hash\_size} and \n{hash\_prime})
%and the hyphenation exception dictionary (parameter \n{hyph\_size}).
\TeX\ has a curious error message: `Fatal format error: I'm stymied',
which is given if \TeX\ tries to load a format that was made
with an incompatible version of \IniTeX. See the point
above about memory sizes, and Chapter~\ref{error} for
the hash size (parameters \n{hash\_size} and \n{hash\_prime})
and the hyphenation exception dictionary (parameter \n{hyph\_size}).

%%\spoint Preloaded fonts
%\subsection{Preloaded fonts}
%\spoint Preloaded fonts
\subsection{Preloaded fonts}

%During a run of \TeX\ the only information needed about fonts
%is the data that is found in the \n{tfm} files (see below).
%Since a run of \TeX, especially if the input contains math material,
%can easily access 30--40 fonts, the disk access for
%all the \n{tfm} files can become significant.
%Therefore the plain format and \LaTeX\ load these 
%metrics files in \IniTeX. A~\TeX\ version using such a format
%does not need to load any \n{tfm} files.
During a run of \TeX\ the only information needed about fonts
is the data that is found in the \n{tfm} files (see below).
Since a run of \TeX, especially if the input contains math material,
can easily access 30--40 fonts, the disk access for
all the \n{tfm} files can become significant.
Therefore the plain format and \LaTeX\ load these 
metrics files in \IniTeX. A~\TeX\ version using such a format
does not need to load any \n{tfm} files.

%On the other hand, if a format has the possibility of accessing
%a range of typefaces, it may be advantageous to have metrics
%files loaded on demand during the actual run of \TeX.
On the other hand, if a format has the possibility of accessing
a range of typefaces, it may be advantageous to have metrics
files loaded on demand during the actual run of \TeX.

%%\spoint The plain format
%\subsection{The plain format}
%\spoint The plain format
\subsection{The plain format}

%The first format written for \TeX, and the basis for all
%later ones,
%is the plain format, described in \TeXbook.
%It is a mixture of 
%\begin{itemize}
%\item definitions and macros one simply cannot live without
%such as the initial \cs{catcode} assignments, 
%all of the math delimiter definitions,
%and the \cs{new...} macros;
%\item constructs that are useful, but for which \LaTeX\ 
%and other packages use
%a different implementation, such as the tabbing environment; and
%\item some macros that are insufficient for any but the
%simplest applications: \cs{item} and \cs{beginsection}
%are in this category.
%\end{itemize}
The first format written for \TeX, and the basis for all
later ones,
is the plain format, described in \TeXbook.
It is a mixture of 
\begin{itemize}
\item definitions and macros one simply cannot live without
such as the initial \cs{catcode} assignments, 
all of the math delimiter definitions,
and the \cs{new...} macros;
\item constructs that are useful, but for which \LaTeX\ 
and other packages use
a different implementation, such as the tabbing environment; and
\item some macros that are insufficient for any but the
simplest applications: \cs{item} and \cs{beginsection}
are in this category.
\end{itemize}

%It is the first category which Knuth meant to serve as a
%foundation for future macro packages, so that they
%can live peacefully together (see  Chapter~\ref{alloc}).
%This idea is reflected in the fact that the name `plain'
%is not capitalized: it is the basic set of macros.
It is the first category which Knuth meant to serve as a
foundation for future macro packages, so that they
can live peacefully together (see  Chapter~\ref{alloc}).
This idea is reflected in the fact that the name `plain'
is not capitalized: it is the basic set of macros.

%%\spoint The \LaTeX\ format
%\subsection{The \LaTeX\ format}
%\spoint The \LaTeX\ format
\subsection{The \LaTeX\ format}

%The \LaTeX\ format\thecstoidx LaTeX\par,
%written by  Leslie Lamport of Digital Equipment Corporation
%and described in~\cite{Lamport:LaTeX},
%was released around 1985.
%The \LaTeX\ format, using its own version
%of \n{plain.tex} (called \n{lplain.tex}),
%is not compatible with plain \TeX;
%a~number of plain macros are not available. Still, it contains
%large parts of the plain format (even when they overlap with
%its own constructs).
The \LaTeX\ format\thecstoidx LaTeX\par,
written by  Leslie Lamport of Digital Equipment Corporation
and described in~\cite{Lamport:LaTeX},
was released around 1985.
The \LaTeX\ format, using its own version
of \n{plain.tex} (called \n{lplain.tex}),
is not compatible with plain \TeX;
a~number of plain macros are not available. Still, it contains
large parts of the plain format (even when they overlap with
its own constructs).

%\LaTeX\ is a powerful format with facilities such as
%marginal notes, floating objects, cross referencing,
%and automatic table of contents generation.
%Its main drawback is that the `style files' which
%define the actual layout are quite hard to write
%(although \LaTeX\ is in the process of a major revision,
%in which this problem will be tackled;
%see \cite{Frank} and~\cite{Frank2}).
%As a result,
%people have had at their disposal mostly the styles
%written by Leslie Lamport, the layout of which is
%rather idiosyncratic. See~\cite{BEP} for a successful
%attempt to replace these styles.
\LaTeX\ is a powerful format with facilities such as
marginal notes, floating objects, cross referencing,
and automatic table of contents generation.
Its main drawback is that the `style files' which
define the actual layout are quite hard to write
(although \LaTeX\ is in the process of a major revision,
in which this problem will be tackled;
see \cite{Frank} and~\cite{Frank2}).
As a result,
people have had at their disposal mostly the styles
written by Leslie Lamport, the layout of which is
rather idiosyncratic. See~\cite{BEP} for a successful
attempt to replace these styles.

%%\spoint Mathematical formats
%\subsection{Mathematical formats}
%\spoint Mathematical formats
\subsection{Mathematical formats}

%There are two formats with extensive facilities for
%mathematics typesetting:
%\AmsTeX~\cite{Ams}
%(which originated at the American Mathematical Society)
%and \LamsTeX~\cite{Lams}.
%The first of these includes more facilities than plain \TeX\
%or \LaTeX\ for typesetting mathematics, but it lacks
%features such as automatic numbering and cross-referencing,
%available in \LaTeX, for instance. \LamsTeX, then, is the
%synthesis of \AmsTeX\ and \LaTeX. Also it includes
%still more features for mathematics, such as complicated
%tables and commutative diagrams.
There are two formats with extensive facilities for
mathematics typesetting:
\AmsTeX~\cite{Ams}
(which originated at the American Mathematical Society)
and \LamsTeX~\cite{Lams}.
The first of these includes more facilities than plain \TeX\
or \LaTeX\ for typesetting mathematics, but it lacks
features such as automatic numbering and cross-referencing,
available in \LaTeX, for instance. \LamsTeX, then, is the
synthesis of \AmsTeX\ and \LaTeX. Also it includes
still more features for mathematics, such as complicated
tables and commutative diagrams.

%%\spoint Other formats
%\subsection{Other formats}
%\spoint Other formats
\subsection{Other formats}

%Other formats than the above exist:
%for instance, \n{Phyzzx}~\cite{Phyzzx}, \n{TeXsis}~\cite{TeXsis}, 
%Macro \TeX~\cite{Amy}, \n{eplain}~\cite{Berry},
%and \n{\TeX T1}~\cite{TeXT1}.
%Typically, such formats provide the facilities of \LaTeX, but
%try to be more easily adaptable by the user.
%Also, in general they
%have been written with the intention of being an
%add-on product to the plain format.
Other formats than the above exist:
for instance, \n{Phyzzx}~\cite{Phyzzx}, \n{TeXsis}~\cite{TeXsis}, 
Macro \TeX~\cite{Amy}, \n{eplain}~\cite{Berry},
and \n{\TeX T1}~\cite{TeXT1}.
Typically, such formats provide the facilities of \LaTeX, but
try to be more easily adaptable by the user.
Also, in general they
have been written with the intention of being an
add-on product to the plain format.

%This book was, in its incarnation published by Addison-Wesley,
%also written in an `other format':
%the \indexterm{Lollipop}  format.
%This format does not contain user macros, but the
%tools with which a style designer can program them; see~\cite{EL}.
%The current version of this book is written in \LaTeX.
This book was, in its incarnation published by Addison-Wesley,
also written in an `other format':
the \indexterm{Lollipop}  format.
This format does not contain user macros, but the
tools with which a style designer can program them; see~\cite{EL}.
The current version of this book is written in \LaTeX.

%%\point The \n{dvi} file
%\section{The \n{dvi} file}
%\point The \n{dvi} file
\section{The \n{dvi} file}

%The \n{dvi} file (this term stands for `device independent')
%\term \n{dvi} file\par
%contains the output of a \TeX\ run: it
%contains compactly dumped representations of boxes that
%\mdqon
%have been sent there by \cs{shipout}""\gr{box}. The act
%\mdqoff
%of shipping out usually occurs inside the output routine,
%but this is not necessarily so.
The \n{dvi} file (this term stands for `device independent')
\term \n{dvi} file\par
contains the output of a \TeX\ run: it
contains compactly dumped representations of boxes that
\mdqon
have been sent there by \cs{shipout}""\gr{box}. The act
\mdqoff
of shipping out usually occurs inside the output routine,
but this is not necessarily so.

%%\spoint The \n{dvi} file format
%\subsection{The \n{dvi} file format}
%\spoint The \n{dvi} file format
\subsection{The \n{dvi} file format}

%A \n{dvi} file is a byte-oriented file,
%consisting of a preamble, a postamble,
%and a list of pages.
A \n{dvi} file is a byte-oriented file,
consisting of a preamble, a postamble,
and a list of pages.

%Access for subsequent software to a completed \n{dvi} file
%is strictly sequential in nature:
%the pages are stored as a backwards linked list. This
%means that only two ways of accessing are possible:
%\begin{itemize} \item given the start of a page, the next can be
%found by reading until an end-of-page code is encountered, and
%\item starting at the end of the file pages can be read
%backwards at higher speed, as each beginning-of-page code
%contains the byte position of the previous one.\end{itemize}
Access for subsequent software to a completed \n{dvi} file
is strictly sequential in nature:
the pages are stored as a backwards linked list. This
means that only two ways of accessing are possible:
\begin{itemize} \item given the start of a page, the next can be
found by reading until an end-of-page code is encountered, and
\item starting at the end of the file pages can be read
backwards at higher speed, as each beginning-of-page code
contains the byte position of the previous one.\end{itemize}

%The preamble and postamble contain
%\begin{itemize}\item the magnification of the document (see below),
%\item the unit of measurement used for the document, and
%\item possibly a comment string.\end{itemize}
%The postamble contains in addition a list of the font definitions
%that appear on the pages of the file.
The preamble and postamble contain
\begin{itemize}\item the magnification of the document (see below),
\item the unit of measurement used for the document, and
\item possibly a comment string.\end{itemize}
The postamble contains in addition a list of the font definitions
that appear on the pages of the file.

%Neither the preamble nor the postamble 
%of the file contains a
%table of byte positions of pages.
%The full definition of the \n{dvi} file format can be found
%in~\cite{Knuth:TeXprogram}.
Neither the preamble nor the postamble 
of the file contains a
table of byte positions of pages.
The full definition of the \n{dvi} file format can be found
in~\cite{Knuth:TeXprogram}.

%%\spoint Page identification
%\subsection{Page identification}
%\spoint Page identification
\subsection{Page identification}

%Whenever a \cs{shipout} occurs, \TeX\ also writes the
%values of counters  0--9 to the \n{dvi} file and the terminal.
%Ordinarily, only counter~0, the page number, is used, and the
%other counters are zero. Those zeros are not output to the
%terminal. The other counters can be used to indicate further
%structure in the document. Log output shows the non-zero
%counters and the zero counters in between.
Whenever a \cs{shipout} occurs, \TeX\ also writes the
values of counters  0--9 to the \n{dvi} file and the terminal.
Ordinarily, only counter~0, the page number, is used, and the
other counters are zero. Those zeros are not output to the
terminal. The other counters can be used to indicate further
structure in the document. Log output shows the non-zero
counters and the zero counters in between.


%%\spoint Magnification 
%\subsection{Magnification }
%\spoint Magnification 
\subsection{Magnification }

%The \indexterm{magnification}
%of a document can be indicated by the \gr{integer
%parameter} 
%\cstoidx mag\par
%\cs{mag}, which specifies 1000 times the magnification
%ratio.
The \indexterm{magnification}
of a document can be indicated by the \gr{integer
parameter} 
\cstoidx mag\par
\cs{mag}, which specifies 1000 times the magnification
ratio.

%The \n{dvi} file contains the value of \cs{mag} for the
%document in its preamble and postamble. 
%If no {\tt true} dimensions are used
%the \n{dvi} file will look the same as when no magnification
%would have been used, except for the \cs{mag} value in the
%preamble and the postamble.
The \n{dvi} file contains the value of \cs{mag} for the
document in its preamble and postamble. 
If no {\tt true} dimensions are used
the \n{dvi} file will look the same as when no magnification
would have been used, except for the \cs{mag} value in the
preamble and the postamble.

%Whenever a {\tt true} dimension is used it is divided
%by the value of \cs{mag}, so that the final output will have
%the dimension as prescribed by the user.
%The \cs{mag} parameter cannot be changed after a
%\n{true} dimension has been used, or after the first
%page has been shipped to the \n{dvi} file.
Whenever a {\tt true} dimension is used it is divided
by the value of \cs{mag}, so that the final output will have
the dimension as prescribed by the user.
The \cs{mag} parameter cannot be changed after a
\n{true} dimension has been used, or after the first
page has been shipped to the \n{dvi} file.

%Plain \TeX\ has the \csidx{magnification} macro for
%globally sizing the document, without changing
% the physical size of the page:
%\begin{verbatim}
%\def\magnification{\afterassignment\m@g\count@}
%\def\m@g{\mag\count@
%  \hsize6.5truein\vsize8.9truein\dimen\footins8truein}
%\end{verbatim}
%The explanation for this is
%as follows: the command \cs{m@g} is saved with an \cs{afterassignment}
%command, and the magnification value (which is 1000 times the
%actual magnification factor) is assigned to \cs{count@}.
%After this assignment, the macro \cs{m@g} assigns
%the magnification value to \cs{mag}, and the horizontal
%and vertical size are reset to 
%their original values {\tt 6.5truein} and {\tt 8.9truein}.
%The \cs{footins} is also reset.
Plain \TeX\ has the \csidx{magnification} macro for
globally sizing the document, without changing
 the physical size of the page:
\begin{verbatim}
\def\magnification{\afterassignment\m@g\count@}
\def\m@g{\mag\count@
  \hsize6.5truein\vsize8.9truein\dimen\footins8truein}
\end{verbatim}
The explanation for this is
as follows: the command \cs{m@g} is saved with an \cs{afterassignment}
command, and the magnification value (which is 1000 times the
actual magnification factor) is assigned to \cs{count@}.
After this assignment, the macro \cs{m@g} assigns
the magnification value to \cs{mag}, and the horizontal
and vertical size are reset to 
their original values {\tt 6.5truein} and {\tt 8.9truein}.
The \cs{footins} is also reset.

%%\point[special] Specials
%\section{Specials}
%\label{special}
%\point[special] Specials
\section{Specials}
\label{special}

%\mdqon
%\TeX\ is to a large degree machine"-independent, but it still needs
%\mdqoff
%a hook for machine-dependent extensions. This is 
%done through \indexterm{specials}. The
%\csidx{special} command writes a \gr{balanced text}
%to the \n{dvi} file which \TeX\ does not interpret like other token lists:
%it assumes that the printer driver knows what to do with it.
%The \cs{special} command is not supposed to change the
%$x$ and $y$ position on the page, so that the implementation
%of \TeX\ remains independent of the actual 
%\indexterm{device driver}
%that handles the \cs{special}.
\mdqon
\TeX\ is to a large degree machine"-independent, but it still needs
\mdqoff
a hook for machine-dependent extensions. This is 
done through \indexterm{specials}. The
\csidx{special} command writes a \gr{balanced text}
to the \n{dvi} file which \TeX\ does not interpret like other token lists:
it assumes that the printer driver knows what to do with it.
The \cs{special} command is not supposed to change the
$x$ and $y$ position on the page, so that the implementation
of \TeX\ remains independent of the actual 
\indexterm{device driver}
that handles the \cs{special}.

%The most popular application of specials is probably the
%inclusion of graphic material, written in some
%page description language, such as \indexterm{PostScript}.
%The size of the graphics can usually be determined from
%the file containing it (in the case of encapsulated
%PostScript through
%the `bounding box' data), so \TeX\ can leave space for
%such material.
The most popular application of specials is probably the
inclusion of graphic material, written in some
page description language, such as \indexterm{PostScript}.
The size of the graphics can usually be determined from
the file containing it (in the case of encapsulated
PostScript through
the `bounding box' data), so \TeX\ can leave space for
such material.

%%\point Time
%\section{Time}
%\point Time
\section{Time}

%\TeX\ has four parameters, \csidx{year}, \csidx{month}, \csidx{day}, and
%\csidx{time}, that tell
%the \indexterm{time} and \indexterm{date}
%when the current job started.
%After this, the parameters are not updated.
%The user can change them without this having any effect.
\TeX\ has four parameters, \csidx{year}, \csidx{month}, \csidx{day}, and
\csidx{time}, that tell
the \indexterm{time} and \indexterm{date}
when the current job started.
After this, the parameters are not updated.
The user can change them without this having any effect.

%All four parameters are integers; the \cs{time} parameter
%gives the number of minutes since midnight that the current
%job started.
All four parameters are integers; the \cs{time} parameter
gives the number of minutes since midnight that the current
job started.

%%\point Fonts
%\section{Fonts}
%\point Fonts
\section{Fonts}

%Font information is split in the \TeX\ system into
%the metric information (how high, wide, and deep is a character),
%and the actual description of the characters in a font.
%\TeX, the formatter, needs only the metric information;
%printer drivers and screen previewers need the character
%descriptions. With this approach it is for instance possible
%for \TeX\ to use with relative ease the resident fonts of
%a printer.
Font information is split in the \TeX\ system into
the metric information (how high, wide, and deep is a character),
and the actual description of the characters in a font.
\TeX, the formatter, needs only the metric information;
printer drivers and screen previewers need the character
descriptions. With this approach it is for instance possible
for \TeX\ to use with relative ease the resident fonts of
a printer.

%%\spoint Font metrics
%\subsection{Font metrics}
%\spoint Font metrics
\subsection{Font metrics}

%The metric information of \TeX's fonts is stored in \n{tfm}
%files, which stands for `\TeX\ \indexterm{font metrics}' files.
%Metrics files contain the following information
%(see \cite{Knuth:TeXprogram} for the full definition):
%\begin{itemize}\item the design size of a font;
%\item the values for the \cs{fontdimen} parameters
%(see Chapter~\ref{font});
%\item the height, depth, width, and italic correction
%     of individual characters;
%\item kerning tables;
%\item ligature tables;
%\item information regarding successors and extensions
%of math characters (see Chapter~\ref{mathchar}).
%\end{itemize}
%Metrics files use a packed format, but they can be converted
%to and from a readable format by the auxiliary programs
%\n{tftopl} and \n{pltotf} (see~\cite{K:Fuchs}). 
%Here \n{pl} stands for `property list',
%a term deriving from the programming language Lisp.
%Files in \n{pl} format are just text, so they can easily be edited;
%after conversion
%they can then again be used as \n{tfm} files.
The metric information of \TeX's fonts is stored in \n{tfm}
files, which stands for `\TeX\ \indexterm{font metrics}' files.
Metrics files contain the following information
(see \cite{Knuth:TeXprogram} for the full definition):
\begin{itemize}\item the design size of a font;
\item the values for the \cs{fontdimen} parameters
(see Chapter~\ref{font});
\item the height, depth, width, and italic correction
     of individual characters;
\item kerning tables;
\item ligature tables;
\item information regarding successors and extensions
of math characters (see Chapter~\ref{mathchar}).
\end{itemize}
Metrics files use a packed format, but they can be converted
to and from a readable format by the auxiliary programs
\n{tftopl} and \n{pltotf} (see~\cite{K:Fuchs}). 
Here \n{pl} stands for `property list',
a term deriving from the programming language Lisp.
Files in \n{pl} format are just text, so they can easily be edited;
after conversion
they can then again be used as \n{tfm} files.

%%\spoint[virtual:fonts] Virtual fonts
%\subsection{Virtual fonts}
%\label{virtual:fonts}
%\spoint[virtual:fonts] Virtual fonts
\subsection{Virtual fonts}
\label{virtual:fonts}

%With 
%\indexterm{virtual fonts}
%(see~\cite{K:virt}) it is possible that
%what looks like one font to \TeX\ resides in more than
%one physical font file. Also, virtual fonts can be used
%to change in effect the internal organization of font files.
With 
\indexterm{virtual fonts}
(see~\cite{K:virt}) it is possible that
what looks like one font to \TeX\ resides in more than
one physical font file. Also, virtual fonts can be used
to change in effect the internal organization of font files.

%For  \TeX\ itself, the
%presence of virtual fonts makes no difference: everything
%is still based on \n{tfm} files containing metric
%information. However, the screen or printer driver that displays
%the resulting \n{dvi} file on the screen or on a printer
%will search for files with extension \n{.vf} to determine
%how characters are to be interpreted.
%The \n{vf} file can, for instance, instruct the driver
%to interpret a character as a certain position
%in a certain font file, to interpret a character as more
%than one position (a~way of forming accented characters),
%or to include \cs{special} information (for
%instance to set gray levels).
For  \TeX\ itself, the
presence of virtual fonts makes no difference: everything
is still based on \n{tfm} files containing metric
information. However, the screen or printer driver that displays
the resulting \n{dvi} file on the screen or on a printer
will search for files with extension \n{.vf} to determine
how characters are to be interpreted.
The \n{vf} file can, for instance, instruct the driver
to interpret a character as a certain position
in a certain font file, to interpret a character as more
than one position (a~way of forming accented characters),
or to include \cs{special} information (for
instance to set gray levels).

%Readable variants of \n{vf} files have extension \n{vpl},
%analogous to the \n{pl} files for the \n{tfm} files; see above.
%Conversion between \n{vf} and \n{vpl} files can be
%performed with the \n{vftovp} and \n{vptovf} programs.
Readable variants of \n{vf} files have extension \n{vpl},
analogous to the \n{pl} files for the \n{tfm} files; see above.
Conversion between \n{vf} and \n{vpl} files can be
performed with the \n{vftovp} and \n{vptovf} programs.

%However, because virtual fonts are a matter for
%\indexterm{device drivers}, no more details will be given in this book.
However, because virtual fonts are a matter for
\indexterm{device drivers}, no more details will be given in this book.

%%\spoint Font files
%\subsection{Font files}
%\index{font files|(}
%\spoint Font files
\subsection{Font files}
\index{font files|(}

%Character descriptions are stored in three types of files.
%\begin{description} \item [gf]
%    Generic Font files.
%This is the file type that the Metafont program generates.
%There are not many previewers or printer drivers that use
%this type of file directly.
%\item [pxl]
%    Pixel files. The \n{pxl} format is a pure bitmap format.
%Thus it is easy to generate \n{pxl} files from, for instance,
%scanner images.
Character descriptions are stored in three types of files.
\begin{description} \item [gf]
    Generic Font files.
This is the file type that the Metafont program generates.
There are not many previewers or printer drivers that use
this type of file directly.
\item [pxl]
    Pixel files. The \n{pxl} format is a pure bitmap format.
Thus it is easy to generate \n{pxl} files from, for instance,
scanner images.

%This format should be superseded by the \n{pk} format.
%Pixel files can become rather big,
%as their size grows quadratically in the size of the characters.
This format should be superseded by the \n{pk} format.
Pixel files can become rather big,
as their size grows quadratically in the size of the characters.

%\item [pk]
%    Packed files. Pixel files can be packed by a form of run-length
%encoding: instead of storing the complete bitmap only the
%starting positions and lengths of `runs' of black and white
%pixels are stored. This makes the size of \n{pk} files
%approximately linear in the size of the characters.
%However, a previewer or printer driver using a packed font file
%has to unpack it before it is   able to use it.
%\end{description}
\item [pk]
    Packed files. Pixel files can be packed by a form of run-length
encoding: instead of storing the complete bitmap only the
starting positions and lengths of `runs' of black and white
pixels are stored. This makes the size of \n{pk} files
approximately linear in the size of the characters.
However, a previewer or printer driver using a packed font file
has to unpack it before it is   able to use it.
\end{description}

%The following conversion programs exist:
%\n{gftopxl}, \n{gftopk}, \n{pktopxl}, \n{pxltopk}.
The following conversion programs exist:
\n{gftopxl}, \n{gftopk}, \n{pktopxl}, \n{pxltopk}.

%\index{font files|)}
\index{font files|)}

%\subsection{Computer Modern}
\subsection{Computer Modern}

%The only family of typefaces that comes with \TeX\ 
%in the standard distribution is the 
%\indexterm{Computer Modern}
%family of typefaces. This is an adaptation (using the terminology
%of~\cite{S}) by Donald Knuth of the Monotype Modern~8A typeface
%that was used for the first volume of his {\italic Art of Computer
%Programming\/} series. The `modern faces' all derive from the
%types that were cut between 1780 and 1800 by Firmin Didot in
%France, Giambattista Bodoni in Italy, and Justus Erich Walbaum
%in Germany. After the first two, these types are also called
%`Didone' types. This name was coined in the Vox classification
%of types \cite{Vox}. Ultimately, the inspiration for the Didone
%types is the `Romain du Roi', the type that was designed by
%Nicolas Jaugeon around 1692 for the French Imprimerie Royale.
The only family of typefaces that comes with \TeX\ 
in the standard distribution is the 
\indexterm{Computer Modern}
family of typefaces. This is an adaptation (using the terminology
of~\cite{S}) by Donald Knuth of the Monotype Modern~8A typeface
that was used for the first volume of his {\italic Art of Computer
Programming\/} series. The `modern faces' all derive from the
types that were cut between 1780 and 1800 by Firmin Didot in
France, Giambattista Bodoni in Italy, and Justus Erich Walbaum
in Germany. After the first two, these types are also called
`Didone' types. This name was coined in the Vox classification
of types \cite{Vox}. Ultimately, the inspiration for the Didone
types is the `Romain du Roi', the type that was designed by
Nicolas Jaugeon around 1692 for the French Imprimerie Royale.

%Didone types are characterized by a strong vertical orientation,
%and thin hairlines. The vertical accent is strengthened by the
%fact that the insides of curves are flattened.
%The result is a clear and brilliant page, provided that the
%printing is done carefully and on good quality paper.
%\message{Reference format}
%However, they are quite vulnerable; \cite{Up}
%compares them to the distinguished but fragile furniture
%from the same period, saying one is afraid to use either,
%`for both seem in danger of breaking in pieces'.
%With the current proliferation of low resolution (around
%300 dot per inch) printers, the Computer Modern is
%a somewhat unfortunate choice.
Didone types are characterized by a strong vertical orientation,
and thin hairlines. The vertical accent is strengthened by the
fact that the insides of curves are flattened.
The result is a clear and brilliant page, provided that the
printing is done carefully and on good quality paper.
\message{Reference format}
However, they are quite vulnerable; \cite{Up}
compares them to the distinguished but fragile furniture
from the same period, saying one is afraid to use either,
`for both seem in danger of breaking in pieces'.
With the current proliferation of low resolution (around
300 dot per inch) printers, the Computer Modern is
a somewhat unfortunate choice.

%Recently, Donald Knuth has developed 
%a new typeface (or rather, a subfamily of typefaces)
%by changing parameters
%in the Computer Modern family. The result is a so-called
%`Egyptian' typeface: Computer Concrete \cite{K:cc}.
%The name derives from the
%fact that it was intended primarily for the book {\italic Concrete
%Mathematics}. Egyptian typefaces (they fall under the `M\'ecanes'
%in the Vox classification, meaning constructed,
%not derived from written letters) have a very uniform line width
%and square serifs. They do not have anything to do with Egypt;
%such types happened to be popular in the first half of the nineteenth
%century when Egyptology was developing and popular.
Recently, Donald Knuth has developed 
a new typeface (or rather, a subfamily of typefaces)
by changing parameters
in the Computer Modern family. The result is a so-called
`Egyptian' typeface: Computer Concrete \cite{K:cc}.
The name derives from the
fact that it was intended primarily for the book {\italic Concrete
Mathematics}. Egyptian typefaces (they fall under the `M\'ecanes'
in the Vox classification, meaning constructed,
not derived from written letters) have a very uniform line width
and square serifs. They do not have anything to do with Egypt;
such types happened to be popular in the first half of the nineteenth
century when Egyptology was developing and popular.

%%\point \TeX\ and web
%\section{\TeX\ and web}
%\point \TeX\ and web
\section{\TeX\ and web}

%The \TeX\ program is written in \web, a programming language
%\thecstoidx web\par
%that can be considered as a subset of \indexterm{Pascal}, augmented with
%a preprocessor.
The \TeX\ program is written in \web, a programming language
\thecstoidx web\par
that can be considered as a subset of \indexterm{Pascal}, augmented with
a preprocessor.

%\TeX\ makes no use of some features of Pascal, in order to
%facilitate porting to Pascal systems other than the one
%it was originally designed for, and even to enable automatic
%translation to other programming languages such as~C.
%For instance, it does not use the Pascal \n{With} construct.
%Also, procedures do not have output parameters; apart from
%writing to global variables, the only way
%values are returned is through
%\n{Function} values.
\TeX\ makes no use of some features of Pascal, in order to
facilitate porting to Pascal systems other than the one
it was originally designed for, and even to enable automatic
translation to other programming languages such as~C.
For instance, it does not use the Pascal \n{With} construct.
Also, procedures do not have output parameters; apart from
writing to global variables, the only way
values are returned is through
\n{Function} values.

%Actually, \web\ is more than a superset of a subset of Pascal
%(and to be more precise, it can also be used with other
%programming languages);
%it is a `system of structured documentation'. This means that
%the \web\ programmer writes pieces of program code,
%interspersed with their documentation, in one file. 
%This idea of `literate programming' was
%introduced  in~\cite{K:literate};
%for more information, see~\cite{Sewell}.
Actually, \web\ is more than a superset of a subset of Pascal
(and to be more precise, it can also be used with other
programming languages);
it is a `system of structured documentation'. This means that
the \web\ programmer writes pieces of program code,
interspersed with their documentation, in one file. 
This idea of `literate programming' was
introduced  in~\cite{K:literate};
for more information, see~\cite{Sewell}.

%Two auxiliary programs,
%Tangle and Weave, can then be used to strip the documentation
%and convert \web\ into regular Pascal, or to convert the
%\web\ file into a \TeX\ file that will typeset the program 
%and documentation.
Two auxiliary programs,
Tangle and Weave, can then be used to strip the documentation
and convert \web\ into regular Pascal, or to convert the
\web\ file into a \TeX\ file that will typeset the program 
and documentation.

%Portability of \web\ programs is achieved by the `change file'
%mechanism. A~change file is a list of changes to be made to
%the \web\ file; a~bit like a stream editor script.
%These changes can comprise both adaptations of the \web\ file
%to the particular Pascal compiler that will be used, and
%bug fixes to \TeX. Thus the \n{TeX.web} file need never be edited.
Portability of \web\ programs is achieved by the `change file'
mechanism. A~change file is a list of changes to be made to
the \web\ file; a~bit like a stream editor script.
These changes can comprise both adaptations of the \web\ file
to the particular Pascal compiler that will be used, and
bug fixes to \TeX. Thus the \n{TeX.web} file need never be edited.


%%\point The \TeX\ Users Group
%\section{The \TeX\ Users Group}
%\point The \TeX\ Users Group
\section{The \TeX\ Users Group}

%\TeX\ users have joined into several users groups
%over the last decade. Many national or language
%users groups exist, and a lot of them publish newsletters.
%The oldest of all \TeX\ users groups is simply called
%that: the \TeX\ Users Group, or \indexterm{TUG},
%and its journal is called \indexterm{TUGboat}.
%You can reach them at
%\begin{disp} \TeX\ Users Group\nl P.O. Box 2311\nl
%     Portland, OR 97208-2311, USA
%\end{disp}
% or electronically at \n{office@tug.org} on the Internet.
\TeX\ users have joined into several users groups
over the last decade. Many national or language
users groups exist, and a lot of them publish newsletters.
The oldest of all \TeX\ users groups is simply called
that: the \TeX\ Users Group, or \indexterm{TUG},
and its journal is called \indexterm{TUGboat}.
You can reach them at
\begin{disp} \TeX\ Users Group\nl P.O. Box 2311\nl
     Portland, OR 97208-2311, USA
\end{disp}
 or electronically at \n{office@tug.org} on the Internet.


%\endofchapter
%%%%% end of input file [run]
\endofchapter
%%%% end of input file [run]
