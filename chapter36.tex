% -*- coding: utf-8 -*-
\documentclass{book}

% -*- coding: utf-8 -*-

\usepackage[b5paper,text={5in,8in},centering]{geometry}

\usepackage[CJKchecksingle]{xeCJK}
\setmainfont[Mapping=tex-text]{TeX Gyre Schola}
%\setsansfont{URW Gothic L Book}
%\setmonofont{Nimbus Mono L}
\setCJKmainfont[BoldFont=FandolHei,ItalicFont=FandolKai]{FandolSong}
\setCJKsansfont{FandolHei}
\setCJKmonofont{FandolFang}
\xeCJKsetup{PunctStyle = kaiming}

\linespread{1.25}
\setlength{\parindent}{2em}

\usepackage{xcolor}
\definecolor{myblue}{rgb}{0,0.2,0.6}

\usepackage{titlesec}
\titleformat{\chapter}
    {\normalfont\Huge\sffamily\color{myblue}}
    {第\thechapter 章}
    {1em}
    {}
%\titlespacing{\chapter}{0pt}{50pt}{40pt}
\titleformat{\section}
    {\normalfont\Large\sffamily\color{myblue}}
    {\thesection}
    {1em}
    {}
%\titlespacing{\section}{0pt}{3.5ex plus 1ex minus .2ex}{2.3ex plus .2ex}
\titleformat{\subsection}
    {\normalfont\large\sffamily\color{myblue}}
    {\thesubsection}
    {1em}
    {}
%\titlespacing{\subsection}{0pt}{3.25ex plus 1ex minus .2ex}{1.5ex plus .2ex}
%
\newpagestyle{special}[\small\sffamily]{
  \headrule
  \sethead[\usepage][][\chaptertitle]
  {\chaptertitle}{}{\usepage}}
\newpagestyle{main}[\small\sffamily]{
  \headrule
  \sethead[\usepage][][第\thechapter 章\quad\chaptertitle]
  {\thesection\quad\sectiontitle}{}{\usepage}}

\usepackage{titletoc}
%\setcounter{tocdepth}{1}
%\titlecontents{标题层次}[左间距]{上间距和整体格式}{标题序号}{标题内容}{指引线和页码}[下间距]
\titlecontents{chapter}[1.5em]{\vspace{.5em}\bfseries\sffamily}{\color{myblue}\contentslabel{1.5em}}{}
    {\titlerule*[20pt]{$\cdot$}\contentspage}[]
\titlecontents{section}[4.5em]{\sffamily}{\color{myblue}\contentslabel{3em}}{}
    {\titlerule*[20pt]{$\cdot$}\contentspage}[]
%\titlecontents{subsection}[8.5em]{\sffamily}{\contentslabel{4em}}{}
%    {\titlerule*[20pt]{$\cdot$}\contentspage}

\usepackage{enumitem}
\setlist{topsep=2pt,itemsep=2pt,parsep=1pt,leftmargin=\parindent}

\usepackage{fancyvrb}
\DefineVerbatimEnvironment{verbatim}{Verbatim}
  {xleftmargin=2em,baselinestretch=1,formatcom=\color{teal}\upshape}

\usepackage{etoolbox}
\makeatletter
\preto{\FV@ListVSpace}{\topsep=2pt \partopsep=0pt }
\makeatother

\usepackage[colorlinks,plainpages,pagebackref]{hyperref}
\hypersetup{
   pdfstartview={FitH},
   citecolor=teal,
   linkcolor=myblue,
   urlcolor=black,
   bookmarksnumbered
}

\usepackage{comment,makeidx,multicol}

%\usepackage{german}
%% german
%\righthyphenmin=3
%\mdqoff
%\captionsenglish
\usepackage[english]{babel}
{\catcode`"=13 \gdef"#1{\ifx#1"\discretionary{}{}{}\fi\relax}}
\def\mdqon{\catcode`"=13\relax}
\def\mdqoff{\catcode`"=12\relax}
\makeindex
\hyphenation{ex-em-pli-fies}

\newdimen\tempdima \newdimen\tempdimb

% these are fine
\def\handbreak{\\ \message{^^JManual break!!!!^^J}}
\def\nl{\protect\\}\def\n#1{{\tt #1}}
\def\cs#1{\texorpdfstring{{\tt\char`\\#1}}{\textbackslash#1}} %\def\cs#1{{\tt\char`\\#1}}
\let\csc\cs
\def\lb{{\tt\char`\{}}\def\rb{{\tt\char`\}}}
\def\gr#1{\texorpdfstring{$\langle$#1$\rangle$}{<#1>}} %\def\gr#1{$\langle$#1$\rangle$}
\def\key#1{{\tt#1}}
\def\alt{}\def\altt{}%this way in manstijl
\def\ldash{\unskip\ --\nobreak\ \ignorespaces}
\def\rdash{\unskip\nobreak\ --\ \ignorespaces}
% check these
\def\hex{{\tt"}}
\def\ascii{{\sc ascii}}
\def\ebcdic{{\sc ebcdic}}
\def\IniTeX{Ini\TeX}\def\LamsTeX{LAMS\TeX}\def\VirTeX{Vir\TeX}
\def\AmsTeX{Ams\TeX}
\def\TeXbook{the \TeX\ book}\def\web{{\sc web}}
% needs major thinking
\newenvironment{myquote}{\list{}{%
    \topsep=2pt \partopsep=0pt%
    \leftmargin=\parindent \rightmargin=\parindent
    }\item[]}{\endlist}
\newenvironment{disp}{\begin{myquote}}{\end{myquote}}
\newenvironment{Disp}{\begin{myquote}}{\end{myquote}}
\newenvironment{tdisp}{\begin{myquote}}{\end{myquote}}
\newenvironment{example}{\begin{myquote}\noindent\itshape 例子:}{\end{myquote}}
\newenvironment{inventory}{\begin{description}\raggedright}{\end{description}}
\newenvironment{glossinventory}{\begin{description}}{\end{description}}
\def\gram#1{\gr{#1}}%???
%
% index
%
\def\indexterm#1{\emph{#1}\index{#1}}
\def\indextermsub#1#2{\emph{#1 #2}\index{#1!#2}}
\def\indextermbus#1#2{\emph{#1 #2}\index{#2!#1}}
\def\term#1\par{\index{#1}}
\def\howto#1\par{}
\def\cstoidx#1\par{\index{#1@\cs{#1}@}}
\def\thecstoidx#1\par{\index{#1@\protect\csname #1\endcsname}}
\def\thecstoidxsub#1#2{\index{#1, #2@\protect\csname #1\endcsname, #2}\ignorespaces}
\def\csterm#1\par{\cstoidx #1\par\cs{#1}}
\def\csidx#1{\cstoidx #1\par\cs{#1}}

\def\tmc{\tracingmacros=2 \tracingcommands\tracingmacros}

%%%%%%%%%%%%%%%%%%%
\makeatletter
\def\snugbox{\hbox\bgroup\setbox\z@\vbox\bgroup
    \leftskip\z@
    \bgroup\aftergroup\make@snug
    \let\next=}
\def\make@snug{\par\sn@gify\egroup \box\z@\egroup}
\def\sn@gify
   {\skip\z@=\lastskip \unskip
    \advance\skip\z@\lastskip \unskip
    \unpenalty
    \setbox\z@\lastbox
    \ifvoid\z@ \nointerlineskip \else {\sn@gify} \fi
    \hbox{\unhbox\z@}\nointerlineskip
    \vskip\skip\z@
    }

\newdimen\fbh \fbh=60pt % dimension for easy scaling:
\newdimen\fbw \fbw=60pt % height and width of character box

\newdimen\dh \newdimen\dw % height and width of current character box
\newdimen\lh % height of previous character box
\newdimen\lw \lw=.4pt % line weight, instead of default .4pt

\def\hdotfill{\noindent
    \leaders\hbox{\vrule width 1pt height\lw
                  \kern4pt
                  \vrule width.5pt height\lw}\hfill\hbox{}
    \par}
\def\hlinefill{\noindent
    \leaders\hbox{\vrule width 5.5pt height\lw         }\hfill\hbox{}
    \par}
\def\stippel{$\qquad\qquad\qquad\qquad$}
\makeatother
%%%%%%%%%%%%%%%%%%%

%\def\SansSerif{\Typeface:macHelvetica }
%\def\SerifFont{\Typeface:macTimes }
%\def\SansSerif{\Typeface:bsGillSans }
%\def\SerifFont{\Typeface:bsBaskerville }
\let\SansSerif\relax \def\italic{\it}
\let\SerifFont\relax \def\MainFont{\rm}
\let\SansSerif\relax
\let\SerifFont\relax
\let\PopIndentLevel\relax \let\PushIndentLevel\relax
\let\ToVerso\relax \let\ToRecto\relax

%\def\stop@command@suffix{stop}
%\let\PopListLevel\PopIndentLevel
%\let\FlushRight\relax
%\let\flushright\FlushRight
%\let\SetListIndent\LevelIndent
%\def\awp{\ifhmode\vadjust{\penalty-10000 }\else
%    \penalty-10000 \fi}
\let\awp\relax
\let\PopIndentLevel\relax \let\PopListLevel\relax

\showboxdepth=-1

%\input figs
\def\endofchapter{\vfill\noindent}

\setcounter{chapter}{35}

\begin{document}

%\chapter{The Grammar of \TeX}\label{gramm}
\chapter{The Grammar of \TeX}\label{gramm}

%Many  chapters in this book contain pieces of the
%grammar that defines the formal syntax of \TeX.
%In this chapter the structure of the rewriting rules of the
%grammar is explained, and some key notions are presented.
Many  chapters in this book contain pieces of the
grammar that defines the formal syntax of \TeX.
In this chapter the structure of the rewriting rules of the
grammar is explained, and some key notions are presented.

%In \TeXbook\ a grammar appears in Chapters~24--27.
%An even more rigorous grammar of \TeX\ can be found in~\cite{Appelt}.
%The grammar presented in this book is virtually identical 
%to that of \TeXbook.
In \TeXbook\ a grammar appears in Chapters~24--27.
An even more rigorous grammar of \TeX\ can be found in~\cite{Appelt}.
The grammar presented in this book is virtually identical 
to that of \TeXbook.

%%\point Notations
%\section{Notations}
%\point Notations
\section{Notations}

%Basic to the grammar are 
%\begin{description}\item [grammatical terms]
%These are enclosed in angle brackets:
%\begin{disp}\gr{term}\end{disp}
%\item [control sequences]
%These are given in typewriter type with a backslash for
%the escape character:
%\begin{disp}\cs{command}\end{disp}
%\end{description}
%Lastly there are 
%\begin{description}\item [keywords]
%Also given in typewriter type
%\begin{disp}\n{keyword}\end{disp}
%This is a limited collection of words that have a special
%meaning for \TeX\ in certain contexts; see below.\end{description}
Basic to the grammar are 
\begin{description}\item [grammatical terms]
These are enclosed in angle brackets:
\begin{disp}\gr{term}\end{disp}
\item [control sequences]
These are given in typewriter type with a backslash for
the escape character:
\begin{disp}\cs{command}\end{disp}
\end{description}
Lastly there are 
\begin{description}\item [keywords]
Also given in typewriter type
\begin{disp}\n{keyword}\end{disp}
This is a limited collection of words that have a special
meaning for \TeX\ in certain contexts; see below.\end{description}

%The three elements of the grammar are used in syntax rules:
%\begin{disp}\gr{snark} $\longrightarrow$ \n{boojum} $|$ \gr{empty}
%\end{disp}
%This rule says that the grammatical entity \gr{snark}
%is either the keyword \n{boojum}, or the grammatical
%entity \gr{empty}.
The three elements of the grammar are used in syntax rules:
\begin{disp}\gr{snark} $\longrightarrow$ \n{boojum} $|$ \gr{empty}
\end{disp}
This rule says that the grammatical entity \gr{snark}
is either the keyword \n{boojum}, or the grammatical
entity \gr{empty}.

%There are two other notational conventions.
%The first is that the double quote
%is used to indicate hexadecimal (base~16) notation.
%For instance \verb>"ab56> stands for $10\times16^3+11\times16^2
%+5\times16^1+6\times16^0$. The second convention
%is that subscripts are used to denote category codes.
%Thus \n{a}$_{12}$ denotes an `a' of category~12.
There are two other notational conventions.
The first is that the double quote
is used to indicate hexadecimal (base~16) notation.
For instance \verb>"ab56> stands for $10\times16^3+11\times16^2
+5\times16^1+6\times16^0$. The second convention
is that subscripts are used to denote category codes.
Thus \n{a}$_{12}$ denotes an `a' of category~12.

%%\point[keywords] Keywords
%\section{Keywords}
%\label{keywords}
%\index{keywords|(}
%\point[keywords] Keywords
\section{Keywords}
\label{keywords}
\index{keywords|(}

%A keyword is sequence of characters (or character tokens)
%of any category code but~13\index{category!13} (active).
%Unlike the situation in control sequences, \TeX\ does not 
%distinguish between lowercase and uppercase characters
%in keywords. Uppercase characters in keywords are converted to
%lowercase by adding 32 to them; the \cs{lccode} and \cs{uccode}
%are not used here. Furthermore, any keyword can be preceded by
%optional spaces.
A keyword is sequence of characters (or character tokens)
of any category code but~13\index{category!13} (active).
Unlike the situation in control sequences, \TeX\ does not 
distinguish between lowercase and uppercase characters
in keywords. Uppercase characters in keywords are converted to
lowercase by adding 32 to them; the \cs{lccode} and \cs{uccode}
are not used here. Furthermore, any keyword can be preceded by
optional spaces.

%Thus both \n{true cm} and \n{truecm} are legal.
%By far the strangest example, however, is provided
%by the grammar rule
%\begin{disp}\gr{fil unit} $\longrightarrow$ \n{fil} $|$ \gr{fil unit}\n l
%\end{disp} which implies that \hbox{\n{fil L l}} is also
%a legal \gr{fil dimen}. Strange errors can ensue from this;
%see page~\pageref{fil:l:l} for an example.
Thus both \n{true cm} and \n{truecm} are legal.
By far the strangest example, however, is provided
by the grammar rule
\begin{disp}\gr{fil unit} $\longrightarrow$ \n{fil} $|$ \gr{fil unit}\n l
\end{disp} which implies that \hbox{\n{fil L l}} is also
a legal \gr{fil dimen}. Strange errors can ensue from this;
see page~\pageref{fil:l:l} for an example.

%Here is the full list of all keywords: \n{at}, \n{bp},
%\n{by}, \n{cc}, \n{cm}, \n{dd}, \n{depth}, \n{em}, \n{ex},
%\n{fil}, \n{height}, \n{in}, \n l, \n{minus}, \n{mm}, \n{mu},
%\n{pc}, \n{plus}, \n{pt}, \n{scaled}, \n{sp}, \n{spread},
%\n{to}, \n{true}, \n{width}.
Here is the full list of all keywords: \n{at}, \n{bp},
\n{by}, \n{cc}, \n{cm}, \n{dd}, \n{depth}, \n{em}, \n{ex},
\n{fil}, \n{height}, \n{in}, \n l, \n{minus}, \n{mm}, \n{mu},
\n{pc}, \n{plus}, \n{pt}, \n{scaled}, \n{sp}, \n{spread},
\n{to}, \n{true}, \n{width}.

%\index{keywords|)}
\index{keywords|)}

%\section{Specific grammatical terms}
\section{Specific grammatical terms}

%Some grammatical terms appear in a lot of rules.
%One such term is \gr{optional spaces}.
%The term \indextermbus{optional}{space} is probably clear enough,
%but here is the formal definition:
%\begin{disp}\gr{optional spaces} $\longrightarrow$
%     \gr{empty} $|$ \gr{space token}\gr{optional spaces}
%     \end{disp}
%which amounts to saying that \gr{optional spaces}
%is zero or more space tokens.
Some grammatical terms appear in a lot of rules.
One such term is \gr{optional spaces}.
The term \indextermbus{optional}{space} is probably clear enough,
but here is the formal definition:
\begin{disp}\gr{optional spaces} $\longrightarrow$
     \gr{empty} $|$ \gr{space token}\gr{optional spaces}
     \end{disp}
which amounts to saying that \gr{optional spaces}
is zero or more space tokens.

%Other terms may not be so immediately obvious.
%Below are some of them.
Other terms may not be so immediately obvious.
Below are some of them.

%%\spoint \gr{equals}
%\subsection{\gr{equals}}
%\spoint \gr{equals}
\subsection{\gr{equals}}

%In assignments the equals sign is optional; therefore there
%is a term
%\begin{disp}\gr{equals} $\longrightarrow$ \gr{optional spaces}
%     $|$ \gr{optional spaces}$=_{12}$\end{disp}
%in \TeX's grammar.
%%% 
%\begin{comment}
%%% One assignment exists where the equals sign cannot
%%% be left out:
%%%
%\begin{verbatim}
%%% \let\spacetoken= %assign a space
%\end{verbatim}
%%% Here the space would have been skipped in \TeX's input processor
%%% if the equals sign had been left out.
%\end{comment}
In assignments the equals sign is optional; therefore there
is a term
\begin{disp}\gr{equals} $\longrightarrow$ \gr{optional spaces}
     $|$ \gr{optional spaces}$=_{12}$\end{disp}
in \TeX's grammar.
%% 
\begin{comment}
%% One assignment exists where the equals sign cannot
%% be left out:
%%
\begin{verbatim}
%% \let\spacetoken= %assign a space
\end{verbatim}
%% Here the space would have been skipped in \TeX's input processor
%% if the equals sign had been left out.
\end{comment}

%%\spoint \gr{filler}, \gr{general text}
%\subsection{\gr{filler}, \gr{general text}}
%\spoint \gr{filler}, \gr{general text}
\subsection{\gr{filler}, \gr{general text}}

%More obscure than the \gr{optional spaces} is the combination
%of spaces and \cs{relax} tokens that is allowed
%in some places, for instance
%\begin{verbatim}
%\setbox0= \relax\box1
%\end{verbatim}
%The quantity involved is 
%\begin{disp}\gr{filler} $\longrightarrow$ \gr{optional spaces}
%     $|$ \gr{filler}\cs{relax}\gr{optional spaces}\end{disp}
%One important occurrence of \gr{filler} is in
%\begin{disp}\gr{general text} $\longrightarrow$
%     \gr{filler}\lb\gr{balanced text}\gr{right brace}
%     \end{disp}
%A \gr{general text} follows such control sequences as
%\cs{message}, \cs{uppercase}, or \cs{mark}. The braces around
%the \gr{balanced text} are explained in the next point.
More obscure than the \gr{optional spaces} is the combination
of spaces and \cs{relax} tokens that is allowed
in some places, for instance
\begin{verbatim}
\setbox0= \relax\box1
\end{verbatim}
The quantity involved is 
\begin{disp}\gr{filler} $\longrightarrow$ \gr{optional spaces}
     $|$ \gr{filler}\cs{relax}\gr{optional spaces}\end{disp}
One important occurrence of \gr{filler} is in
\begin{disp}\gr{general text} $\longrightarrow$
     \gr{filler}\lb\gr{balanced text}\gr{right brace}
     \end{disp}
A \gr{general text} follows such control sequences as
\cs{message}, \cs{uppercase}, or \cs{mark}. The braces around
the \gr{balanced text} are explained in the next point.

%%\spoint \lb\rb\ and \gr{left brace}\gr{right brace}
%\subsection{\lb\rb\ and \gr{left brace}\gr{right brace}}
%\spoint \lb\rb\ and \gr{left brace}\gr{right brace}
\subsection{\lb\rb\ and \gr{left brace}\gr{right brace}}

%The \TeX\ grammar uses a perhaps somewhat unfortunate
%convention for braces. First of all 
%\begin{disp}\lb\ and \rb\end{disp}
%stand for braces that are either explicit open/close group
%characters, or control sequences defined by \cs{let},
%such as
%\begin{verbatim}
%\let\bgroup={ \let\egroup=}
%\end{verbatim}
%The grammatical terms 
%\begin{disp}\gr{left brace} and \gr{right brace}
%\end{disp} stand for explicit open/close group characters,
%that is, characters of categories 1 and~2 respectively.
The \TeX\ grammar uses a perhaps somewhat unfortunate
convention for braces. First of all 
\begin{disp}\lb\ and \rb\end{disp}
stand for braces that are either explicit open/close group
characters, or control sequences defined by \cs{let},
such as
\begin{verbatim}
\let\bgroup={ \let\egroup=}
\end{verbatim}
The grammatical terms 
\begin{disp}\gr{left brace} and \gr{right brace}
\end{disp} stand for explicit open/close group characters,
that is, characters of categories 1 and~2 respectively.

%Various combinations of these two kinds of braces exist.
%Braces around boxes can be implicit:
%\begin{disp}\cs{hbox}\gr{box specification}\lb
%     \gr{horizontal mode material}\rb\end{disp}
%Around a macro definition there must be explicit braces:
%\begin{disp}\gr{definition text} $\longrightarrow$
%     \gr{parameter text}\gr{left brace}\gr{balanced text}\gr{right brace}
%     \end{disp}
%Finally, the \gr{general text} that was mentioned above
%has to be explicitly closed, but it can be implicitly opened:
%\begin{disp}\gr{general text} $\longrightarrow$
%     \gr{filler}\lb\gr{balanced text}\gr{right brace}
%     \end{disp}
%The closing brace of a \gr{general text} has to be explicit,
%since a general text is a token list, which may
%contain \cs{egroup} tokens.
%\TeX\ performs expansion to find the opening 
%brace of a \gr{general text}.
Various combinations of these two kinds of braces exist.
Braces around boxes can be implicit:
\begin{disp}\cs{hbox}\gr{box specification}\lb
     \gr{horizontal mode material}\rb\end{disp}
Around a macro definition there must be explicit braces:
\begin{disp}\gr{definition text} $\longrightarrow$
     \gr{parameter text}\gr{left brace}\gr{balanced text}\gr{right brace}
     \end{disp}
Finally, the \gr{general text} that was mentioned above
has to be explicitly closed, but it can be implicitly opened:
\begin{disp}\gr{general text} $\longrightarrow$
     \gr{filler}\lb\gr{balanced text}\gr{right brace}
     \end{disp}
The closing brace of a \gr{general text} has to be explicit,
since a general text is a token list, which may
contain \cs{egroup} tokens.
\TeX\ performs expansion to find the opening 
brace of a \gr{general text}.

%%\spoint \gr{math field}
%\subsection{\gr{math field}}
%\spoint \gr{math field}
\subsection{\gr{math field}}

%In math mode various operations such as subscripting
%or applying \cs{underline} take an argument that
%is a \gr{math field}: either a single symbol, or
%a group. Here is the exact definition.
%\begin{disp}\gr{math field} $\longrightarrow$
%    \gr{math symbol} $|$ \gr{filler}\lb\gr{math mode material}\rb\nl
% \gr{math symbol}  $\longrightarrow$ \gr{character} $|$
%    \gr{math character}
%\end{disp}
%See page~\pageref{character} for \gr{character},
%\alt
%and page~\pageref{math:character} for \gr{math character}.
In math mode various operations such as subscripting
or applying \cs{underline} take an argument that
is a \gr{math field}: either a single symbol, or
a group. Here is the exact definition.
\begin{disp}\gr{math field} $\longrightarrow$
    \gr{math symbol} $|$ \gr{filler}\lb\gr{math mode material}\rb\nl
 \gr{math symbol}  $\longrightarrow$ \gr{character} $|$
    \gr{math character}
\end{disp}
See page~\pageref{character} for \gr{character},
\alt
and page~\pageref{math:character} for \gr{math character}.

%%\point[2vs3] Differences between \TeX\ versions 2 and 3
%\section{Differences between \TeX\ versions 2 and 3}
%\label{2vs3}
%\point[2vs3] Differences between \TeX\ versions 2 and 3
\section{Differences between \TeX\ versions 2 and 3}
\label{2vs3}

%In 1989 Knuth released \TeX\ version~3.0, which is 
%\thecstoidxsub{TeX}{version 2}
%\thecstoidxsub{TeX}{version 3}
%the first real change in \TeX\ since version~2.0,
%which was released in~1986 (version~0 of \TeX\ was
%released in 1982; see~\cite{Knuth:TeXerrors} for more about
%the history of \TeX).
%All intermediate versions were merely bug fixes.
In 1989 Knuth released \TeX\ version~3.0, which is 
\thecstoidxsub{TeX}{version 2}
\thecstoidxsub{TeX}{version 3}
the first real change in \TeX\ since version~2.0,
which was released in~1986 (version~0 of \TeX\ was
released in 1982; see~\cite{Knuth:TeXerrors} for more about
the history of \TeX).
All intermediate versions were merely bug fixes.

%The main difference between versions 2~and~3 lies
%in the fact that 8-bit input has become possible.
%Associated with this, various quantities that
%used to be 127 or~128 have been raised to 255
%or~256 respectively. Here is a short list.
%The full description is in~\cite{K:TeX23}.
The main difference between versions 2~and~3 lies
in the fact that 8-bit input has become possible.
Associated with this, various quantities that
used to be 127 or~128 have been raised to 255
or~256 respectively. Here is a short list.
The full description is in~\cite{K:TeX23}.

%\begin{inventory}\message{Remove other TeX3 refernces!}
%\item All `codes' (\cs{catcode}, \cs{sfcode}, et cetera;
%    see page~\pageref{codename})
%    now apply to 256 character codes instead of~128.
%\item A character with code \cs{endlinechar}
%    is appended to the line unless this parameter is negative
%or more than~255 (this was~127) (see page~\pageref{append:elc}).
%\item No escape character is output by \cs{write} and
%    other commands if \cs{escapechar} is negative or more than~255
%(this was~127) (see page~\pageref{use:escape}).
%\item The \verb>^^> replacement mechanism has been extended
%    (see page~\pageref{hathat}).
%\item Parameters \cs{language}, \cs{inputlineno},
%    \cs{errorcontextlines}, \cs{lefthyphenmin}, \cs{righthyphenmin},
%\cs{badness}, \cs{holdinginserts}, \cs{emergencystretch},
%and commands \cs{noboundary}, \cs{setlanguage}
%have been added.
%\item The value of \cs{outputpenalty} is no longer zero
%    if the page break was not at a penalty item;
%    instead it is~$10\,000$ (see page~\pageref{break:penalty}).
%\end{inventory}
\begin{inventory}\message{Remove other TeX3 refernces!}
\item All `codes' (\cs{catcode}, \cs{sfcode}, et cetera;
    see page~\pageref{codename})
    now apply to 256 character codes instead of~128.
\item A character with code \cs{endlinechar}
    is appended to the line unless this parameter is negative
or more than~255 (this was~127) (see page~\pageref{append:elc}).
\item No escape character is output by \cs{write} and
    other commands if \cs{escapechar} is negative or more than~255
(this was~127) (see page~\pageref{use:escape}).
\item The \verb>^^> replacement mechanism has been extended
    (see page~\pageref{hathat}).
\item Parameters \cs{language}, \cs{inputlineno},
    \cs{errorcontextlines}, \cs{lefthyphenmin}, \cs{righthyphenmin},
\cs{badness}, \cs{holdinginserts}, \cs{emergencystretch},
and commands \cs{noboundary}, \cs{setlanguage}
have been added.
\item The value of \cs{outputpenalty} is no longer zero
    if the page break was not at a penalty item;
    instead it is~$10\,000$ (see page~\pageref{break:penalty}).
\end{inventory}

%The plain format has also been updated, mostly
%with default settings for parameters such as
%\cs{lefthyphenmin}, but also a few macros have been added.
The plain format has also been updated, mostly
with default settings for parameters such as
\cs{lefthyphenmin}, but also a few macros have been added.

%\endofchapter
%%%%% end of input file [syntax]
\endofchapter
%%%% end of input file [syntax]

\end{document}
