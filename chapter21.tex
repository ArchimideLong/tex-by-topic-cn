% -*- coding: utf-8 -*-
\documentclass{book}

% -*- coding: utf-8 -*-

\usepackage[b5paper,text={5in,8in},centering]{geometry}

\usepackage[CJKchecksingle]{xeCJK}
\setmainfont[Mapping=tex-text]{TeX Gyre Schola}
%\setsansfont{URW Gothic L Book}
%\setmonofont{Nimbus Mono L}
\setCJKmainfont[BoldFont=FandolHei,ItalicFont=FandolKai]{FandolSong}
\setCJKsansfont{FandolHei}
\setCJKmonofont{FandolFang}
\xeCJKsetup{PunctStyle = kaiming}

\linespread{1.25}
\setlength{\parindent}{2em}

\usepackage{xcolor}
\definecolor{myblue}{rgb}{0,0.2,0.6}

\usepackage{titlesec}
\titleformat{\chapter}
    {\normalfont\Huge\sffamily\color{myblue}}
    {第\thechapter 章}
    {1em}
    {}
%\titlespacing{\chapter}{0pt}{50pt}{40pt}
\titleformat{\section}
    {\normalfont\Large\sffamily\color{myblue}}
    {\thesection}
    {1em}
    {}
%\titlespacing{\section}{0pt}{3.5ex plus 1ex minus .2ex}{2.3ex plus .2ex}
\titleformat{\subsection}
    {\normalfont\large\sffamily\color{myblue}}
    {\thesubsection}
    {1em}
    {}
%\titlespacing{\subsection}{0pt}{3.25ex plus 1ex minus .2ex}{1.5ex plus .2ex}
%
\newpagestyle{special}[\small\sffamily]{
  \headrule
  \sethead[\usepage][][\chaptertitle]
  {\chaptertitle}{}{\usepage}}
\newpagestyle{main}[\small\sffamily]{
  \headrule
  \sethead[\usepage][][第\thechapter 章\quad\chaptertitle]
  {\thesection\quad\sectiontitle}{}{\usepage}}

\usepackage{titletoc}
%\setcounter{tocdepth}{1}
%\titlecontents{标题层次}[左间距]{上间距和整体格式}{标题序号}{标题内容}{指引线和页码}[下间距]
\titlecontents{chapter}[1.5em]{\vspace{.5em}\bfseries\sffamily}{\color{myblue}\contentslabel{1.5em}}{}
    {\titlerule*[20pt]{$\cdot$}\contentspage}[]
\titlecontents{section}[4.5em]{\sffamily}{\color{myblue}\contentslabel{3em}}{}
    {\titlerule*[20pt]{$\cdot$}\contentspage}[]
%\titlecontents{subsection}[8.5em]{\sffamily}{\contentslabel{4em}}{}
%    {\titlerule*[20pt]{$\cdot$}\contentspage}

\usepackage{enumitem}
\setlist{topsep=2pt,itemsep=2pt,parsep=1pt,leftmargin=\parindent}

\usepackage{fancyvrb}
\DefineVerbatimEnvironment{verbatim}{Verbatim}
  {xleftmargin=2em,baselinestretch=1,formatcom=\color{teal}\upshape}

\usepackage{etoolbox}
\makeatletter
\preto{\FV@ListVSpace}{\topsep=2pt \partopsep=0pt }
\makeatother

\usepackage[colorlinks,plainpages,pagebackref]{hyperref}
\hypersetup{
   pdfstartview={FitH},
   citecolor=teal,
   linkcolor=myblue,
   urlcolor=black,
   bookmarksnumbered
}

\usepackage{comment,makeidx,multicol}

%\usepackage{german}
%% german
%\righthyphenmin=3
%\mdqoff
%\captionsenglish
\usepackage[english]{babel}
{\catcode`"=13 \gdef"#1{\ifx#1"\discretionary{}{}{}\fi\relax}}
\def\mdqon{\catcode`"=13\relax}
\def\mdqoff{\catcode`"=12\relax}
\makeindex
\hyphenation{ex-em-pli-fies}

\newdimen\tempdima \newdimen\tempdimb

% these are fine
\def\handbreak{\\ \message{^^JManual break!!!!^^J}}
\def\nl{\protect\\}\def\n#1{{\tt #1}}
\def\cs#1{\texorpdfstring{{\tt\char`\\#1}}{\textbackslash#1}} %\def\cs#1{{\tt\char`\\#1}}
\let\csc\cs
\def\lb{{\tt\char`\{}}\def\rb{{\tt\char`\}}}
\def\gr#1{\texorpdfstring{$\langle$#1$\rangle$}{<#1>}} %\def\gr#1{$\langle$#1$\rangle$}
\def\key#1{{\tt#1}}
\def\alt{}\def\altt{}%this way in manstijl
\def\ldash{\unskip\ --\nobreak\ \ignorespaces}
\def\rdash{\unskip\nobreak\ --\ \ignorespaces}
% check these
\def\hex{{\tt"}}
\def\ascii{{\sc ascii}}
\def\ebcdic{{\sc ebcdic}}
\def\IniTeX{Ini\TeX}\def\LamsTeX{LAMS\TeX}\def\VirTeX{Vir\TeX}
\def\AmsTeX{Ams\TeX}
\def\TeXbook{the \TeX\ book}\def\web{{\sc web}}
% needs major thinking
\newenvironment{myquote}{\list{}{%
    \topsep=2pt \partopsep=0pt%
    \leftmargin=\parindent \rightmargin=\parindent
    }\item[]}{\endlist}
\newenvironment{disp}{\begin{myquote}}{\end{myquote}}
\newenvironment{Disp}{\begin{myquote}}{\end{myquote}}
\newenvironment{tdisp}{\begin{myquote}}{\end{myquote}}
\newenvironment{example}{\begin{myquote}\noindent\itshape 例子:}{\end{myquote}}
\newenvironment{inventory}{\begin{description}\raggedright}{\end{description}}
\newenvironment{glossinventory}{\begin{description}}{\end{description}}
\def\gram#1{\gr{#1}}%???
%
% index
%
\def\indexterm#1{\emph{#1}\index{#1}}
\def\indextermsub#1#2{\emph{#1 #2}\index{#1!#2}}
\def\indextermbus#1#2{\emph{#1 #2}\index{#2!#1}}
\def\term#1\par{\index{#1}}
\def\howto#1\par{}
\def\cstoidx#1\par{\index{#1@\cs{#1}@}}
\def\thecstoidx#1\par{\index{#1@\protect\csname #1\endcsname}}
\def\thecstoidxsub#1#2{\index{#1, #2@\protect\csname #1\endcsname, #2}\ignorespaces}
\def\csterm#1\par{\cstoidx #1\par\cs{#1}}
\def\csidx#1{\cstoidx #1\par\cs{#1}}

\def\tmc{\tracingmacros=2 \tracingcommands\tracingmacros}

%%%%%%%%%%%%%%%%%%%
\makeatletter
\def\snugbox{\hbox\bgroup\setbox\z@\vbox\bgroup
    \leftskip\z@
    \bgroup\aftergroup\make@snug
    \let\next=}
\def\make@snug{\par\sn@gify\egroup \box\z@\egroup}
\def\sn@gify
   {\skip\z@=\lastskip \unskip
    \advance\skip\z@\lastskip \unskip
    \unpenalty
    \setbox\z@\lastbox
    \ifvoid\z@ \nointerlineskip \else {\sn@gify} \fi
    \hbox{\unhbox\z@}\nointerlineskip
    \vskip\skip\z@
    }

\newdimen\fbh \fbh=60pt % dimension for easy scaling:
\newdimen\fbw \fbw=60pt % height and width of character box

\newdimen\dh \newdimen\dw % height and width of current character box
\newdimen\lh % height of previous character box
\newdimen\lw \lw=.4pt % line weight, instead of default .4pt

\def\hdotfill{\noindent
    \leaders\hbox{\vrule width 1pt height\lw
                  \kern4pt
                  \vrule width.5pt height\lw}\hfill\hbox{}
    \par}
\def\hlinefill{\noindent
    \leaders\hbox{\vrule width 5.5pt height\lw         }\hfill\hbox{}
    \par}
\def\stippel{$\qquad\qquad\qquad\qquad$}
\makeatother
%%%%%%%%%%%%%%%%%%%

%\def\SansSerif{\Typeface:macHelvetica }
%\def\SerifFont{\Typeface:macTimes }
%\def\SansSerif{\Typeface:bsGillSans }
%\def\SerifFont{\Typeface:bsBaskerville }
\let\SansSerif\relax \def\italic{\it}
\let\SerifFont\relax \def\MainFont{\rm}
\let\SansSerif\relax
\let\SerifFont\relax
\let\PopIndentLevel\relax \let\PushIndentLevel\relax
\let\ToVerso\relax \let\ToRecto\relax

%\def\stop@command@suffix{stop}
%\let\PopListLevel\PopIndentLevel
%\let\FlushRight\relax
%\let\flushright\FlushRight
%\let\SetListIndent\LevelIndent
%\def\awp{\ifhmode\vadjust{\penalty-10000 }\else
%    \penalty-10000 \fi}
\let\awp\relax
\let\PopIndentLevel\relax \let\PopListLevel\relax

\showboxdepth=-1

%\input figs
\def\endofchapter{\vfill\noindent}

\setcounter{chapter}{20}

\begin{document}

%\chapter{Characters in Math Mode}\label{mathchar}
\chapter{Characters in Math Mode}\label{mathchar}

%In math mode every character specifies by its
%\cs{mathcode} what position of
%a font to access, among other things.
%For delimiters this story is a bit
%more complicated. This chapter explains the concept
%of math codes, and shows how \TeX\ implements variable
%size delimiters.
% 
%\label{cschap:mathcode}\label{cschap:mathchar}\label{cschap:mathchardef}\label{cschap:delcode}\label{cschap:delimiter}\label{cschap:delimiterfactor}\label{cschap:delimitershortfall}\label{cschap:nulldelimiterspace}\label{cschap:left}\label{cschap:right}\label{cschap:big}\label{cschap:Big}\label{cschap:bigg}\label{cschap:Bigg}\label{cschap:bigl}\label{cschap:bigm}\label{cschap:bigr}\label{cschap:radical}\label{cschap:mathaccent}\label{cschap:skewchar}\label{cschap:defaultskewchar}\label{cschap:skew}\label{cschap:widehat}\label{cschap:widetilde}
%\begin{inventory}
%\item [\cs{mathcode}] 
%      Code of a character determining its treatment in math mode.
In math mode every character specifies by its
\cs{mathcode} what position of
a font to access, among other things.
For delimiters this story is a bit
more complicated. This chapter explains the concept
of math codes, and shows how \TeX\ implements variable
size delimiters.
 
\label{cschap:mathcode}\label{cschap:mathchar}\label{cschap:mathchardef}\label{cschap:delcode}\label{cschap:delimiter}\label{cschap:delimiterfactor}\label{cschap:delimitershortfall}\label{cschap:nulldelimiterspace}\label{cschap:left}\label{cschap:right}\label{cschap:big}\label{cschap:Big}\label{cschap:bigg}\label{cschap:Bigg}\label{cschap:bigl}\label{cschap:bigm}\label{cschap:bigr}\label{cschap:radical}\label{cschap:mathaccent}\label{cschap:skewchar}\label{cschap:defaultskewchar}\label{cschap:skew}\label{cschap:widehat}\label{cschap:widetilde}
\begin{inventory}
\item [\cs{mathcode}] 
      Code of a character determining its treatment in math mode.

%\item [\cs{mathchar}] 
%      Explicit denotation of a mathematical character.
\item [\cs{mathchar}] 
      Explicit denotation of a mathematical character.

%\item [\cs{mathchardef}] 
%      Define a control sequence to be a synonym for
%      a~math character code.
\item [\cs{mathchardef}] 
      Define a control sequence to be a synonym for
      a~math character code.

%\item [\cs{delcode}] 
%      Code specifying how a character should be used as delimiter.
\item [\cs{delcode}] 
      Code specifying how a character should be used as delimiter.

%\item [\cs{delimiter}] 
%      Explicit denotation of a delimiter.
\item [\cs{delimiter}] 
      Explicit denotation of a delimiter.

%\item [\cs{delimiterfactor}] 
%      1000 times the fraction of a delimited formula that should be
%      covered by a delimiter.
%      Plain \TeX\ default:~\n{901}
\item [\cs{delimiterfactor}] 
      1000 times the fraction of a delimited formula that should be
      covered by a delimiter.
      Plain \TeX\ default:~\n{901}

%\item [\cs{delimitershortfall}] 
%      Size of the part of a delimited formula that is allowed 
%      to go uncovered by a delimiter.
%      Plain \TeX\ default:~\n{5pt}
\item [\cs{delimitershortfall}] 
      Size of the part of a delimited formula that is allowed 
      to go uncovered by a delimiter.
      Plain \TeX\ default:~\n{5pt}

%\item [\cs{nulldelimiterspace}] 
%      Width taken for empty delimiters. 
%      Plain \TeX\ default:~\n{1.2pt}
\item [\cs{nulldelimiterspace}] 
      Width taken for empty delimiters. 
      Plain \TeX\ default:~\n{1.2pt}

%\item [\cs{left}] 
%      Use the following character as an open delimiter.
\item [\cs{left}] 
      Use the following character as an open delimiter.

%\item [\cs{right}] 
%      Use the following character as a closing delimiter.
\item [\cs{right}] 
      Use the following character as a closing delimiter.

%\item [\cs{big}] 
%      One line high delimiter.
\item [\cs{big}] 
      One line high delimiter.

%\item [\cs{Big}] 
%      One and a half line high delimiter.
\item [\cs{Big}] 
      One and a half line high delimiter.

%\item [\cs{bigg}] 
%      Two lines high delimiter.
\item [\cs{bigg}] 
      Two lines high delimiter.

%\item [\cs{Bigg}] 
%      Two and a half lines high delimiter.
\item [\cs{Bigg}] 
      Two and a half lines high delimiter.

%\item [\cs{bigl {\MainFont etc.}}]
%      Left delimiters.
\item [\cs{bigl {\MainFont etc.}}]
      Left delimiters.

%\item [\cs{bigm {\MainFont etc.}}]
%      Delimiters used as binary relations.
\item [\cs{bigm {\MainFont etc.}}]
      Delimiters used as binary relations.

%\item [\cs{bigr {\MainFont etc.}}]
%      Right delimiters.
\item [\cs{bigr {\MainFont etc.}}]
      Right delimiters.

%\item [\cs{radical}] 
%      Command for setting things such as root signs.
\item [\cs{radical}] 
      Command for setting things such as root signs.

%\item [\cs{mathaccent}] 
%      Place an accent in math mode.
\item [\cs{mathaccent}] 
      Place an accent in math mode.

%\item [\cs{skewchar}] 
%      Font position of an after-placed accent.
\item [\cs{skewchar}] 
      Font position of an after-placed accent.

%\item [\cs{defaultskewchar}] 
%      Value of \cs{skewchar} when a font is loaded.
\item [\cs{defaultskewchar}] 
      Value of \cs{skewchar} when a font is loaded.

%\item [\cs{skew}] 
%      Macro to shift accents on top of characters explicitly.
\item [\cs{skew}] 
      Macro to shift accents on top of characters explicitly.

%\item [\cs{widehat}]
%      Hat accent that can
%      accommodate wide expressions.
\item [\cs{widehat}]
      Hat accent that can
      accommodate wide expressions.

%\item [\cs{widetilde}]
%      Tilde accent that can
%      accommodate wide expressions.
\item [\cs{widetilde}]
      Tilde accent that can
      accommodate wide expressions.

%\end{inventory}
\end{inventory}

%\section{Mathematical characters}
%\index{math characters|(}
\section{Mathematical characters}
\index{math characters|(}

%Each of the 256 permissible character codes has
%an associated \csidx{mathcode}, which can be assigned by
%\begin{disp}\cs{mathcode}\gr{8-bit number}\gr{equals}\gr{15-bit number}\end{disp}
%When processing in math mode, \TeX\ replaces all characters of
%categories 11 and~12, and \cs{char} and \cs{chardef} characters,
%by their associated mathcode.
Each of the 256 permissible character codes has
an associated \csidx{mathcode}, which can be assigned by
\begin{disp}\cs{mathcode}\gr{8-bit number}\gr{equals}\gr{15-bit number}\end{disp}
When processing in math mode, \TeX\ replaces all characters of
categories 11 and~12, and \cs{char} and \cs{chardef} characters,
by their associated mathcode.

%The  15-bit math code is most conveniently denoted hexadecimally
%as \verb-"xyzz-, where
%\begin{disp}
%\n x${}\leq7$ is the class (see page~\pageref{math:class}),\nl
%\n y is the font family number \alt
%(see Chapter~\ref{mathfont}), and \nl
%\n{zz} is the position of the character in the font.\end{disp}
The  15-bit math code is most conveniently denoted hexadecimally
as \verb-"xyzz-, where
\begin{disp}
\n x${}\leq7$ is the class (see page~\pageref{math:class}),\nl
\n y is the font family number \alt
(see Chapter~\ref{mathfont}), and \nl
\n{zz} is the position of the character in the font.\end{disp}

%Math codes can also be specified directly by 
%\cstoidx mathchar\par\cstoidx mathchardef\par
%a \gr{math character}, which can be\label{math:character}
%\begin{itemize}\item\cs{mathchar}\gr{15-bit number}; 
%\item \gr{mathchardef token}, a control sequence that was defined by
%\begin{disp}\cs{mathchardef}\gr{control sequence}\gr{equals}\gr{15-bit number}
%\end{disp}
% or
%\item a delimiter command\alt
%\begin{disp}\cs{delimiter}\gr{27-bit number}\end{disp}
% where the last 12 bits
%are discarded.\end{itemize}
%The commands \cs{mathchar} and \cs{mathchardef}
%are analogous to \cs{char} and \cs{char\-def} in text mode.
%Delimiters are treated below.
%A~\gr{mathchardef token} 
%can be used as a \gr{number}, even outside math mode.
Math codes can also be specified directly by 
\cstoidx mathchar\par\cstoidx mathchardef\par
a \gr{math character}, which can be\label{math:character}
\begin{itemize}\item\cs{mathchar}\gr{15-bit number}; 
\item \gr{mathchardef token}, a control sequence that was defined by
\begin{disp}\cs{mathchardef}\gr{control sequence}\gr{equals}\gr{15-bit number}
\end{disp}
 or
\item a delimiter command\alt
\begin{disp}\cs{delimiter}\gr{27-bit number}\end{disp}
 where the last 12 bits
are discarded.\end{itemize}
The commands \cs{mathchar} and \cs{mathchardef}
are analogous to \cs{char} and \cs{char\-def} in text mode.
Delimiters are treated below.
A~\gr{mathchardef token} 
can be used as a \gr{number}, even outside math mode.

%In \IniTeX\ all letters receive \cs{mathcode} \verb-"71zz- and
%all digits receive \verb-"70zz-, where \verb-"zz- is the 
%hexadecimal position of the character in the font.
%Thus, letters are initially from family~1
%(math italic in plain \TeX), and digits are from family~0
%(roman).
%For all other characters, \IniTeX\ assigns
%\begin{disp}\cs{mathcode}$\,x=x$,\end{disp}
%thereby placing them also in family~0.
In \IniTeX\ all letters receive \cs{mathcode} \verb-"71zz- and
all digits receive \verb-"70zz-, where \verb-"zz- is the 
hexadecimal position of the character in the font.
Thus, letters are initially from family~1
(math italic in plain \TeX), and digits are from family~0
(roman).
For all other characters, \IniTeX\ assigns
\begin{disp}\cs{mathcode}$\,x=x$,\end{disp}
thereby placing them also in family~0.

%If the mathcode is \verb-"8000-,
%\label{mcode:8000}the smallest integer that is
%not a \gr{15-bit number}, the character is treated as an active
%character with the original character code. Plain \TeX\
%assigns a \cs{mathcode} of \verb-"8000- to the space, underscore and prime.
If the mathcode is \verb-"8000-,
\label{mcode:8000}the smallest integer that is
not a \gr{15-bit number}, the character is treated as an active
character with the original character code. Plain \TeX\
assigns a \cs{mathcode} of \verb-"8000- to the space, underscore and prime.

%\index{math characters|)}
\index{math characters|)}

%\section{Delimiters}
%\index{delimiters|(}
\section{Delimiters}
\index{delimiters|(}

%After \csidx{left} and \csidx{right}
%commands \TeX\ looks for a delimiter. A~delimiter
%is either an explicit \cs{delimiter} command (or a
%macro abbreviation for it), or a character with a non-zero
%delimiter code.
After \csidx{left} and \csidx{right}
commands \TeX\ looks for a delimiter. A~delimiter
is either an explicit \cs{delimiter} command (or a
macro abbreviation for it), or a character with a non-zero
delimiter code.

%The \cs{left} and \cs{right} commands
%implicitly delimit a group, which is considered as a subformula.
%Since the enclosed formula can
%be arbitrarily large, the quest for the proper delimiter is
%a complicated story of looking at variants in two different
%fonts, linked chains of variants in a font, and building
%extendable delimiters from repeatable pieces.
The \cs{left} and \cs{right} commands
implicitly delimit a group, which is considered as a subformula.
Since the enclosed formula can
be arbitrarily large, the quest for the proper delimiter is
a complicated story of looking at variants in two different
fonts, linked chains of variants in a font, and building
extendable delimiters from repeatable pieces.

%The fact that a group enclosed in \verb>\left...\right> is
%treated as an independent subformula implies that a
%sub- or superscript at the start of this formula is
%not considered to belong to the delimiter. 
%For example, \TeX\ acts as if 
%\verb>\left(_2> is equivalent to \verb>\left({}_2>.
%(A~subscript after a \cs{right} delimiter is positioned
%with respect to that delimiter.)
The fact that a group enclosed in \verb>\left...\right> is
treated as an independent subformula implies that a
sub- or superscript at the start of this formula is
not considered to belong to the delimiter. 
For example, \TeX\ acts as if 
\verb>\left(_2> is equivalent to \verb>\left({}_2>.
(A~subscript after a \cs{right} delimiter is positioned
with respect to that delimiter.)

%%\spoint[delcodes] Delimiter codes 
%\subsection{Delimiter codes }
%\label{delcodes}
%\spoint[delcodes] Delimiter codes 
\subsection{Delimiter codes }
\label{delcodes}

%To each character code there corresponds a 
%\cstoidx delcode\par\indexterm{delimiter code},
%assigned by
%\begin{disp}\cs{delcode}\gr{8-bit number}\gr{equals}%
%          \gr{24-bit number}\end{disp}
%A delimiter code thus consists of six hexadecimal digits
%\verb-"uvvxyy-, where
%\begin{disp}
%\n{uvv} is the small variant of the delimiter, and\nl
%\n{xyy} is the large variant;\nl
%\n u, \n x are the font families of the variants, and\nl
%\n{vv}, \n{yy} are the locations in those fonts.\end{disp}
%Delimiter codes are used after \cs{left} and \cs{right}
%commands.
%\IniTeX\ sets all delimiter codes to~$-1$,
%except\label{ini:del}
%\verb-\delcode`.=0-, which makes the period an empty delimiter.
%In plain \TeX\ delimiters have typically \n{u}${}=2$ and~\n{x}${}=3$,
%that is, first family~2 is tried, and if no big
%enough delimiter turns up family~3 is tried.
To each character code there corresponds a 
\cstoidx delcode\par\indexterm{delimiter code},
assigned by
\begin{disp}\cs{delcode}\gr{8-bit number}\gr{equals}%
          \gr{24-bit number}\end{disp}
A delimiter code thus consists of six hexadecimal digits
\verb-"uvvxyy-, where
\begin{disp}
\n{uvv} is the small variant of the delimiter, and\nl
\n{xyy} is the large variant;\nl
\n u, \n x are the font families of the variants, and\nl
\n{vv}, \n{yy} are the locations in those fonts.\end{disp}
Delimiter codes are used after \cs{left} and \cs{right}
commands.
\IniTeX\ sets all delimiter codes to~$-1$,
except\label{ini:del}
\verb-\delcode`.=0-, which makes the period an empty delimiter.
In plain \TeX\ delimiters have typically \n{u}${}=2$ and~\n{x}${}=3$,
that is, first family~2 is tried, and if no big
enough delimiter turns up family~3 is tried.


%%\spoint Explicit \cs{delimiter} commands
%\subsection{Explicit \cs{delimiter} commands}
%\spoint Explicit \cs{delimiter} commands
\subsection{Explicit \cs{delimiter} commands}

%Delimiters can also be denoted 
%\cstoidx delimiter\par
%explicitly by a \gr{27-bit number},
%\begin{verbatim}
%\delimiter"tuvvxyy
%\end{verbatim}
%where \n{uvvxyy} are the small and large variant of the
%delimiter as above;
%the extra digit \n{t} (which is~$<8$) denotes the class
%(see page~\pageref{math:class}).
%For instance, the \cs{langle} macro is defined as
%\begin{verbatim}
%\def\langle{\delimiter "426830A }
%\end{verbatim}
%which means it belongs to class~4, opening. Similarly,
%\cs{rangle} is of class~5, closing; and \cs{uparrow} is of class~3,
%relation.
Delimiters can also be denoted 
\cstoidx delimiter\par
explicitly by a \gr{27-bit number},
\begin{verbatim}
\delimiter"tuvvxyy
\end{verbatim}
where \n{uvvxyy} are the small and large variant of the
delimiter as above;
the extra digit \n{t} (which is~$<8$) denotes the class
(see page~\pageref{math:class}).
For instance, the \cs{langle} macro is defined as
\begin{verbatim}
\def\langle{\delimiter "426830A }
\end{verbatim}
which means it belongs to class~4, opening. Similarly,
\cs{rangle} is of class~5, closing; and \cs{uparrow} is of class~3,
relation.

%After \cs{left} and \cs{right} \ldash that is, when \TeX\
%is looking for a delimiter \rdash  the class digit is ignored;
%otherwise \ldash when \TeX\ is not looking for a delimiter \rdash 
%the rightmost three digits are ignored, and the
%four remaining digits are treated as a~\cs{mathchar}; see above.
After \cs{left} and \cs{right} \ldash that is, when \TeX\
is looking for a delimiter \rdash  the class digit is ignored;
otherwise \ldash when \TeX\ is not looking for a delimiter \rdash 
the rightmost three digits are ignored, and the
four remaining digits are treated as a~\cs{mathchar}; see above.

%%\spoint[successor] Finding a delimiter; successors
%\subsection{Finding a delimiter; successors}
%\label{successor}
%\index{delimiter!size|(}
%\spoint[successor] Finding a delimiter; successors
\subsection{Finding a delimiter; successors}
\label{successor}
\index{delimiter!size|(}

%Typesetting a delimiter is a somewhat involved affair.
%First \TeX\ determines the size $y$ of the formula to be covered,
%which is twice the maximum of the height and depth of the
%formula. Thus the formula may not look optimal if
%it is not centred itself.
Typesetting a delimiter is a somewhat involved affair.
First \TeX\ determines the size $y$ of the formula to be covered,
which is twice the maximum of the height and depth of the
formula. Thus the formula may not look optimal if
it is not centred itself.

%The size of the delimiter should be at least 
%\csidx{delimiterfactor}${}\times y/1000$ and at least 
%$y-{}$\csidx{delimitershortfall}.
%\TeX\ then tries first the small variant, and if that one
%is not satisfactory (or if the \n{uvv} part of the delimiter
%is~\n{000}) it tries the large variant. If trying the large variant
%does not meet with success, \TeX\ takes the largest delimiter
%encountered in this search; if no delimiter at all was found
%(which can happen if  the \n{xyy} part is
%\altt
%also~\n{000}),
%an empty box of width~\csidx{nulldelimiterspace} is taken.
The size of the delimiter should be at least 
\csidx{delimiterfactor}${}\times y/1000$ and at least 
$y-{}$\csidx{delimitershortfall}.
\TeX\ then tries first the small variant, and if that one
is not satisfactory (or if the \n{uvv} part of the delimiter
is~\n{000}) it tries the large variant. If trying the large variant
does not meet with success, \TeX\ takes the largest delimiter
encountered in this search; if no delimiter at all was found
(which can happen if  the \n{xyy} part is
\altt
also~\n{000}),
an empty box of width~\csidx{nulldelimiterspace} is taken.

%Investigating a variant means, in sequence,
%\begin{itemize}
%\item if the current style (see page~\pageref{math:styles}) is
%  scriptscriptstyle, the \cs{scriptscriptfont} of the family is tried;
%\item if the current style is scriptstyle or smaller, the
%  \cs{scriptfont} of the family is tried;
%\item otherwise the \cs{textfont} of the family is tried.
%\end{itemize}
%The plain format puts the \verb-cmex10- font  in all three
%styles of family~3.
Investigating a variant means, in sequence,
\begin{itemize}
\item if the current style (see page~\pageref{math:styles}) is
  scriptscriptstyle, the \cs{scriptscriptfont} of the family is tried;
\item if the current style is scriptstyle or smaller, the
  \cs{scriptfont} of the family is tried;
\item otherwise the \cs{textfont} of the family is tried.
\end{itemize}
The plain format puts the \verb-cmex10- font  in all three
styles of family~3.

%Looking for a delimiter at a certain position in a certain font
%means
%\begin{itemize}
%\item if the character is large enough, accept it;
%\item if the character is
%  \emph{extendable}\index{character!extendable}, accept it;
%\item otherwise, if the character has a \indexterm{successor}, that
%  is, it is part of a chain of increasingly bigger delimiters in the
%  same font, try the successor.
%\end{itemize}
%Information about
%successors and extensibility of a delimiter is coded in the font
%metric file of the font.  An extendable character has a top, a bottom,
%possibly a mid piece, and a piece which is repeated directly below the
%top piece, and directly above the bottom piece if there is a mid
%piece.
Looking for a delimiter at a certain position in a certain font
means
\begin{itemize}
\item if the character is large enough, accept it;
\item if the character is
  \emph{extendable}\index{character!extendable}, accept it;
\item otherwise, if the character has a \indexterm{successor}, that
  is, it is part of a chain of increasingly bigger delimiters in the
  same font, try the successor.
\end{itemize}
Information about
successors and extensibility of a delimiter is coded in the font
metric file of the font.  An extendable character has a top, a bottom,
possibly a mid piece, and a piece which is repeated directly below the
top piece, and directly above the bottom piece if there is a mid
piece.

%\index{delimiter!size|)}
\index{delimiter!size|)}

%\subsection{\cs{big}, \cs{Big}, \cs{bigg}, and \cs{Bigg}
%delimiter macros}
\subsection{\cs{big}, \cs{Big}, \cs{bigg}, and \cs{Bigg}
delimiter macros}

%In order to be able to use a delimiter outside the 
%\verb-\left...\right- context, or to specify a delimiter of
%a different size than \TeX\ would have chosen,
%four macros for `big' delimiters exist: \cs{big},
%\cs{Big}, \cs{bigg}, and \cs{Bigg}. These can be used with
%anything that can follow \cs{left} or \cs{right}.
In order to be able to use a delimiter outside the 
\verb-\left...\right- context, or to specify a delimiter of
a different size than \TeX\ would have chosen,
four macros for `big' delimiters exist: \cs{big},
\cs{Big}, \cs{bigg}, and \cs{Bigg}. These can be used with
anything that can follow \cs{left} or \cs{right}.

%Twelve further macros (for instance \cs{bigl}, \cs{bigm},
%\cstoidx big \rm etc.\par
%%\csterm big \Style:roman etc.\par
%and~\cs{bigr}) force such delimiters in the context of
%an opening symbol, a binary relation, and a closing symbol
%respectively:
%\begin{verbatim}
%\def\bigl{\mathopen\big}
%\def\bigm{\mathrel\big} \def\bigr{\mathclose\big}
%\end{verbatim}
Twelve further macros (for instance \cs{bigl}, \cs{bigm},
\cstoidx big \rm etc.\par
%\csterm big \Style:roman etc.\par
and~\cs{bigr}) force such delimiters in the context of
an opening symbol, a binary relation, and a closing symbol
respectively:
\begin{verbatim}
\def\bigl{\mathopen\big}
\def\bigm{\mathrel\big} \def\bigr{\mathclose\big}
\end{verbatim}

%The `big' macros themselves put the requested delimiter and
%a null delimiter around an empty vertical box:
%\begin{verbatim}
%\def\big#1{{\nulldelimiterspace=0pt \mathsurround=0pt
%            \hbox{$\left#1\vbox to 8.5pt{}\right.$}}}
%\end{verbatim}
%As an approximate measure,
%the \n{Big} delimiters are one and a half times as large (11.5pt) as
%\n{big} delimiters; \n{bigg} ones are twice (14.5pt), and \n{Bigg}
%ones are two and a half times as large (17.5pt).
The `big' macros themselves put the requested delimiter and
a null delimiter around an empty vertical box:
\begin{verbatim}
\def\big#1{{\nulldelimiterspace=0pt \mathsurround=0pt
            \hbox{$\left#1\vbox to 8.5pt{}\right.$}}}
\end{verbatim}
As an approximate measure,
the \n{Big} delimiters are one and a half times as large (11.5pt) as
\n{big} delimiters; \n{bigg} ones are twice (14.5pt), and \n{Bigg}
ones are two and a half times as large (17.5pt).

%\index{delimiters|)}
\index{delimiters|)}

%\section{Radicals}
\section{Radicals}

%A \indexterm{radical} is a compound of a left delimiter and an overlined
%math expression.
%\cstoidx radical\par
%The overlined expression is set in the
%cramped version of the surrounding style
%\alt
%(see page~\pageref{math:styles}).
A \indexterm{radical} is a compound of a left delimiter and an overlined
math expression.
\cstoidx radical\par
The overlined expression is set in the
cramped version of the surrounding style
\alt
(see page~\pageref{math:styles}).

%In the plain format and the Computer Modern
%math fonts there is only one radical: the square root
%construct
%\begin{verbatim}
%\def\sqrt{\radical"270370 }
%\end{verbatim}
%The control sequence \cs{radical} is followed by a \gr{24-bit number}
%which specifies a small and a large variant of the left delimiter
%as was explained above. Joining the delimiter and the rule
%is done by letting the delimiter have a large depth, and a height
%which is equal to the desired rule thickness. The rule can then
%be placed on the current baseline. After the delimiter and the
%ruled expression have been joined the whole is shifted 
%vertically to achieve the usual vertical centring 
%(see Chapter~\ref{math}).
In the plain format and the Computer Modern
math fonts there is only one radical: the square root
construct
\begin{verbatim}
\def\sqrt{\radical"270370 }
\end{verbatim}
The control sequence \cs{radical} is followed by a \gr{24-bit number}
which specifies a small and a large variant of the left delimiter
as was explained above. Joining the delimiter and the rule
is done by letting the delimiter have a large depth, and a height
which is equal to the desired rule thickness. The rule can then
be placed on the current baseline. After the delimiter and the
ruled expression have been joined the whole is shifted 
vertically to achieve the usual vertical centring 
(see Chapter~\ref{math}).

%%\point Math accents
%\section{Math accents}
%\index{accents in math mode|(}
%\point Math accents
\section{Math accents}
\index{accents in math mode|(}

%Accents in math mode are specified by
%\cstoidx mathaccent\par
%\begin{disp}\cs{mathaccent}\gr{15-bit number}\gr{math field}\end{disp}
%Representing the 15-bit number as \verb>"xyzz>,
%only the family~\n{y} and the character position~\n{zz}
%are used: an accented expression acts as \cs{mathord} expression
%(see Chapter~\ref{math}).
Accents in math mode are specified by
\cstoidx mathaccent\par
\begin{disp}\cs{mathaccent}\gr{15-bit number}\gr{math field}\end{disp}
Representing the 15-bit number as \verb>"xyzz>,
only the family~\n{y} and the character position~\n{zz}
are used: an accented expression acts as \cs{mathord} expression
(see Chapter~\ref{math}).

%In math mode whole expressions can be accented,
%\alt
%whereas in text mode only characters can be accented.
%Thus in math mode accents can be stacked. However, the top
%accent may (or, more likely, will) not be properly positioned
%horizontally. Therefore the plain format has a macro \csidx{skew}
%that effectively shifts the top accent. Its definition is
%\begin{verbatim}
%\def\skew#1#2#3{{#2{#3\mkern#1mu}\mkern-#1mu}{}}
%\end{verbatim}
%and it is used for instance like
%\begin{verbatim}
%$\skew4\hat{\hat x}$
%\end{verbatim}
%\message{skew thing.}
%%which gives~{\font\tmp=cmmi10 $\textfont\VMIfam=\tmp\skew4\hat{\hat x}$}.
%which gives~{$\skew4\hat{\hat x}$}.
In math mode whole expressions can be accented,
\alt
whereas in text mode only characters can be accented.
Thus in math mode accents can be stacked. However, the top
accent may (or, more likely, will) not be properly positioned
horizontally. Therefore the plain format has a macro \csidx{skew}
that effectively shifts the top accent. Its definition is
\begin{verbatim}
\def\skew#1#2#3{{#2{#3\mkern#1mu}\mkern-#1mu}{}}
\end{verbatim}
and it is used for instance like
\begin{verbatim}
$\skew4\hat{\hat x}$
\end{verbatim}
\message{skew thing.}
%which gives~{\font\tmp=cmmi10 $\textfont\VMIfam=\tmp\skew4\hat{\hat x}$}.
which gives~{$\skew4\hat{\hat x}$}.

%For the correct positioning of accents over single characters
%the symbol and extension font have a \csidx{skewchar}:
%this is the largest accent that adds to the width of an
%accented character. Positioning of any accent
%is based on the width of the character to be accented,
%followed by the skew character. 
For the correct positioning of accents over single characters
the symbol and extension font have a \csidx{skewchar}:
this is the largest accent that adds to the width of an
accented character. Positioning of any accent
is based on the width of the character to be accented,
followed by the skew character. 

%The skew characters of the Computer Modern
%math italic and symbol fonts are character \n{\hex7F},
%\alt
%`$\mathchar"7F$',\message{skew characters}
%and \n{\hex30}, `$\mathchar"30$', respectively. The \csidx{defaultskewchar}
%value is assigned to the \cs{skewchar} when a font is loaded.
%In plain \TeX\ this is~\n{-1}, so fonts ordinarily have no
%\cs{skewchar}.
The skew characters of the Computer Modern
math italic and symbol fonts are character \n{\hex7F},
\alt
`$\mathchar"7F$',\message{skew characters}
and \n{\hex30}, `$\mathchar"30$', respectively. The \csidx{defaultskewchar}
value is assigned to the \cs{skewchar} when a font is loaded.
In plain \TeX\ this is~\n{-1}, so fonts ordinarily have no
\cs{skewchar}.

%Math accents can adapt themselves to the size of the accented
%expression: \TeX\ will look for a successor of an accent
%in the same way that it looks for a successor of a delimiter.
%In the Computer Modern math fonts this mechanism is used in
%\cstoidx widehat\par\cstoidx widetilde\par
%the \cs{widehat} and \cs{widetilde} macros.
%For example,
%\begin{disp}\verb>\widehat x>, \verb>\widehat{xy}>, \verb>\widehat{xyz}>
%\end{disp} give
%\begin{disp}$\widehat x$, $\widehat{xy}$, $\widehat{xyz}$
%\end{disp} respectively.
Math accents can adapt themselves to the size of the accented
expression: \TeX\ will look for a successor of an accent
in the same way that it looks for a successor of a delimiter.
In the Computer Modern math fonts this mechanism is used in
\cstoidx widehat\par\cstoidx widetilde\par
the \cs{widehat} and \cs{widetilde} macros.
For example,
\begin{disp}\verb>\widehat x>, \verb>\widehat{xy}>, \verb>\widehat{xyz}>
\end{disp} give
\begin{disp}$\widehat x$, $\widehat{xy}$, $\widehat{xyz}$
\end{disp} respectively.

%\index{accents in math mode|)}
%\endofchapter
\index{accents in math mode|)}
\endofchapter

\end{document}
