% -*- coding: utf-8 -*-
\documentclass[twoside,letterpaper,openright]{rapport3}

% -*- coding: utf-8 -*-

\usepackage[b5paper,text={5in,8in},centering]{geometry}

\usepackage[CJKchecksingle]{xeCJK}
\setmainfont[Mapping=tex-text]{TeX Gyre Schola}
%\setsansfont{URW Gothic L Book}
%\setmonofont{Nimbus Mono L}
\setCJKmainfont[BoldFont=FandolHei,ItalicFont=FandolKai]{FandolSong}
\setCJKsansfont{FandolHei}
\setCJKmonofont{FandolFang}
\xeCJKsetup{PunctStyle = kaiming}

\linespread{1.25}
\setlength{\parindent}{2em}

\usepackage{xcolor}
\definecolor{myblue}{rgb}{0,0.2,0.6}

\usepackage{titlesec}
\titleformat{\chapter}
    {\normalfont\Huge\sffamily\color{myblue}}
    {第\thechapter 章}
    {1em}
    {}
%\titlespacing{\chapter}{0pt}{50pt}{40pt}
\titleformat{\section}
    {\normalfont\Large\sffamily\color{myblue}}
    {\thesection}
    {1em}
    {}
%\titlespacing{\section}{0pt}{3.5ex plus 1ex minus .2ex}{2.3ex plus .2ex}
\titleformat{\subsection}
    {\normalfont\large\sffamily\color{myblue}}
    {\thesubsection}
    {1em}
    {}
%\titlespacing{\subsection}{0pt}{3.25ex plus 1ex minus .2ex}{1.5ex plus .2ex}
%
\newpagestyle{special}[\small\sffamily]{
  \headrule
  \sethead[\usepage][][\chaptertitle]
  {\chaptertitle}{}{\usepage}}
\newpagestyle{main}[\small\sffamily]{
  \headrule
  \sethead[\usepage][][第\thechapter 章\quad\chaptertitle]
  {\thesection\quad\sectiontitle}{}{\usepage}}

\usepackage{titletoc}
%\setcounter{tocdepth}{1}
%\titlecontents{标题层次}[左间距]{上间距和整体格式}{标题序号}{标题内容}{指引线和页码}[下间距]
\titlecontents{chapter}[1.5em]{\vspace{.5em}\bfseries\sffamily}{\color{myblue}\contentslabel{1.5em}}{}
    {\titlerule*[20pt]{$\cdot$}\contentspage}[]
\titlecontents{section}[4.5em]{\sffamily}{\color{myblue}\contentslabel{3em}}{}
    {\titlerule*[20pt]{$\cdot$}\contentspage}[]
%\titlecontents{subsection}[8.5em]{\sffamily}{\contentslabel{4em}}{}
%    {\titlerule*[20pt]{$\cdot$}\contentspage}

\usepackage{enumitem}
\setlist{topsep=2pt,itemsep=2pt,parsep=1pt,leftmargin=\parindent}

\usepackage{fancyvrb}
\DefineVerbatimEnvironment{verbatim}{Verbatim}
  {xleftmargin=2em,baselinestretch=1,formatcom=\color{teal}\upshape}

\usepackage{etoolbox}
\makeatletter
\preto{\FV@ListVSpace}{\topsep=2pt \partopsep=0pt }
\makeatother

\usepackage[colorlinks,plainpages,pagebackref]{hyperref}
\hypersetup{
   pdfstartview={FitH},
   citecolor=teal,
   linkcolor=myblue,
   urlcolor=black,
   bookmarksnumbered
}

\usepackage{comment,makeidx,multicol}

%\usepackage{german}
%% german
%\righthyphenmin=3
%\mdqoff
%\captionsenglish
\usepackage[english]{babel}
{\catcode`"=13 \gdef"#1{\ifx#1"\discretionary{}{}{}\fi\relax}}
\def\mdqon{\catcode`"=13\relax}
\def\mdqoff{\catcode`"=12\relax}
\makeindex
\hyphenation{ex-em-pli-fies}

\newdimen\tempdima \newdimen\tempdimb

% these are fine
\def\handbreak{\\ \message{^^JManual break!!!!^^J}}
\def\nl{\protect\\}\def\n#1{{\tt #1}}
\def\cs#1{\texorpdfstring{{\tt\char`\\#1}}{\textbackslash#1}} %\def\cs#1{{\tt\char`\\#1}}
\let\csc\cs
\def\lb{{\tt\char`\{}}\def\rb{{\tt\char`\}}}
\def\gr#1{\texorpdfstring{$\langle$#1$\rangle$}{<#1>}} %\def\gr#1{$\langle$#1$\rangle$}
\def\key#1{{\tt#1}}
\def\alt{}\def\altt{}%this way in manstijl
\def\ldash{\unskip\ --\nobreak\ \ignorespaces}
\def\rdash{\unskip\nobreak\ --\ \ignorespaces}
% check these
\def\hex{{\tt"}}
\def\ascii{{\sc ascii}}
\def\ebcdic{{\sc ebcdic}}
\def\IniTeX{Ini\TeX}\def\LamsTeX{LAMS\TeX}\def\VirTeX{Vir\TeX}
\def\AmsTeX{Ams\TeX}
\def\TeXbook{the \TeX\ book}\def\web{{\sc web}}
% needs major thinking
\newenvironment{myquote}{\list{}{%
    \topsep=2pt \partopsep=0pt%
    \leftmargin=\parindent \rightmargin=\parindent
    }\item[]}{\endlist}
\newenvironment{disp}{\begin{myquote}}{\end{myquote}}
\newenvironment{Disp}{\begin{myquote}}{\end{myquote}}
\newenvironment{tdisp}{\begin{myquote}}{\end{myquote}}
\newenvironment{example}{\begin{myquote}\noindent\itshape 例子:}{\end{myquote}}
\newenvironment{inventory}{\begin{description}\raggedright}{\end{description}}
\newenvironment{glossinventory}{\begin{description}}{\end{description}}
\def\gram#1{\gr{#1}}%???
%
% index
%
\def\indexterm#1{\emph{#1}\index{#1}}
\def\indextermsub#1#2{\emph{#1 #2}\index{#1!#2}}
\def\indextermbus#1#2{\emph{#1 #2}\index{#2!#1}}
\def\term#1\par{\index{#1}}
\def\howto#1\par{}
\def\cstoidx#1\par{\index{#1@\cs{#1}@}}
\def\thecstoidx#1\par{\index{#1@\protect\csname #1\endcsname}}
\def\thecstoidxsub#1#2{\index{#1, #2@\protect\csname #1\endcsname, #2}\ignorespaces}
\def\csterm#1\par{\cstoidx #1\par\cs{#1}}
\def\csidx#1{\cstoidx #1\par\cs{#1}}

\def\tmc{\tracingmacros=2 \tracingcommands\tracingmacros}

%%%%%%%%%%%%%%%%%%%
\makeatletter
\def\snugbox{\hbox\bgroup\setbox\z@\vbox\bgroup
    \leftskip\z@
    \bgroup\aftergroup\make@snug
    \let\next=}
\def\make@snug{\par\sn@gify\egroup \box\z@\egroup}
\def\sn@gify
   {\skip\z@=\lastskip \unskip
    \advance\skip\z@\lastskip \unskip
    \unpenalty
    \setbox\z@\lastbox
    \ifvoid\z@ \nointerlineskip \else {\sn@gify} \fi
    \hbox{\unhbox\z@}\nointerlineskip
    \vskip\skip\z@
    }

\newdimen\fbh \fbh=60pt % dimension for easy scaling:
\newdimen\fbw \fbw=60pt % height and width of character box

\newdimen\dh \newdimen\dw % height and width of current character box
\newdimen\lh % height of previous character box
\newdimen\lw \lw=.4pt % line weight, instead of default .4pt

\def\hdotfill{\noindent
    \leaders\hbox{\vrule width 1pt height\lw
                  \kern4pt
                  \vrule width.5pt height\lw}\hfill\hbox{}
    \par}
\def\hlinefill{\noindent
    \leaders\hbox{\vrule width 5.5pt height\lw         }\hfill\hbox{}
    \par}
\def\stippel{$\qquad\qquad\qquad\qquad$}
\makeatother
%%%%%%%%%%%%%%%%%%%

%\def\SansSerif{\Typeface:macHelvetica }
%\def\SerifFont{\Typeface:macTimes }
%\def\SansSerif{\Typeface:bsGillSans }
%\def\SerifFont{\Typeface:bsBaskerville }
\let\SansSerif\relax \def\italic{\it}
\let\SerifFont\relax \def\MainFont{\rm}
\let\SansSerif\relax
\let\SerifFont\relax
\let\PopIndentLevel\relax \let\PushIndentLevel\relax
\let\ToVerso\relax \let\ToRecto\relax

%\def\stop@command@suffix{stop}
%\let\PopListLevel\PopIndentLevel
%\let\FlushRight\relax
%\let\flushright\FlushRight
%\let\SetListIndent\LevelIndent
%\def\awp{\ifhmode\vadjust{\penalty-10000 }\else
%    \penalty-10000 \fi}
\let\awp\relax
\let\PopIndentLevel\relax \let\PopListLevel\relax

\showboxdepth=-1

%\input figs
\def\endofchapter{\vfill\noindent}

\setcounter{chapter}{14}

\begin{document}

%\chapter{Baseline Distances}\label{baseline}
\chapter{Baseline Distances}\label{baseline}

%%\hbox{}\vfil\vfil\hbox{}
%Lines of text are in most cases not of equal height or depth.
%Therefore \TeX\ adds interline glue to keep baselines at a uniform
%distance from one another. 
%This chapter treats the computation of such
%interline glue.
%\hbox{}\vfil\vfil\hbox{}
Lines of text are in most cases not of equal height or depth.
Therefore \TeX\ adds interline glue to keep baselines at a uniform
distance from one another. 
This chapter treats the computation of such
interline glue.

%\label{cschap:baselineskip}\label{cschap:lineskiplimit}\label{cschap:lineskip}\label{cschap:prevdepth}\label{cschap:nointerlineskip}\label{cschap:offinterlineskip}\label{cschap:openup}
%\begin{inventory}
\label{cschap:baselineskip}\label{cschap:lineskiplimit}\label{cschap:lineskip}\label{cschap:prevdepth}\label{cschap:nointerlineskip}\label{cschap:offinterlineskip}\label{cschap:openup}
\begin{inventory}

%\item [\cs{baselineskip}] 
%      The `ideal' baseline distance between neighbouring 
%      boxes on a vertical list. Plain \TeX\ default:~\n{12pt}.
\item [\cs{baselineskip}] 
      The `ideal' baseline distance between neighbouring 
      boxes on a vertical list. Plain \TeX\ default:~\n{12pt}.

%\item [\cs{lineskiplimit}] 
%      Distance to be maintained between the bottom and top of 
%      neighbouring boxes on a vertical list.
%      Plain \TeX\ default:~\n{0pt}.
\item [\cs{lineskiplimit}] 
      Distance to be maintained between the bottom and top of 
      neighbouring boxes on a vertical list.
      Plain \TeX\ default:~\n{0pt}.

%\item [\cs{lineskip}]  
%      Glue added if the distance between bottom
%      and top of neighbouring boxes 
%      is less than \cs{lineskiplimit}.
%      Plain \TeX\ default:~\n{1pt}.
\item [\cs{lineskip}]  
      Glue added if the distance between bottom
      and top of neighbouring boxes 
      is less than \cs{lineskiplimit}.
      Plain \TeX\ default:~\n{1pt}.

%\item [\cs{prevdepth}]  
%      Depth of the last box added to a vertical list as it is 
%      perceived by \TeX.
\item [\cs{prevdepth}]  
      Depth of the last box added to a vertical list as it is 
      perceived by \TeX.

%\item [\cs{nointerlineskip}]
%      Macro to prevent interline glue insertion once.
\item [\cs{nointerlineskip}]
      Macro to prevent interline glue insertion once.

%\item [\cs{offinterlineskip}]
%      Macro to prevent interline glue globally
%      henceforth.
\item [\cs{offinterlineskip}]
      Macro to prevent interline glue globally
      henceforth.

%\item [\cs{openup}]
%      Increase \cs{baselineskip}, \cs{lineskip}, 
%      and \cs{lineskiplimit} by specified amount.
\item [\cs{openup}]
      Increase \cs{baselineskip}, \cs{lineskip}, 
      and \cs{lineskiplimit} by specified amount.

%\end{inventory}
\end{inventory}


%\section{Interline glue}
\section{Interline glue}

%\input figs17
%%\message{fig17 missing}
\input figs17
%\message{fig17 missing}

%\TeX\ tries to keep a certain distance between the reference
%points of boxes that are added to a vertical list
%by inserting \indextermbus{interline}{glue}.
%In particular it tries to keep a constant \indextermsub{baseline}{distance}
%between lines of ordinary text: the \csidx{baselineskip}. Actually,
%the \cs{baselineskip} is  a \gr{glue}, so line distances can
%stretch or shrink. However, the natural sizes,
%as well as the stretch and the shrink, are the same
%between all lines.
\TeX\ tries to keep a certain distance between the reference
points of boxes that are added to a vertical list
by inserting \indextermbus{interline}{glue}.
In particular it tries to keep a constant \indextermsub{baseline}{distance}
between lines of ordinary text: the \csidx{baselineskip}. Actually,
the \cs{baselineskip} is  a \gr{glue}, so line distances can
stretch or shrink. However, the natural sizes,
as well as the stretch and the shrink, are the same
between all lines.

%When boxes, whether they are lines of a paragraph or explicit boxes,
%are appended to a vertical list, glue 
%is added usually so that the depth of the preceding box
%and the height of the current one add up to the \cs{baselineskip}.
%This has the effect of keeping the reference points 
%of subsequent lines at regular intervals.
When boxes, whether they are lines of a paragraph or explicit boxes,
are appended to a vertical list, glue 
is added usually so that the depth of the preceding box
and the height of the current one add up to the \cs{baselineskip}.
This has the effect of keeping the reference points 
of subsequent lines at regular intervals.

%%\eject
%\eject

%\begin{quotation}
%  \figone
%\end{quotation}
%%\begin{disp}\leavevmode\hbox{}\nl\figone\end{disp}
\begin{quotation}
  \figone
\end{quotation}
%\begin{disp}\leavevmode\hbox{}\nl\figone\end{disp}

%However, this process can bring the bottom and top of two
%subsequent boxes to be less than \cs{lineskiplimit} apart:
%\begin{quotation}
%  \figtwo
%\end{quotation}
%%\begin{disp}\leavevmode\figtwo\end{disp}
However, this process can bring the bottom and top of two
subsequent boxes to be less than \cs{lineskiplimit} apart:
\begin{quotation}
  \figtwo
\end{quotation}
%\begin{disp}\leavevmode\figtwo\end{disp}

%In that case, \cs{lineskip} glue is added:
%\begin{quotation}
%  \figthree
%\end{quotation}
%%\begin{disp}\leavevmode\figthree\end{disp}
%Note that this will usually increase the distance
%between the baselines of the boxes to more than the
%\cs{baselineskip}.
In that case, \cs{lineskip} glue is added:
\begin{quotation}
  \figthree
\end{quotation}
%\begin{disp}\leavevmode\figthree\end{disp}
Note that this will usually increase the distance
between the baselines of the boxes to more than the
\cs{baselineskip}.

%The exact process is this:
%\begin{itemize}
%\item if \cs{prevdepth} is \n{-1000pt} or less,
%no glue is added, otherwise
%\item \TeX\ calculates the distance between the bottom of the previous box
%and the top of the current one as the natural width of the
%\cs{baselineskip} minus \cs{prev\-depth} (the
%depth of the last box) and minus the height of the current box;
%\item if this distance is at least \csidx{lineskiplimit}, 
%glue is added with the calculated distance as natural size,
%and with the stretch and shrink of the \cs{baselineskip},
%\item otherwise \csidx{lineskip} glue is added.
%\item \csidx{prevdepth} is set to the depth of the
%current item.
%\end{itemize}
The exact process is this:
\begin{itemize}
\item if \cs{prevdepth} is \n{-1000pt} or less,
no glue is added, otherwise
\item \TeX\ calculates the distance between the bottom of the previous box
and the top of the current one as the natural width of the
\cs{baselineskip} minus \cs{prev\-depth} (the
depth of the last box) and minus the height of the current box;
\item if this distance is at least \csidx{lineskiplimit}, 
glue is added with the calculated distance as natural size,
and with the stretch and shrink of the \cs{baselineskip},
\item otherwise \csidx{lineskip} glue is added.
\item \csidx{prevdepth} is set to the depth of the
current item.
\end{itemize}

%There are two exceptional  situations:
%no interline glue is added before and after a rule,
%and the \cs{prevdepth} is not updated by an \cs{unvbox}
%or \cs{unvcopy} command. After a rule interline glue
%is prevented by a value of \n{-1000pt} of the \cs{prevdepth}.
There are two exceptional  situations:
no interline glue is added before and after a rule,
and the \cs{prevdepth} is not updated by an \cs{unvbox}
or \cs{unvcopy} command. After a rule interline glue
is prevented by a value of \n{-1000pt} of the \cs{prevdepth}.

%The above process is carried out, irrespective of what extra
%glue may have been inserted in between the boxes.
%Thus a skip in between boxes in vertical mode will not
%affect the distance calculated from the baseline distances,
%and therefore also not the amount of baselineskip glue.
%The same holds for glue added with \cs{vadjust} inside
%a paragraph.
The above process is carried out, irrespective of what extra
glue may have been inserted in between the boxes.
Thus a skip in between boxes in vertical mode will not
affect the distance calculated from the baseline distances,
and therefore also not the amount of baselineskip glue.
The same holds for glue added with \cs{vadjust} inside
a paragraph.

%\begin{example}
%\begin{verbatim}
%\baselineskip=10pt \lineskiplimit=2pt \lineskip=2pt
%\setbox0=\vbox{\hbox{\vrule depth4pt}
%               \hbox{\vrule height 3pt}}
%\showbox0
%\end{verbatim}
%gives
%\begin{verbatim}
%\box0=
%\vbox(10.0+0.0)x0.4
%.\hbox(0.0+4.0)x0.4
%..\rule(*+4.0)x0.4
%.\glue(\baselineskip) 3.0
%.\hbox(3.0+0.0)x0.4
%..\rule(3.0+*)x0
%\end{verbatim}
%Bringing the boxes to within  \cs{lineskiplimit}
%of each other, that is
%\begin{verbatim}
%\setbox0\vbox{\hbox{\vrule depth4pt}
%              \hbox{\vrule height 5pt}}
%\showbox0
%\end{verbatim}
%gives
%\begin{verbatim}
%\box0=
%\vbox(11.0+0.0)x0.4
%.\hbox(0.0+4.0)x0.4
%..\rule(*+4.0)x0.4
%.\glue(\lineskip) 2.0
%.\hbox(5.0+0.0)x0.4
%..\rule(5.0+*)x0.4
%\end{verbatim}
%where \cs{lineskip} glue has been inserted
%instead of the usual \cs{baselineskip} glue.
%\end{example}
\begin{example}
\begin{verbatim}
\baselineskip=10pt \lineskiplimit=2pt \lineskip=2pt
\setbox0=\vbox{\hbox{\vrule depth4pt}
               \hbox{\vrule height 3pt}}
\showbox0
\end{verbatim}
gives
\begin{verbatim}
\box0=
\vbox(10.0+0.0)x0.4
.\hbox(0.0+4.0)x0.4
..\rule(*+4.0)x0.4
.\glue(\baselineskip) 3.0
.\hbox(3.0+0.0)x0.4
..\rule(3.0+*)x0
\end{verbatim}
Bringing the boxes to within  \cs{lineskiplimit}
of each other, that is
\begin{verbatim}
\setbox0\vbox{\hbox{\vrule depth4pt}
              \hbox{\vrule height 5pt}}
\showbox0
\end{verbatim}
gives
\begin{verbatim}
\box0=
\vbox(11.0+0.0)x0.4
.\hbox(0.0+4.0)x0.4
..\rule(*+4.0)x0.4
.\glue(\lineskip) 2.0
.\hbox(5.0+0.0)x0.4
..\rule(5.0+*)x0.4
\end{verbatim}
where \cs{lineskip} glue has been inserted
instead of the usual \cs{baselineskip} glue.
\end{example}

%The plain \TeX\ default values are
%\begin{verbatim}
%\lineskiplimit=0pt lineskip=1pt
%\end{verbatim}
%so, when boxes start to touch each other, they are
%moved one point apart.
The plain \TeX\ default values are
\begin{verbatim}
\lineskiplimit=0pt lineskip=1pt
\end{verbatim}
so, when boxes start to touch each other, they are
moved one point apart.

%\section{The perceived depth of boxes}
\section{The perceived depth of boxes}

%The decision process for interline glue uses \csidx{prevdepth}
%as the perceived depth of the preceding box on the vertical
%list. The \cs{prevdepth} parameter can be used only in
%vertical mode.
The decision process for interline glue uses \csidx{prevdepth}
as the perceived depth of the preceding box on the vertical
list. The \cs{prevdepth} parameter can be used only in
vertical mode.

%The \cs{prevdepth} is set to the depth of boxes added to the
%vertical list, but it is not affected by \cs{unvbox}
%or \cs{unvcopy}. After an \cs{hrule} it is set to
%\n{-1000pt} to prevent interline glue before the next box.
The \cs{prevdepth} is set to the depth of boxes added to the
vertical list, but it is not affected by \cs{unvbox}
or \cs{unvcopy}. After an \cs{hrule} it is set to
\n{-1000pt} to prevent interline glue before the next box.

%At the beginning of a vertical list \cs{prevdepth}
%is set to \n{-1000pt}, except in an \cs{halign}
%and \cs{noalign} code contained therein, where it
%is carried over from the surrounding list.
%At the end of the alignment the value of \cs{prevdepth}
%set by the last alignment row is carried to the outer list.
At the beginning of a vertical list \cs{prevdepth}
is set to \n{-1000pt}, except in an \cs{halign}
and \cs{noalign} code contained therein, where it
is carried over from the surrounding list.
At the end of the alignment the value of \cs{prevdepth}
set by the last alignment row is carried to the outer list.

%In order to prevent interline glue just once, all that
%is needed is to alter the \cs{prevdepth}.
%\cstoidx nointerlineskip\par
%\begin{verbatim}
%\def\nointerlineskip{\prevdepth=-1000pt}
%\end{verbatim}
In order to prevent interline glue just once, all that
is needed is to alter the \cs{prevdepth}.
\cstoidx nointerlineskip\par
\begin{verbatim}
\def\nointerlineskip{\prevdepth=-1000pt}
\end{verbatim}

%The \csidx{offinterlineskip} macro is much more drastic:
%it prevents {\sl all\/} interline glue from the moment
%of its call onwards, or, if it is used inside a paragraph,
%from the start of that paragraph.
%Its definition is
%\begin{verbatim}
%\baselineskip=-1000pt \lineskip=0pt 
%\lineskiplimit\maxdimen
%\end{verbatim}
%where the second line is the essential one: it
%causes \TeX\ to add \cs{lineskip} glue (which is zero)
%always. 
%Settings for \cs{baselineskip} do not matter any more then.
The \csidx{offinterlineskip} macro is much more drastic:
it prevents {\sl all\/} interline glue from the moment
of its call onwards, or, if it is used inside a paragraph,
from the start of that paragraph.
Its definition is
\begin{verbatim}
\baselineskip=-1000pt \lineskip=0pt 
\lineskiplimit\maxdimen
\end{verbatim}
where the second line is the essential one: it
causes \TeX\ to add \cs{lineskip} glue (which is zero)
always. 
Settings for \cs{baselineskip} do not matter any more then.

%The \cs{offinterlineskip} macro has an important application
%in alignments (see Chapter~\ref{align}).
The \cs{offinterlineskip} macro has an important application
in alignments (see Chapter~\ref{align}).

%By setting
%\begin{verbatim}
%\lineskiplimit=-\maxdimen
%\end{verbatim}
%you can force \TeX\  to apply the \cs{baselineskip}
%always, regardless of whether this would bring boxes too close
%together or, indeed, if this would make them overlap.
By setting
\begin{verbatim}
\lineskiplimit=-\maxdimen
\end{verbatim}
you can force \TeX\  to apply the \cs{baselineskip}
always, regardless of whether this would bring boxes too close
together or, indeed, if this would make them overlap.

%%\point Terminology
%\section{Terminology}
%\point Terminology
\section{Terminology}

%In hot metal typesetting, all letters of a particular font
%were on a `body' of the same
%size. Thus every line of type had the same height and depth, and
%the resulting distance between the baselines would be some suitable
%value for that type. If for some reason this distance should
%be larger (see~\cite{White:line} for a discussion of this),
%strips of lead would be inserted. The extra distance was
%called the `leading' (pronounced `ledding').
In hot metal typesetting, all letters of a particular font
were on a `body' of the same
size. Thus every line of type had the same height and depth, and
the resulting distance between the baselines would be some suitable
value for that type. If for some reason this distance should
be larger (see~\cite{White:line} for a discussion of this),
strips of lead would be inserted. The extra distance was
called the `leading' (pronounced `ledding').

%With phototypesetting, when the baseline distance was sometimes
%called the `film transport', this terminology blurred, and the
%term `leading' was also used for the baseline distance. Some of this
%confusion is also present in \TeX: the parameter \cs{baselineskip}
%specifies the baseline distance, but in the trace output
%(see the examples above) the glue inserted to make the 
%baseline distance equal to \cs{baselineskip} is called
%\cs{baselineskip}.
With phototypesetting, when the baseline distance was sometimes
called the `film transport', this terminology blurred, and the
term `leading' was also used for the baseline distance. Some of this
confusion is also present in \TeX: the parameter \cs{baselineskip}
specifies the baseline distance, but in the trace output
(see the examples above) the glue inserted to make the 
baseline distance equal to \cs{baselineskip} is called
\cs{baselineskip}.

%%\point Additional remarks
%\section{Additional remarks}
%\point Additional remarks
\section{Additional remarks}

%In general, for documents longer than one page it is desirable
%to have the same baseline distance throughout. However,
%for one-page documents you may add stretchability to the
%baselineskip, for instance if the text has to be flush bottom.
In general, for documents longer than one page it is desirable
to have the same baseline distance throughout. However,
for one-page documents you may add stretchability to the
baselineskip, for instance if the text has to be flush bottom.

%Increasing the distance between just one pair of lines
%can be done with \cs{vadjust}. The argument of this
%command is vertical material that\vadjust{\kern2pt}
%will be inserted in the
%vertical list right after the line where this command was given.
%The second line of this paragraph, for instance, 
%contains the command \verb-\vadjust{\kern2pt}-.
Increasing the distance between just one pair of lines
can be done with \cs{vadjust}. The argument of this
command is vertical material that\vadjust{\kern2pt}
will be inserted in the
vertical list right after the line where this command was given.
The second line of this paragraph, for instance, 
contains the command \verb-\vadjust{\kern2pt}-.

%The amount of leading cannot be changed in the middle of
%a paragraph, because the value for \cs{baselineskip}
%that is used is the one that
%is current when the paragraph is finally broken and
%added to the main vertical list. The same holds
%for the \cs{lineskip} and \cs{lineskiplimit}.
The amount of leading cannot be changed in the middle of
a paragraph, because the value for \cs{baselineskip}
that is used is the one that
is current when the paragraph is finally broken and
added to the main vertical list. The same holds
for the \cs{lineskip} and \cs{lineskiplimit}.

%The plain \TeX\ macro \csidx{openup}
%increases the \cs{baselineskip}, \cs{lineskip}, and
%\cs{lineskiplimit} by the amount of the argument
%to the macro. In effect, this increases line distances
%by this amount regardless of whether they are governed
%by \cs{baselineskip} or \cs{lineskip}.
The plain \TeX\ macro \csidx{openup}
increases the \cs{baselineskip}, \cs{lineskip}, and
\cs{lineskiplimit} by the amount of the argument
to the macro. In effect, this increases line distances
by this amount regardless of whether they are governed
by \cs{baselineskip} or \cs{lineskip}.


%\endofchapter
%%%%% end of input file [baseline]
\endofchapter
%%%% end of input file [baseline]

\end{document}
