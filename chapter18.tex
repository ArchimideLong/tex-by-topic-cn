% -*- coding: utf-8 -*-
\documentclass{book}

% -*- coding: utf-8 -*-

\usepackage[b5paper,text={5in,8in},centering]{geometry}

\usepackage[CJKchecksingle]{xeCJK}
\setmainfont[Mapping=tex-text]{TeX Gyre Schola}
%\setsansfont{URW Gothic L Book}
%\setmonofont{Nimbus Mono L}
\setCJKmainfont[BoldFont=FandolHei,ItalicFont=FandolKai]{FandolSong}
\setCJKsansfont{FandolHei}
\setCJKmonofont{FandolFang}
\xeCJKsetup{PunctStyle = kaiming}

\linespread{1.25}
\setlength{\parindent}{2em}

\usepackage{xcolor}
\definecolor{myblue}{rgb}{0,0.2,0.6}

\usepackage{titlesec}
\titleformat{\chapter}
    {\normalfont\Huge\sffamily\color{myblue}}
    {第\thechapter 章}
    {1em}
    {}
%\titlespacing{\chapter}{0pt}{50pt}{40pt}
\titleformat{\section}
    {\normalfont\Large\sffamily\color{myblue}}
    {\thesection}
    {1em}
    {}
%\titlespacing{\section}{0pt}{3.5ex plus 1ex minus .2ex}{2.3ex plus .2ex}
\titleformat{\subsection}
    {\normalfont\large\sffamily\color{myblue}}
    {\thesubsection}
    {1em}
    {}
%\titlespacing{\subsection}{0pt}{3.25ex plus 1ex minus .2ex}{1.5ex plus .2ex}
%
\newpagestyle{special}[\small\sffamily]{
  \headrule
  \sethead[\usepage][][\chaptertitle]
  {\chaptertitle}{}{\usepage}}
\newpagestyle{main}[\small\sffamily]{
  \headrule
  \sethead[\usepage][][第\thechapter 章\quad\chaptertitle]
  {\thesection\quad\sectiontitle}{}{\usepage}}

\usepackage{titletoc}
%\setcounter{tocdepth}{1}
%\titlecontents{标题层次}[左间距]{上间距和整体格式}{标题序号}{标题内容}{指引线和页码}[下间距]
\titlecontents{chapter}[1.5em]{\vspace{.5em}\bfseries\sffamily}{\color{myblue}\contentslabel{1.5em}}{}
    {\titlerule*[20pt]{$\cdot$}\contentspage}[]
\titlecontents{section}[4.5em]{\sffamily}{\color{myblue}\contentslabel{3em}}{}
    {\titlerule*[20pt]{$\cdot$}\contentspage}[]
%\titlecontents{subsection}[8.5em]{\sffamily}{\contentslabel{4em}}{}
%    {\titlerule*[20pt]{$\cdot$}\contentspage}

\usepackage{enumitem}
\setlist{topsep=2pt,itemsep=2pt,parsep=1pt,leftmargin=\parindent}

\usepackage{fancyvrb}
\DefineVerbatimEnvironment{verbatim}{Verbatim}
  {xleftmargin=2em,baselinestretch=1,formatcom=\color{teal}\upshape}

\usepackage{etoolbox}
\makeatletter
\preto{\FV@ListVSpace}{\topsep=2pt \partopsep=0pt }
\makeatother

\usepackage[colorlinks,plainpages,pagebackref]{hyperref}
\hypersetup{
   pdfstartview={FitH},
   citecolor=teal,
   linkcolor=myblue,
   urlcolor=black,
   bookmarksnumbered
}

\usepackage{comment,makeidx,multicol}

%\usepackage{german}
%% german
%\righthyphenmin=3
%\mdqoff
%\captionsenglish
\usepackage[english]{babel}
{\catcode`"=13 \gdef"#1{\ifx#1"\discretionary{}{}{}\fi\relax}}
\def\mdqon{\catcode`"=13\relax}
\def\mdqoff{\catcode`"=12\relax}
\makeindex
\hyphenation{ex-em-pli-fies}

\newdimen\tempdima \newdimen\tempdimb

% these are fine
\def\handbreak{\\ \message{^^JManual break!!!!^^J}}
\def\nl{\protect\\}\def\n#1{{\tt #1}}
\def\cs#1{\texorpdfstring{{\tt\char`\\#1}}{\textbackslash#1}} %\def\cs#1{{\tt\char`\\#1}}
\let\csc\cs
\def\lb{{\tt\char`\{}}\def\rb{{\tt\char`\}}}
\def\gr#1{\texorpdfstring{$\langle$#1$\rangle$}{<#1>}} %\def\gr#1{$\langle$#1$\rangle$}
\def\key#1{{\tt#1}}
\def\alt{}\def\altt{}%this way in manstijl
\def\ldash{\unskip\ --\nobreak\ \ignorespaces}
\def\rdash{\unskip\nobreak\ --\ \ignorespaces}
% check these
\def\hex{{\tt"}}
\def\ascii{{\sc ascii}}
\def\ebcdic{{\sc ebcdic}}
\def\IniTeX{Ini\TeX}\def\LamsTeX{LAMS\TeX}\def\VirTeX{Vir\TeX}
\def\AmsTeX{Ams\TeX}
\def\TeXbook{the \TeX\ book}\def\web{{\sc web}}
% needs major thinking
\newenvironment{myquote}{\list{}{%
    \topsep=2pt \partopsep=0pt%
    \leftmargin=\parindent \rightmargin=\parindent
    }\item[]}{\endlist}
\newenvironment{disp}{\begin{myquote}}{\end{myquote}}
\newenvironment{Disp}{\begin{myquote}}{\end{myquote}}
\newenvironment{tdisp}{\begin{myquote}}{\end{myquote}}
\newenvironment{example}{\begin{myquote}\noindent\itshape 例子:}{\end{myquote}}
\newenvironment{inventory}{\begin{description}\raggedright}{\end{description}}
\newenvironment{glossinventory}{\begin{description}}{\end{description}}
\def\gram#1{\gr{#1}}%???
%
% index
%
\def\indexterm#1{\emph{#1}\index{#1}}
\def\indextermsub#1#2{\emph{#1 #2}\index{#1!#2}}
\def\indextermbus#1#2{\emph{#1 #2}\index{#2!#1}}
\def\term#1\par{\index{#1}}
\def\howto#1\par{}
\def\cstoidx#1\par{\index{#1@\cs{#1}@}}
\def\thecstoidx#1\par{\index{#1@\protect\csname #1\endcsname}}
\def\thecstoidxsub#1#2{\index{#1, #2@\protect\csname #1\endcsname, #2}\ignorespaces}
\def\csterm#1\par{\cstoidx #1\par\cs{#1}}
\def\csidx#1{\cstoidx #1\par\cs{#1}}

\def\tmc{\tracingmacros=2 \tracingcommands\tracingmacros}

%%%%%%%%%%%%%%%%%%%
\makeatletter
\def\snugbox{\hbox\bgroup\setbox\z@\vbox\bgroup
    \leftskip\z@
    \bgroup\aftergroup\make@snug
    \let\next=}
\def\make@snug{\par\sn@gify\egroup \box\z@\egroup}
\def\sn@gify
   {\skip\z@=\lastskip \unskip
    \advance\skip\z@\lastskip \unskip
    \unpenalty
    \setbox\z@\lastbox
    \ifvoid\z@ \nointerlineskip \else {\sn@gify} \fi
    \hbox{\unhbox\z@}\nointerlineskip
    \vskip\skip\z@
    }

\newdimen\fbh \fbh=60pt % dimension for easy scaling:
\newdimen\fbw \fbw=60pt % height and width of character box

\newdimen\dh \newdimen\dw % height and width of current character box
\newdimen\lh % height of previous character box
\newdimen\lw \lw=.4pt % line weight, instead of default .4pt

\def\hdotfill{\noindent
    \leaders\hbox{\vrule width 1pt height\lw
                  \kern4pt
                  \vrule width.5pt height\lw}\hfill\hbox{}
    \par}
\def\hlinefill{\noindent
    \leaders\hbox{\vrule width 5.5pt height\lw         }\hfill\hbox{}
    \par}
\def\stippel{$\qquad\qquad\qquad\qquad$}
\makeatother
%%%%%%%%%%%%%%%%%%%

%\def\SansSerif{\Typeface:macHelvetica }
%\def\SerifFont{\Typeface:macTimes }
%\def\SansSerif{\Typeface:bsGillSans }
%\def\SerifFont{\Typeface:bsBaskerville }
\let\SansSerif\relax \def\italic{\it}
\let\SerifFont\relax \def\MainFont{\rm}
\let\SansSerif\relax
\let\SerifFont\relax
\let\PopIndentLevel\relax \let\PushIndentLevel\relax
\let\ToVerso\relax \let\ToRecto\relax

%\def\stop@command@suffix{stop}
%\let\PopListLevel\PopIndentLevel
%\let\FlushRight\relax
%\let\flushright\FlushRight
%\let\SetListIndent\LevelIndent
%\def\awp{\ifhmode\vadjust{\penalty-10000 }\else
%    \penalty-10000 \fi}
\let\awp\relax
\let\PopIndentLevel\relax \let\PopListLevel\relax

\showboxdepth=-1

%\input figs
\def\endofchapter{\vfill\noindent}

\setcounter{chapter}{17}

\begin{document}

%\chapter{Paragraph Shape}\label{par:shape}
%\index{paragraph!shape|(}
\chapter{Paragraph Shape}\label{par:shape}
\index{paragraph!shape|(}

%This chapter discusses the parameters and commands that influence the
%shape of a paragraph.
This chapter discusses the parameters and commands that influence the
shape of a paragraph.

%\label{cschap:parindent2}\label{cschap:hsize2}\label{cschap:leftskip}\label{cschap:rightskip}\label{cschap:hangindent}\label{cschap:hangafter}\label{cschap:parshape}
%\begin{inventory}
%\item [\cs{parindent}] 
%      Width of the indentation box added in front of a paragraph.
%      Plain \TeX\ default:~\n{20pt}.
\label{cschap:parindent2}\label{cschap:hsize2}\label{cschap:leftskip}\label{cschap:rightskip}\label{cschap:hangindent}\label{cschap:hangafter}\label{cschap:parshape}
\begin{inventory}
\item [\cs{parindent}] 
      Width of the indentation box added in front of a paragraph.
      Plain \TeX\ default:~\n{20pt}.

%\item [\cs{hsize}] 
%      Line width used for typesetting a paragraph.
%      Plain \TeX\ default:~\n{6.5in}.
\item [\cs{hsize}] 
      Line width used for typesetting a paragraph.
      Plain \TeX\ default:~\n{6.5in}.

%\item [\cs{leftskip}] 
%      Glue that is placed to the left of all lines of a paragraph.
\item [\cs{leftskip}] 
      Glue that is placed to the left of all lines of a paragraph.


%\item [\cs{rightskip}] 
%      Glue that is placed to the right of all lines of a paragraph.
\item [\cs{rightskip}] 
      Glue that is placed to the right of all lines of a paragraph.


%\item [\cs{hangindent}]   
%      If positive, this indicates indentation from the left margin; 
%      if negative, this is the negative of the indentation 
%      from the right margin.
\item [\cs{hangindent}]   
      If positive, this indicates indentation from the left margin; 
      if negative, this is the negative of the indentation 
      from the right margin.

%\item [\cs{hangafter}]   
%      If positive, this denotes the number of lines 
%      before indenting starts; 
%      if negative, the absolute value of this is the number 
%      of indented lines starting with the first line of the paragraph.
%      Default:~\n1.
\item [\cs{hangafter}]   
      If positive, this denotes the number of lines 
      before indenting starts; 
      if negative, the absolute value of this is the number 
      of indented lines starting with the first line of the paragraph.
      Default:~\n1.

%\item [\cs{parshape}]
%      Command for general paragraph shapes.
\item [\cs{parshape}]
      Command for general paragraph shapes.

%\end{inventory}
\end{inventory}


%\begin{figure}[ht]
%  \input figs20
%\end{figure}
\begin{figure}[ht]
  \input figs20
\end{figure}

%%\point The width of text lines
%\section{The width of text lines}
%\index{line!width|(}
%\point The width of text lines
\section{The width of text lines}
\index{line!width|(}

%When \TeX\ has finished absorbing a paragraph, 
%it has formed a horizontal list, starting with an indentation
%box, and ending with \cs{parfillskip} glue.
%This list is then broken into lines of length \cs{hsize}.
%\cstoidx hsize\par\cstoidx leftskip\par\cstoidx rightskip\par
%Each line of a paragraph is padded left and right with
%certain amounts of glue, the \cs{leftskip} and \cs{rightskip},
%which are taken into account in reaching \cs{hsize}.
When \TeX\ has finished absorbing a paragraph, 
it has formed a horizontal list, starting with an indentation
box, and ending with \cs{parfillskip} glue.
This list is then broken into lines of length \cs{hsize}.
\cstoidx hsize\par\cstoidx leftskip\par\cstoidx rightskip\par
Each line of a paragraph is padded left and right with
certain amounts of glue, the \cs{leftskip} and \cs{rightskip},
which are taken into account in reaching \cs{hsize}.

%The values of \cs{leftskip} and \cs{rightskip} are taken 
%into account in the line-breaking algorithm.
%Thus the main point about the \csidx{raggedright} 
%macro in plain \TeX\ and the \LaTeX\ `flushleft'
%environment is that they
%set the \cs{rightskip} to  zero plus some stretch.
The values of \cs{leftskip} and \cs{rightskip} are taken 
into account in the line-breaking algorithm.
Thus the main point about the \csidx{raggedright} 
macro in plain \TeX\ and the \LaTeX\ `flushleft'
environment is that they
set the \cs{rightskip} to  zero plus some stretch.

%The commands \cs{parshape} and \cs{hangindent}
%also affect line width. They work by altering the
%\cs{hsize} and afterwards shifting the boxes 
%containing the lines.
The commands \cs{parshape} and \cs{hangindent}
also affect line width. They work by altering the
\cs{hsize} and afterwards shifting the boxes 
containing the lines.

%\index{line!width|)}
\index{line!width|)}

%%\point Shape parameters
%\section{Shape parameters}
%\point Shape parameters
\section{Shape parameters}

%%\spoint Hanging indentation
%\subsection{Hanging indentation}
%\index{indentation!hanging|(}
%\spoint Hanging indentation
\subsection{Hanging indentation}
\index{indentation!hanging|(}

%\message{twolines?}
%A simple, and frequently occurring, paragraph shape is that
%\cstoidx hangafter\par\cstoidx hangindent\par
%with a number of starting or trailing lines indented.
%\TeX\ can realize such shapes using two parameters:
%\cs{hangafter} and \cs{hangindent}.
%Both can assume positive and negative values.
\message{twolines?}
A simple, and frequently occurring, paragraph shape is that
\cstoidx hangafter\par\cstoidx hangindent\par
with a number of starting or trailing lines indented.
\TeX\ can realize such shapes using two parameters:
\cs{hangafter} and \cs{hangindent}.
Both can assume positive and negative values.

%The \cs{hangindent} controls the amount of indentation:
%\begin{itemize}\item \cs{hangindent}${}>0$: the paragraph
%is indented at the left margin by this amount.
%\item\cs{hangindent}${}<0$: the paragraph is indented
%at the right margin by the absolute value of this amount.
%\end{itemize}
%\def\exnul{\leftskip=0pt \rightskip=0pt \relax}
%For example (assume \cs{parindent=0pt}),
%\begin{disp}\leavevmode\message{Check parshape example!}%
%\hbox{%\Distance:verbatimwhiteleft=0pt
%$\vcenter{\hsize=3.5in \snugbox{\begin{verbatim}
% a a a a a a a a a a a a ...
%
% \hangindent=10pt
% a a a a a a a a a a a a ...
%
% \hangindent=-10pt
% a a a a a a a a a a a a ...
%\end{verbatim}
%}}$\quad gives\quad\quad
The \cs{hangindent} controls the amount of indentation:
\begin{itemize}\item \cs{hangindent}${}>0$: the paragraph
is indented at the left margin by this amount.
\item\cs{hangindent}${}<0$: the paragraph is indented
at the right margin by the absolute value of this amount.
\end{itemize}
\def\exnul{\leftskip=0pt \rightskip=0pt \relax}
For example (assume \cs{parindent=0pt}),
\begin{disp}\leavevmode\message{Check parshape example!}%
\hbox{%\Distance:verbatimwhiteleft=0pt
$\vcenter{\hsize=3.5in \snugbox{\begin{verbatim}
 a a a a a a a a a a a a ...

 \hangindent=10pt
 a a a a a a a a a a a a ...

 \hangindent=-10pt
 a a a a a a a a a a a a ...
\end{verbatim}
}}$\quad gives\quad\quad
%$\vcenter{\parindent0pt \setbox0\hbox{a a a a a}\hsize\wd0
% \leftskip=0pt %\parskip6pt 
% a a a a a a a a a a a a \dots\par%\vskip\baselineskip
% \hangindent=10pt
% a a a a a a a a a a a a \dots\par%\vskip\baselineskip
% \hangindent=-10pt
% a a a a a a a a a a a a \dots\par}$
%}\end{disp}
%The default value of \cs{hangindent} is~\n{0pt}.
$\vcenter{\parindent0pt \setbox0\hbox{a a a a a}\hsize\wd0
 \leftskip=0pt %\parskip6pt 
 a a a a a a a a a a a a \dots\par%\vskip\baselineskip
 \hangindent=10pt
 a a a a a a a a a a a a \dots\par%\vskip\baselineskip
 \hangindent=-10pt
 a a a a a a a a a a a a \dots\par}$
}\end{disp}
The default value of \cs{hangindent} is~\n{0pt}.

%The \cs{hangafter} parameter determines the number of
%lines that is indented:
%\begin{itemize}\item \cs{hangafter}${}\geq0$: 
%after this number of lines the rest of the lines will be
%indented; in other words, this many lines from the
%start of the paragraph will not be indented.
%\item \cs{hangafter}${}<0$: the absolute value of this
%is the number of lines that will be indented starting
%at the beginning of the paragraph.\end{itemize}
%For example,
%\message{check left align}
%\begin{disp}\leavevmode\hbox{%\Distance:verbatimwhiteleft=0pt
%$\vcenter{\hsize=3.5in \snugbox{\begin{verbatim}
% a a a a a a a a a a a a ...
%
% \hangindent=10pt \hangafter=2
% a a a a a a a a a a a a ...
%
% \hangindent=10pt \hangafter=-2
% a a a a a a a a a a a a ...
%\end{verbatim}
%}}$\quad looks like\quad\quad
%$\vcenter{\parindent0pt \setbox0\hbox{a a a a a}\hsize\wd0
% \leftskip=0pt %\parskip6pt
% a a a a a a a a a a a a \dots\par%\vskip\baselineskip
% \hangindent=10pt \hangafter=2
% a a a a a a a a a a a a \dots\par%\vskip\baselineskip
% \hangindent=10pt \hangafter=-2
% a a a a a a a a a a a a \dots\par}$
%}\end{disp}
The \cs{hangafter} parameter determines the number of
lines that is indented:
\begin{itemize}\item \cs{hangafter}${}\geq0$: 
after this number of lines the rest of the lines will be
indented; in other words, this many lines from the
start of the paragraph will not be indented.
\item \cs{hangafter}${}<0$: the absolute value of this
is the number of lines that will be indented starting
at the beginning of the paragraph.\end{itemize}
For example,
\message{check left align}
\begin{disp}\leavevmode\hbox{%\Distance:verbatimwhiteleft=0pt
$\vcenter{\hsize=3.5in \snugbox{\begin{verbatim}
 a a a a a a a a a a a a ...

 \hangindent=10pt \hangafter=2
 a a a a a a a a a a a a ...

 \hangindent=10pt \hangafter=-2
 a a a a a a a a a a a a ...
\end{verbatim}
}}$\quad looks like\quad\quad
$\vcenter{\parindent0pt \setbox0\hbox{a a a a a}\hsize\wd0
 \leftskip=0pt %\parskip6pt
 a a a a a a a a a a a a \dots\par%\vskip\baselineskip
 \hangindent=10pt \hangafter=2
 a a a a a a a a a a a a \dots\par%\vskip\baselineskip
 \hangindent=10pt \hangafter=-2
 a a a a a a a a a a a a \dots\par}$
}\end{disp}
%The default value for \cs{hangafter} is~\n1.
The default value for \cs{hangafter} is~\n1.

%With both parameters having the possibility to
%be positive and negative,
%four ways of hanging indentation result. See below
%for hanging indentation into the margin (`outdent').
With both parameters having the possibility to
be positive and negative,
four ways of hanging indentation result. See below
for hanging indentation into the margin (`outdent').

%Hanging indentation is implemented as follows.
%The amount of hanging indentation is subtracted
%from the \cs{hsize} for the lines that indent;
%after the paragraph has been broken into horizontal
%boxes, the lines that should indent on the left are
%shifted right.
Hanging indentation is implemented as follows.
The amount of hanging indentation is subtracted
from the \cs{hsize} for the lines that indent;
after the paragraph has been broken into horizontal
boxes, the lines that should indent on the left are
shifted right.

%Regular indentation of size \cs{parindent} is not
%influenced by hanging indentation. Thus you should
%start a paragraph with hanging indentation 
%explicitly by~\cs{noindent} if the extra
%indentation is unwanted.
Regular indentation of size \cs{parindent} is not
influenced by hanging indentation. Thus you should
start a paragraph with hanging indentation 
explicitly by~\cs{noindent} if the extra
indentation is unwanted.

%The default values of \cs{hangindent} and \cs{hangafter} are
%restored after every \cs{par} command.
The default values of \cs{hangindent} and \cs{hangafter} are
restored after every \cs{par} command.

%\index{indentation!hanging|)}
\index{indentation!hanging|)}

%\subsection{General paragraph shapes: \cs{parshape}}
\subsection{General paragraph shapes: \cs{parshape}}

%Quite general paragraph shapes can be implemented
%using \csidx{parshape}. With this command line lengths and indentation
%for the first $n$ lines
%of a paragraph can be specified. Thus this command
%takes $2n+1$ parameters: the number of lines $n$, followed
%by $n$ pairs of an indentation   and a line length.
%\begin{disp} \cs{parshape}\gr{equals}
%    $n$ $i_1$ $\ell_1$ $\ldots$ $i_n$ $\ell_n$\end{disp}
%The   specification for the last line is repeated if the
%paragraph following has more than $n$ lines. If there are fewer
%than $n$ lines the remaining specifications are ignored.
%The default value is (naturally) \cs{parshape${}={}$0}.
Quite general paragraph shapes can be implemented
using \csidx{parshape}. With this command line lengths and indentation
for the first $n$ lines
of a paragraph can be specified. Thus this command
takes $2n+1$ parameters: the number of lines $n$, followed
by $n$ pairs of an indentation   and a line length.
\begin{disp} \cs{parshape}\gr{equals}
    $n$ $i_1$ $\ell_1$ $\ldots$ $i_n$ $\ell_n$\end{disp}
The   specification for the last line is repeated if the
paragraph following has more than $n$ lines. If there are fewer
than $n$ lines the remaining specifications are ignored.
The default value is (naturally) \cs{parshape${}={}$0}.

%A \cs{parshape} command takes precedence over a \cs{hangindent}
%if both have been specified. 
%%Regular \cs{parindent} indentation
%%is suppressed if \cs{parshape} is in effect.
%Regular \cs{parindent}, \cs{leftskip}, 
%and \cs{rightskip} are still obeyed if \cs{parshape} is in effect.
A \cs{parshape} command takes precedence over a \cs{hangindent}
if both have been specified. 
%Regular \cs{parindent} indentation
%is suppressed if \cs{parshape} is in effect.
Regular \cs{parindent}, \cs{leftskip}, 
and \cs{rightskip} are still obeyed if \cs{parshape} is in effect.

%The \cs{parshape} parameter is, like \cs{hangindent}, \cs{hangafter},
%and \cs{looseness} (see Chapter~\ref{line:break}),
%cleared after a \cs{par}
%command. Since every empty line generates a \cs{par} token,
%one should not leave an empty line
%between a paragraph shape (or hanging indentation)
%declaration and the following paragraph.
The \cs{parshape} parameter is, like \cs{hangindent}, \cs{hangafter},
and \cs{looseness} (see Chapter~\ref{line:break}),
cleared after a \cs{par}
command. Since every empty line generates a \cs{par} token,
one should not leave an empty line
between a paragraph shape (or hanging indentation)
declaration and the following paragraph.

%The control sequence
%\alt
%\cs{parshape} is an \gr{internal integer}:
%its value is the number of lines $n$ with which
%it was set.
The control sequence
\alt
\cs{parshape} is an \gr{internal integer}:
its value is the number of lines $n$ with which
it was set.

%%\point Assorted remarks
%\section{Assorted remarks}
%\point Assorted remarks
\section{Assorted remarks}

%%\spoint Centred last lines
%\subsection{Centred last lines}
%\spoint Centred last lines
\subsection{Centred last lines}

%Equal stretch and shrink amounts for the \cs{leftskip} and 
%\cs{rightskip}
%give centred texts, in the sense that each line is
%centred. 
%For proper centring of the first
%and last lines of a paragraph the \cs{parindent} and
%\cs{parfillskip} have to be made zero.
%However, the margins are ragged.
Equal stretch and shrink amounts for the \cs{leftskip} and 
\cs{rightskip}
give centred texts, in the sense that each line is
centred. 
For proper centring of the first
and last lines of a paragraph the \cs{parindent} and
\cs{parfillskip} have to be made zero.
However, the margins are ragged.

%A surprising application of \cs{leftskip} and \cs{rightskip}
%\mdqon
%\howto Centre the first/""last line of a paragraph\par
%\mdqoff
%leads to paragraphs with flush margins and a centred
%last line.
%\begin{verbatim}
%\leftskip=0cm plus 0.5fil \rightskip=0cm plus -0.5fil
%\parfillskip=0cm plus 1fil
%\end{verbatim}
A surprising application of \cs{leftskip} and \cs{rightskip}
\mdqon
\howto Centre the first/""last line of a paragraph\par
\mdqoff
leads to paragraphs with flush margins and a centred
last line.
\begin{verbatim}
\leftskip=0cm plus 0.5fil \rightskip=0cm plus -0.5fil
\parfillskip=0cm plus 1fil
\end{verbatim}

%For all lines of a paragraph but the 
%last one the stretch components
%add up to zero so the \cs{leftskip} and \cs{rightskip}
%inserted are zero.
%On the last line the \cs{parfillskip} adds \hbox{\n{plus 1fil}}
%of stretch; therefore there is a total of
%\hbox{\n{plus 0.5fil}} of stretch at both the left and right
%end of the line.
For all lines of a paragraph but the 
last one the stretch components
add up to zero so the \cs{leftskip} and \cs{rightskip}
inserted are zero.
On the last line the \cs{parfillskip} adds \hbox{\n{plus 1fil}}
of stretch; therefore there is a total of
\hbox{\n{plus 0.5fil}} of stretch at both the left and right
end of the line.

%It would have been incorrect to specify
%\begin{verbatim}
%\leftskip=0cm plus 0.5fil \rightskip=0cm minus 0.5fil
%\end{verbatim}
%\TeX\ gives an error about this: it complains about
%`infinite shrinkage'.
It would have been incorrect to specify
\begin{verbatim}
\leftskip=0cm plus 0.5fil \rightskip=0cm minus 0.5fil
\end{verbatim}
\TeX\ gives an error about this: it complains about
`infinite shrinkage'.

%Centring not only the last line, but also the
%first line of a paragraph can be done by
%the parameter settings
%\begin{verbatim}
%\parindent=0pt \everypar{\hskip 0pt plus -1fil}
%\leftskip=0pt plus .5fil
%\rightskip=0pt plus -.5fil
%\end{verbatim}
%This time a horizontal skip inserted by \cs{everypar}
%combines with the \cs{leftskip} to give the same
%amount of stretchability on both sides of the
%first line of the paragraph.
Centring not only the last line, but also the
first line of a paragraph can be done by
the parameter settings
\begin{verbatim}
\parindent=0pt \everypar{\hskip 0pt plus -1fil}
\leftskip=0pt plus .5fil
\rightskip=0pt plus -.5fil
\end{verbatim}
This time a horizontal skip inserted by \cs{everypar}
combines with the \cs{leftskip} to give the same
amount of stretchability on both sides of the
first line of the paragraph.

%%\spoint Indenting into the margin
%\subsection{Indenting into the margin}
%\spoint Indenting into the margin
\subsection{Indenting into the margin}

%Suppose you want a hanging indent of \n{1cm} {\sl into\/}
%\howto Indent into the margin\par
%the left margin after the first two lines of a paragraph. 
%Specifying \verb/\hangindent=-1cm/ will give
%a hanging indentation of one centimetre from the {\sl right\/}
%margin, so another approach is necessary. The following does the
%job:
%\begin{verbatim}
% \leftskip=-1cm \hangindent=1cm \hangafter=-2
%\end{verbatim}
%The only problem with this is that
%the leftskip needs to be reset after the paragraph.
%Suitable redefinition of \cs{par} removes this objection:
%\begin{verbatim}
%\def\hangintomargin{\bgroup
%    \leftskip=-1cm \hangindent=1cm \hangafter=-2
%    \def\par{\endgraf\egroup}}
%\end{verbatim}
%The redefinition of \cs{par} is here local to the paragraph that
%should be outdented.
Suppose you want a hanging indent of \n{1cm} {\sl into\/}
\howto Indent into the margin\par
the left margin after the first two lines of a paragraph. 
Specifying \verb/\hangindent=-1cm/ will give
a hanging indentation of one centimetre from the {\sl right\/}
margin, so another approach is necessary. The following does the
job:
\begin{verbatim}
 \leftskip=-1cm \hangindent=1cm \hangafter=-2
\end{verbatim}
The only problem with this is that
the leftskip needs to be reset after the paragraph.
Suitable redefinition of \cs{par} removes this objection:
\begin{verbatim}
\def\hangintomargin{\bgroup
    \leftskip=-1cm \hangindent=1cm \hangafter=-2
    \def\par{\endgraf\egroup}}
\end{verbatim}
The redefinition of \cs{par} is here local to the paragraph that
should be outdented.

%Another, elegant, solution uses \cs{parshape}:
%\begin{verbatim}
% 
%\dimen0=\hsize \advance\dimen0 by 1cm
%\parshape=3        % three lines:
%    0cm\hsize      % first  line specification
%    0cm\hsize      % second line specification
%    -1cm\dimen0    % third  line specification
%\end{verbatim}
Another, elegant, solution uses \cs{parshape}:
\begin{verbatim}
 
\dimen0=\hsize \advance\dimen0 by 1cm
\parshape=3        % three lines:
    0cm\hsize      % first  line specification
    0cm\hsize      % second line specification
    -1cm\dimen0    % third  line specification
\end{verbatim}

%%\spoint Hang a paragraph from an object
%\subsection{Hang a paragraph from an object}
%\spoint Hang a paragraph from an object
\subsection{Hang a paragraph from an object}

%The \LaTeX\ format has a macro, \cs{@hangfrom}, to have
%\howto Hang a paragraph from an object\par
%one paragraph of text hanging from some object, usually a box
%or a short line of text. 
The \LaTeX\ format has a macro, \cs{@hangfrom}, to have
\howto Hang a paragraph from an object\par
one paragraph of text hanging from some object, usually a box
or a short line of text. 

%\begingroup
%\medskip
%\def\hangobject{Example \ }
%\setbox0=\hbox{\hangobject}
%\hangindent \wd0 \noindent \hangobject
%This paragraph is an example of the \cs{hangfrom} macro
%defined below.
%In the \LaTeX\ document
%styles, the \cs{@hangfrom} macro (which is similar to this)
%is used for multi-line section headings.\par
%\endgroup
\begingroup
\medskip
\def\hangobject{Example \ }
\setbox0=\hbox{\hangobject}
\hangindent \wd0 \noindent \hangobject
This paragraph is an example of the \cs{hangfrom} macro
defined below.
In the \LaTeX\ document
styles, the \cs{@hangfrom} macro (which is similar to this)
is used for multi-line section headings.\par
\endgroup

%Consider then the macro \cs{hangfrom}:
%\begin{verbatim}
% 
%\def\hangfrom#1{\def\hangobject{#1}\setbox0=\hbox{\hangobject}%
%    \hangindent \wd0 \noindent \hangobject \ignorespaces}
%\end{verbatim}
%Because of the default \cs{hangafter=1}, this 
%will produce one line of width \cs{hsize}, after which the
%rest of the paragraph will be left indented  by the width of the
%\cs{hangobject}.
Consider then the macro \cs{hangfrom}:
\begin{verbatim}
 
\def\hangfrom#1{\def\hangobject{#1}\setbox0=\hbox{\hangobject}%
    \hangindent \wd0 \noindent \hangobject \ignorespaces}
\end{verbatim}
Because of the default \cs{hangafter=1}, this 
will produce one line of width \cs{hsize}, after which the
rest of the paragraph will be left indented  by the width of the
\cs{hangobject}.

%%\spoint Another approach to hanging indentation
%\subsection{Another approach to hanging indentation}
%\spoint Another approach to hanging indentation
\subsection{Another approach to hanging indentation}

%Hanging indentation can also be attained by a combination
%of shifting the left margin and outdenting.
%Itemized lists can for instance be implemented in this manner:
%\begin{verbatim}
%\newdimen\listindent
%\def\itemize{\begingroup
%    \advance\leftskip by \listindent
%    \parindent=-\listindent}
%\def\stopitemize{\par\endgroup}
%\def\item#1{\par\leavevmode
%    \hbox to \listindent{#1\hfil}\ignorespaces
%    }
%\end{verbatim}
%If an item should encompass more than one paragraph, the
%implementation could be
%\begin{verbatim}
%\newdimen\listindent \newdimen\listparindent
%\def\itemize{\begingroup
%    \advance\leftskip by \listindent
%    \parindent=\listparindent}
%\def\stopitemize{\par\endgroup}
%\def\item#1{\par\noindent
%    \hbox to 0cm{\kern-\listindent #1\hfil}\ignorespaces
%    }
%\end{verbatim}
Hanging indentation can also be attained by a combination
of shifting the left margin and outdenting.
Itemized lists can for instance be implemented in this manner:
\begin{verbatim}
\newdimen\listindent
\def\itemize{\begingroup
    \advance\leftskip by \listindent
    \parindent=-\listindent}
\def\stopitemize{\par\endgroup}
\def\item#1{\par\leavevmode
    \hbox to \listindent{#1\hfil}\ignorespaces
    }
\end{verbatim}
If an item should encompass more than one paragraph, the
implementation could be
\begin{verbatim}
\newdimen\listindent \newdimen\listparindent
\def\itemize{\begingroup
    \advance\leftskip by \listindent
    \parindent=\listparindent}
\def\stopitemize{\par\endgroup}
\def\item#1{\par\noindent
    \hbox to 0cm{\kern-\listindent #1\hfil}\ignorespaces
    }
\end{verbatim}

%\begin{example}
%\begin{verbatim}
%\itemize\item{1.}First item\par
%Is two paragraphs long.
%\item{2.}Second item.\stopitemize
%\end{verbatim}
%gives
%\begin{disp}
%\def\itemize{\begingroup
%    \advance\leftskip by \parindent
%    \parindent=1em\relax}
%\def\stopitemize{\par\endgroup}
%\def\item#1{\par\noindent
%    \hbox to 0cm{\kern-\parindent #1\hfil}\ignorespaces
%    }
%\itemize\item{1.}First item\par
%Is two paragraphs long.
%\item{2.}Second item.\stopitemize
%\end{disp}
%\end{example}
\begin{example}
\begin{verbatim}
\itemize\item{1.}First item\par
Is two paragraphs long.
\item{2.}Second item.\stopitemize
\end{verbatim}
gives
\begin{disp}
\def\itemize{\begingroup
    \advance\leftskip by \parindent
    \parindent=1em\relax}
\def\stopitemize{\par\endgroup}
\def\item#1{\par\noindent
    \hbox to 0cm{\kern-\parindent #1\hfil}\ignorespaces
    }
\itemize\item{1.}First item\par
Is two paragraphs long.
\item{2.}Second item.\stopitemize
\end{disp}
\end{example}

%%\spoint Hanging indentation versus \cs{leftskip} shifting
%\subsection{Hanging indentation versus \cs{leftskip} shifting}
%\spoint Hanging indentation versus \cs{leftskip} shifting
\subsection{Hanging indentation versus \cs{leftskip} shifting}

%From the above examples it would seem that
%hanging indentation and modifying the \cs{leftskip} and \cs{rightskip}
%are interchangeable. They are, but only to a certain   extent.
%\altt
From the above examples it would seem that
hanging indentation and modifying the \cs{leftskip} and \cs{rightskip}
are interchangeable. They are, but only to a certain   extent.
\altt

%Setting \cs{leftskip} to some positive value for a paragraph
%means that the \cs{hsize} stays the same, but every line
%starts with a glue item. Hanging indentation, on the other hand,
%is implemented by decreasing the \cs{hsize} value for the
%lines that hang, and shifting the finished 
%horizontal boxes horizontally in the surrounding vertical list.
Setting \cs{leftskip} to some positive value for a paragraph
means that the \cs{hsize} stays the same, but every line
starts with a glue item. Hanging indentation, on the other hand,
is implemented by decreasing the \cs{hsize} value for the
lines that hang, and shifting the finished 
horizontal boxes horizontally in the surrounding vertical list.

%The difference between the two approaches becomes visible
%mainly in the fact that display formulas are not shifted
%when the \cs{leftskip} is altered.
%See Chapter~\ref{rules} for an example showing how leaders
%are affected by margin shifting.
The difference between the two approaches becomes visible
mainly in the fact that display formulas are not shifted
when the \cs{leftskip} is altered.
See Chapter~\ref{rules} for an example showing how leaders
are affected by margin shifting.

%%\spoint More examples
%\subsection{More examples}
%\spoint More examples
\subsection{More examples}

%Some more examples of paragraph shapes (effected by
%various means) can be found in~\cite{E1}. One example
%from that article appears on page~\pageref{varioset}.
Some more examples of paragraph shapes (effected by
various means) can be found in~\cite{E1}. One example
from that article appears on page~\pageref{varioset}.

%\index{paragraph!shape|)}
%\endofchapter
\index{paragraph!shape|)}
\endofchapter

\end{document}
