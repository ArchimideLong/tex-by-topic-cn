% -*- coding: utf-8 -*-
\documentclass[twoside,letterpaper,openright]{rapport3}

% -*- coding: utf-8 -*-

\usepackage[b5paper,text={5in,8in},centering]{geometry}

\usepackage[CJKchecksingle]{xeCJK}
\setmainfont[Mapping=tex-text]{TeX Gyre Schola}
%\setsansfont{URW Gothic L Book}
%\setmonofont{Nimbus Mono L}
\setCJKmainfont[BoldFont=FandolHei,ItalicFont=FandolKai]{FandolSong}
\setCJKsansfont{FandolHei}
\setCJKmonofont{FandolFang}
\xeCJKsetup{PunctStyle = kaiming}

\linespread{1.25}
\setlength{\parindent}{2em}

\usepackage{xcolor}
\definecolor{myblue}{rgb}{0,0.2,0.6}

\usepackage{titlesec}
\titleformat{\chapter}
    {\normalfont\Huge\sffamily\color{myblue}}
    {第\thechapter 章}
    {1em}
    {}
%\titlespacing{\chapter}{0pt}{50pt}{40pt}
\titleformat{\section}
    {\normalfont\Large\sffamily\color{myblue}}
    {\thesection}
    {1em}
    {}
%\titlespacing{\section}{0pt}{3.5ex plus 1ex minus .2ex}{2.3ex plus .2ex}
\titleformat{\subsection}
    {\normalfont\large\sffamily\color{myblue}}
    {\thesubsection}
    {1em}
    {}
%\titlespacing{\subsection}{0pt}{3.25ex plus 1ex minus .2ex}{1.5ex plus .2ex}
%
\newpagestyle{special}[\small\sffamily]{
  \headrule
  \sethead[\usepage][][\chaptertitle]
  {\chaptertitle}{}{\usepage}}
\newpagestyle{main}[\small\sffamily]{
  \headrule
  \sethead[\usepage][][第\thechapter 章\quad\chaptertitle]
  {\thesection\quad\sectiontitle}{}{\usepage}}

\usepackage{titletoc}
%\setcounter{tocdepth}{1}
%\titlecontents{标题层次}[左间距]{上间距和整体格式}{标题序号}{标题内容}{指引线和页码}[下间距]
\titlecontents{chapter}[1.5em]{\vspace{.5em}\bfseries\sffamily}{\color{myblue}\contentslabel{1.5em}}{}
    {\titlerule*[20pt]{$\cdot$}\contentspage}[]
\titlecontents{section}[4.5em]{\sffamily}{\color{myblue}\contentslabel{3em}}{}
    {\titlerule*[20pt]{$\cdot$}\contentspage}[]
%\titlecontents{subsection}[8.5em]{\sffamily}{\contentslabel{4em}}{}
%    {\titlerule*[20pt]{$\cdot$}\contentspage}

\usepackage{enumitem}
\setlist{topsep=2pt,itemsep=2pt,parsep=1pt,leftmargin=\parindent}

\usepackage{fancyvrb}
\DefineVerbatimEnvironment{verbatim}{Verbatim}
  {xleftmargin=2em,baselinestretch=1,formatcom=\color{teal}\upshape}

\usepackage{etoolbox}
\makeatletter
\preto{\FV@ListVSpace}{\topsep=2pt \partopsep=0pt }
\makeatother

\usepackage[colorlinks,plainpages,pagebackref]{hyperref}
\hypersetup{
   pdfstartview={FitH},
   citecolor=teal,
   linkcolor=myblue,
   urlcolor=black,
   bookmarksnumbered
}

\usepackage{comment,makeidx,multicol}

%\usepackage{german}
%% german
%\righthyphenmin=3
%\mdqoff
%\captionsenglish
\usepackage[english]{babel}
{\catcode`"=13 \gdef"#1{\ifx#1"\discretionary{}{}{}\fi\relax}}
\def\mdqon{\catcode`"=13\relax}
\def\mdqoff{\catcode`"=12\relax}
\makeindex
\hyphenation{ex-em-pli-fies}

\newdimen\tempdima \newdimen\tempdimb

% these are fine
\def\handbreak{\\ \message{^^JManual break!!!!^^J}}
\def\nl{\protect\\}\def\n#1{{\tt #1}}
\def\cs#1{\texorpdfstring{{\tt\char`\\#1}}{\textbackslash#1}} %\def\cs#1{{\tt\char`\\#1}}
\let\csc\cs
\def\lb{{\tt\char`\{}}\def\rb{{\tt\char`\}}}
\def\gr#1{\texorpdfstring{$\langle$#1$\rangle$}{<#1>}} %\def\gr#1{$\langle$#1$\rangle$}
\def\key#1{{\tt#1}}
\def\alt{}\def\altt{}%this way in manstijl
\def\ldash{\unskip\ --\nobreak\ \ignorespaces}
\def\rdash{\unskip\nobreak\ --\ \ignorespaces}
% check these
\def\hex{{\tt"}}
\def\ascii{{\sc ascii}}
\def\ebcdic{{\sc ebcdic}}
\def\IniTeX{Ini\TeX}\def\LamsTeX{LAMS\TeX}\def\VirTeX{Vir\TeX}
\def\AmsTeX{Ams\TeX}
\def\TeXbook{the \TeX\ book}\def\web{{\sc web}}
% needs major thinking
\newenvironment{myquote}{\list{}{%
    \topsep=2pt \partopsep=0pt%
    \leftmargin=\parindent \rightmargin=\parindent
    }\item[]}{\endlist}
\newenvironment{disp}{\begin{myquote}}{\end{myquote}}
\newenvironment{Disp}{\begin{myquote}}{\end{myquote}}
\newenvironment{tdisp}{\begin{myquote}}{\end{myquote}}
\newenvironment{example}{\begin{myquote}\noindent\itshape 例子:}{\end{myquote}}
\newenvironment{inventory}{\begin{description}\raggedright}{\end{description}}
\newenvironment{glossinventory}{\begin{description}}{\end{description}}
\def\gram#1{\gr{#1}}%???
%
% index
%
\def\indexterm#1{\emph{#1}\index{#1}}
\def\indextermsub#1#2{\emph{#1 #2}\index{#1!#2}}
\def\indextermbus#1#2{\emph{#1 #2}\index{#2!#1}}
\def\term#1\par{\index{#1}}
\def\howto#1\par{}
\def\cstoidx#1\par{\index{#1@\cs{#1}@}}
\def\thecstoidx#1\par{\index{#1@\protect\csname #1\endcsname}}
\def\thecstoidxsub#1#2{\index{#1, #2@\protect\csname #1\endcsname, #2}\ignorespaces}
\def\csterm#1\par{\cstoidx #1\par\cs{#1}}
\def\csidx#1{\cstoidx #1\par\cs{#1}}

\def\tmc{\tracingmacros=2 \tracingcommands\tracingmacros}

%%%%%%%%%%%%%%%%%%%
\makeatletter
\def\snugbox{\hbox\bgroup\setbox\z@\vbox\bgroup
    \leftskip\z@
    \bgroup\aftergroup\make@snug
    \let\next=}
\def\make@snug{\par\sn@gify\egroup \box\z@\egroup}
\def\sn@gify
   {\skip\z@=\lastskip \unskip
    \advance\skip\z@\lastskip \unskip
    \unpenalty
    \setbox\z@\lastbox
    \ifvoid\z@ \nointerlineskip \else {\sn@gify} \fi
    \hbox{\unhbox\z@}\nointerlineskip
    \vskip\skip\z@
    }

\newdimen\fbh \fbh=60pt % dimension for easy scaling:
\newdimen\fbw \fbw=60pt % height and width of character box

\newdimen\dh \newdimen\dw % height and width of current character box
\newdimen\lh % height of previous character box
\newdimen\lw \lw=.4pt % line weight, instead of default .4pt

\def\hdotfill{\noindent
    \leaders\hbox{\vrule width 1pt height\lw
                  \kern4pt
                  \vrule width.5pt height\lw}\hfill\hbox{}
    \par}
\def\hlinefill{\noindent
    \leaders\hbox{\vrule width 5.5pt height\lw         }\hfill\hbox{}
    \par}
\def\stippel{$\qquad\qquad\qquad\qquad$}
\makeatother
%%%%%%%%%%%%%%%%%%%

%\def\SansSerif{\Typeface:macHelvetica }
%\def\SerifFont{\Typeface:macTimes }
%\def\SansSerif{\Typeface:bsGillSans }
%\def\SerifFont{\Typeface:bsBaskerville }
\let\SansSerif\relax \def\italic{\it}
\let\SerifFont\relax \def\MainFont{\rm}
\let\SansSerif\relax
\let\SerifFont\relax
\let\PopIndentLevel\relax \let\PushIndentLevel\relax
\let\ToVerso\relax \let\ToRecto\relax

%\def\stop@command@suffix{stop}
%\let\PopListLevel\PopIndentLevel
%\let\FlushRight\relax
%\let\flushright\FlushRight
%\let\SetListIndent\LevelIndent
%\def\awp{\ifhmode\vadjust{\penalty-10000 }\else
%    \penalty-10000 \fi}
\let\awp\relax
\let\PopIndentLevel\relax \let\PopListLevel\relax

\showboxdepth=-1

%\input figs
\def\endofchapter{\vfill\noindent}

\setcounter{chapter}{22}

\begin{document}

%\chapter{Mathematics Typesetting}\label{math}
\chapter{Mathematics Typesetting}\label{math}

%\TeX\ has two math modes, display and non-display, and
%four styles, display, text, script, and scriptscript style, and
%\altt
%every object in math mode belongs to one of eight classes.
%This chapter treats these concepts.
\TeX\ has two math modes, display and non-display, and
four styles, display, text, script, and scriptscript style, and
\altt
every object in math mode belongs to one of eight classes.
This chapter treats these concepts.


%\label{cschap:everymath}\label{cschap:displaystyle}\label{cschap:textstyle}\label{cschap:scriptstyle}\label{cschap:scriptscriptstyle}\label{cschap:mathchoice}\label{cschap:mathord}\label{cschap:mathop}\label{cschap:mathbin}\label{cschap:mathrel}\label{cschap:mathopen}\label{cschap:mathclose}\label{cschap:mathpunct}\label{cschap:mathinner}\label{cschap:mathaccent2}\label{cschap:vcenter}\label{cschap:limits}\label{cschap:nolimits}\label{cschap:displaylimits}\label{cschap:scriptspace}\label{cschap:nonscript}\label{cschap:mkern}\label{cschap:mskip}\label{cschap:muskip}\label{cschap:muskipdef}\label{cschap:newmuskip}\label{cschap:thinmuskip}\label{cschap:medmuskip}\label{cschap:thickmuskip}\label{cschap:mathsurround}\label{cschap:over}\label{cschap:atop}\label{cschap:above}\label{cschap:overwithdelims}\label{cschap:atopwithdelims}\label{cschap:abovewithdelims}\label{cschap:underline}\label{cschap:overline}\label{cschap:relpenalty}\label{cschap:binoppenalty}
%\begin{inventory}
%\item [\cs{everymath}] 
%      Token list inserted at the start of a non-display formula.
\label{cschap:everymath}\label{cschap:displaystyle}\label{cschap:textstyle}\label{cschap:scriptstyle}\label{cschap:scriptscriptstyle}\label{cschap:mathchoice}\label{cschap:mathord}\label{cschap:mathop}\label{cschap:mathbin}\label{cschap:mathrel}\label{cschap:mathopen}\label{cschap:mathclose}\label{cschap:mathpunct}\label{cschap:mathinner}\label{cschap:mathaccent2}\label{cschap:vcenter}\label{cschap:limits}\label{cschap:nolimits}\label{cschap:displaylimits}\label{cschap:scriptspace}\label{cschap:nonscript}\label{cschap:mkern}\label{cschap:mskip}\label{cschap:muskip}\label{cschap:muskipdef}\label{cschap:newmuskip}\label{cschap:thinmuskip}\label{cschap:medmuskip}\label{cschap:thickmuskip}\label{cschap:mathsurround}\label{cschap:over}\label{cschap:atop}\label{cschap:above}\label{cschap:overwithdelims}\label{cschap:atopwithdelims}\label{cschap:abovewithdelims}\label{cschap:underline}\label{cschap:overline}\label{cschap:relpenalty}\label{cschap:binoppenalty}
\begin{inventory}
\item [\cs{everymath}] 
      Token list inserted at the start of a non-display formula.

%\item [\cs{everydisplay}]
%      Token list inserted at the start of a display formula.
\item [\cs{everydisplay}]
      Token list inserted at the start of a display formula.

%\item [\cs{displaystyle}] 
%      Select the display style of mathematics typesetting.
\item [\cs{displaystyle}] 
      Select the display style of mathematics typesetting.

%\item [\cs{textstyle}] 
%      Select the text style of mathematics typesetting.
\item [\cs{textstyle}] 
      Select the text style of mathematics typesetting.

%\item [\cs{scriptstyle}] 
%      Select the script style of mathematics typesetting.
\item [\cs{scriptstyle}] 
      Select the script style of mathematics typesetting.

%\item [\cs{scriptscriptstyle}] 
%      Select the scriptscript style of mathematics typesetting.
\item [\cs{scriptscriptstyle}] 
      Select the scriptscript style of mathematics typesetting.

%\item [\cs{mathchoice}] 
%      Give four variants of a formula for the four styles
%      of mathematics typesetting.
\item [\cs{mathchoice}] 
      Give four variants of a formula for the four styles
      of mathematics typesetting.

%\item [\cs{mathord}] 
%      Let the following character or subformula function 
%      as an ordinary object.
\item [\cs{mathord}] 
      Let the following character or subformula function 
      as an ordinary object.

%\item [\cs{mathop}] 
%      Let the following character or subformula function 
%      as a large operator.
\item [\cs{mathop}] 
      Let the following character or subformula function 
      as a large operator.

%\item [\cs{mathbin}] 
%      Let the following character or subformula function 
%      as a binary operation.
\item [\cs{mathbin}] 
      Let the following character or subformula function 
      as a binary operation.

%\item [\cs{mathrel}] 
%      Let the following character or subformula function as a relation.
\item [\cs{mathrel}] 
      Let the following character or subformula function as a relation.

%\item [\cs{mathopen}] 
%      Let the following character or subformula function 
%      as a opening symbol.
\item [\cs{mathopen}] 
      Let the following character or subformula function 
      as a opening symbol.

%\item [\cs{mathclose}] 
%      Let the following character or subformula function
%      as a closing symbol.
\item [\cs{mathclose}] 
      Let the following character or subformula function
      as a closing symbol.

%\item [\cs{mathpunct}] 
%      Let the following character or subformula function 
%      as a punctuation symbol.
\item [\cs{mathpunct}] 
      Let the following character or subformula function 
      as a punctuation symbol.

%\item [\cs{mathinner}] 
%      Let the following character or subformula function 
%      as an inner formula.
\item [\cs{mathinner}] 
      Let the following character or subformula function 
      as an inner formula.

%\item [\cs{mathaccent}] 
%      Place an accent in math mode.
\item [\cs{mathaccent}] 
      Place an accent in math mode.

%\item [\cs{vcenter}] 
%      Construct a vertical box, vertically centred
%      on the math axis.
\item [\cs{vcenter}] 
      Construct a vertical box, vertically centred
      on the math axis.

%\item [\cs{limits}] 
%      Place limits over and under a large operator.
\item [\cs{limits}] 
      Place limits over and under a large operator.

%\item [\cs{nolimits}] 
%      Place limits of a large operator as subscript and 
%      superscript expressions.
\item [\cs{nolimits}] 
      Place limits of a large operator as subscript and 
      superscript expressions.

%\item [\cs{displaylimits}] 
%      Restore default placement for limits.
\item [\cs{displaylimits}] 
      Restore default placement for limits.

%\item [\cs{scriptspace}] 
%      Extra space after subscripts and superscripts.
%      Plain \TeX\ default:~\n{0.5pt}
\item [\cs{scriptspace}] 
      Extra space after subscripts and superscripts.
      Plain \TeX\ default:~\n{0.5pt}

%\item [\cs{nonscript}] 
%      Cancel the next glue item if it occurs in 
%      scriptstyle or scriptscriptstyle.
\item [\cs{nonscript}] 
      Cancel the next glue item if it occurs in 
      scriptstyle or scriptscriptstyle.

%\item [\cs{mkern}] 
%      Insert a kern measured in mu units.
\item [\cs{mkern}] 
      Insert a kern measured in mu units.

%\item [\cs{mskip}] 
%      Insert glue measured in mu units.
\item [\cs{mskip}] 
      Insert glue measured in mu units.

%\item [\cs{muskip}] 
%      Prefix for skips measured in mu units. 
\item [\cs{muskip}] 
      Prefix for skips measured in mu units. 

%\item [\cs{muskipdef}] 
%      Define a control sequence to be a synonym for
%      a~\cs{muskip} register.
\item [\cs{muskipdef}] 
      Define a control sequence to be a synonym for
      a~\cs{muskip} register.

%\item [\cs{newmuskip}] 
%      Allocate a new muskip register.
\item [\cs{newmuskip}] 
      Allocate a new muskip register.

%\item [\cs{thinmuskip}] 
%      Small amount of mu glue.
\item [\cs{thinmuskip}] 
      Small amount of mu glue.

%\item [\cs{medmuskip}] 
%      Medium amount of mu glue.
\item [\cs{medmuskip}] 
      Medium amount of mu glue.

%\item [\cs{thickmuskip}] 
%      Large amount of mu glue. 
\item [\cs{thickmuskip}] 
      Large amount of mu glue. 

%\item [\cs{mathsurround}] 
%      Kern amount placed before and after in-line formulas.
\item [\cs{mathsurround}] 
      Kern amount placed before and after in-line formulas.

%\item [\cs{over}]
%      Fraction.
\item [\cs{over}]
      Fraction.

%\item [\cs{atop}]
%      Place objects over one another.
\item [\cs{atop}]
      Place objects over one another.

%\item [\cs{above}]
%      Fraction with specified bar width. 
\item [\cs{above}]
      Fraction with specified bar width. 

%\item [\cs{overwithdelims}]
%      Fraction with delimiters.
\item [\cs{overwithdelims}]
      Fraction with delimiters.

%\item [\cs{atopwithdelims}]
%      Place objects over one another with delimiters.
\item [\cs{atopwithdelims}]
      Place objects over one another with delimiters.

%\item [\cs{abovewithdelims}]
%      Generalized fraction with delimiters.
\item [\cs{abovewithdelims}]
      Generalized fraction with delimiters.

%\item [\cs{underline}] 
%      Underline the following \gr{math symbol} or group.
\item [\cs{underline}] 
      Underline the following \gr{math symbol} or group.

%\item [\cs{overline}] 
%      Overline the following \gr{math symbol} or group.
\item [\cs{overline}] 
      Overline the following \gr{math symbol} or group.


%\item [\cs{relpenalty}] 
%      Penalty for breaking after a binary relation
%      not enclosed in a subformula.
%      Plain \TeX\ default:~\n{500}
\item [\cs{relpenalty}] 
      Penalty for breaking after a binary relation
      not enclosed in a subformula.
      Plain \TeX\ default:~\n{500}

%\item [\cs{binoppenalty}] 
%      Penalty for breaking after a binary operator not enclosed in
%      a subformula.
%      Plain \TeX\ default:~\n{700}
\item [\cs{binoppenalty}] 
      Penalty for breaking after a binary operator not enclosed in
      a subformula.
      Plain \TeX\ default:~\n{700}

%\item [\cs{allowbreak}] 
%      Macro for creating a breakpoint.
\item [\cs{allowbreak}] 
      Macro for creating a breakpoint.

%\end{inventory}
\end{inventory}

%%\point[math:modes] Math modes
%\section{Math modes}
%\label{math:modes}
%\point[math:modes] Math modes
\section{Math modes}
\label{math:modes}

%\TeX\ changes to \indexterm{math mode} when it encounters a
%\indexterm{math shift character}, category~3\index{category!3},
%in the input. After such
%an opening math shift it investigates (without expansion) the next
%token to see whether this is another math shift.  In the latter case
%\TeX\ starts processing in \indextermbus{display}{math mode} until a
%closing double math shift is encountered:
%\begin{disp}\verb> .. $$ >{\italic displayed formula}\verb> $$ ..>\end{disp}
%Otherwise it starts processing an in-line formula in
%\indextermbus{non-display}{math mode}:
%\begin{disp}\verb> .. $ >{\italic in-line formula}\verb> $ ..>\end{disp}
%The single math shift character is a \gr{horizontal command}.
\TeX\ changes to \indexterm{math mode} when it encounters a
\indexterm{math shift character}, category~3\index{category!3},
in the input. After such
an opening math shift it investigates (without expansion) the next
token to see whether this is another math shift.  In the latter case
\TeX\ starts processing in \indextermbus{display}{math mode} until a
closing double math shift is encountered:
\begin{disp}\verb> .. $$ >{\italic displayed formula}\verb> $$ ..>\end{disp}
Otherwise it starts processing an in-line formula in
\indextermbus{non-display}{math mode}:
\begin{disp}\verb> .. $ >{\italic in-line formula}\verb> $ ..>\end{disp}
The single math shift character is a \gr{horizontal command}.

%Exception: displays are not possible in restricted horizontal
%mode, so inside an \cs{hbox} the sequence
%\verb>$$> is an empty math formula and
%not the start of a displayed formula.
Exception: displays are not possible in restricted horizontal
mode, so inside an \cs{hbox} the sequence
\verb>$$> is an empty math formula and
not the start of a displayed formula.

%Associated with the two math modes are two \gr{token parameter}
%registers (see also Chapter~\ref{token}):
%at the start of an in-line formula the \csidx{everymath} tokens
%are inserted; at the start of a displayed formula the
%\cs{everydisplay} tokens are inserted.
%Display math is treated further in the next chapter.
Associated with the two math modes are two \gr{token parameter}
registers (see also Chapter~\ref{token}):
at the start of an in-line formula the \csidx{everymath} tokens
are inserted; at the start of a displayed formula the
\cs{everydisplay} tokens are inserted.
Display math is treated further in the next chapter.

%Math modes can be tested for: \cs{ifmmode} is true
%in display and non-display math mode, and \cs{ifinner}
%is true in non-display mode, but not in display mode.
Math modes can be tested for: \cs{ifmmode} is true
in display and non-display math mode, and \cs{ifinner}
is true in non-display mode, but not in display mode.

%%\point[math:styles] Styles in math mode
%\section{Styles in math mode}
%\label{math:styles}
%\point[math:styles] Styles in math mode
\section{Styles in math mode}
\label{math:styles}

%Math formulas are set in any of eight 
%\indexterm{math styles}:
%\begin{description} 
%\item [D] display style,
%\item [T] text style,
%\item [S] script style,
%\item [SS] scriptscript style,
%\end{description}
%and the four \emph{cramped}\indexterm{cramped styles}
%variants $D'$, $T'$, $S'$, $SS'$ of
%these. The cramped styles differ mainly in the
%fact that superscripts are not raised as far as in
%the original styles.
Math formulas are set in any of eight 
\indexterm{math styles}:
\begin{description} 
\item [D] display style,
\item [T] text style,
\item [S] script style,
\item [SS] scriptscript style,
\end{description}
and the four \emph{cramped}\indexterm{cramped styles}
variants $D'$, $T'$, $S'$, $SS'$ of
these. The cramped styles differ mainly in the
fact that superscripts are not raised as far as in
the original styles.

%%\spoint Superscripts and subscripts
%\subsection{Superscripts and subscripts}
%\index{superscript|(}\index{subscript|(}
%\spoint Superscripts and subscripts
\subsection{Superscripts and subscripts}
\index{superscript|(}\index{subscript|(}

%\TeX\ can typeset a symbol or group
%as a superscript (or subscript) to the preceding
%symbol or group, if that preceding item
%does not already have a superscript
%(subscript). Superscripts (subscripts) are specified by
%the syntax
%\begin{disp}\gr{superscript}\gr{math field}\end{disp}
%or 
%\begin{disp}\gr{subscript}\gr{math field}\end{disp}
%where a \gr{superscript} (\gr{subscript}) is either a character
%of category~7\index{category!7} (8\index{category!8}),
%or a control sequence \cs{let} to such
%a character.
%The plain format has the control
%\cstoidx\char94\par\cstoidx\char95\par
%sequences
%\begin{verbatim}
%\let\sp=^ \let\sb=_
%\end{verbatim}
%as implicit superscript
%and subscript characters.
\TeX\ can typeset a symbol or group
as a superscript (or subscript) to the preceding
symbol or group, if that preceding item
does not already have a superscript
(subscript). Superscripts (subscripts) are specified by
the syntax
\begin{disp}\gr{superscript}\gr{math field}\end{disp}
or 
\begin{disp}\gr{subscript}\gr{math field}\end{disp}
where a \gr{superscript} (\gr{subscript}) is either a character
of category~7\index{category!7} (8\index{category!8}),
or a control sequence \cs{let} to such
a character.
The plain format has the control
\cstoidx\char94\par\cstoidx\char95\par
sequences
\begin{verbatim}
\let\sp=^ \let\sb=_
\end{verbatim}
as implicit superscript
and subscript characters.

%Specifying a superscript (subscript) expression as the first
%item in an empty math list is equivalent to specifying
%it as the superscript (subscript) of an empty expression.
%For instance, 
%\begin{disp}
%\verb>$^{...}>\quad is equivalent to\quad \verb>${}^{...}>\end{disp}
Specifying a superscript (subscript) expression as the first
item in an empty math list is equivalent to specifying
it as the superscript (subscript) of an empty expression.
For instance, 
\begin{disp}
\verb>$^{...}>\quad is equivalent to\quad \verb>${}^{...}>\end{disp}

%For \TeX's internal calculations, superscript and subscript
%expressions are made wider by \csidx{scriptspace};
%the value of this in plain \TeX\ is~\n{0.5pt}.
For \TeX's internal calculations, superscript and subscript
expressions are made wider by \csidx{scriptspace};
the value of this in plain \TeX\ is~\n{0.5pt}.

%\index{superscript|)}\index{subscript|)}
\index{superscript|)}\index{subscript|)}

%\subsection{Choice of styles}
\subsection{Choice of styles}

%Ordering the four styles $D$, $T$, $S$, and~$SS$, and
%considering the other four as mere variants, the
%style rules for math mode are as follows:
%\begin{itemize}\item In any style superscripts and subscripts
%are taken from the next smaller style. Exception:
%in display style they are taken in script style.
%\item Subscripts are always in the cramped variant of
%the style; superscripts are only cramped if the original
%style was cramped.
%\item In an \verb-{..\over..}- formula in any style
%the numerator and denominator are taken from the next
%smaller style.
%\item The denominator is always in cramped style;
%the numerator is only in cramped style if the original
%style was cramped.
%\item Formulas under a \cs{sqrt} or \cs{overline}
%are in cramped style.\end{itemize}
Ordering the four styles $D$, $T$, $S$, and~$SS$, and
considering the other four as mere variants, the
style rules for math mode are as follows:
\begin{itemize}\item In any style superscripts and subscripts
are taken from the next smaller style. Exception:
in display style they are taken in script style.
\item Subscripts are always in the cramped variant of
the style; superscripts are only cramped if the original
style was cramped.
\item In an \verb-{..\over..}- formula in any style
the numerator and denominator are taken from the next
smaller style.
\item The denominator is always in cramped style;
the numerator is only in cramped style if the original
style was cramped.
\item Formulas under a \cs{sqrt} or \cs{overline}
are in cramped style.\end{itemize}

%Styles can be forced by the explicit commands
%\alt
%\cstoidx displaystyle\par\cstoidx textstyle\par
%\cstoidx scriptstyle\par\cstoidx scriptscriptstyle\par
%\cs{displaystyle}, \cs{textstyle}, \cs{scriptstyle},
%and \cs{scriptscriptstyle}.
Styles can be forced by the explicit commands
\alt
\cstoidx displaystyle\par\cstoidx textstyle\par
\cstoidx scriptstyle\par\cstoidx scriptscriptstyle\par
\cs{displaystyle}, \cs{textstyle}, \cs{scriptstyle},
and \cs{scriptscriptstyle}.


%In display style and text style the \cs{textfont} of the
%current family is used, 
%in scriptstyle the \cs{scriptfont} is used, and in
%\alt
%scriptscriptstyle the \cs{scriptscriptfont} is used.
In display style and text style the \cs{textfont} of the
current family is used, 
in scriptstyle the \cs{scriptfont} is used, and in
\alt
scriptscriptstyle the \cs{scriptscriptfont} is used.

%The primitive command
%\cstoidx mathchoice\par
%\begin{disp}\cs{mathchoice}\lb {\it D\/\rb\lb T\/\rb\lb S\/\rb\lb SS\/}\rb
%\end{disp}
%lets the user specify four variants of a formula for the
%four styles. 
%\TeX\ constructs all four and inserts the appropriate one.
The primitive command
\cstoidx mathchoice\par
\begin{disp}\cs{mathchoice}\lb {\it D\/\rb\lb T\/\rb\lb S\/\rb\lb SS\/}\rb
\end{disp}
lets the user specify four variants of a formula for the
four styles. 
\TeX\ constructs all four and inserts the appropriate one.

%%\point[math:class] Classes of mathematical objects
%\section{Classes of mathematical objects}
%\label{math:class}
%\point[math:class] Classes of mathematical objects
\section{Classes of mathematical objects}
\label{math:class}

%Objects in math mode belong to one of eight \indexterm{math
%  classes}. Depending on the class the object may be surrounded by
%some amount of white space, or treated specially in some way. Commands
%exist to force symbols, or sequences of symbols, to act as belonging
%to a certain class.  In the hexadecimal representation \verb>"xyzz>
%the class is the \gr{3-bit number}~\n x.
Objects in math mode belong to one of eight \indexterm{math
  classes}. Depending on the class the object may be surrounded by
some amount of white space, or treated specially in some way. Commands
exist to force symbols, or sequences of symbols, to act as belonging
to a certain class.  In the hexadecimal representation \verb>"xyzz>
the class is the \gr{3-bit number}~\n x.

%This is the list of classes and commands that force those
%classes. The examples are from the plain format 
%(see the tables starting at page~\pageref{math:sym:tables}).
%\begin{enumerate} \message{set a counter here!}%\SetCounter:item=-1
%\item {\em ordinary\/}: lowercase Greek characters and those symbols
%      that are `just symbols'; 
%      the command \csidx{mathord} forces this class.
%\item {\em large operator\/}: integral and sum signs,
%      and `big' objects such as \cs{bigcap} or\handbreak \cs{bigotimes};
%      the command \csidx{mathop} forces this class.
%      Characters that are
%      large operators are centred vertically, and they
%      may behave differently in display style from in the
%      other styles; see below.
%\item {\em binary operation\/}: plus and minus,
%      and things such as \cs{cap} or \cs{otimes};
%      the command \csidx{mathbin} forces this class.
%\item {\em relation\/} (also called {\em binary relation\/}): 
%      equals, less than, and greater than signs, subset and
%      superset, perpendicular, parallel;
%      the command \csidx{mathrel} forces this class.
%\item {\em opening symbol\/}: opening brace, bracket, parenthesis, angle,
% \altt
%      floor, ceiling;
%      the command \csidx{mathopen} forces this class.
%\item {\em closing symbol\/}: closing brace, bracket, parenthesis, angle,
% \altt
%      floor, ceiling;
%      the command \csidx{mathclose} forces this class.
%\item {\em punctuation\/}: most punctuation marks, but
%      \n:~is a relation, the \cs{colon} is a punctuation colon;
%      the command \csidx{mathpunct} forces this class.
%\item {\em variable family\/}: symbols in this class change font
%      with the \cs{fam} parameter; in plain \TeX\ uppercase
%      Greek letters and ordinary letters and digits are
%      in this class.
%\end{enumerate}
This is the list of classes and commands that force those
classes. The examples are from the plain format 
(see the tables starting at page~\pageref{math:sym:tables}).
\begin{enumerate} \message{set a counter here!}%\SetCounter:item=-1
\item {\em ordinary\/}: lowercase Greek characters and those symbols
      that are `just symbols'; 
      the command \csidx{mathord} forces this class.
\item {\em large operator\/}: integral and sum signs,
      and `big' objects such as \cs{bigcap} or\handbreak \cs{bigotimes};
      the command \csidx{mathop} forces this class.
      Characters that are
      large operators are centred vertically, and they
      may behave differently in display style from in the
      other styles; see below.
\item {\em binary operation\/}: plus and minus,
      and things such as \cs{cap} or \cs{otimes};
      the command \csidx{mathbin} forces this class.
\item {\em relation\/} (also called {\em binary relation\/}): 
      equals, less than, and greater than signs, subset and
      superset, perpendicular, parallel;
      the command \csidx{mathrel} forces this class.
\item {\em opening symbol\/}: opening brace, bracket, parenthesis, angle,
 \altt
      floor, ceiling;
      the command \csidx{mathopen} forces this class.
\item {\em closing symbol\/}: closing brace, bracket, parenthesis, angle,
 \altt
      floor, ceiling;
      the command \csidx{mathclose} forces this class.
\item {\em punctuation\/}: most punctuation marks, but
      \n:~is a relation, the \cs{colon} is a punctuation colon;
      the command \csidx{mathpunct} forces this class.
\item {\em variable family\/}: symbols in this class change font
      with the \cs{fam} parameter; in plain \TeX\ uppercase
      Greek letters and ordinary letters and digits are
      in this class.
\end{enumerate}

%There is one further class: the {\em inner\/} subformulas.
%No characters can be assigned to this class, but characters and
%subformulas can be forced into it by \csidx{mathinner}.
%The \gr{generalized fraction}s and \verb-\left...\right- groups
%are inner formulas. Inner formulas are surrounded
%by some white space; see the table below.
There is one further class: the {\em inner\/} subformulas.
No characters can be assigned to this class, but characters and
subformulas can be forced into it by \csidx{mathinner}.
The \gr{generalized fraction}s and \verb-\left...\right- groups
are inner formulas. Inner formulas are surrounded
by some white space; see the table below.

%Other subformulas than those that are inner are treated as
%ordinary symbols. In particular, subformulas enclosed in
%braces are ordinary: \verb-$a+b$- looks like `$a\mathop+b$', but
%\message{Check a+b look}%
%\verb-$a{+}b$- looks like~`$a{+}b$'. Note, however, that
%in \verb-${a+b}$- the whole subformula is treated as an
%ordinary symbol, not its components; 
%therefore the result is~`${a+b}$'.
Other subformulas than those that are inner are treated as
ordinary symbols. In particular, subformulas enclosed in
braces are ordinary: \verb-$a+b$- looks like `$a\mathop+b$', but
\message{Check a+b look}%
\verb-$a{+}b$- looks like~`$a{+}b$'. Note, however, that
in \verb-${a+b}$- the whole subformula is treated as an
ordinary symbol, not its components; 
therefore the result is~`${a+b}$'.

%%\point Large operators and their limits
%\section{Large operators and their limits}
%\point Large operators and their limits
\section{Large operators and their limits}

%The large operators in the Computer Modern fonts come in
%two sizes: one for text style and one for display style.
%Control sequences such as \cs{sum} are simply defined by
%\cs{mathchardef} to correspond to a position in a font:
%\begin{verbatim}
%\mathchardef\sum="1350
%\end{verbatim}
%but if the
%current style is display style, \TeX\ looks to see whether
%that character has a successor in the font.
The large operators in the Computer Modern fonts come in
two sizes: one for text style and one for display style.
Control sequences such as \cs{sum} are simply defined by
\cs{mathchardef} to correspond to a position in a font:
\begin{verbatim}
\mathchardef\sum="1350
\end{verbatim}
but if the
current style is display style, \TeX\ looks to see whether
that character has a successor in the font.

%Large operators in text style behave as if they are followed
%\cstoidx limits\par\cstoidx nolimits\par
%by \cs{nolimits}, which places the limits as sub/superscript
%expressions after the
%operator:
%\begin{disp}$\sum_{k=1}^\infty$\end{disp}
%In display style they behave as if they are followed by
%\cs{limits}, which places the limits over and under
%the operator:
%\begin{disp}$\displaystyle\sum_{k=1}^\infty$\end{disp}
%The successor mechanism (see page~\pageref{successor})
%\alt
%lets \TeX\ take a larger variant
%of the delimiter here.
Large operators in text style behave as if they are followed
\cstoidx limits\par\cstoidx nolimits\par
by \cs{nolimits}, which places the limits as sub/superscript
expressions after the
operator:
\begin{disp}$\sum_{k=1}^\infty$\end{disp}
In display style they behave as if they are followed by
\cs{limits}, which places the limits over and under
the operator:
\begin{disp}$\displaystyle\sum_{k=1}^\infty$\end{disp}
The successor mechanism (see page~\pageref{successor})
\alt
lets \TeX\ take a larger variant
of the delimiter here.

%The integral sign has been defined in plain \TeX\ as
%\begin{verbatim}
%\mathchardef\intop="1352 \def\int{\intop\nolimits}
%\end{verbatim}
%which places the limits after the operator, even in display style:
%\begin{disp}$\displaystyle\int_0^\infty e^{-x^2}\,dx=\sqrt{\pi}/2$
%\end{disp}
The integral sign has been defined in plain \TeX\ as
\begin{verbatim}
\mathchardef\intop="1352 \def\int{\intop\nolimits}
\end{verbatim}
which places the limits after the operator, even in display style:
\begin{disp}$\displaystyle\int_0^\infty e^{-x^2}\,dx=\sqrt{\pi}/2$
\end{disp}

%With \verb-\limits\nolimits- or \verb-\nolimits\limits- the
%last specification has precedence; the default placement
%can be restored by \csidx{displaylimits}. For instance,
%\begin{verbatim}
%$ ... \sum\limits\displaylimits ... $
%\end{verbatim}
%is equivalent to
%\begin{verbatim}
%$ ... \sum ... $
%\end{verbatim}
%and 
%\begin{verbatim}
%$$ ... \sum\nolimits\displaylimits ... $$
%\end{verbatim}
%is equivalent to
%\begin{verbatim}
%$$ ... \sum ... $$
%\end{verbatim}
With \verb-\limits\nolimits- or \verb-\nolimits\limits- the
last specification has precedence; the default placement
can be restored by \csidx{displaylimits}. For instance,
\begin{verbatim}
$ ... \sum\limits\displaylimits ... $
\end{verbatim}
is equivalent to
\begin{verbatim}
$ ... \sum ... $
\end{verbatim}
and 
\begin{verbatim}
$$ ... \sum\nolimits\displaylimits ... $$
\end{verbatim}
is equivalent to
\begin{verbatim}
$$ ... \sum ... $$
\end{verbatim}

%%\point Vertical centering: \cs{vcenter}
%\section{Vertical centring: \protect\cs{vcenter}}
%\point Vertical centering: \cs{vcenter}
\section{Vertical centring: \protect\cs{vcenter}}

%Each formula has an \emph{axis}\index{formula!axis
%  of}\index{formula!centring of}, which is for an in-line formula
%about half the x-height of the surrounding text; the exact value is
%the \cs{fontdimen22} of the font in family~2, the symbol font, in the
%current style.
Each formula has an \emph{axis}\index{formula!axis
  of}\index{formula!centring of}, which is for an in-line formula
about half the x-height of the surrounding text; the exact value is
the \cs{fontdimen22} of the font in family~2, the symbol font, in the
current style.

%The bar line in fractions is placed on the axis; large
%operators, delimiters and \cs{vcenter} boxes are centred on it.
The bar line in fractions is placed on the axis; large
operators, delimiters and \cs{vcenter} boxes are centred on it.

%A \csidx{vcenter}\label{vcenter}
%box is a vertical box that is arranged
%so that it is centred on the math axis.
%It is possible to give a \n{spread} or \n{to}
%specification with a \cs{vcenter} box.
A \csidx{vcenter}\label{vcenter}
box is a vertical box that is arranged
so that it is centred on the math axis.
It is possible to give a \n{spread} or \n{to}
specification with a \cs{vcenter} box.

%The \cs{vcenter} box is allowed only in math mode, and
%it does not behave like other boxes; for instance, it can
%not be stored in a box register. It does not qualify as
%a~\gr{box}. See page~\pageref{tvcenter} for a macro that
%repairs this.
The \cs{vcenter} box is allowed only in math mode, and
it does not behave like other boxes; for instance, it can
not be stored in a box register. It does not qualify as
a~\gr{box}. See page~\pageref{tvcenter} for a macro that
repairs this.

%%\point[muglue] Mathematical spacing: \n{mu} glue
%\section{Mathematical spacing: \n{mu} glue}
%\label{muglue}
%\index{math spacing|(}
%\point[muglue] Mathematical spacing: \n{mu} glue
\section{Mathematical spacing: \n{mu} glue}
\label{muglue}
\index{math spacing|(}

%Spacing around mathematical objects is measured in \indexterm{math units}:
%multiples of a~\n{mu}. A~\n{mu} is $1/18$th part of \cs{fontdimen6} of
%the font in family~2 in the current style, the \indexterm{quad} value of the
%symbol font.
Spacing around mathematical objects is measured in \indexterm{math units}:
multiples of a~\n{mu}. A~\n{mu} is $1/18$th part of \cs{fontdimen6} of
the font in family~2 in the current style, the \indexterm{quad} value of the
symbol font.

%%\spoint Classification of \n{mu} glue
%\subsection{Classification of \n{mu} glue}
%\spoint Classification of \n{mu} glue
\subsection{Classification of \n{mu} glue}

%The user can specify \n{mu} spacing by \cs{mkern} or~\cs{mskip},
%\cstoidx mkern\par\cstoidx mskip\par
%but most \indexterm{mu glue} is inserted automatically by \TeX,
%based on the classes to which objects belong (see above).
%First, here are some rules of thumb describing the global 
%behaviour.
The user can specify \n{mu} spacing by \cs{mkern} or~\cs{mskip},
\cstoidx mkern\par\cstoidx mskip\par
but most \indexterm{mu glue} is inserted automatically by \TeX,
based on the classes to which objects belong (see above).
First, here are some rules of thumb describing the global 
behaviour.

%\begin{itemize} \item A \cs{thickmuskip} (default value in plain
%\TeX: \n{5mu plus 5mu})
%\cstoidx thickmuskip\par
%is inserted around (binary) relations, except where these are
%preceded or followed by other relations or punctuation, and
%except if they follow an open, or precede a close symbol.
%\item A \csidx{medmuskip} (default value in plain
%\TeX: \n{4mu plus 2mu minus 4mu}) 
%is put around binary operators.
%\item A \csidx{thinmuskip} 
%(default value in plain \TeX: \n{3mu}) follows after
%punctuation, and is put around inner objects, except where these
%are followed by a close or preceded by an open symbol, and
%except if the other object is a large operator or a
%binary relation.
%\item No \n{mu} glue is inserted after an open or before a close
%symbol except where the latter is preceded by punctuation;
%no \n{mu} glue is inserted also before punctuation, except where
%the preceding object is punctuation or an inner object.
%\end{itemize} 
\begin{itemize} \item A \cs{thickmuskip} (default value in plain
\TeX: \n{5mu plus 5mu})
\cstoidx thickmuskip\par
is inserted around (binary) relations, except where these are
preceded or followed by other relations or punctuation, and
except if they follow an open, or precede a close symbol.
\item A \csidx{medmuskip} (default value in plain
\TeX: \n{4mu plus 2mu minus 4mu}) 
is put around binary operators.
\item A \csidx{thinmuskip} 
(default value in plain \TeX: \n{3mu}) follows after
punctuation, and is put around inner objects, except where these
are followed by a close or preceded by an open symbol, and
except if the other object is a large operator or a
binary relation.
\item No \n{mu} glue is inserted after an open or before a close
symbol except where the latter is preceded by punctuation;
no \n{mu} glue is inserted also before punctuation, except where
the preceding object is punctuation or an inner object.
\end{itemize} 

%The following table gives the complete definition of mu glue
%between math objects.
%\begin{disp}\leavevmode
%\vbox{\offinterlineskip
%    \halign{#\enspace\hfil&#\enspace\hfil\vrule
%           &&\hfil\enspace#\hfil\strut\cr
%    \omit\hfil&\omit\hfil& 0:& 1:& 2:& 3:& 4:& 5:& 6:\cr
%    \omit\hfil&\omit\hfil&\hfill Ord&\hfill Op&\hfill Bin&\hfill Rel&
%                  \hfill Open&\hfill Close&\hfill Punct&\hfill Inner\cr
%    \omit\hfil&\omit\hfil&\multispan8\hrulefill\cr
%    0:&Ord&    0&  1&(2)&(3)&  0&  0&  0&(1)\cr
%    1:&Op&     1&  1&  *&(3)&  0&  0&  0&(1)\cr
%    2:&Bin&  (2)&(2)&  *&  *&(2)&  *&  *&(2)\cr
%    3:&Rel&  (3)&(3)&  *&  0&(2)&  *&  *&(2)\cr
%    4:&Open&   0&  0&  *&  0&  0&  0&  0&  0\cr
%    5:&Close&  0&  1&(2)&(3)&  0&  0&  0&(1)\cr
%    6:&Punct&(1)&(1)&  *&(1)&(1)&(1)&(1)&(1)\cr
%      &Inner&(1)&  1&(2)&(3)&(1)&  0&(1)&(1)\cr
%%    \omit\hfil&\omit\hfil&\multispan8\hrulefil\cr
%}}
%\end{disp}
The following table gives the complete definition of mu glue
between math objects.
\begin{disp}\leavevmode
\vbox{\offinterlineskip
    \halign{#\enspace\hfil&#\enspace\hfil\vrule
           &&\hfil\enspace#\hfil\strut\cr
    \omit\hfil&\omit\hfil& 0:& 1:& 2:& 3:& 4:& 5:& 6:\cr
    \omit\hfil&\omit\hfil&\hfill Ord&\hfill Op&\hfill Bin&\hfill Rel&
                  \hfill Open&\hfill Close&\hfill Punct&\hfill Inner\cr
    \omit\hfil&\omit\hfil&\multispan8\hrulefill\cr
    0:&Ord&    0&  1&(2)&(3)&  0&  0&  0&(1)\cr
    1:&Op&     1&  1&  *&(3)&  0&  0&  0&(1)\cr
    2:&Bin&  (2)&(2)&  *&  *&(2)&  *&  *&(2)\cr
    3:&Rel&  (3)&(3)&  *&  0&(2)&  *&  *&(2)\cr
    4:&Open&   0&  0&  *&  0&  0&  0&  0&  0\cr
    5:&Close&  0&  1&(2)&(3)&  0&  0&  0&(1)\cr
    6:&Punct&(1)&(1)&  *&(1)&(1)&(1)&(1)&(1)\cr
      &Inner&(1)&  1&(2)&(3)&(1)&  0&(1)&(1)\cr
%    \omit\hfil&\omit\hfil&\multispan8\hrulefil\cr
}}
\end{disp}

%where the symbols have the following meanings:
%\begin{itemize}\item 0, no space; 1, thin space; 2, medium space;
%     3, thick space;
%\item $(\cdot)$, insert only in text and display
%     mode, not in script or scriptscript mode;
%\item    cases * cannot occur, because a Bin object is converted
%    to Ord if it is the first in the list, preceded by
%    Bin, Op, Open, Punct, Rel, or followed by Close,
%    Punct, and Rel; also, a Rel is converted to Ord when
%    \alt
%    it is followed by Close or Punct.
%\end{itemize}
where the symbols have the following meanings:
\begin{itemize}\item 0, no space; 1, thin space; 2, medium space;
     3, thick space;
\item $(\cdot)$, insert only in text and display
     mode, not in script or scriptscript mode;
\item    cases * cannot occur, because a Bin object is converted
    to Ord if it is the first in the list, preceded by
    Bin, Op, Open, Punct, Rel, or followed by Close,
    Punct, and Rel; also, a Rel is converted to Ord when
    \alt
    it is followed by Close or Punct.
\end{itemize}

%Stretchable \n{mu} glue is set according to the same rules that
%govern ordinary glue. However, only \n{mu} glue on the outer
%level can be stretched or shrunk; any \n{mu} glue enclosed
%in  a~group is set at natural width.
Stretchable \n{mu} glue is set according to the same rules that
govern ordinary glue. However, only \n{mu} glue on the outer
level can be stretched or shrunk; any \n{mu} glue enclosed
in  a~group is set at natural width.

%%\spoint Muskip registers
%\subsection{Muskip registers}
%\spoint Muskip registers
\subsection{Muskip registers}

%Like ordinary glue, \n{mu} glue can be stored in registers,
%\cstoidx muskip\par\cstoidx muskipdef\par\cstoidx newmuskip\par
%the \cs{muskip} registers,
%of which there are 256 in \TeX. 
%The registers are denoted by
%\begin{disp}\cs{muskip}\gr{8-bit number}\end{disp}
%and they can be assigned to a control sequence by
%\begin{disp}\cs{muskipdef}\gr{control sequence}\gr{equals}\gr{8-bit number}
%\end{disp}
%and there is a macro that allocates unused registers:
%\begin{disp}\cs{newmuskip}\gr{control sequence}\end{disp}
%Arithmetic for mu glue exists as for glue; see
%Chapter~\ref{glue}.
Like ordinary glue, \n{mu} glue can be stored in registers,
\cstoidx muskip\par\cstoidx muskipdef\par\cstoidx newmuskip\par
the \cs{muskip} registers,
of which there are 256 in \TeX. 
The registers are denoted by
\begin{disp}\cs{muskip}\gr{8-bit number}\end{disp}
and they can be assigned to a control sequence by
\begin{disp}\cs{muskipdef}\gr{control sequence}\gr{equals}\gr{8-bit number}
\end{disp}
and there is a macro that allocates unused registers:
\begin{disp}\cs{newmuskip}\gr{control sequence}\end{disp}
Arithmetic for mu glue exists as for glue; see
Chapter~\ref{glue}.

%%\spoint Other spaces in math mode
%\subsection{Other spaces in math mode}
%\spoint Other spaces in math mode
\subsection{Other spaces in math mode}

%In math mode space tokens are ignored; however,
%the math code of the space character is \verb-"8000-
%in plain \TeX,
%so if its category is made `letter' or `other character', it
%will behave like an active character in math mode.
%See also page~\pageref{mcode:8000}.
In math mode space tokens are ignored; however,
the math code of the space character is \verb-"8000-
in plain \TeX,
so if its category is made `letter' or `other character', it
will behave like an active character in math mode.
See also page~\pageref{mcode:8000}.

%Admissible glue in math mode is of type~\gr{mathematical skip},
%which is either a \gr{horizontal skip} (see Chapter~\ref{hvmode}) 
%or~\cs{mskip}\gr{muglue}. Leaders in math mode can be specified
%with a \gr{mathematical skip}.
Admissible glue in math mode is of type~\gr{mathematical skip},
which is either a \gr{horizontal skip} (see Chapter~\ref{hvmode}) 
or~\cs{mskip}\gr{muglue}. Leaders in math mode can be specified
with a \gr{mathematical skip}.

%A glue item preceded by \csidx{nonscript}
%is cancelled if it occurs in scriptstyle or scriptscriptstyle.
A glue item preceded by \csidx{nonscript}
is cancelled if it occurs in scriptstyle or scriptscriptstyle.

%Control space functions in math mode
%\alt
%as it does in horizontal mode.
Control space functions in math mode
\alt
as it does in horizontal mode.

%In-line formulas are surrounded by kerns of size
%\csidx{mathsurround}, the so-called `math-on' and
%`math-off' items. Line breaking can occur at the front of
%the math-off kern if it is followed by glue.
In-line formulas are surrounded by kerns of size
\csidx{mathsurround}, the so-called `math-on' and
`math-off' items. Line breaking can occur at the front of
the math-off kern if it is followed by glue.

%\index{math spacing|)}
\index{math spacing|)}

%\section{Generalized fractions}
\section{Generalized fractions}

%Fraction-like objects can be set with six primitive commands
%of type \gr{generalized fraction}.
%Each of these \indexterm{generalized fractions}
%takes the preceding and the following subformulas
%and puts them over one another, if necessary with a fraction
%bar and with delimiters.
%\begin{description} \item [\csidx{over}]
%   is the ordinary fraction; the bar thickness is \cs{fontdimen8}
%   of the extension font: 
%   
%\begin{disp}\verb>$\pi\over2$>\quad gives\quad `$\pi\over2$'\message{pi over 2}\end{disp}
%\item [\csidx{atop}]
%   is equivalent to a fraction with zero bar thickness:
%   
%\begin{disp}\verb>$\pi\atop2$>\quad gives\quad `$\pi\atop2$'\end{disp}
%\item [\csidx{above}\gr{dimen}]
%   specifies the thickness
%   of the bar line explicitly:
%   
%\begin{disp}\verb>$\pi\above 1pt 2$>\quad gives\quad `$\pi\above 1pt 2$'\end{disp}
%\end{description} 
Fraction-like objects can be set with six primitive commands
of type \gr{generalized fraction}.
Each of these \indexterm{generalized fractions}
takes the preceding and the following subformulas
and puts them over one another, if necessary with a fraction
bar and with delimiters.
\begin{description} \item [\csidx{over}]
   is the ordinary fraction; the bar thickness is \cs{fontdimen8}
   of the extension font: 
   
\begin{disp}\verb>$\pi\over2$>\quad gives\quad `$\pi\over2$'\message{pi over 2}\end{disp}
\item [\csidx{atop}]
   is equivalent to a fraction with zero bar thickness:
   
\begin{disp}\verb>$\pi\atop2$>\quad gives\quad `$\pi\atop2$'\end{disp}
\item [\csidx{above}\gr{dimen}]
   specifies the thickness
   of the bar line explicitly:
   
\begin{disp}\verb>$\pi\above 1pt 2$>\quad gives\quad `$\pi\above 1pt 2$'\end{disp}
\end{description} 

%To each of these three there corresponds a \cs{...withdelims} variant
%\cstoidx overwithdelims\par\cstoidx atopwithdelims\par
%\cstoidx abovewithdelims\par
%that lets the user specify delimiters for the expression.
%For example, the most general command, in terms of which
%all five others could have been defined, is
%\begin{disp}\cs{abovewithdelims}\gr{delim$_1$}\gr{delim$_2$}\gr{dimen}.
%\end{disp}
%Delimiters in these generalized fractions do not grow with the
%enclosed expression: in display mode a delimiter is taken
%which is at least \cs{fontdimen20} high, otherwise
%\alt
%it has to be
%at least \cs{fontdimen21} high.
%These dimensions are taken
%from the font in family~2, the symbol font, in the current style.
To each of these three there corresponds a \cs{...withdelims} variant
\cstoidx overwithdelims\par\cstoidx atopwithdelims\par
\cstoidx abovewithdelims\par
that lets the user specify delimiters for the expression.
For example, the most general command, in terms of which
all five others could have been defined, is
\begin{disp}\cs{abovewithdelims}\gr{delim$_1$}\gr{delim$_2$}\gr{dimen}.
\end{disp}
Delimiters in these generalized fractions do not grow with the
enclosed expression: in display mode a delimiter is taken
which is at least \cs{fontdimen20} high, otherwise
\alt
it has to be
at least \cs{fontdimen21} high.
These dimensions are taken
from the font in family~2, the symbol font, in the current style.

%The control sequences \cs{over}, \cs{atop}, and \cs{above}
%are primitives, although they could have been defined
%as \cs{...withdelims..}, that is, with two null delimiters.
%Because of these implied surrounding null delimiters,
%there is a kern of size \cs{nulldelimiterspace} before and after
%these simple generalized fractions. 
The control sequences \cs{over}, \cs{atop}, and \cs{above}
are primitives, although they could have been defined
as \cs{...withdelims..}, that is, with two null delimiters.
Because of these implied surrounding null delimiters,
there is a kern of size \cs{nulldelimiterspace} before and after
these simple generalized fractions. 

%%\point Underlining, overlining
%\section{Underlining, overlining}
%\point Underlining, overlining
\section{Underlining, overlining}

%The primitive commands \csidx{underline} and \csidx{overline} take a 
%\gr{math field} argument, that is, a \gr{math symbol} or
%a group, and draw a line under or over it.
%The result is an `Under' or `Over' atom, which
%is appended to the current math list.
%The line thickness is font dimension~8 of the extension font,
%which also determines the clearance between the line and
%the \gr{math field}.
The primitive commands \csidx{underline} and \csidx{overline} take a 
\gr{math field} argument, that is, a \gr{math symbol} or
a group, and draw a line under or over it.
The result is an `Under' or `Over' atom, which
is appended to the current math list.
The line thickness is font dimension~8 of the extension font,
which also determines the clearance between the line and
the \gr{math field}.

%Various other \cs{over...} and \cs{under...} commands exist
%in plain \TeX;
%these are all macros
%that use the \TeX\ \cs{halign} command.
Various other \cs{over...} and \cs{under...} commands exist
in plain \TeX;
these are all macros
that use the \TeX\ \cs{halign} command.

%%\point Line breaking in math formulas
%\section{Line breaking in math formulas}
%\point Line breaking in math formulas
\section{Line breaking in math formulas}

%In-line formulas can be broken after relations and binary operators.
%\cstoidx relpenalty\par\cstoidx binoppenaly\par
%The respective \emph{penalties}\index{penalties in math mode}
%are the \cs{relpenalty} 
%and the~\cs{binoppenalty}. However, \TeX\ will only break
%after such symbols if they are not enclosed in braces.
%Other \emph{breakpoints}\index{breakpoints in math lists}
%can be created with~\cs{allowbreak},
%\cstoidx allowbreak\par
%which is an abbreviation for~\cs{penalty0}.
In-line formulas can be broken after relations and binary operators.
\cstoidx relpenalty\par\cstoidx binoppenaly\par
The respective \emph{penalties}\index{penalties in math mode}
are the \cs{relpenalty} 
and the~\cs{binoppenalty}. However, \TeX\ will only break
after such symbols if they are not enclosed in braces.
Other \emph{breakpoints}\index{breakpoints in math lists}
can be created with~\cs{allowbreak},
\cstoidx allowbreak\par
which is an abbreviation for~\cs{penalty0}.

%Unlike in horizontal or vertical mode where putting two penalties
%in a row is equivalent to just placing the smallest one,
%in math mode a penalty placed at a break point \ldash that is,
%after a relation or binary operator \rdash  will effectively
%replace the old penalty by the new one.
Unlike in horizontal or vertical mode where putting two penalties
in a row is equivalent to just placing the smallest one,
in math mode a penalty placed at a break point \ldash that is,
after a relation or binary operator \rdash  will effectively
replace the old penalty by the new one.

%%\point[fam23:fontdims] Font dimensions of families 2 and 3
%\section{Font dimensions of families 2 and 3}
%\label{fam23:fontdims}
%\point[fam23:fontdims] Font dimensions of families 2 and 3
\section{Font dimensions of families 2 and 3}
\label{fam23:fontdims}

%If a font is used in text mode, \TeX\ will look at its
%first 7 \cs{fontdimen} parameters
%(see page~\pageref{font:dims}), for instance to
%control spacing.
%In math, however, more font dimensions are needed.
%\TeX\ will look at the first 22 parameters of the
%fonts in family~2, and the first 13 of the fonts in
%family~3, to control various
%aspects of math typesetting. The next two subsections
%have been quoted loosely from~\cite{BB:ISO}.
If a font is used in text mode, \TeX\ will look at its
first 7 \cs{fontdimen} parameters
(see page~\pageref{font:dims}), for instance to
control spacing.
In math, however, more font dimensions are needed.
\TeX\ will look at the first 22 parameters of the
fonts in family~2, and the first 13 of the fonts in
family~3, to control various
aspects of math typesetting. The next two subsections
have been quoted loosely from~\cite{BB:ISO}.

%%\spoint Symbol font attributes
%\subsection{Symbol font attributes}
%\index{symbol font|(}
%\spoint Symbol font attributes
\subsection{Symbol font attributes}
\index{symbol font|(}

%Attributes of the font in family~2, the \emph{symbol font},
%mainly specify the
%initial vertical positioning
%of parts of fractions, subscripts, superscripts, et cetera.
%The position determined by applying these
%attributes may be further modified because of other
%conditions, for example the presence of a fraction bar.
Attributes of the font in family~2, the \emph{symbol font},
mainly specify the
initial vertical positioning
of parts of fractions, subscripts, superscripts, et cetera.
The position determined by applying these
attributes may be further modified because of other
conditions, for example the presence of a fraction bar.

%One text font dimension, number~6,
%the quad, determines the size of mu glue;
%see above.
One text font dimension, number~6,
the quad, determines the size of mu glue;
see above.

%Fraction numerator attributes: minimum shift up, from
%the main baseline, of the baseline of the numerator
%of a generalized fraction,
%\begin{enumerate} \message{set a counter here!}%\SetCounter:item=7
%\item num1:
% for display style,
%\item num2:
% for text style or smaller if a fraction bar is present,
%\item num3:
% for text style or smaller if no fraction bar is present.
%\end{enumerate}
Fraction numerator attributes: minimum shift up, from
the main baseline, of the baseline of the numerator
of a generalized fraction,
\begin{enumerate} \message{set a counter here!}%\SetCounter:item=7
\item num1:
 for display style,
\item num2:
 for text style or smaller if a fraction bar is present,
\item num3:
 for text style or smaller if no fraction bar is present.
\end{enumerate}

%Fraction denominator attributes: minimum shift down, from
%the main baseline, of the baseline of the denominator
%of a generalized fraction,
%\begin{enumerate} \message{set a counter here!}%\SetCounter:item=10
%\item denom1:
%for display style,
%\item denom2:
%for text style or smaller.
%\end{enumerate}
Fraction denominator attributes: minimum shift down, from
the main baseline, of the baseline of the denominator
of a generalized fraction,
\begin{enumerate} \message{set a counter here!}%\SetCounter:item=10
\item denom1:
for display style,
\item denom2:
for text style or smaller.
\end{enumerate}

%Superscript attributes: minimum shift up, from the main baseline,
%of the baseline of a superscript,
%\begin{enumerate} \message{set a counter here!}%\SetCounter:item=12
%\item sup1:
%for display style,
%\item sup2:
%for text style or smaller, non-cramped,
%\item sup3:
%for text style or smaller, cramped.
%\end{enumerate}
Superscript attributes: minimum shift up, from the main baseline,
of the baseline of a superscript,
\begin{enumerate} \message{set a counter here!}%\SetCounter:item=12
\item sup1:
for display style,
\item sup2:
for text style or smaller, non-cramped,
\item sup3:
for text style or smaller, cramped.
\end{enumerate}

%Subscript attributes: minimum shift down, from the main baseline,
%of the baseline of a subscript,
%\begin{enumerate} \message{set a counter here!}%\SetCounter:item=15
%\item sub1:
%when no superscript is present,
%\item sub2:
%when a superscript is present.
%\end{enumerate}
Subscript attributes: minimum shift down, from the main baseline,
of the baseline of a subscript,
\begin{enumerate} \message{set a counter here!}%\SetCounter:item=15
\item sub1:
when no superscript is present,
\item sub2:
when a superscript is present.
\end{enumerate}

%Script adjustment attributes: for use only with non-glyph,
%that is, composite, objects.
%\begin{enumerate} \message{set a counter here!}%\SetCounter:item=17
%\item sup\_drop:
%maximum distance of superscript baseline below top of nucleus
%\item sub\_drop:
%minimum distance of subscript baseline below bottom of nucleus.
%\end{enumerate}
Script adjustment attributes: for use only with non-glyph,
that is, composite, objects.
\begin{enumerate} \message{set a counter here!}%\SetCounter:item=17
\item sup\_drop:
maximum distance of superscript baseline below top of nucleus
\item sub\_drop:
minimum distance of subscript baseline below bottom of nucleus.
\end{enumerate}

%Delimiter span attributes: height plus depth of delimiter enclosing
%a generalized fraction,
%\begin{enumerate} \message{set a counter here!}%\SetCounter:item=19
%\item delim1:
%in display style,
%\item delim2:
%in text style or smaller.
%\end{enumerate}
Delimiter span attributes: height plus depth of delimiter enclosing
a generalized fraction,
\begin{enumerate} \message{set a counter here!}%\SetCounter:item=19
\item delim1:
in display style,
\item delim2:
in text style or smaller.
\end{enumerate}

%A parameter with many uses, the height of the math axis,
%\begin{enumerate} \message{set a counter here!}%\SetCounter:item=21
%\item axis\_height:
%the height above the baseline
%of the fraction bar, and the centre of large delimiters
%and most operators and relations. This position is
%used in vertical centring operations.
%\end{enumerate}
A parameter with many uses, the height of the math axis,
\begin{enumerate} \message{set a counter here!}%\SetCounter:item=21
\item axis\_height:
the height above the baseline
of the fraction bar, and the centre of large delimiters
and most operators and relations. This position is
used in vertical centring operations.
\end{enumerate}

%\index{symbol font|)}
\index{symbol font|)}

%\subsection{Extension font attributes}
\subsection{Extension font attributes}

%\index{extension font|(}
\index{extension font|(}

%Attributes of the font in family~3, the \emph{extension font},
%mostly specify
%the way the limits of large operators are set.
Attributes of the font in family~3, the \emph{extension font},
mostly specify
the way the limits of large operators are set.

%The first parameter, number 8, default\_rule\_thickness,
%serves many purposes. It
%is the thickness of the rule used for overlines,
%underlines, radical extenders (square root), 
%and fraction bars. Various clearances are  also specified
%in terms of this dimension: between the fraction bar and
%the numerator and denominator, between an object and
%the rule drawn by an underline, overline, or radical,
%and between the bottom of superscripts and top of subscripts.
The first parameter, number 8, default\_rule\_thickness,
serves many purposes. It
is the thickness of the rule used for overlines,
underlines, radical extenders (square root), 
and fraction bars. Various clearances are  also specified
in terms of this dimension: between the fraction bar and
the numerator and denominator, between an object and
the rule drawn by an underline, overline, or radical,
and between the bottom of superscripts and top of subscripts.

%Minimum clearances around large operators are as follows:
%\begin{enumerate} \message{set a counter here!}%\SetCounter:item=8
%\item big\_op\_spacing1:
%minimum clearance between baseline of upper limit and top
%of large operator; see below.
%\item big\_op\_spacing2:
%minimum clearance between bottom of large operator and top of 
%lower limit.
%\item big\_op\_spacing3:
%minimum clearance between baseline of
%upper limit and top of large operator,
%taking into account depth of upper limit; see below.
%\item big\_op\_spacing4:
%minimum clearance between bottom of large operator and top of lower
%limit, taking into account height of lower limit; see below.
%\item big\_op\_spacing5:
%clearance above upper limit or below lower limit of a large operator.
%\end{enumerate}
%The resulting clearance above an operator is the maximum
%of parameter~7, and parameter~11 minus the depth of the
%upper limit.
%The resulting clearance below an operator is the maximum
%of parameter~10, and parameter~12 minus the height of the
%lower limit.
Minimum clearances around large operators are as follows:
\begin{enumerate} \message{set a counter here!}%\SetCounter:item=8
\item big\_op\_spacing1:
minimum clearance between baseline of upper limit and top
of large operator; see below.
\item big\_op\_spacing2:
minimum clearance between bottom of large operator and top of 
lower limit.
\item big\_op\_spacing3:
minimum clearance between baseline of
upper limit and top of large operator,
taking into account depth of upper limit; see below.
\item big\_op\_spacing4:
minimum clearance between bottom of large operator and top of lower
limit, taking into account height of lower limit; see below.
\item big\_op\_spacing5:
clearance above upper limit or below lower limit of a large operator.
\end{enumerate}
The resulting clearance above an operator is the maximum
of parameter~7, and parameter~11 minus the depth of the
upper limit.
The resulting clearance below an operator is the maximum
of parameter~10, and parameter~12 minus the height of the
lower limit.

%\index{extension font|)}
\index{extension font|)}

%\subsection{Example: subscript lowering}
\subsection{Example: subscript lowering}

%The location of a subscript depends on whether there
%\alt
%\howto Adjust subscript lowering\par
%is a superscript; for instance
%\begin{disp} $X_1+Y^2_1=1$\end{disp}
%If you would rather have that look like
%\begin{disp} $\global\tempdima=\fontdimen16\textfont2\relax
%       \global\tempdimb=\fontdimen17\textfont2\relax
%       \fontdimen16\textfont2=3pt \fontdimen17\textfont2=3pt
%       X_1+Y^2_1=1$,$\fontdimen16\textfont2=\tempdima\relax
%                     \fontdimen17\textfont2=\tempdimb\relax$
%\end{disp}
%\message{check lowering}
%it suffices to specify
%\begin{verbatim}
%\fontdimen16\textfont2=3pt \fontdimen17\textfont2=3pt
%\end{verbatim}
%which makes the subscript drop equal in both cases.
%Since font dimension assignments are global, you 
%have to specify this only once in your document.
The location of a subscript depends on whether there
\alt
\howto Adjust subscript lowering\par
is a superscript; for instance
\begin{disp} $X_1+Y^2_1=1$\end{disp}
If you would rather have that look like
\begin{disp} $\global\tempdima=\fontdimen16\textfont2\relax
       \global\tempdimb=\fontdimen17\textfont2\relax
       \fontdimen16\textfont2=3pt \fontdimen17\textfont2=3pt
       X_1+Y^2_1=1$,$\fontdimen16\textfont2=\tempdima\relax
                     \fontdimen17\textfont2=\tempdimb\relax$
\end{disp}
\message{check lowering}
it suffices to specify
\begin{verbatim}
\fontdimen16\textfont2=3pt \fontdimen17\textfont2=3pt
\end{verbatim}
which makes the subscript drop equal in both cases.
Since font dimension assignments are global, you 
have to specify this only once in your document.


%\endofchapter
\endofchapter

\end{document}
