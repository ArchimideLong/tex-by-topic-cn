% -*- coding: utf-8 -*-

%\chapter{Running \TeX}\label{run}
\chapter{Running \TeX}\label{run}

%This chapter treats the run modes of \TeX, and some
%other commands associated with the job being processed.
This chapter treats the run modes of \TeX, and some
other commands associated with the job being processed.

%\label{cschap:everyjob}\label{cschap:jobname}\label{cschap:end}\label{cschap:bye}\label{cschap:pausing}\label{cschap:errorstopmode}\label{cschap:scrollmode}\label{cschap:nonstopmode}\label{cschap:batchmode}
%\begin{inventory}
%\item [\cs{everyjob}] 
%      Token list that is inserted at the start of each new job.
\label{cschap:everyjob}\label{cschap:jobname}\label{cschap:end}\label{cschap:bye}\label{cschap:pausing}\label{cschap:errorstopmode}\label{cschap:scrollmode}\label{cschap:nonstopmode}\label{cschap:batchmode}
\begin{inventory}
\item [\cs{everyjob}] 
      Token list that is inserted at the start of each new job.

%\item [\cs{jobname}] 
%      Name of the main \TeX\ file being processed.
\item [\cs{jobname}] 
      Name of the main \TeX\ file being processed.

%\item [\cs{end}] 
%      Command to finish off a run of \TeX.
\item [\cs{end}] 
      Command to finish off a run of \TeX.

%\item [\cs{bye}]
%      Plain \TeX\ macro to force the final output.
\item [\cs{bye}]
      Plain \TeX\ macro to force the final output.

%\item [\cs{pausing}] 
%      Specify that \TeX\ should pause after each line that is 
%      read from a file.
\item [\cs{pausing}] 
      Specify that \TeX\ should pause after each line that is 
      read from a file.

%\item [\cs{errorstopmode}]
%      \TeX\ will ask for user input on the occurrence of an error.
\item [\cs{errorstopmode}]
      \TeX\ will ask for user input on the occurrence of an error.

%\item [\cs{scrollmode}] 
%      \TeX\ fixes errors itself,
%      but will ask the user for missing files.
\item [\cs{scrollmode}] 
      \TeX\ fixes errors itself,
      but will ask the user for missing files.

%\item [\cs{nonstopmode}] 
%      \TeX\ fixes errors itself, 
%      and performs an emergency stop on serious errors 
%      such as missing input files.
\item [\cs{nonstopmode}] 
      \TeX\ fixes errors itself, 
      and performs an emergency stop on serious errors 
      such as missing input files.

%\item [\cs{batchmode}] 
%      \TeX\ fixes errors itself 
%      and performs an emergency stop on serious errors 
%      such as missing input files,
%      but no terminal output is generated.
\item [\cs{batchmode}] 
      \TeX\ fixes errors itself 
      and performs an emergency stop on serious errors 
      such as missing input files,
      but no terminal output is generated.

%\end{inventory}
\end{inventory}

%%\point Jobs
%\section{Jobs}
%\index{job|(}
%\point Jobs
\section{Jobs}
\index{job|(}

%\TeX\ associates with each run a name for the file
%being processed: the \csidx{jobname}. If \TeX\ is run
%interactively
%\ldash meaning that it has been invoked without a file argument,
%and the user types commands \rdash
%the jobname is \n{texput}.
\TeX\ associates with each run a name for the file
being processed: the \csidx{jobname}. If \TeX\ is run
interactively
\ldash meaning that it has been invoked without a file argument,
and the user types commands \rdash
the jobname is \n{texput}.

%The \cs{jobname} can be used to generate
%the names of auxiliary files to be read or
%written during the run. For instance, for a file \n{story.tex}
%the \cs{jobname} is \n{story}, and writing
%\begin{verbatim}
%\openout\Auxiliary=\jobname.aux
%\openout\TableOfContents=\jobname.toc
%\end{verbatim}
%will create the files \n{story.aux} and \n{story.toc}.
The \cs{jobname} can be used to generate
the names of auxiliary files to be read or
written during the run. For instance, for a file \n{story.tex}
the \cs{jobname} is \n{story}, and writing
\begin{verbatim}
\openout\Auxiliary=\jobname.aux
\openout\TableOfContents=\jobname.toc
\end{verbatim}
will create the files \n{story.aux} and \n{story.toc}.

%%\spoint Start of the job
%\subsection{Start of the job}
%\spoint Start of the job
\subsection{Start of the job}

%\TeX\ starts each job by inserting the \csidx{everyjob} token
%list into the command stream.
%Setting this variable during a run of \TeX\ has no use,
%but a format can use it to identify itself to the user.
%If a
%format fills the token list, the commands therein are automatically
%executed when \TeX\ is run using that format.
\TeX\ starts each job by inserting the \csidx{everyjob} token
list into the command stream.
Setting this variable during a run of \TeX\ has no use,
but a format can use it to identify itself to the user.
If a
format fills the token list, the commands therein are automatically
executed when \TeX\ is run using that format.

%%\spoint End of the job
%\subsection{End of the job}
%\spoint End of the job
\subsection{End of the job}

%A \TeX\ job is terminated by the \csidx{end} command. This
%may involve first forcing the output routine to process any
%remaining material (see Chapter~\ref{page:break}).
%If the end of job occurs inside a group
%\TeX\ will give a diagnostic
%message. The \cs{end} command is not allowed in internal
%vertical mode, because this would be inside a vertical box.
A \TeX\ job is terminated by the \csidx{end} command. This
may involve first forcing the output routine to process any
remaining material (see Chapter~\ref{page:break}).
If the end of job occurs inside a group
\TeX\ will give a diagnostic
message. The \cs{end} command is not allowed in internal
vertical mode, because this would be inside a vertical box.

%Usually some sugar coating of the \cs{end} command is necessary.
%For instance the plain \TeX\ macro \csidx{bye} is defined
%as
%\begin{verbatim}
%\def\bye{\par\vfill\supereject\end}
%\end{verbatim}
%where the \cs{supereject} takes care of any leftover insertions.
Usually some sugar coating of the \cs{end} command is necessary.
For instance the plain \TeX\ macro \csidx{bye} is defined
as
\begin{verbatim}
\def\bye{\par\vfill\supereject\end}
\end{verbatim}
where the \cs{supereject} takes care of any leftover insertions.

%%\spoint The log file
%\subsection{The log file}
%\spoint The log file
\subsection{The log file}

%For each run \TeX\ creates a \indexterm{log file}. Usually this will be
%a file with as name the value of \cs{jobname}, and the
%extension \n{.log}. Other extensions such as \n{.lis}
%are used by some implementations.
%This log file contains all information that
%is displayed on the screen during the run of \TeX, but
%it will display some information more elaborately, and it
%can contain statistics that are usually not displayed on
%the screen. If the parameter \cs{tracingonline}
%has a positive value, all the log file information will be
%shown on the screen.
For each run \TeX\ creates a \indexterm{log file}. Usually this will be
a file with as name the value of \cs{jobname}, and the
extension \n{.log}. Other extensions such as \n{.lis}
are used by some implementations.
This log file contains all information that
is displayed on the screen during the run of \TeX, but
it will display some information more elaborately, and it
can contain statistics that are usually not displayed on
the screen. If the parameter \cs{tracingonline}
has a positive value, all the log file information will be
shown on the screen.

%Overfull and underfull boxes are reported on the terminal
%screen, and they are dumped using the parameters
%\cs{showboxdepth} and \cs{showboxbreadth} in the log file
%(see Chapter~\ref{trace}). These parameters are also used
%for box dumps caused by the \cs{showbox} command, and
%for the dump of boxes written by \cs{shipout} 
%if \cs{tracingoutput} is set to a positive value.
Overfull and underfull boxes are reported on the terminal
screen, and they are dumped using the parameters
\cs{showboxdepth} and \cs{showboxbreadth} in the log file
(see Chapter~\ref{trace}). These parameters are also used
for box dumps caused by the \cs{showbox} command, and
for the dump of boxes written by \cs{shipout} 
if \cs{tracingoutput} is set to a positive value.

%Statistics generated by commands such as \cs{tracingparagraphs}
%will be written to the log file; if \cs{tracingonline} is positive
%they will also be shown on the screen.
Statistics generated by commands such as \cs{tracingparagraphs}
will be written to the log file; if \cs{tracingonline} is positive
they will also be shown on the screen.

%Output operations to a stream that is not open, or to a
%stream with a number that is not in the range 0--15,
%go to the log file. If the stream number is positive,
%they also go to the terminal.
Output operations to a stream that is not open, or to a
stream with a number that is not in the range 0--15,
go to the log file. If the stream number is positive,
they also go to the terminal.

%\index{job|)}
\index{job|)}

%\section{Run modes}
%\index{run modes|(}
\section{Run modes}
\index{run modes|(}

%By default, \TeX\ goes into \cs{errorstopmode} if an error occurs:
%\cstoidx errorstopmode\par
%it stops and asks for input from the user. Some implementations 
% have a way of forcing \TeX\ into errorstopmode
%when the user interrupts \TeX, so that
%the internal state of \TeX\ can be inspected (and altered). 
%See page~\pageref{interaction} for ways to switch the run
%mode when \TeX\ has been interrupted.
By default, \TeX\ goes into \cs{errorstopmode} if an error occurs:
\cstoidx errorstopmode\par
it stops and asks for input from the user. Some implementations 
 have a way of forcing \TeX\ into errorstopmode
when the user interrupts \TeX, so that
the internal state of \TeX\ can be inspected (and altered). 
See page~\pageref{interaction} for ways to switch the run
mode when \TeX\ has been interrupted.

%Often, \TeX\ can
%fix an error itself if the user asks \TeX\ just to continue
%(usually by hitting the return key),
%but sometimes (for instance in alignments)
%it may take a while before \TeX\ is on the
%right track again (and sometimes it never is).
%In such cases the user may want to
%turn on \csidx{scrollmode},
%which instructs \TeX\ to fix as best it can any
%occurring error without confirmation from the user.
%This is usually done by typing `s' when \TeX\ asks
%for input.
Often, \TeX\ can
fix an error itself if the user asks \TeX\ just to continue
(usually by hitting the return key),
but sometimes (for instance in alignments)
it may take a while before \TeX\ is on the
right track again (and sometimes it never is).
In such cases the user may want to
turn on \csidx{scrollmode},
which instructs \TeX\ to fix as best it can any
occurring error without confirmation from the user.
This is usually done by typing `s' when \TeX\ asks
for input.

%In \cs{scrollmode}, \TeX\ also does not ask for input
%after \cs{show...} commands.
%\alt
%However, some errors, such as a file that could not be
%found for \cs{input}, are not so easily remedied, so
%the user will still be asked for input.
In \cs{scrollmode}, \TeX\ also does not ask for input
after \cs{show...} commands.
\alt
However, some errors, such as a file that could not be
found for \cs{input}, are not so easily remedied, so
the user will still be asked for input.

%With \csidx{nonstopmode} \TeX\ will scroll through errors and,
%in the case of the kind of error that cannot be recovered from,
%it will make an emergency stop, aborting the run.
%Also \TeX\ will abort the run if a \cs{read} is attempted
%from the terminal.
%The \csidx{batchmode} differs only from nonstopmode in that
%it gives messages only to the log file, not to the terminal.
With \csidx{nonstopmode} \TeX\ will scroll through errors and,
in the case of the kind of error that cannot be recovered from,
it will make an emergency stop, aborting the run.
Also \TeX\ will abort the run if a \cs{read} is attempted
from the terminal.
The \csidx{batchmode} differs only from nonstopmode in that
it gives messages only to the log file, not to the terminal.

%\index{run modes|)}
%\endofchapter
\index{run modes|)}
\endofchapter
