
\chapter{Page Shape}\label{page:shape}

This chapter treats some of the parameters that 
determine the size of the page and how it appears on paper.

\label{cschap:topskip}\label{cschap:hoffset}\label{cschap:voffset}\label{cschap:vsize2}\label{cschap:maxdepth}\label{cschap:splitmaxdepth2}
\begin{inventory}
\item [\cs{topskip}] 
      Minimum distance between the top of the page box
      and the baseline of the first box on the page. 
      Plain \TeX\ default:~\n{10pt}

\item [\cs{hoffset \cs{voffset}}]
\mdqon
      Distance by which the page is shifted right/""down 
\mdqoff
      with respect to the reference point.

\item [\cs{vsize}] 
      Height of the page box.
      Plain \TeX\ default:~\n{8.9in}

\item [\cs{maxdepth}] 
      Maximum depth of the page box.
      Plain \TeX\ default:~\n{4pt}

\item [\cs{splitmaxdepth}] 
      Maximum depth of a box split off by a \cs{vsplit} operation. 
      Plain \TeX\ default: \cs{maxdimen}

\end{inventory}

%\point The reference point for global positioning
\section{The reference point for global positioning}

The \indexterm{page positioning} on the paper is governed by
a \TeX\ convention, to which output device drivers
must adhere, that the top left point of the page is
one inch from the page edges. Unfortunately this
may lead to lots of trouble, for instance if a printer
(or the page description language it uses)
takes, say, the {\em lower\/} left corner as the
reference point, and is factory set to US paper sizes,
but is used with European standard A4 paper.

The page is shifted on the paper if one assigns non-zero
values to \csidx{hoffset} or \csidx{voffset}: positive values
shift to the right and down respectively.

%\point \cs{topskip}
\section{\protect\cs{topskip}}

The \csidx{topskip} ensures to a certain point
that the first baseline of a page
will be at the same location from page to page,
even if font sizes
are switched between pages or if the first line has
no ascenders.

Before the first box on each page some glue is inserted.
This glue has the same stretch and shrink as \cs{topskip}, but
the natural size is the natural size of \cs{topskip}
minus the height of the first box, or zero if this
would be negative.

Plain \TeX\ sets \cs{topskip} to {\tt 10pt}.
Thus the top lines of pages will have their baselines
at the same place if
the top portion of the characters is ten point or less.
For the Computer Modern fonts this condition is satisfied
if the font size is less than (about) 13~points; 
for larger fonts
the baseline of the top line will drop.

The height of the page box for a page containing only
text (and assuming a zero \cs{parskip})
will be the \cs{topskip} plus a number of times
the \cs{baselineskip}. Thus one can define a macro
to compute the \cs{vsize} from the number of lines
on a page:
\howto Specify page height in lines\par
\begin{verbatim}
\def\HeightInLines#1{\count@=#1\relax
    \advance\count@ by -1\relax
    \vsize=\baselineskip
    \multiply\vsize by \count@
    \advance\vsize by \topskip}
\end{verbatim}
Calculating the \cs{vsize} this way will prevent
underfull boxes for text-only pages.

In cases where the page does not start with a line of text
(for instance a rule), the topskip may give unwanted effects.
To prevent these, start the page with
\begin{verbatim}
\hbox{}\kern-\topskip
\end{verbatim}
followed by what you wanted on top. 

Analogous to the \cs{topskip}, there is a \cs{splittopskip}
for pages generated by a \cs{vsplit} operation; see
the next chapter.

%\point Page height and depth
\section{Page height and depth}

\index{page!height|(}
\index{page!depth|(}

\TeX\ tries to build pages as a \cs{vbox} of height \csidx{vsize};
\alt
see also \cs{pagegoal} in the next chapter.

If the last item on a page has an excessive depth,
that page would be noticeably longer than other pages.
To prevent this phenomenon \TeX\ uses \csidx{maxdepth} as
the maximum depth of the page box. If adding an item to the
page would make the depth exceed this quantity, then the
reference point of the page is moved down to make the depth
exactly \cs{maxdepth}.

The `raggedbottom' effect is obtained in plain \TeX\
\cstoidx raggedbottom\par
by giving the \cs{topskip} some finite stretchability:
\hbox{\n{10pt plus 60pt}}.
Thus the natural height of box~255 can vary when it reaches
the output routine.
Pages are then shipped out (more or less) as
\begin{verbatim}
\dimen0=\dp255 \unvbox255
\ifraggedbottom \kern-\dimen0 \vfil \fi 
\end{verbatim}
The \cs{vfil} causes the topskip to be set at natural
width, so the effect is one of a fixed top line  and a
variable bottom line of the page.

Before \cs{box255} is unboxed in the plain \TeX\ output routine,
\cs{boxmaxdepth} is set to \cs{maxdepth}
so that this box will made under the same assumptions
that the page builder used when putting together \cs{box255}.

The depth of box split off by a \cs{vsplit} operation
is controlled by the \cs{splitmaxdepth} parameter.

\index{page!height|)}
\index{page!depth|)}

\endofchapter
