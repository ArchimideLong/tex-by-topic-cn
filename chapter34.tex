% -*- coding: utf-8 -*-
\documentclass{book}

% -*- coding: utf-8 -*-

\usepackage[b5paper,text={5in,8in},centering]{geometry}

\usepackage[CJKchecksingle]{xeCJK}
\setmainfont[Mapping=tex-text]{TeX Gyre Schola}
%\setsansfont{URW Gothic L Book}
%\setmonofont{Nimbus Mono L}
\setCJKmainfont[BoldFont=FandolHei,ItalicFont=FandolKai]{FandolSong}
\setCJKsansfont{FandolHei}
\setCJKmonofont{FandolFang}
\xeCJKsetup{PunctStyle = kaiming}

\linespread{1.25}
\setlength{\parindent}{2em}

\usepackage{xcolor}
\definecolor{myblue}{rgb}{0,0.2,0.6}

\usepackage{titlesec}
\titleformat{\chapter}
    {\normalfont\Huge\sffamily\color{myblue}}
    {第\thechapter 章}
    {1em}
    {}
%\titlespacing{\chapter}{0pt}{50pt}{40pt}
\titleformat{\section}
    {\normalfont\Large\sffamily\color{myblue}}
    {\thesection}
    {1em}
    {}
%\titlespacing{\section}{0pt}{3.5ex plus 1ex minus .2ex}{2.3ex plus .2ex}
\titleformat{\subsection}
    {\normalfont\large\sffamily\color{myblue}}
    {\thesubsection}
    {1em}
    {}
%\titlespacing{\subsection}{0pt}{3.25ex plus 1ex minus .2ex}{1.5ex plus .2ex}
%
\newpagestyle{special}[\small\sffamily]{
  \headrule
  \sethead[\usepage][][\chaptertitle]
  {\chaptertitle}{}{\usepage}}
\newpagestyle{main}[\small\sffamily]{
  \headrule
  \sethead[\usepage][][第\thechapter 章\quad\chaptertitle]
  {\thesection\quad\sectiontitle}{}{\usepage}}

\usepackage{titletoc}
%\setcounter{tocdepth}{1}
%\titlecontents{标题层次}[左间距]{上间距和整体格式}{标题序号}{标题内容}{指引线和页码}[下间距]
\titlecontents{chapter}[1.5em]{\vspace{.5em}\bfseries\sffamily}{\color{myblue}\contentslabel{1.5em}}{}
    {\titlerule*[20pt]{$\cdot$}\contentspage}[]
\titlecontents{section}[4.5em]{\sffamily}{\color{myblue}\contentslabel{3em}}{}
    {\titlerule*[20pt]{$\cdot$}\contentspage}[]
%\titlecontents{subsection}[8.5em]{\sffamily}{\contentslabel{4em}}{}
%    {\titlerule*[20pt]{$\cdot$}\contentspage}

\usepackage{enumitem}
\setlist{topsep=2pt,itemsep=2pt,parsep=1pt,leftmargin=\parindent}

\usepackage{fancyvrb}
\DefineVerbatimEnvironment{verbatim}{Verbatim}
  {xleftmargin=2em,baselinestretch=1,formatcom=\color{teal}\upshape}

\usepackage{etoolbox}
\makeatletter
\preto{\FV@ListVSpace}{\topsep=2pt \partopsep=0pt }
\makeatother

\usepackage[colorlinks,plainpages,pagebackref]{hyperref}
\hypersetup{
   pdfstartview={FitH},
   citecolor=teal,
   linkcolor=myblue,
   urlcolor=black,
   bookmarksnumbered
}

\usepackage{comment,makeidx,multicol}

%\usepackage{german}
%% german
%\righthyphenmin=3
%\mdqoff
%\captionsenglish
\usepackage[english]{babel}
{\catcode`"=13 \gdef"#1{\ifx#1"\discretionary{}{}{}\fi\relax}}
\def\mdqon{\catcode`"=13\relax}
\def\mdqoff{\catcode`"=12\relax}
\makeindex
\hyphenation{ex-em-pli-fies}

\newdimen\tempdima \newdimen\tempdimb

% these are fine
\def\handbreak{\\ \message{^^JManual break!!!!^^J}}
\def\nl{\protect\\}\def\n#1{{\tt #1}}
\def\cs#1{\texorpdfstring{{\tt\char`\\#1}}{\textbackslash#1}} %\def\cs#1{{\tt\char`\\#1}}
\let\csc\cs
\def\lb{{\tt\char`\{}}\def\rb{{\tt\char`\}}}
\def\gr#1{\texorpdfstring{$\langle$#1$\rangle$}{<#1>}} %\def\gr#1{$\langle$#1$\rangle$}
\def\key#1{{\tt#1}}
\def\alt{}\def\altt{}%this way in manstijl
\def\ldash{\unskip\ --\nobreak\ \ignorespaces}
\def\rdash{\unskip\nobreak\ --\ \ignorespaces}
% check these
\def\hex{{\tt"}}
\def\ascii{{\sc ascii}}
\def\ebcdic{{\sc ebcdic}}
\def\IniTeX{Ini\TeX}\def\LamsTeX{LAMS\TeX}\def\VirTeX{Vir\TeX}
\def\AmsTeX{Ams\TeX}
\def\TeXbook{the \TeX\ book}\def\web{{\sc web}}
% needs major thinking
\newenvironment{myquote}{\list{}{%
    \topsep=2pt \partopsep=0pt%
    \leftmargin=\parindent \rightmargin=\parindent
    }\item[]}{\endlist}
\newenvironment{disp}{\begin{myquote}}{\end{myquote}}
\newenvironment{Disp}{\begin{myquote}}{\end{myquote}}
\newenvironment{tdisp}{\begin{myquote}}{\end{myquote}}
\newenvironment{example}{\begin{myquote}\noindent\itshape 例子:}{\end{myquote}}
\newenvironment{inventory}{\begin{description}\raggedright}{\end{description}}
\newenvironment{glossinventory}{\begin{description}}{\end{description}}
\def\gram#1{\gr{#1}}%???
%
% index
%
\def\indexterm#1{\emph{#1}\index{#1}}
\def\indextermsub#1#2{\emph{#1 #2}\index{#1!#2}}
\def\indextermbus#1#2{\emph{#1 #2}\index{#2!#1}}
\def\term#1\par{\index{#1}}
\def\howto#1\par{}
\def\cstoidx#1\par{\index{#1@\cs{#1}@}}
\def\thecstoidx#1\par{\index{#1@\protect\csname #1\endcsname}}
\def\thecstoidxsub#1#2{\index{#1, #2@\protect\csname #1\endcsname, #2}\ignorespaces}
\def\csterm#1\par{\cstoidx #1\par\cs{#1}}
\def\csidx#1{\cstoidx #1\par\cs{#1}}

\def\tmc{\tracingmacros=2 \tracingcommands\tracingmacros}

%%%%%%%%%%%%%%%%%%%
\makeatletter
\def\snugbox{\hbox\bgroup\setbox\z@\vbox\bgroup
    \leftskip\z@
    \bgroup\aftergroup\make@snug
    \let\next=}
\def\make@snug{\par\sn@gify\egroup \box\z@\egroup}
\def\sn@gify
   {\skip\z@=\lastskip \unskip
    \advance\skip\z@\lastskip \unskip
    \unpenalty
    \setbox\z@\lastbox
    \ifvoid\z@ \nointerlineskip \else {\sn@gify} \fi
    \hbox{\unhbox\z@}\nointerlineskip
    \vskip\skip\z@
    }

\newdimen\fbh \fbh=60pt % dimension for easy scaling:
\newdimen\fbw \fbw=60pt % height and width of character box

\newdimen\dh \newdimen\dw % height and width of current character box
\newdimen\lh % height of previous character box
\newdimen\lw \lw=.4pt % line weight, instead of default .4pt

\def\hdotfill{\noindent
    \leaders\hbox{\vrule width 1pt height\lw
                  \kern4pt
                  \vrule width.5pt height\lw}\hfill\hbox{}
    \par}
\def\hlinefill{\noindent
    \leaders\hbox{\vrule width 5.5pt height\lw         }\hfill\hbox{}
    \par}
\def\stippel{$\qquad\qquad\qquad\qquad$}
\makeatother
%%%%%%%%%%%%%%%%%%%

%\def\SansSerif{\Typeface:macHelvetica }
%\def\SerifFont{\Typeface:macTimes }
%\def\SansSerif{\Typeface:bsGillSans }
%\def\SerifFont{\Typeface:bsBaskerville }
\let\SansSerif\relax \def\italic{\it}
\let\SerifFont\relax \def\MainFont{\rm}
\let\SansSerif\relax
\let\SerifFont\relax
\let\PopIndentLevel\relax \let\PushIndentLevel\relax
\let\ToVerso\relax \let\ToRecto\relax

%\def\stop@command@suffix{stop}
%\let\PopListLevel\PopIndentLevel
%\let\FlushRight\relax
%\let\flushright\FlushRight
%\let\SetListIndent\LevelIndent
%\def\awp{\ifhmode\vadjust{\penalty-10000 }\else
%    \penalty-10000 \fi}
\let\awp\relax
\let\PopIndentLevel\relax \let\PopListLevel\relax

\showboxdepth=-1

%\input figs
\def\endofchapter{\vfill\noindent}

\setcounter{chapter}{33}

\begin{document}

%\chapter{Tracing}\label{trace}
%\index{tracing|(}
\chapter{Tracing}\label{trace}
\index{tracing|(}

%\TeX's workings are often quite different from what
%the programmer expected, so there are ways to discover how \TeX\
%arrived at the result it did. The \cs{tracing...} 
%commands all write
%\indexterm{statistics}
%information of a certain kind to the log file 
%(and to the terminal if \cs{tracingonline} is positive),
%and a number of \cs{show...} commands can be used to ask the
%current status or value of various items of \TeX.
\TeX's workings are often quite different from what
the programmer expected, so there are ways to discover how \TeX\
arrived at the result it did. The \cs{tracing...} 
commands all write
\indexterm{statistics}
information of a certain kind to the log file 
(and to the terminal if \cs{tracingonline} is positive),
and a number of \cs{show...} commands can be used to ask the
current status or value of various items of \TeX.

%In the following list, only \cs{show} and \cs{showthe}
%display their output on the terminal by default,
%other \cs{show...} and \cs{tracing...} commands
%write to the log file. They will write in addition to
%the terminal if \cs{tracingonline} is positive.
In the following list, only \cs{show} and \cs{showthe}
display their output on the terminal by default,
other \cs{show...} and \cs{tracing...} commands
write to the log file. They will write in addition to
the terminal if \cs{tracingonline} is positive.

%\label{cschap:meaning}\label{cschap:show}\label{cschap:showthe}\label{cschap:showbox}\label{cschap:tracingoutput}\label{cschap:showboxdepth}\label{cschap:showboxbreadth}\label{cschap:tracingparagraphs}\label{cschap:tracingpages}\label{cschap:tracinglostchars}\label{cschap:tracingrestores}\label{cschap:tracingstats}\label{cschap:tracingcommands}\label{cschap:tracingmacros}\label{cschap:tracingonline}
%\begin{inventory}
%\item [\cs{meaning}] 
%      Give the meaning of a control sequence as a string of characters.
\label{cschap:meaning}\label{cschap:show}\label{cschap:showthe}\label{cschap:showbox}\label{cschap:tracingoutput}\label{cschap:showboxdepth}\label{cschap:showboxbreadth}\label{cschap:tracingparagraphs}\label{cschap:tracingpages}\label{cschap:tracinglostchars}\label{cschap:tracingrestores}\label{cschap:tracingstats}\label{cschap:tracingcommands}\label{cschap:tracingmacros}\label{cschap:tracingonline}
\begin{inventory}
\item [\cs{meaning}] 
      Give the meaning of a control sequence as a string of characters.

%\item [\cs{show}] 
%      Display the meaning of a control sequence.
\item [\cs{show}] 
      Display the meaning of a control sequence.

%\item [\cs{showthe}] 
%      Display the result of prefixing a token with \cs{the}.
\item [\cs{showthe}] 
      Display the result of prefixing a token with \cs{the}.

%\item [\cs{showbox}] 
%      Display the contents of a box.
\item [\cs{showbox}] 
      Display the contents of a box.

%\item [\cs{showlists}] 
%      Display
%      the contents of the partial lists
%      currently built in all modes.
%      This is treated on
%      page~\pageref{showlists}.
\item [\cs{showlists}] 
      Display
      the contents of the partial lists
      currently built in all modes.
      This is treated on
      page~\pageref{showlists}.

%\item [\csidx{tracingcommands}]
%      If this is~1 \TeX\ displays primitive commands executed; 
%      if this is 2~or more the outcome of conditionals is also recorded.
\item [\csidx{tracingcommands}]
      If this is~1 \TeX\ displays primitive commands executed; 
      if this is 2~or more the outcome of conditionals is also recorded.

%\item [\csidx{tracingmacros}] 
%      If this is~1, \TeX\ shows expansion of macros 
%      that are performed and the actual values of the arguments; 
%      if this is 2~or more \gr{token parameter}s such as
%      \cs{output} and \cs{everypar} are also traced.
\item [\csidx{tracingmacros}] 
      If this is~1, \TeX\ shows expansion of macros 
      that are performed and the actual values of the arguments; 
      if this is 2~or more \gr{token parameter}s such as
      \cs{output} and \cs{everypar} are also traced.

%\item [\cs{tracingoutput}] 
%      If this is positive, the log file shows a dump of boxes 
%      that are shipped to the \n{dvi} file.
\item [\cs{tracingoutput}] 
      If this is positive, the log file shows a dump of boxes 
      that are shipped to the \n{dvi} file.

%\item [\cs{showboxdepth}]  
%      The number of levels of box dump that are shown when 
%      boxes are displayed.
\item [\cs{showboxdepth}]  
      The number of levels of box dump that are shown when 
      boxes are displayed.

%\item [\cs{showboxbreadth}] 
%      Number of successive elements on each level that are shown when 
%      boxes are displayed.
\item [\cs{showboxbreadth}] 
      Number of successive elements on each level that are shown when 
      boxes are displayed.

%\item [\csidx{tracingonline}] 
%      If this parameter is positive, \TeX\ will write trace      
%      information to the terminal in addition to the log file.
\item [\csidx{tracingonline}] 
      If this parameter is positive, \TeX\ will write trace      
      information to the terminal in addition to the log file.

%\item [\cs{tracingparagraphs}] 
%      If this parameter is positive, \TeX\ generates      
%      a trace of the line breaking algorithm.
\item [\cs{tracingparagraphs}] 
      If this parameter is positive, \TeX\ generates      
      a trace of the line breaking algorithm.

%\item [\csidx{tracingpages}] 
%      If this parameter is positive, \TeX\ generates      
%      a trace of the page breaking algorithm.
\item [\csidx{tracingpages}] 
      If this parameter is positive, \TeX\ generates      
      a trace of the page breaking algorithm.

%\item [\csidx{tracinglostchars}] 
%      If this parameter is positive, \TeX\ gives      
%      diagnostic messages whenever a character is accessed that      
%      is not present in a font.
%      Plain default:~1.
\item [\csidx{tracinglostchars}] 
      If this parameter is positive, \TeX\ gives      
      diagnostic messages whenever a character is accessed that      
      is not present in a font.
      Plain default:~1.

%\item [\csidx{tracingrestores}] 
%      If this parameter is positive, \TeX\ will report      
%      all values that are restored when a group ends.
\item [\csidx{tracingrestores}] 
      If this parameter is positive, \TeX\ will report      
      all values that are restored when a group ends.

%\item [\cs{tracingstats}] 
%      If this parameter is~1, \TeX\ reports at the      
%      end of the job the usage of various internal arrays;
%      if it is~2, the memory demands are given whenever
%      a page is shipped out.
\item [\cs{tracingstats}] 
      If this parameter is~1, \TeX\ reports at the      
      end of the job the usage of various internal arrays;
      if it is~2, the memory demands are given whenever
      a page is shipped out.

%\end{inventory}
\end{inventory}

%\section{Meaning and content: \protect\cs{show}, \protect\cs{showthe}, \protect\cs{meaning}}
\section{Meaning and content: \protect\cs{show}, \protect\cs{showthe}, \protect\cs{meaning}}

%The meaning of control sequences, and the contents of those
%that represent internal quantities, can be obtained
%by the primitive commands \cs{show}, \cs{showthe},
%and~\cs{meaning}.
The meaning of control sequences, and the contents of those
that represent internal quantities, can be obtained
by the primitive commands \cs{show}, \cs{showthe},
and~\cs{meaning}.

%The control sequences \cs{show} and \cs{meaning} are similar:
%\alt
%the former will give
%\cstoidx show\par\cstoidx meaning\par
%output to the log file and the terminal, whereas the latter
%will produce the same tokens, but they are placed in \TeX's
%input stream.
The control sequences \cs{show} and \cs{meaning} are similar:
\alt
the former will give
\cstoidx show\par\cstoidx meaning\par
output to the log file and the terminal, whereas the latter
will produce the same tokens, but they are placed in \TeX's
input stream.

%The meaning of a primitive command of \TeX\ is that command itself:
%\begin{verbatim}
%\show\baselineskip
%\end{verbatim}
%gives
%\begin{verbatim}
%\baselineskip=\baselineskip
%\end{verbatim}
%The meaning of a defined  quantity is its definition:
%\begin{verbatim}
%\show\pageno
%\end{verbatim}
%gives
%\begin{verbatim}
%\pageno=\count0
%\end{verbatim}
%The meaning of a macro is its parameter text and replacement text:
%\begin{verbatim}
%\def\foo#1?#2\par{\set{#1!}\set{#2?}}
%\show\foo
%\end{verbatim}
%gives
%\begin{verbatim}
%\foo=macro:
%#1?#2\par ->\set {#1!}\set {#2?}
%\end{verbatim}
%For macros without parameters the part before the arrow
%(the parameter text) is empty.
The meaning of a primitive command of \TeX\ is that command itself:
\begin{verbatim}
\show\baselineskip
\end{verbatim}
gives
\begin{verbatim}
\baselineskip=\baselineskip
\end{verbatim}
The meaning of a defined  quantity is its definition:
\begin{verbatim}
\show\pageno
\end{verbatim}
gives
\begin{verbatim}
\pageno=\count0
\end{verbatim}
The meaning of a macro is its parameter text and replacement text:
\begin{verbatim}
\def\foo#1?#2\par{\set{#1!}\set{#2?}}
\show\foo
\end{verbatim}
gives
\begin{verbatim}
\foo=macro:
#1?#2\par ->\set {#1!}\set {#2?}
\end{verbatim}
For macros without parameters the part before the arrow
(the parameter text) is empty.

%The \csidx{showthe} command will display on the log file and terminal 
%the tokens that \cs{the} produces. 
%After \cs{show}, \cs{showthe}, \cs{showbox}, and \cs{showlists}
%\TeX\ asks the user for input; this can be prevented
%by specifying \cs{scrollmode}.
%Characters generated
%by \cs{meaning} and \cs{the} have category~12\index{category!12},
%except for spaces
%(see page~\pageref{cat12});
%the value of \cs{escapechar} is used when control sequences
%are represented.
The \csidx{showthe} command will display on the log file and terminal 
the tokens that \cs{the} produces. 
After \cs{show}, \cs{showthe}, \cs{showbox}, and \cs{showlists}
\TeX\ asks the user for input; this can be prevented
by specifying \cs{scrollmode}.
Characters generated
by \cs{meaning} and \cs{the} have category~12\index{category!12},
except for spaces
(see page~\pageref{cat12});
the value of \cs{escapechar} is used when control sequences
are represented.

%%\point Show boxes: \cs{showbox}, \cs{tracingoutput}
%\section{Show boxes: \protect\cs{showbox}, \protect\cs{tracingoutput}}
%\point Show boxes: \cs{showbox}, \cs{tracingoutput}
\section{Show boxes: \protect\cs{showbox}, \protect\cs{tracingoutput}}

%If \cs{tracingoutput} is positive the log file will
%\cstoidx tracingoutput\par\cstoidx showbox\par
%receive a dumped representation of all boxes that are
%written to the \n{dvi} file with \cs{shipout}.
%The same representation is used
%by the command \cs{showbox}\gr{8-bit number}.
If \cs{tracingoutput} is positive the log file will
\cstoidx tracingoutput\par\cstoidx showbox\par
receive a dumped representation of all boxes that are
written to the \n{dvi} file with \cs{shipout}.
The same representation is used
by the command \cs{showbox}\gr{8-bit number}.

%In the first case \TeX\ will report `Completed box being shipped out';
%in the second case it will enter \cs{errorstopmode}, and
%tell the user `OK. (see the transcript file)'.
%If \cs{tracingonline} is positive, the box is also displayed
%on the terminal; if \cs{scrollmode} has been specified,
%\TeX\ does not stop for input.
In the first case \TeX\ will report `Completed box being shipped out';
in the second case it will enter \cs{errorstopmode}, and
tell the user `OK. (see the transcript file)'.
If \cs{tracingonline} is positive, the box is also displayed
on the terminal; if \cs{scrollmode} has been specified,
\TeX\ does not stop for input.

%The upper bound on the
%number of nested boxes that is dumped is \cs{showboxdepth};
%\cstoidx showboxdepth\par\cstoidx showboxbreadth\par
%each time a level is visited at most \cs{showboxbreadth}
%items are shown, the remainder of the list is summarized
%with~\n{etc.}
%For each box its height, depth, and width
%are indicated in that order, and for characters it is
%stated from what font they were taken. 
The upper bound on the
number of nested boxes that is dumped is \cs{showboxdepth};
\cstoidx showboxdepth\par\cstoidx showboxbreadth\par
each time a level is visited at most \cs{showboxbreadth}
items are shown, the remainder of the list is summarized
with~\n{etc.}
For each box its height, depth, and width
are indicated in that order, and for characters it is
stated from what font they were taken. 

%\begin{example} After
%\begin{verbatim}
%\font\tenroman=cmr10 \tenroman
%\setbox0=\hbox{g}
%\showbox0
%\end{verbatim} 
%the log file will show
%\begin{verbatim}
%\hbox(4.30554+1.94444)x5.00002
%.\tenroman g
%\end{verbatim}
%indicating that the box was \n{4.30554pt} high,
%\n{1.94444pt} deep, and \n{5.00002pt} wide, and that it contained
%a character `g' from the font \cs{tenroman}. 
%Note that the fifth decimal of all sizes may be rounded
%because \TeX\ works with multiples of $2^{-16}$\n{pt}.
%\message{ifmath: scriptfont fam0 fill!}
%\end{example}
\begin{example} After
\begin{verbatim}
\font\tenroman=cmr10 \tenroman
\setbox0=\hbox{g}
\showbox0
\end{verbatim} 
the log file will show
\begin{verbatim}
\hbox(4.30554+1.94444)x5.00002
.\tenroman g
\end{verbatim}
indicating that the box was \n{4.30554pt} high,
\n{1.94444pt} deep, and \n{5.00002pt} wide, and that it contained
a character `g' from the font \cs{tenroman}. 
Note that the fifth decimal of all sizes may be rounded
because \TeX\ works with multiples of $2^{-16}$\n{pt}.
\message{ifmath: scriptfont fam0 fill!}
\end{example}

%The next example has nested boxes, 
%\begin{verbatim}
%\vbox{\hbox{g}\hbox{o}}
%\end{verbatim}
%and it contains \cs{baselineskip} glue between the boxes.
%After a \cs{showbox} command the log file output is:
%\begin{verbatim}
%\vbox(16.30554+0.0)x5.00002
%.\hbox(4.30554+1.94444)x5.00002
%..\tenroman g
%.\glue(\baselineskip) 5.75002
%.\hbox(4.30554+0.0)x5.00002
%..\tenroman o
%\end{verbatim}
%Each time a new level is entered an extra dot is added to
%the front of the line. Note that \TeX\ tells explicitly
%that the glue is \cs{baselineskip} glue;
%it inserts names like this for all automatically inserted glue.
%The value of
%the baselineskip glue here is such that the baselines of
%the boxes are at 12 point distance.
The next example has nested boxes, 
\begin{verbatim}
\vbox{\hbox{g}\hbox{o}}
\end{verbatim}
and it contains \cs{baselineskip} glue between the boxes.
After a \cs{showbox} command the log file output is:
\begin{verbatim}
\vbox(16.30554+0.0)x5.00002
.\hbox(4.30554+1.94444)x5.00002
..\tenroman g
.\glue(\baselineskip) 5.75002
.\hbox(4.30554+0.0)x5.00002
..\tenroman o
\end{verbatim}
Each time a new level is entered an extra dot is added to
the front of the line. Note that \TeX\ tells explicitly
that the glue is \cs{baselineskip} glue;
it inserts names like this for all automatically inserted glue.
The value of
the baselineskip glue here is such that the baselines of
the boxes are at 12 point distance.

%Now let us look at explicit (user) glue. \TeX\ indicates the ratio
%by which it is stretched or shrunk. 
Now let us look at explicit (user) glue. \TeX\ indicates the ratio
by which it is stretched or shrunk. 

%\begin{example}
%\begin{verbatim}
%\hbox to 20pt {\kern10pt \hskip0pt plus 5pt}
%\end{verbatim}
%gives (indicating that the available stretch has been 
%multiplied by~\n{2.0}):
%\begin{verbatim}
%\hbox(0.0+0.0)x20.0, glue set 2.0
%.\kern 10.0
%.\glue 0.0 plus 5.0
%\end{verbatim}
%and
%\begin{verbatim}
%\hbox to 0pt {\kern10pt \hskip0pt minus 20pt}
%\end{verbatim}
%gives (the shrink has been multiplied by~\n{0.5})
%\begin{verbatim}
%\hbox(0.0+0.0)x0.0, glue set - 0.5
%.\kern 10.0
%.\glue 0.0 minus 20.0
%\end{verbatim}
%respectively.
%\end{example}
\begin{example}
\begin{verbatim}
\hbox to 20pt {\kern10pt \hskip0pt plus 5pt}
\end{verbatim}
gives (indicating that the available stretch has been 
multiplied by~\n{2.0}):
\begin{verbatim}
\hbox(0.0+0.0)x20.0, glue set 2.0
.\kern 10.0
.\glue 0.0 plus 5.0
\end{verbatim}
and
\begin{verbatim}
\hbox to 0pt {\kern10pt \hskip0pt minus 20pt}
\end{verbatim}
gives (the shrink has been multiplied by~\n{0.5})
\begin{verbatim}
\hbox(0.0+0.0)x0.0, glue set - 0.5
.\kern 10.0
.\glue 0.0 minus 20.0
\end{verbatim}
respectively.
\end{example}

%This is an example with infinitely stretchable or shrinkable
%glue:
%\begin{verbatim}
%\hbox(4.00000+0.14000)x15.0, glue set 9.00000fil
%\end{verbatim}
%This means that the horizontal box contained \n{fil} glue, and
%it was set such that its resulting width was \n{9pt}.
This is an example with infinitely stretchable or shrinkable
glue:
\begin{verbatim}
\hbox(4.00000+0.14000)x15.0, glue set 9.00000fil
\end{verbatim}
This means that the horizontal box contained \n{fil} glue, and
it was set such that its resulting width was \n{9pt}.

%Underfull boxes are dumped like all other boxes, but
%the usual `\n{Underfull hbox detected at line...}'
%is given. Overfull horizontal boxes contain a vertical rule
%of width \cs{overfullrule}:
%\begin{verbatim}
%\hbox to 5pt {\kern10pt}
%\end{verbatim}
%gives
%\begin{verbatim}
%\hbox(0.0+0.0)x5.0
%.\kern 10.0
%.\rule(*+*)x5.0
%\end{verbatim}
Underfull boxes are dumped like all other boxes, but
the usual `\n{Underfull hbox detected at line...}'
is given. Overfull horizontal boxes contain a vertical rule
of width \cs{overfullrule}:
\begin{verbatim}
\hbox to 5pt {\kern10pt}
\end{verbatim}
gives
\begin{verbatim}
\hbox(0.0+0.0)x5.0
.\kern 10.0
.\rule(*+*)x5.0
\end{verbatim}


%Box leaders are not dumped completely:
%\begin{verbatim}
%.\leaders 40.0
%..\hbox(4.77313+0.14581)x15.0, glue set 9.76852fil
%...\tenrm a
%...\glue 0.0 plus 1.0fil
%\end{verbatim}
%is the dump for
%\begin{verbatim}
%\leaders\hbox to 15pt{\tenrm a\hfil}\hskip 40pt
%\end{verbatim}
%Preceding or trailing glue around the leader
%boxes is also not indicated.
Box leaders are not dumped completely:
\begin{verbatim}
.\leaders 40.0
..\hbox(4.77313+0.14581)x15.0, glue set 9.76852fil
...\tenrm a
...\glue 0.0 plus 1.0fil
\end{verbatim}
is the dump for
\begin{verbatim}
\leaders\hbox to 15pt{\tenrm a\hfil}\hskip 40pt
\end{verbatim}
Preceding or trailing glue around the leader
boxes is also not indicated.

%%\point Global statistics
%\section{Global statistics}
%\point Global statistics
\section{Global statistics}

%The parameter \csidx{tracingstats} can be used to force \TeX\
%to report at the end of the job the global use of resources.
%Some production versions of \TeX\ may not have this option.
The parameter \csidx{tracingstats} can be used to force \TeX\
to report at the end of the job the global use of resources.
Some production versions of \TeX\ may not have this option.

%As an example, here are the statistics for this book:
%\begin{verbatim}
%Here is how much of TeX's memory you used:
%\end{verbatim}
%String memory (bounded by `pool size'):
%\begin{verbatim}
% 877 strings out of 4649
% 9928 string characters out of 61781
%\end{verbatim}
%Main memory, control sequences, font memory:
%\begin{verbatim}
% 53071 words of memory out of 262141
% 2528 multiletter control sequences out of 9500
% 20137 words of font info for 70 fonts,
%             out of 72000 for 255
%\end{verbatim}
%Hyphenation:
%\begin{verbatim}
% 14 hyphenation exceptions out of 607
%\end{verbatim}
%Stacks: input, nest, parameter, buffer, and save stack respectively,
%\begin{verbatim}
% 17i,6n,19p,245b,422s stack positions out of 
% 300i,40n,60p,3000b,4000s
%\end{verbatim}
As an example, here are the statistics for this book:
\begin{verbatim}
Here is how much of TeX's memory you used:
\end{verbatim}
String memory (bounded by `pool size'):
\begin{verbatim}
 877 strings out of 4649
 9928 string characters out of 61781
\end{verbatim}
Main memory, control sequences, font memory:
\begin{verbatim}
 53071 words of memory out of 262141
 2528 multiletter control sequences out of 9500
 20137 words of font info for 70 fonts,
             out of 72000 for 255
\end{verbatim}
Hyphenation:
\begin{verbatim}
 14 hyphenation exceptions out of 607
\end{verbatim}
Stacks: input, nest, parameter, buffer, and save stack respectively,
\begin{verbatim}
 17i,6n,19p,245b,422s stack positions out of 
 300i,40n,60p,3000b,4000s
\end{verbatim}

%%\input tracingparagraphs
%\input tracingparagraphs

%\index{tracing|)}
%\endofchapter
%%%%% end of input file [trace]
\index{tracing|)}
\endofchapter
%%%% end of input file [trace]

\end{document}
