
\pagebreak
\mark{}
\addcontentsline{toc}{section}{Preface}

\paragraph*{\bf Preface}
To the casual observer, \TeX\
is not a state-of-the-art typesetting system.
No flashy multilevel menus and interactive manipulation
of text and graphics dazzle the onlooker.
On a less superficial level, however, \TeX\ is a very sophisticated
program, first of all because of the ingeniousness of its
built-in algorithms for such things as paragraph breaking
and make-up of mathematical formulas, and
second because of its almost complete programmability.
The combination of these factors makes it possible for \TeX\
to realize almost every imaginable layout in a highly automated
fashion.

Unfortunately, it also means that \TeX\ has an
unusually large number of commands and parameters,
and that programming \TeX\ can be far from easy.
Anyone wanting to program in \TeX, and maybe
even the ordinary user, would seem to need two books:
a~tutorial that gives a first glimpse of the many
nuts and bolts of \TeX, and after that
a~systematic, complete reference manual.
This book tries to fulfil the latter function.
A~\TeX er who has already made a start
(using any of a number of introductory books
on the market)
should be able to use this book indefinitely thereafter.

In this volume the universe of \TeX\ is presented as
about forty different subjects, each in a separate
chapter.
Each chapter starts out with a list of control sequences
relevant to the topic of that chapter
and proceeds to treat the 
theory of the topic. 
Most chapters conclude with remarks and examples.

Globally, the chapters are ordered as follows. 
The chapters on basic mechanisms are first,
the chapters on text treatment and mathematics are next,
and finally there are some
chapters on output and aspects of \TeX's connections to
the outside world.
%
The book also contains a glossary of \TeX\
commands, tables,
and indexes by example, by control sequence, and by subject.
The subject index refers for most concepts to
only one page, where most of the information
on that topic can be found, as well as references
to the locations of related information.

This book does not treat any specific \TeX\ macro package.
Any parts of the plain format that are treated are those
parts that belong to the `core' of plain \TeX: they
are also present in, for instance, \LaTeX.
Therefore, most remarks about the plain format
are true for \LaTeX, as well as most other formats.
Putting it differently,
if the text refers to the plain format, this should be taken
as a contrast to pure \IniTeX, not to \LaTeX.
By way of illustration, occasionally macros from plain \TeX\
are explained that do not belong to the core.

\medskip\noindent
{\bf Acknowledgment}\nl
I am indebted to Barbara Beeton, Karl Berry, and Nico Poppelier,
who read previous versions of this book. Their comments
helped to improve the presentation.
Also I~would like to thank the participants of
the discussion lists \TeX hax, \TeX-nl, and {\tt comp.text.tex}.
Their questions and answers gave me much food for thought.
Finally, any acknowledgement in a book about \TeX\ ought to
include Donald Knuth for inventing \TeX\ in the
first place. This book is no exception.

\begin{flushright}
 Victor Eijkhout\\
 Urbana, Illinois, August 1991\\
 Knoxville, Tennessee, May 2001
\end{flushright}
\pagebreak
\normalheads
