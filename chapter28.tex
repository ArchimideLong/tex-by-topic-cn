% -*- coding: utf-8 -*-

%\chapter{Output Routines}\label{output}
%\index{output routine|(}
\chapter{Output Routines}\label{output}
\index{output routine|(}

%The final stages of page processing are performed by the
%output routine. The page builder cuts off a certain portion
%of the main vertical list and hands it to the output routine
%in \cs{box255}. This chapter treats the commands and parameters
%that pertain to the output routine, and it explains how
%output routines can receive information through marks.
The final stages of page processing are performed by the
output routine. The page builder cuts off a certain portion
of the main vertical list and hands it to the output routine
in \cs{box255}. This chapter treats the commands and parameters
that pertain to the output routine, and it explains how
output routines can receive information through marks.

%\label{cschap:output}\label{cschap:shipout}\label{cschap:mark}\label{cschap:topmark}\label{cschap:botmark}\label{cschap:firstmark}\label{cschap:splitbotmark}\label{cschap:splitfirstmark}\label{cschap:deadcycles}\label{cschap:maxdeadcycles}\label{cschap:outputpenalty2}
%\begin{inventory}
%\item [\cs{output}] 
%      Token list with instructions for shipping out pages.
\label{cschap:output}\label{cschap:shipout}\label{cschap:mark}\label{cschap:topmark}\label{cschap:botmark}\label{cschap:firstmark}\label{cschap:splitbotmark}\label{cschap:splitfirstmark}\label{cschap:deadcycles}\label{cschap:maxdeadcycles}\label{cschap:outputpenalty2}
\begin{inventory}
\item [\cs{output}] 
      Token list with instructions for shipping out pages.

%\item [\cs{shipout}] 
%      Ship a box to the \n{dvi} file.
\item [\cs{shipout}] 
      Ship a box to the \n{dvi} file.


%\item [\cs{mark}] 
%      Specify a mark text.
\item [\cs{mark}] 
      Specify a mark text.

%\item [\cs{topmark}] 
%      The last mark on the previous page.
\item [\cs{topmark}] 
      The last mark on the previous page.

%\item [\cs{botmark}] 
%      The last mark on the current page.
\item [\cs{botmark}] 
      The last mark on the current page.

%\item [\cs{firstmark}] 
%      The first mark on the current page.
\item [\cs{firstmark}] 
      The first mark on the current page.

%\item [\cs{splitbotmark}] 
%      The last mark on a split-off page.
\item [\cs{splitbotmark}] 
      The last mark on a split-off page.

%\item [\cs{splitfirstmark}] 
%      The first mark on a split-off page.
\item [\cs{splitfirstmark}] 
      The first mark on a split-off page.

%\item [\cs{deadcycles}] 
%      Counter that keeps track of how many times 
%      the output routine has been called without a \cs{shipout} 
%      taking place. 
\item [\cs{deadcycles}] 
      Counter that keeps track of how many times 
      the output routine has been called without a \cs{shipout} 
      taking place. 

%\item [\cs{maxdeadcycles}] 
%      The maximum number of times that the output routine is allowed to
%      be called without a \cs{shipout} occurring.
\item [\cs{maxdeadcycles}] 
      The maximum number of times that the output routine is allowed to
      be called without a \cs{shipout} occurring.

%\item [\cs{outputpenalty}] 
%      Value of the penalty at the current page break,
% \alt
%      or $10\,000$ if the break was not at a penalty.
\item [\cs{outputpenalty}] 
      Value of the penalty at the current page break,
 \alt
      or $10\,000$ if the break was not at a penalty.

%\end{inventory}
\end{inventory}


%%\point The \cs{output} token list
%\section{The \protect\cs{output} token list}
%\point The \cs{output} token list
\section{The \protect\cs{output} token list}

%Common parlance has it that
%`the output routine is called' when \TeX\ has found a place
%to break the main vertical list.
%Actually, \cs{output} is not a macro but a token list that
%is inserted into \TeX's command stream.
Common parlance has it that
`the output routine is called' when \TeX\ has found a place
to break the main vertical list.
Actually, \cs{output} is not a macro but a token list that
is inserted into \TeX's command stream.

%Insertion of the \cs{output} token list happens
%\cstoidx output\par
%inside a group that is implicitly opened.
%Also, \TeX\ enters internal vertical mode.
%Because of the group, non-local assignments 
%(to the page number, for instance)
%have to be prefixed with \cs{global}.
%The vertical mode implies that during the workings of the 
%output routine 
%spaces are mostly harmless.
Insertion of the \cs{output} token list happens
\cstoidx output\par
inside a group that is implicitly opened.
Also, \TeX\ enters internal vertical mode.
Because of the group, non-local assignments 
(to the page number, for instance)
have to be prefixed with \cs{global}.
The vertical mode implies that during the workings of the 
output routine 
spaces are mostly harmless.

%The \cs{output} token list belongs
%to the class of the
%\gr{token parameter}s. These behave the same as
%\cs{toks}$nnn$ token lists; see Chapter~\ref{token}.
%Assigning an output routine can therefore take the following
%forms:
%\begin{disp}\cs{output}\gr{equals}\gr{general text}\quad
%or\quad 
%\cs{output}\gr{equals}\gr{filler}\gr{token variable}
%\end{disp}
The \cs{output} token list belongs
to the class of the
\gr{token parameter}s. These behave the same as
\cs{toks}$nnn$ token lists; see Chapter~\ref{token}.
Assigning an output routine can therefore take the following
forms:
\begin{disp}\cs{output}\gr{equals}\gr{general text}\quad
or\quad 
\cs{output}\gr{equals}\gr{filler}\gr{token variable}
\end{disp}


%%\point[output255] Output and \cs{box255}
%\section{Output and \protect\cs{box255}}
%\label{output255}
%\point[output255] Output and \cs{box255}
\section{Output and \protect\cs{box255}}
\label{output255}

%\TeX's page builder breaks the current page at the optimal point,
%and stores everything above that in \cs{box255};
%then, the \cs{output} tokens are inserted into the input stream.
%Any remaining material on the main vertical list
%is pushed back to the recent 
%contributions. 
%If the page is broken at a penalty,
%\alt
%that value is recorded in \cs{outputpenalty}, and 
%a penalty of size $10\,000$ is placed on top of the
%recent contributions; otherwise, \cs{outputpenalty}
%is set to~$10\,000$.
%When the output routine is finished, \cs{box255} is
%supposed to be empty.
%If it is not, \TeX\ gives an error message.
\TeX's page builder breaks the current page at the optimal point,
and stores everything above that in \cs{box255};
then, the \cs{output} tokens are inserted into the input stream.
Any remaining material on the main vertical list
is pushed back to the recent 
contributions. 
If the page is broken at a penalty,
\alt
that value is recorded in \cs{outputpenalty}, and 
a penalty of size $10\,000$ is placed on top of the
recent contributions; otherwise, \cs{outputpenalty}
is set to~$10\,000$.
When the output routine is finished, \cs{box255} is
supposed to be empty.
If it is not, \TeX\ gives an error message.

%Usually, the output routine will take the pagebox,
%\cstoidx shipout\par
%\mdqon
%append a headline and/""or footline,
%\mdqoff
%maybe merge in some insertions such as footnotes,
%and ship the page to the \n{dvi} file:
%\begin{verbatim}
%\output={\setbox255=\vbox
%           {\someheadline 
%            \vbox to \vsize{\unvbox255 \unvbox\footins}
%            \somefootline}
%         \shipout\box255}
%\end{verbatim}
%When box 255 reaches the output routine, its height has
%been set to \cs{vsize}.
%However, the material in it can have considerably
%smaller height.
%Thus, the above output routine may lead to underfull boxes.
%This can be remedied with a \cs{vfil}.
Usually, the output routine will take the pagebox,
\cstoidx shipout\par
\mdqon
append a headline and/""or footline,
\mdqoff
maybe merge in some insertions such as footnotes,
and ship the page to the \n{dvi} file:
\begin{verbatim}
\output={\setbox255=\vbox
           {\someheadline 
            \vbox to \vsize{\unvbox255 \unvbox\footins}
            \somefootline}
         \shipout\box255}
\end{verbatim}
When box 255 reaches the output routine, its height has
been set to \cs{vsize}.
However, the material in it can have considerably
smaller height.
Thus, the above output routine may lead to underfull boxes.
This can be remedied with a \cs{vfil}.

%The output routine is under no obligation to
%\cstoidx deadcycles\par
%do anything useful with \cs{box255}; it can empty it, or
%unbox it to let \TeX\ have another go at finding a page
%break. The number of times
%that the output routing  postpones the \cs{shipout}
%is recorded in \cs{deadcycles}: this parameter is set to~0
%by \cs{shipout}, and increased by~1 just before 
%every \cs{output}.
The output routine is under no obligation to
\cstoidx deadcycles\par
do anything useful with \cs{box255}; it can empty it, or
unbox it to let \TeX\ have another go at finding a page
break. The number of times
that the output routing  postpones the \cs{shipout}
is recorded in \cs{deadcycles}: this parameter is set to~0
by \cs{shipout}, and increased by~1 just before 
every \cs{output}.

%When  the number of dead cycles reaches
%\csidx{maxdeadcycles}, \TeX\ gives an error message,
%and performs the default output routine
%\begin{verbatim}
%\shipout\box255
%\end{verbatim}
%instead of the routine it was about
%to start.
%The \LaTeX\ format has a much higher value for \cs{maxdeadcycles}
%than plain \TeX, because the output routine in \LaTeX\
%is often called for
%intermediate handling of floats and marginal notes.
When  the number of dead cycles reaches
\csidx{maxdeadcycles}, \TeX\ gives an error message,
and performs the default output routine
\begin{verbatim}
\shipout\box255
\end{verbatim}
instead of the routine it was about
to start.
The \LaTeX\ format has a much higher value for \cs{maxdeadcycles}
than plain \TeX, because the output routine in \LaTeX\
is often called for
intermediate handling of floats and marginal notes.

%The \cs{shipout} command can send any \gr{box} to the \n{dvi} file;
%this need not be box 255, or even a box
%containing the current page.
%It does not have to be called inside the output routine, either.
The \cs{shipout} command can send any \gr{box} to the \n{dvi} file;
this need not be box 255, or even a box
containing the current page.
It does not have to be called inside the output routine, either.

%If the output routine produces any material, for instance
%by calling
%\begin{verbatim}
%\unvbox255
%\end{verbatim}
%this is put on top
%of the recent contributions.
If the output routine produces any material, for instance
by calling
\begin{verbatim}
\unvbox255
\end{verbatim}
this is put on top
of the recent contributions.

%After the output routine finishes, the page builder is
%activated. In particular, because the current page
%has been emptied, the \cs{vsize} is read again.
%Changes made to this parameter inside the output
%routine (using \cs{global}) will therefore take effect.
After the output routine finishes, the page builder is
activated. In particular, because the current page
has been emptied, the \cs{vsize} is read again.
Changes made to this parameter inside the output
routine (using \cs{global}) will therefore take effect.

%%\point Marks
%\section{Marks}
%\point Marks
\section{Marks}

%Information can be passed to the output routine through the
%\cstoidx mark\par
%mechanism of \indexterm{marks}. The user can specify a token list
%with 
%\begin{disp}\cs{mark}\lb\gr{mark text}\rb\end{disp}
%which is put in a mark item on the current vertical list.
%The mark text is subject to expansion as in \cs{edef}.
Information can be passed to the output routine through the
\cstoidx mark\par
mechanism of \indexterm{marks}. The user can specify a token list
with 
\begin{disp}\cs{mark}\lb\gr{mark text}\rb\end{disp}
which is put in a mark item on the current vertical list.
The mark text is subject to expansion as in \cs{edef}.

%If the mark is given in horizontal mode it migrates to
%the surrounding vertical lists like an insertion item
%(see page~\pageref{migrate});
%however, if this is not the external vertical list, the
%output routine will not find the mark.
If the mark is given in horizontal mode it migrates to
the surrounding vertical lists like an insertion item
(see page~\pageref{migrate});
however, if this is not the external vertical list, the
output routine will not find the mark.

%Marks are the main mechanism through which the output routine
%can obtain information about the contents of the currently
%broken-off page, in particular its top and bottom.
%\TeX\ sets three variables:
%\begin{description} 
%\item [\csidx{botmark}]
%   the last mark occurring on the current page;
%\item [\csidx{firstmark}]
%   the first mark occurring on the current page;
%\item [\csidx{topmark}]
%   the last mark of the previous page,
%   that is, the value of \cs{botmark}
%   on the previous page.
%\end{description}
%If no marks have occurred yet, all three are empty;
%if no marks occurred on the current page, 
%all three mark variables are equal
%to the \cs{botmark} of the previous page.
Marks are the main mechanism through which the output routine
can obtain information about the contents of the currently
broken-off page, in particular its top and bottom.
\TeX\ sets three variables:
\begin{description} 
\item [\csidx{botmark}]
   the last mark occurring on the current page;
\item [\csidx{firstmark}]
   the first mark occurring on the current page;
\item [\csidx{topmark}]
   the last mark of the previous page,
   that is, the value of \cs{botmark}
   on the previous page.
\end{description}
If no marks have occurred yet, all three are empty;
if no marks occurred on the current page, 
all three mark variables are equal
to the \cs{botmark} of the previous page.

%For boxes generated by a \cs{vsplit} command (see previous chapter),
%the \cs{splitbotmark} and \cs{splitfirstmark}
%\cstoidx splitbotmark\par\cstoidx splitfirstmark\par
%contain the marks of the split-off part; \cs{firstmark}
%and \cs{botmark} reflect the state of what remains in the register.
For boxes generated by a \cs{vsplit} command (see previous chapter),
the \cs{splitbotmark} and \cs{splitfirstmark}
\cstoidx splitbotmark\par\cstoidx splitfirstmark\par
contain the marks of the split-off part; \cs{firstmark}
and \cs{botmark} reflect the state of what remains in the register.

%\begin{example} Marks can be used to get a section heading into
%\howto Do tricks with headlines\par
%the headline or footline of the page.
%\begin{verbatim}
%\def\section#1{ ... \mark{#1} ... }
%\def\rightheadline{\hbox to \hsize
%   {\headlinefont \botmark\hfil\pagenumber}}
%\def\leftheadline{\hbox to \hsize
%   {\headlinefont \pagenumber\hfil\firstmark}}
%\end{verbatim}
%This places the title of the first section that starts on a 
%left page in the left headline, and the title of the last section
%that starts on the right page in the right headline.
%Placing the headlines on the page is the job of the output routine;
%see below.
\begin{example} Marks can be used to get a section heading into
\howto Do tricks with headlines\par
the headline or footline of the page.
\begin{verbatim}
\def\section#1{ ... \mark{#1} ... }
\def\rightheadline{\hbox to \hsize
   {\headlinefont \botmark\hfil\pagenumber}}
\def\leftheadline{\hbox to \hsize
   {\headlinefont \pagenumber\hfil\firstmark}}
\end{verbatim}
This places the title of the first section that starts on a 
left page in the left headline, and the title of the last section
that starts on the right page in the right headline.
Placing the headlines on the page is the job of the output routine;
see below.

%It is important that no page breaks can occur in between the
%mark and the box that places the title:
%\begin{verbatim}
%\def\section#1{ ...
%    \penalty\beforesectionpenalty
%    \mark{#1}
%    \hbox{ ... #1 ...}
%    \nobreak
%    \vskip\aftersectionskip
%    \noindent}
%\end{verbatim}
%\end{example}
It is important that no page breaks can occur in between the
mark and the box that places the title:
\begin{verbatim}
\def\section#1{ ...
    \penalty\beforesectionpenalty
    \mark{#1}
    \hbox{ ... #1 ...}
    \nobreak
    \vskip\aftersectionskip
    \noindent}
\end{verbatim}
\end{example}

%Let us consider
%another example with headlines: often a page looks better if
%the headline is omitted on pages where a chapter starts.
%This can be implemented as follows:
%\begin{verbatim}
%\def\endofchapter
%\chapter#1{ ... \def\chtitle{#1}\mark{1}\mark{0} ... }
%\def\theheadline{\expandafter\ifx\firstmark1 
%    \else \chapheadline \fi}
%\end{verbatim}
%Only on the page where a chapter starts will the mark be~1,
%and on all other pages a headline is placed.
Let us consider
another example with headlines: often a page looks better if
the headline is omitted on pages where a chapter starts.
This can be implemented as follows:
\begin{verbatim}
\def\endofchapter
\chapter#1{ ... \def\chtitle{#1}\mark{1}\mark{0} ... }
\def\theheadline{\expandafter\ifx\firstmark1 
    \else \chapheadline \fi}
\end{verbatim}
Only on the page where a chapter starts will the mark be~1,
and on all other pages a headline is placed.

%%\point Assorted remarks
%\section{Assorted remarks}
%\point Assorted remarks
\section{Assorted remarks}

%%\spoint Hazards in non-trivial output routines
%\subsection{Hazards in non-trivial output routines}
%\spoint Hazards in non-trivial output routines
\subsection{Hazards in non-trivial output routines}

%If the final call to the output routine does not
%perform a \cs{shipout}, \TeX\ will call the output
%routine endlessly, since a run will only stop if both
%the vertical list is empty, and \cs{deadcycles}
%is zero. The output routine can set \cs{deadcycles}
%to zero to prevent this.
If the final call to the output routine does not
perform a \cs{shipout}, \TeX\ will call the output
routine endlessly, since a run will only stop if both
the vertical list is empty, and \cs{deadcycles}
is zero. The output routine can set \cs{deadcycles}
to zero to prevent this.

%%\spoint Page numbering
%\subsection{Page numbering}
%\index{page!numbering|(}
%\spoint Page numbering
\subsection{Page numbering}
\index{page!numbering|(}

%The page number is not an intrinsic property of the output
%routine; in plain \TeX\ it is  the value of \cs{count0}.
%The output routine is responsible for increasing the
%page number when a shipout of a  page occurs.
The page number is not an intrinsic property of the output
routine; in plain \TeX\ it is  the value of \cs{count0}.
The output routine is responsible for increasing the
page number when a shipout of a  page occurs.

%Apart from   \cs{count0}, counter registers~1--9 are also used
%for page identification: at shipout \TeX\ writes the values
%of these ten counters to the \n{dvi} file (see Chapter~\ref{TeXcomm}).
%Terminal and log file output display only the non-zero counters,
%and the zero counters for which a non-zero counter with
%a higher number exists, that is, if \cs{count0}${}=1$ and
%\cs{count3}${}=5$ are the only non-zero counters, the
%displayed list of counters is~\n{[1.0.0.5]}.
Apart from   \cs{count0}, counter registers~1--9 are also used
for page identification: at shipout \TeX\ writes the values
of these ten counters to the \n{dvi} file (see Chapter~\ref{TeXcomm}).
Terminal and log file output display only the non-zero counters,
and the zero counters for which a non-zero counter with
a higher number exists, that is, if \cs{count0}${}=1$ and
\cs{count3}${}=5$ are the only non-zero counters, the
displayed list of counters is~\n{[1.0.0.5]}.

%\index{page!numbering|)}
\index{page!numbering|)}

%\subsection{Headlines and footlines in plain \TeX}
\subsection{Headlines and footlines in plain \TeX}

%Plain \TeX\ has token lists \cs{headline} and
%\cs{footline}; these are used in the macros
%\handbreak \cs{makeheadline} and \cs{makefootline}.
%The page is shipped out as (more or less)
%\begin{verbatim}
%\vbox{\makeheadline\pagebody\makefootline}
%\end{verbatim}
Plain \TeX\ has token lists \cs{headline} and
\cs{footline}; these are used in the macros
\handbreak \cs{makeheadline} and \cs{makefootline}.
The page is shipped out as (more or less)
\begin{verbatim}
\vbox{\makeheadline\pagebody\makefootline}
\end{verbatim}

%Both headline and footline are inserted inside a \cs{line}.
%For non-standard headers and footers it is easier to
%redefine the macros \cs{makeheadline} and \cs{makefootline}
%than to tinker with the token lists.
Both headline and footline are inserted inside a \cs{line}.
For non-standard headers and footers it is easier to
redefine the macros \cs{makeheadline} and \cs{makefootline}
than to tinker with the token lists.

%%\spoint Example: no widow lines
%\subsection{Example: no widow lines}
%\spoint Example: no widow lines
\subsection{Example: no widow lines}

%Suppose that one does not want to allow widow lines,
%but pages have in general no stretch or shrink,
%for instance because they only contain plain text.
%A~solution would be to increase the page length
%by one line if a page turns out to be broken
%at a widow line.
Suppose that one does not want to allow widow lines,
but pages have in general no stretch or shrink,
for instance because they only contain plain text.
A~solution would be to increase the page length
by one line if a page turns out to be broken
at a widow line.

%\TeX's output routine can perform this sort of
%trick: if the \cs{widowpenalty} is set to
%some recognizable value, the output routine
%can see by the \cs{outputpenalty} if a widow
%line occurred. In that case, the output routine
%can temporarily increase the \cs{vsize}, and
%let the page builder have another go at
%finding a break point.
\TeX's output routine can perform this sort of
trick: if the \cs{widowpenalty} is set to
some recognizable value, the output routine
can see by the \cs{outputpenalty} if a widow
line occurred. In that case, the output routine
can temporarily increase the \cs{vsize}, and
let the page builder have another go at
finding a break point.

%Here is the skeleton of such an output routine.
%No headers or footers are provided for.
%\begin{verbatim}
%\newif\ifLargePage \widowpenalty=147
%\newdimen\oldvsize \oldvsize=\vsize
%\output={
%    \ifLargePage \shipout\box255 
%          \global\LargePagefalse
%          \global\vsize=\oldvsize
%    \else \ifnum \outputpenalty=\widowpenalty
%             \global\LargePagetrue
%             \global\advance\vsize\baselineskip
%             \unvbox255 \penalty\outputpenalty
%          \else  \shipout\box255
%    \fi   \fi}
%\end{verbatim}
%The test \cs{ifLargePage} is set to true by the
%output routine if the \cs{outputpenalty}
%equals the \cs{widowpenalty}. The page box
%is then \cs{unvbox}$\,$ed, so that the page builder
%will tackle the same material once more.
Here is the skeleton of such an output routine.
No headers or footers are provided for.
\begin{verbatim}
\newif\ifLargePage \widowpenalty=147
\newdimen\oldvsize \oldvsize=\vsize
\output={
    \ifLargePage \shipout\box255 
          \global\LargePagefalse
          \global\vsize=\oldvsize
    \else \ifnum \outputpenalty=\widowpenalty
             \global\LargePagetrue
             \global\advance\vsize\baselineskip
             \unvbox255 \penalty\outputpenalty
          \else  \shipout\box255
    \fi   \fi}
\end{verbatim}
The test \cs{ifLargePage} is set to true by the
output routine if the \cs{outputpenalty}
equals the \cs{widowpenalty}. The page box
is then \cs{unvbox}$\,$ed, so that the page builder
will tackle the same material once more.

%%\spoint Example: no indentation top of page
%\subsection{Example: no indentation top of page}
%\spoint Example: no indentation top of page
\subsection{Example: no indentation top of page}

%Some  output routines can be classified
%\howto Prevent indentation on top of page\par
%as abuse of the output routine mechanism.
%The output routine in this section is a good example of this.
Some  output routines can be classified
\howto Prevent indentation on top of page\par
as abuse of the output routine mechanism.
The output routine in this section is a good example of this.

%It is imaginable that one wishes paragraphs not to indent
%if they start at the top of a page. (There are plenty of objections
%to this layout, but occasionally it is used.)
%This problem can be solved using the output routine to
%investigate whether the page is still empty and, if so,
%to give a signal that a paragraph should not indent.
It is imaginable that one wishes paragraphs not to indent
if they start at the top of a page. (There are plenty of objections
to this layout, but occasionally it is used.)
This problem can be solved using the output routine to
investigate whether the page is still empty and, if so,
to give a signal that a paragraph should not indent.

%Note that we cannot use the fact here
%that the page builder comes into play after
%the insertion of \cs{everypar}: even if we could
%force the output routine to be activated here,
%there is no way for it to remove the indentation box.
Note that we cannot use the fact here
that the page builder comes into play after
the insertion of \cs{everypar}: even if we could
force the output routine to be activated here,
there is no way for it to remove the indentation box.

%The solution given here lets the \cs{everypar}
%terminate the paragraph immediately
%with
%\begin{verbatim}
%\par\penalty-\specialpenalty
%\end{verbatim}
%which activates the output routine.
%Seeing whether the pagebox is empty (after removing
%the empty line and any \cs{parskip} glue),
%the output routine then can set a switch
%signalling whether the retry of the paragraph
%should indent.
The solution given here lets the \cs{everypar}
terminate the paragraph immediately
with
\begin{verbatim}
\par\penalty-\specialpenalty
\end{verbatim}
which activates the output routine.
Seeing whether the pagebox is empty (after removing
the empty line and any \cs{parskip} glue),
the output routine then can set a switch
signalling whether the retry of the paragraph
should indent.

%There are some minor matters in the following
%routines, the sense of which is left
%for the reader to ponder.
%\begin{verbatim}
%\mathchardef\specialpenalty=10001
%\newif\ifPreventSwitch 
%\newbox\testbox 
%\topskip=10pt

%\everypar{\begingroup \par
%    \penalty-\specialpenalty
%    \everypar{\endgroup}\parskip0pt 
%    \ifPreventSwitch \noindent \else \indent \fi
%    \global\PreventSwitchfalse
%    }
%\output{
%    \ifnum\outputpenalty=-\specialpenalty 
%        \setbox\testbox\vbox{\unvbox255 
%                  {\setbox0=\lastbox}\unskip}
%        \ifdim\ht\testbox=0pt \global\PreventSwitchtrue
%        \else \topskip=0pt \unvbox\testbox \fi
%    \else \shipout\box255 \global\advance\pageno1 \fi}
%\end{verbatim}
There are some minor matters in the following
routines, the sense of which is left
for the reader to ponder.
\begin{verbatim}
\mathchardef\specialpenalty=10001
\newif\ifPreventSwitch 
\newbox\testbox 
\topskip=10pt

\everypar{\begingroup \par
    \penalty-\specialpenalty
    \everypar{\endgroup}\parskip0pt 
    \ifPreventSwitch \noindent \else \indent \fi
    \global\PreventSwitchfalse
    }
\output{
    \ifnum\outputpenalty=-\specialpenalty 
        \setbox\testbox\vbox{\unvbox255 
                  {\setbox0=\lastbox}\unskip}
        \ifdim\ht\testbox=0pt \global\PreventSwitchtrue
        \else \topskip=0pt \unvbox\testbox \fi
    \else \shipout\box255 \global\advance\pageno1 \fi}
\end{verbatim}

%%\spoint More examples of output routines
%\subsection{More examples of output routines}
%\spoint More examples of output routines
\subsection{More examples of output routines}

%A large number of examples of output routines
%can be found in~\cite{Sal1} and~\cite{Sal2}.
A large number of examples of output routines
can be found in~\cite{Sal1} and~\cite{Sal2}.

%\index{output routine|)}
\index{output routine|)}

%\endofchapter
\endofchapter
