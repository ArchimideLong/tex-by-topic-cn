% -*- coding: utf-8 -*-
\documentclass[letterpaper]{book}

% -*- coding: utf-8 -*-

\usepackage[b5paper,text={5in,8in},centering]{geometry}

\usepackage[CJKchecksingle]{xeCJK}
\setmainfont[Mapping=tex-text]{TeX Gyre Schola}
%\setsansfont{URW Gothic L Book}
%\setmonofont{Nimbus Mono L}
\setCJKmainfont[BoldFont=FandolHei,ItalicFont=FandolKai]{FandolSong}
\setCJKsansfont{FandolHei}
\setCJKmonofont{FandolFang}
\xeCJKsetup{PunctStyle = kaiming}

\linespread{1.25}
\setlength{\parindent}{2em}

\usepackage{xcolor}
\definecolor{myblue}{rgb}{0,0.2,0.6}

\usepackage{titlesec}
\titleformat{\chapter}
    {\normalfont\Huge\sffamily\color{myblue}}
    {第\thechapter 章}
    {1em}
    {}
%\titlespacing{\chapter}{0pt}{50pt}{40pt}
\titleformat{\section}
    {\normalfont\Large\sffamily\color{myblue}}
    {\thesection}
    {1em}
    {}
%\titlespacing{\section}{0pt}{3.5ex plus 1ex minus .2ex}{2.3ex plus .2ex}
\titleformat{\subsection}
    {\normalfont\large\sffamily\color{myblue}}
    {\thesubsection}
    {1em}
    {}
%\titlespacing{\subsection}{0pt}{3.25ex plus 1ex minus .2ex}{1.5ex plus .2ex}
%
\newpagestyle{special}[\small\sffamily]{
  \headrule
  \sethead[\usepage][][\chaptertitle]
  {\chaptertitle}{}{\usepage}}
\newpagestyle{main}[\small\sffamily]{
  \headrule
  \sethead[\usepage][][第\thechapter 章\quad\chaptertitle]
  {\thesection\quad\sectiontitle}{}{\usepage}}

\usepackage{titletoc}
%\setcounter{tocdepth}{1}
%\titlecontents{标题层次}[左间距]{上间距和整体格式}{标题序号}{标题内容}{指引线和页码}[下间距]
\titlecontents{chapter}[1.5em]{\vspace{.5em}\bfseries\sffamily}{\color{myblue}\contentslabel{1.5em}}{}
    {\titlerule*[20pt]{$\cdot$}\contentspage}[]
\titlecontents{section}[4.5em]{\sffamily}{\color{myblue}\contentslabel{3em}}{}
    {\titlerule*[20pt]{$\cdot$}\contentspage}[]
%\titlecontents{subsection}[8.5em]{\sffamily}{\contentslabel{4em}}{}
%    {\titlerule*[20pt]{$\cdot$}\contentspage}

\usepackage{enumitem}
\setlist{topsep=2pt,itemsep=2pt,parsep=1pt,leftmargin=\parindent}

\usepackage{fancyvrb}
\DefineVerbatimEnvironment{verbatim}{Verbatim}
  {xleftmargin=2em,baselinestretch=1,formatcom=\color{teal}\upshape}

\usepackage{etoolbox}
\makeatletter
\preto{\FV@ListVSpace}{\topsep=2pt \partopsep=0pt }
\makeatother

\usepackage[colorlinks,plainpages,pagebackref]{hyperref}
\hypersetup{
   pdfstartview={FitH},
   citecolor=teal,
   linkcolor=myblue,
   urlcolor=black,
   bookmarksnumbered
}

\usepackage{comment,makeidx,multicol}

%\usepackage{german}
%% german
%\righthyphenmin=3
%\mdqoff
%\captionsenglish
\usepackage[english]{babel}
{\catcode`"=13 \gdef"#1{\ifx#1"\discretionary{}{}{}\fi\relax}}
\def\mdqon{\catcode`"=13\relax}
\def\mdqoff{\catcode`"=12\relax}
\makeindex
\hyphenation{ex-em-pli-fies}

\newdimen\tempdima \newdimen\tempdimb

% these are fine
\def\handbreak{\\ \message{^^JManual break!!!!^^J}}
\def\nl{\protect\\}\def\n#1{{\tt #1}}
\def\cs#1{\texorpdfstring{{\tt\char`\\#1}}{\textbackslash#1}} %\def\cs#1{{\tt\char`\\#1}}
\let\csc\cs
\def\lb{{\tt\char`\{}}\def\rb{{\tt\char`\}}}
\def\gr#1{\texorpdfstring{$\langle$#1$\rangle$}{<#1>}} %\def\gr#1{$\langle$#1$\rangle$}
\def\key#1{{\tt#1}}
\def\alt{}\def\altt{}%this way in manstijl
\def\ldash{\unskip\ --\nobreak\ \ignorespaces}
\def\rdash{\unskip\nobreak\ --\ \ignorespaces}
% check these
\def\hex{{\tt"}}
\def\ascii{{\sc ascii}}
\def\ebcdic{{\sc ebcdic}}
\def\IniTeX{Ini\TeX}\def\LamsTeX{LAMS\TeX}\def\VirTeX{Vir\TeX}
\def\AmsTeX{Ams\TeX}
\def\TeXbook{the \TeX\ book}\def\web{{\sc web}}
% needs major thinking
\newenvironment{myquote}{\list{}{%
    \topsep=2pt \partopsep=0pt%
    \leftmargin=\parindent \rightmargin=\parindent
    }\item[]}{\endlist}
\newenvironment{disp}{\begin{myquote}}{\end{myquote}}
\newenvironment{Disp}{\begin{myquote}}{\end{myquote}}
\newenvironment{tdisp}{\begin{myquote}}{\end{myquote}}
\newenvironment{example}{\begin{myquote}\noindent\itshape 例子:}{\end{myquote}}
\newenvironment{inventory}{\begin{description}\raggedright}{\end{description}}
\newenvironment{glossinventory}{\begin{description}}{\end{description}}
\def\gram#1{\gr{#1}}%???
%
% index
%
\def\indexterm#1{\emph{#1}\index{#1}}
\def\indextermsub#1#2{\emph{#1 #2}\index{#1!#2}}
\def\indextermbus#1#2{\emph{#1 #2}\index{#2!#1}}
\def\term#1\par{\index{#1}}
\def\howto#1\par{}
\def\cstoidx#1\par{\index{#1@\cs{#1}@}}
\def\thecstoidx#1\par{\index{#1@\protect\csname #1\endcsname}}
\def\thecstoidxsub#1#2{\index{#1, #2@\protect\csname #1\endcsname, #2}\ignorespaces}
\def\csterm#1\par{\cstoidx #1\par\cs{#1}}
\def\csidx#1{\cstoidx #1\par\cs{#1}}

\def\tmc{\tracingmacros=2 \tracingcommands\tracingmacros}

%%%%%%%%%%%%%%%%%%%
\makeatletter
\def\snugbox{\hbox\bgroup\setbox\z@\vbox\bgroup
    \leftskip\z@
    \bgroup\aftergroup\make@snug
    \let\next=}
\def\make@snug{\par\sn@gify\egroup \box\z@\egroup}
\def\sn@gify
   {\skip\z@=\lastskip \unskip
    \advance\skip\z@\lastskip \unskip
    \unpenalty
    \setbox\z@\lastbox
    \ifvoid\z@ \nointerlineskip \else {\sn@gify} \fi
    \hbox{\unhbox\z@}\nointerlineskip
    \vskip\skip\z@
    }

\newdimen\fbh \fbh=60pt % dimension for easy scaling:
\newdimen\fbw \fbw=60pt % height and width of character box

\newdimen\dh \newdimen\dw % height and width of current character box
\newdimen\lh % height of previous character box
\newdimen\lw \lw=.4pt % line weight, instead of default .4pt

\def\hdotfill{\noindent
    \leaders\hbox{\vrule width 1pt height\lw
                  \kern4pt
                  \vrule width.5pt height\lw}\hfill\hbox{}
    \par}
\def\hlinefill{\noindent
    \leaders\hbox{\vrule width 5.5pt height\lw         }\hfill\hbox{}
    \par}
\def\stippel{$\qquad\qquad\qquad\qquad$}
\makeatother
%%%%%%%%%%%%%%%%%%%

%\def\SansSerif{\Typeface:macHelvetica }
%\def\SerifFont{\Typeface:macTimes }
%\def\SansSerif{\Typeface:bsGillSans }
%\def\SerifFont{\Typeface:bsBaskerville }
\let\SansSerif\relax \def\italic{\it}
\let\SerifFont\relax \def\MainFont{\rm}
\let\SansSerif\relax
\let\SerifFont\relax
\let\PopIndentLevel\relax \let\PushIndentLevel\relax
\let\ToVerso\relax \let\ToRecto\relax

%\def\stop@command@suffix{stop}
%\let\PopListLevel\PopIndentLevel
%\let\FlushRight\relax
%\let\flushright\FlushRight
%\let\SetListIndent\LevelIndent
%\def\awp{\ifhmode\vadjust{\penalty-10000 }\else
%    \penalty-10000 \fi}
\let\awp\relax
\let\PopIndentLevel\relax \let\PopListLevel\relax

\showboxdepth=-1

%\input figs
\def\endofchapter{\vfill\noindent}

\setcounter{chapter}{15}

\begin{document}

%\chapter{Paragraph Start}\label{par:start}
%\index{paragraph!start|(}
\chapter{Paragraph Start}\label{par:start}
\index{paragraph!start|(}

%At the start of a paragraph \TeX\ inserts a vertical skip
%as a separation from the preceding paragraph, and a horizontal
%skip as an indentation for the current paragraph.
%This chapter explains the exact sequence
%of actions,
%and it discusses how \TeX's decisions can be altered.
At the start of a paragraph \TeX\ inserts a vertical skip
as a separation from the preceding paragraph, and a horizontal
skip as an indentation for the current paragraph.
This chapter explains the exact sequence
of actions,
and it discusses how \TeX's decisions can be altered.

%\label{cschap:indent}\label{cschap:noindent}\label{cschap:parskip}\label{cschap:parindent}\label{cschap:everypar}\label{cschap:leavevmode}
%\begin{inventory}
%\item [\cs{indent}] 
%      Switch to horizontal mode and insert a box of width \cs{parindent}.
\label{cschap:indent}\label{cschap:noindent}\label{cschap:parskip}\label{cschap:parindent}\label{cschap:everypar}\label{cschap:leavevmode}
\begin{inventory}
\item [\cs{indent}] 
      Switch to horizontal mode and insert a box of width \cs{parindent}.

%\item [\cs{noindent}]  
%      Switch to horizontal mode with an empty horizontal list.
\item [\cs{noindent}]  
      Switch to horizontal mode with an empty horizontal list.

%\item [\cs{parskip}] 
%      Amount of glue added to 
%      the surrounding vertical list when a paragraph starts.
%      Plain \TeX\ default:~\n{0pt plus 1pt}.
\item [\cs{parskip}] 
      Amount of glue added to 
      the surrounding vertical list when a paragraph starts.
      Plain \TeX\ default:~\n{0pt plus 1pt}.

%\item [\cs{parindent}]  
%      Size of the indentation box added in front of a paragraph.
%      Plain \TeX\ default:~\n{20pt}.
\item [\cs{parindent}]  
      Size of the indentation box added in front of a paragraph.
      Plain \TeX\ default:~\n{20pt}.

%\item [\cs{everypar}] 
%      Token list inserted in front of paragraph text; 
\item [\cs{everypar}] 
      Token list inserted in front of paragraph text; 

%\item [\cs{leavevmode}] 
%      Macro to switch to horizontal mode if necessary.
\item [\cs{leavevmode}] 
      Macro to switch to horizontal mode if necessary.

%\end{inventory}
\end{inventory}


%%\point When does a paragraph start
%\section{When does a paragraph start}
%\point When does a paragraph start
\section{When does a paragraph start}

%\TeX\ starts a paragraph whenever it switches from 
%vertical mode to (unrestricted) horizontal mode. This switch can
%be effected by one of the commands
%\cs{indent} and 
%\cs{noindent}, for example
%\begin{verbatim}
%{\bf And now~\dots}
%\vskip3pt
%\noindent It's~\dots
%\end{verbatim}
%or by any \gram{horizontal command}.
%Horizontal commands include characters, in-line formulas,
%and horizontal skips, but not boxes.
%Consider the following examples.
%\alt
%The character `I' is a horizontal command:
%\begin{verbatim}
%\vskip3pt
%It's~\dots
%\end{verbatim}
%A single \n\$ is a horizontal command:
%\begin{verbatim}
%$x$ is supposed~\dots
%\end{verbatim} 
%The control sequence \cs{hskip} is a horizontal command:
%\begin{verbatim}
%\hskip .5\hsize Long indentation~\dots
%\end{verbatim}
%The full list of horizontal commands is given on
%page~\pageref{h:com:list}.
\TeX\ starts a paragraph whenever it switches from 
vertical mode to (unrestricted) horizontal mode. This switch can
be effected by one of the commands
\cs{indent} and 
\cs{noindent}, for example
\begin{verbatim}
{\bf And now~\dots}
\vskip3pt
\noindent It's~\dots
\end{verbatim}
or by any \gram{horizontal command}.
Horizontal commands include characters, in-line formulas,
and horizontal skips, but not boxes.
Consider the following examples.
\alt
The character `I' is a horizontal command:
\begin{verbatim}
\vskip3pt
It's~\dots
\end{verbatim}
A single \n\$ is a horizontal command:
\begin{verbatim}
$x$ is supposed~\dots
\end{verbatim} 
The control sequence \cs{hskip} is a horizontal command:
\begin{verbatim}
\hskip .5\hsize Long indentation~\dots
\end{verbatim}
The full list of horizontal commands is given on
page~\pageref{h:com:list}.

%Upon recognizing a horizontal command in vertical mode,
%\TeX\ will perform an \cs{indent} command (and all the actions
%associated with it; see below), 
%and after that it will reexamine the horizontal command,
%this time executing it.
Upon recognizing a horizontal command in vertical mode,
\TeX\ will perform an \cs{indent} command (and all the actions
associated with it; see below), 
and after that it will reexamine the horizontal command,
this time executing it.



%%\point What happens when a paragraph starts
%\section{What happens when a paragraph starts}
%\point What happens when a paragraph starts
\section{What happens when a paragraph starts}

%The \csidx{indent} and \csidx{noindent}  commands 
%cause a paragraph to be started.
%An~\cs{indent} command can either be placed explicitly by
%the user or a macro, or it can be inserted by \TeX\ when
%a \gr{horizontal command} occurs in vertical mode;
%a~\cs{noindent} command can only be placed explicitly.
The \csidx{indent} and \csidx{noindent}  commands 
cause a paragraph to be started.
An~\cs{indent} command can either be placed explicitly by
the user or a macro, or it can be inserted by \TeX\ when
a \gr{horizontal command} occurs in vertical mode;
a~\cs{noindent} command can only be placed explicitly.

%After  either command is encountered,
%\csidx{parskip} glue is appended to the surrounding vertical
%list
%unless \TeX\ is in internal vertical mode 
%and that list is empty
%(for example, at the start of a \cs{vbox} or \cs{vtop}).
%\TeX\ then switches to unrestricted horizontal mode
%with an empty horizontal list.
%In the case of \cs{indent} (which may be inserted
%implicitly) an empty \cs{hbox} of width
%\cstoidx parindent\par
%\cs{parindent} is placed at the start of the horizontal list; 
%after \cs{noindent} no indentation
%box is inserted. 
After  either command is encountered,
\csidx{parskip} glue is appended to the surrounding vertical
list
unless \TeX\ is in internal vertical mode 
and that list is empty
(for example, at the start of a \cs{vbox} or \cs{vtop}).
\TeX\ then switches to unrestricted horizontal mode
with an empty horizontal list.
In the case of \cs{indent} (which may be inserted
implicitly) an empty \cs{hbox} of width
\cstoidx parindent\par
\cs{parindent} is placed at the start of the horizontal list; 
after \cs{noindent} no indentation
box is inserted. 

%The contents of the \csidx{everypar} \gr{token parameter}
%are then inserted into the input (see some applications below).
%After that,
%the page builder is exercised (see Chapter~\ref{page:break}).
%Note that this happens in horizontal mode: this is to 
%move the \cs{parskip} glue to the current page.
The contents of the \csidx{everypar} \gr{token parameter}
are then inserted into the input (see some applications below).
After that,
the page builder is exercised (see Chapter~\ref{page:break}).
Note that this happens in horizontal mode: this is to 
move the \cs{parskip} glue to the current page.

%If an \cs{indent} command is given while \TeX\ is already in
%horizontal mode, the indentation box is inserted just the same.
%This is not very useful.
If an \cs{indent} command is given while \TeX\ is already in
horizontal mode, the indentation box is inserted just the same.
This is not very useful.

%%\point Assorted remarks
%\section{Assorted remarks}
%\point Assorted remarks
\section{Assorted remarks}

%%\spoint Starting a paragraph with a box
%\subsection{Starting a paragraph with a box}
%\spoint Starting a paragraph with a box
\subsection{Starting a paragraph with a box}

%An \cs{hbox} does not imply horizontal mode, so 
%an attempt to start a paragraph with a box, for instance
%\begin{verbatim}
%\hbox to 0cm{\hss$\bullet$\hskip1em}Text ....
%\end{verbatim}
%will make the text following the box
%wind up one line below the box.
%It is necessary to switch to horizontal mode
%explicitly, using for instance \cs{noindent} or
%\cs{leavevmode}. 
%The latter is defined using \cs{unhbox},
%which is a horizontal command.
An \cs{hbox} does not imply horizontal mode, so 
an attempt to start a paragraph with a box, for instance
\begin{verbatim}
\hbox to 0cm{\hss$\bullet$\hskip1em}Text ....
\end{verbatim}
will make the text following the box
wind up one line below the box.
It is necessary to switch to horizontal mode
explicitly, using for instance \cs{noindent} or
\cs{leavevmode}. 
The latter is defined using \cs{unhbox},
which is a horizontal command.

%%\spoint Starting a paragraph with a group
%\subsection{Starting a paragraph with a group}
%\spoint Starting a paragraph with a group
\subsection{Starting a paragraph with a group}

%If the first \gram{horizontal command} of a paragraph
%is enclosed in braces, the \cs{everypar} is evaluated
%inside the group. This may give unexpected results.
%Consider this example:
%\begin{verbatim}
%\everypar={\setbox0=\vbox\bgroup\def\par{\egroup}}
%{\bf Start} a paragraph ... \par
%\end{verbatim}
%The \gr{horizontal command} starting the paragraph is the
%character~`S', so when \cs{everypar} has been inserted
%the input is essentially
%\begin{verbatim}
%{\bf \indent\setbox0=\vbox\bgroup
%    \def\par{\egroup}Start} a paragraph ... \par
%\end{verbatim}
%which is equivalent to
%\begin{verbatim}
%{\bf \setbox0=\vbox{Start} a paragraph ... \par
%\end{verbatim}
%The effect of this is rather different from what was intended.
%\alt
%Also, \TeX\ will probably end the job inside a group.
If the first \gram{horizontal command} of a paragraph
is enclosed in braces, the \cs{everypar} is evaluated
inside the group. This may give unexpected results.
Consider this example:
\begin{verbatim}
\everypar={\setbox0=\vbox\bgroup\def\par{\egroup}}
{\bf Start} a paragraph ... \par
\end{verbatim}
The \gr{horizontal command} starting the paragraph is the
character~`S', so when \cs{everypar} has been inserted
the input is essentially
\begin{verbatim}
{\bf \indent\setbox0=\vbox\bgroup
    \def\par{\egroup}Start} a paragraph ... \par
\end{verbatim}
which is equivalent to
\begin{verbatim}
{\bf \setbox0=\vbox{Start} a paragraph ... \par
\end{verbatim}
The effect of this is rather different from what was intended.
\alt
Also, \TeX\ will probably end the job inside a group.

%%\point Examples
%\section{Examples}
%\point Examples
\section{Examples}

%%\spoint Stretchable indentation 
%\subsection{Stretchable indentation }
%\spoint Stretchable indentation 
\subsection{Stretchable indentation }

%Considering that \cs{parindent} is a \gram{dimen}, not a \gram{glue},
%it is not possible to declare
%\begin{verbatim}
%\parindent=1cm plus 1fil
%\end{verbatim}
%in order to get
%a variable indentation at the start of a paragraph.
%This problem may be solved by putting
%\begin{verbatim}
%\everypar={\nobreak\hskip 1cm plus 1fil\relax}
%\end{verbatim}
%The \cs{nobreak} serves to prevent (in rare cases) a line break
%at the stretchable glue.
Considering that \cs{parindent} is a \gram{dimen}, not a \gram{glue},
it is not possible to declare
\begin{verbatim}
\parindent=1cm plus 1fil
\end{verbatim}
in order to get
a variable indentation at the start of a paragraph.
This problem may be solved by putting
\begin{verbatim}
\everypar={\nobreak\hskip 1cm plus 1fil\relax}
\end{verbatim}
The \cs{nobreak} serves to prevent (in rare cases) a line break
at the stretchable glue.

%%\spoint Suppressing indentation
%\subsection{Suppressing indentation}
%\spoint Suppressing indentation
\subsection{Suppressing indentation}

%Inserting 
%\verb.{\setbox0=\lastbox}. in the horizontal list
%at the beginning of the paragraph
%removes the indentation:
%indentation consists of a box, which is available through
%\cs{lastbox}. Assigning it effectively removes it from the list.
Inserting 
\verb.{\setbox0=\lastbox}. in the horizontal list
at the beginning of the paragraph
removes the indentation:
indentation consists of a box, which is available through
\cs{lastbox}. Assigning it effectively removes it from the list.

%However, this command sequence
%has to be inserted at a moment when \TeX\ has
%already switched to horizontal mode, so explicit insertion
%of these commands in front of the first \gram{horizontal
%command} of the paragraph does not work. 
%The moment of insertion of the \cs{everypar} tokens 
%is a better candidate: specifying
%\begin{verbatim}
%\everypar={{\setbox0=\lastbox}}
%\end{verbatim}
%leads to unindented paragraphs, even if \cs{parindent} is 
%not zero. 
However, this command sequence
has to be inserted at a moment when \TeX\ has
already switched to horizontal mode, so explicit insertion
of these commands in front of the first \gram{horizontal
command} of the paragraph does not work. 
The moment of insertion of the \cs{everypar} tokens 
is a better candidate: specifying
\begin{verbatim}
\everypar={{\setbox0=\lastbox}}
\end{verbatim}
leads to unindented paragraphs, even if \cs{parindent} is 
not zero. 


%%\spoint[indent:scheme] An indentation scheme
%\subsection{An indentation scheme}
%\label{indent:scheme}
%\spoint[indent:scheme] An indentation scheme
\subsection{An indentation scheme}
\label{indent:scheme}

%The above idea of letting the indentation box be removed
%\howto Control indentation systematically\par
%by \cs{everypar} can be put to use in a systematic approach
%to indentation, where two conditionals
%\begin{verbatim}
%\newif\ifNeedIndent %as a rule
%\newif\ifneedindent %special cases
%\end{verbatim}
%control whether paragraphs should indent as a rule, and
%whether in special cases indentation is needed.
%This section is taken from~\cite{E3}.
The above idea of letting the indentation box be removed
\howto Control indentation systematically\par
by \cs{everypar} can be put to use in a systematic approach
to indentation, where two conditionals
\begin{verbatim}
\newif\ifNeedIndent %as a rule
\newif\ifneedindent %special cases
\end{verbatim}
control whether paragraphs should indent as a rule, and
whether in special cases indentation is needed.
This section is taken from~\cite{E3}.

%We take a fixed \cs{everypar}:
%\begin{verbatim}
%\everypar={\ControlledIndentation}
%\end{verbatim}
%which executes in some cases the macro \cs{RemoveIndentation}
%\begin{verbatim}
%\def\RemoveIndentation{{\setbox0=\lastbox}}
%\end{verbatim}
%The implementation of \cs{ControlledIndentation} is:
%\begin{verbatim}
%\def\ControlledIndentation
%   {\ifNeedIndent \ifneedindent 
%                  \else \RemoveIndentation\needindenttrue \fi
%    \else \ifneedindent \needindentfalse
%          \else         \RemoveIndentation
%    \fi   \fi}
%\end{verbatim}
%In order to regulate indentation for a whole document,
%the user now once specifies, for instance,
%\begin{verbatim}
%\NeedIndenttrue
%\end{verbatim}
%to indicate that, in principle,
%all paragraphs should indent.
%Macros such as \cs{section} can then prevent
%indentation in individual cases:
%\begin{verbatim}
%\def\section#1{ ... \needindentfalse}
%\end{verbatim}
%    
We take a fixed \cs{everypar}:
\begin{verbatim}
\everypar={\ControlledIndentation}
\end{verbatim}
which executes in some cases the macro \cs{RemoveIndentation}
\begin{verbatim}
\def\RemoveIndentation{{\setbox0=\lastbox}}
\end{verbatim}
The implementation of \cs{ControlledIndentation} is:
\begin{verbatim}
\def\ControlledIndentation
   {\ifNeedIndent \ifneedindent 
                  \else \RemoveIndentation\needindenttrue \fi
    \else \ifneedindent \needindentfalse
          \else         \RemoveIndentation
    \fi   \fi}
\end{verbatim}
In order to regulate indentation for a whole document,
the user now once specifies, for instance,
\begin{verbatim}
\NeedIndenttrue
\end{verbatim}
to indicate that, in principle,
all paragraphs should indent.
Macros such as \cs{section} can then prevent
indentation in individual cases:
\begin{verbatim}
\def\section#1{ ... \needindentfalse}
\end{verbatim}
    

%%\spoint[skip:scheme] A paragraph skip scheme
%\subsection{A paragraph skip scheme}
%\label{skip:scheme}
%\spoint[skip:scheme] A paragraph skip scheme
\subsection{A paragraph skip scheme}
\label{skip:scheme}

%The use of \cs{everypar} to control indentation,
%\howto Control vertical white space systematically\par
%as was sketched above, can be extended to the
%paragraph skip.
The use of \cs{everypar} to control indentation,
\howto Control vertical white space systematically\par
as was sketched above, can be extended to the
paragraph skip.

%A visible white space between paragraphs can be
%created by the \cs{parskip} parameter, but, once this
%parameter has been set to some value, it is difficult
%to prevent paragraph skip in certain places elegantly.
%Usually, white space above and below environments
%and section headings should be specifiable independently
%of the paragraph skip. This section sketches an
%approach where \cs{parskip} is set to zero directly
%above and below certain constructs, while the \cs{everypar}
%is used to restore former values. This section is
%taken from~\cite{E4}.
A visible white space between paragraphs can be
created by the \cs{parskip} parameter, but, once this
parameter has been set to some value, it is difficult
to prevent paragraph skip in certain places elegantly.
Usually, white space above and below environments
and section headings should be specifiable independently
of the paragraph skip. This section sketches an
approach where \cs{parskip} is set to zero directly
above and below certain constructs, while the \cs{everypar}
is used to restore former values. This section is
taken from~\cite{E4}.

%First of all, here are two tools. The control sequence
%\cs{csarg} will be used only inside other macros;
%a typical call will look like
%\begin{verbatim}
%\csarg\vskip{#1Parskip}
%\end{verbatim}
%Here is the definition:
%\begin{verbatim}
%\def\csarg#1#2{\expandafter#1\csname#2\endcsname}
%\end{verbatim}
%Next follows a generalization of \cs{vskip}: the macro
%\cs{vspace} will not place its argument if the previous glue item
%is larger; otherwise it will eliminate the preceding
%glue, and place its argument.\begin{verbatim}
%\newskip\tempskipa
%\def\vspace#1{\tempskipa=#1\relax
%    \ifvmode \ifdim\tempskipa<\lastskip 
%             \else \vskip-\lastskip \vskip\tempskipa \fi
%    \else    \vskip\tempskipa \fi}
%\end{verbatim}
%    
%Now assume that any construct \n{foo} 
%with surrounding white space
%starts and ends with macro calls \verb>\StartEnvironment{foo}> and
%\verb>\EndEnvironment{foo}> respectively.
%Furthermore, assume that to this environment there correspond
%three glue registers:
%the \cs{fooStartskip} (glue
%above the environment), \cs{fooParskip} (the paragraph skip
%inside the environment), and the \cs{fooEndskip} (glue below
%the environment).
First of all, here are two tools. The control sequence
\cs{csarg} will be used only inside other macros;
a typical call will look like
\begin{verbatim}
\csarg\vskip{#1Parskip}
\end{verbatim}
Here is the definition:
\begin{verbatim}
\def\csarg#1#2{\expandafter#1\csname#2\endcsname}
\end{verbatim}
Next follows a generalization of \cs{vskip}: the macro
\cs{vspace} will not place its argument if the previous glue item
is larger; otherwise it will eliminate the preceding
glue, and place its argument.\begin{verbatim}
\newskip\tempskipa
\def\vspace#1{\tempskipa=#1\relax
    \ifvmode \ifdim\tempskipa<\lastskip 
             \else \vskip-\lastskip \vskip\tempskipa \fi
    \else    \vskip\tempskipa \fi}
\end{verbatim}
    
Now assume that any construct \n{foo} 
with surrounding white space
starts and ends with macro calls \verb>\StartEnvironment{foo}> and
\verb>\EndEnvironment{foo}> respectively.
Furthermore, assume that to this environment there correspond
three glue registers:
the \cs{fooStartskip} (glue
above the environment), \cs{fooParskip} (the paragraph skip
inside the environment), and the \cs{fooEndskip} (glue below
the environment).

%For restoring the value of the paragraph skip
%a conditional and a glue register are needed:
%\begin{verbatim}
%\newskip\TempParskip \newif\ifParskipNeedsRestoring
%\end{verbatim}
%The basic sequence for the
%starting and ending macros for the environments is then
%\begin{verbatim}
%\TempParskip=\parskip\parskip=0cm\relax
%\ParskipNeedsRestoringtrue
%\end{verbatim}
For restoring the value of the paragraph skip
a conditional and a glue register are needed:
\begin{verbatim}
\newskip\TempParskip \newif\ifParskipNeedsRestoring
\end{verbatim}
The basic sequence for the
starting and ending macros for the environments is then
\begin{verbatim}
\TempParskip=\parskip\parskip=0cm\relax
\ParskipNeedsRestoringtrue
\end{verbatim}

%The implementations can now be given as:
%\begin{verbatim}
%\def\StartEnvironment#1{\csarg\vspace{#1Startskip}
%    
%\begingroup % make changes local
%    \csarg\TempParskip{#1Parskip} \parskip=0cm\relax
%    \ParskipNeedsRestoringtrue}
%\def\EndEnvironment#1{\csarg\vspace{#1Endskip}
%    \endgroup % restore global values
%    \ifParskipNeedsRestoring
%    \else \TempParskip=\parskip \parskip=0cm\relax
%          \ParskipNeedsRestoringtrue
%    \fi}
%\end{verbatim}
%The \cs{EndEnvironment} macro needs a little comment:
%if an environment is used inside another one, and
%it occurs before the first paragraph in that environment,
%the value of the paragraph skip for the outer environment
%has already been saved. Therefore no further actions are
%required in that case.
The implementations can now be given as:
\begin{verbatim}
\def\StartEnvironment#1{\csarg\vspace{#1Startskip}
    
\begingroup % make changes local
    \csarg\TempParskip{#1Parskip} \parskip=0cm\relax
    \ParskipNeedsRestoringtrue}
\def\EndEnvironment#1{\csarg\vspace{#1Endskip}
    \endgroup % restore global values
    \ifParskipNeedsRestoring
    \else \TempParskip=\parskip \parskip=0cm\relax
          \ParskipNeedsRestoringtrue
    \fi}
\end{verbatim}
The \cs{EndEnvironment} macro needs a little comment:
if an environment is used inside another one, and
it occurs before the first paragraph in that environment,
the value of the paragraph skip for the outer environment
has already been saved. Therefore no further actions are
required in that case.

%Note that both macros start with a vertical skip. This prevents
%the \cs{begingroup} and \cs{endgroup} statements from
%occurring in a paragraph. 
Note that both macros start with a vertical skip. This prevents
the \cs{begingroup} and \cs{endgroup} statements from
occurring in a paragraph. 

%We now come to the main point: if necessary, the
%\cs{everypar} will restore the value of the paragraph skip.
%\begin{verbatim}
%\everypar={\ControlledIndentation\ControlledParskip}
%\def\ControlledParskip
%   {\ifParskipNeedsRestoring
%       \parskip=\TempParskip \ParskipNeedsRestoringfalse
%    \fi}
%\end{verbatim}
We now come to the main point: if necessary, the
\cs{everypar} will restore the value of the paragraph skip.
\begin{verbatim}
\everypar={\ControlledIndentation\ControlledParskip}
\def\ControlledParskip
   {\ifParskipNeedsRestoring
       \parskip=\TempParskip \ParskipNeedsRestoringfalse
    \fi}
\end{verbatim}

%\index{paragraph!start|)}
%\endofchapter
\index{paragraph!start|)}
\endofchapter

\end{document}
