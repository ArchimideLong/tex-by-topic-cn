% -*- coding: utf-8 -*-
\documentclass[letterpaper]{book}

% -*- coding: utf-8 -*-

\usepackage[b5paper,text={5in,8in},centering]{geometry}

\usepackage[CJKchecksingle]{xeCJK}
\setmainfont[Mapping=tex-text]{TeX Gyre Schola}
%\setsansfont{URW Gothic L Book}
%\setmonofont{Nimbus Mono L}
\setCJKmainfont[BoldFont=FandolHei,ItalicFont=FandolKai]{FandolSong}
\setCJKsansfont{FandolHei}
\setCJKmonofont{FandolFang}
\xeCJKsetup{PunctStyle = kaiming}

\linespread{1.25}
\setlength{\parindent}{2em}

\usepackage{xcolor}
\definecolor{myblue}{rgb}{0,0.2,0.6}

\usepackage{titlesec}
\titleformat{\chapter}
    {\normalfont\Huge\sffamily\color{myblue}}
    {第\thechapter 章}
    {1em}
    {}
%\titlespacing{\chapter}{0pt}{50pt}{40pt}
\titleformat{\section}
    {\normalfont\Large\sffamily\color{myblue}}
    {\thesection}
    {1em}
    {}
%\titlespacing{\section}{0pt}{3.5ex plus 1ex minus .2ex}{2.3ex plus .2ex}
\titleformat{\subsection}
    {\normalfont\large\sffamily\color{myblue}}
    {\thesubsection}
    {1em}
    {}
%\titlespacing{\subsection}{0pt}{3.25ex plus 1ex minus .2ex}{1.5ex plus .2ex}
%
\newpagestyle{special}[\small\sffamily]{
  \headrule
  \sethead[\usepage][][\chaptertitle]
  {\chaptertitle}{}{\usepage}}
\newpagestyle{main}[\small\sffamily]{
  \headrule
  \sethead[\usepage][][第\thechapter 章\quad\chaptertitle]
  {\thesection\quad\sectiontitle}{}{\usepage}}

\usepackage{titletoc}
%\setcounter{tocdepth}{1}
%\titlecontents{标题层次}[左间距]{上间距和整体格式}{标题序号}{标题内容}{指引线和页码}[下间距]
\titlecontents{chapter}[1.5em]{\vspace{.5em}\bfseries\sffamily}{\color{myblue}\contentslabel{1.5em}}{}
    {\titlerule*[20pt]{$\cdot$}\contentspage}[]
\titlecontents{section}[4.5em]{\sffamily}{\color{myblue}\contentslabel{3em}}{}
    {\titlerule*[20pt]{$\cdot$}\contentspage}[]
%\titlecontents{subsection}[8.5em]{\sffamily}{\contentslabel{4em}}{}
%    {\titlerule*[20pt]{$\cdot$}\contentspage}

\usepackage{enumitem}
\setlist{topsep=2pt,itemsep=2pt,parsep=1pt,leftmargin=\parindent}

\usepackage{fancyvrb}
\DefineVerbatimEnvironment{verbatim}{Verbatim}
  {xleftmargin=2em,baselinestretch=1,formatcom=\color{teal}\upshape}

\usepackage{etoolbox}
\makeatletter
\preto{\FV@ListVSpace}{\topsep=2pt \partopsep=0pt }
\makeatother

\usepackage[colorlinks,plainpages,pagebackref]{hyperref}
\hypersetup{
   pdfstartview={FitH},
   citecolor=teal,
   linkcolor=myblue,
   urlcolor=black,
   bookmarksnumbered
}

\usepackage{comment,makeidx,multicol}

%\usepackage{german}
%% german
%\righthyphenmin=3
%\mdqoff
%\captionsenglish
\usepackage[english]{babel}
{\catcode`"=13 \gdef"#1{\ifx#1"\discretionary{}{}{}\fi\relax}}
\def\mdqon{\catcode`"=13\relax}
\def\mdqoff{\catcode`"=12\relax}
\makeindex
\hyphenation{ex-em-pli-fies}

\newdimen\tempdima \newdimen\tempdimb

% these are fine
\def\handbreak{\\ \message{^^JManual break!!!!^^J}}
\def\nl{\protect\\}\def\n#1{{\tt #1}}
\def\cs#1{\texorpdfstring{{\tt\char`\\#1}}{\textbackslash#1}} %\def\cs#1{{\tt\char`\\#1}}
\let\csc\cs
\def\lb{{\tt\char`\{}}\def\rb{{\tt\char`\}}}
\def\gr#1{\texorpdfstring{$\langle$#1$\rangle$}{<#1>}} %\def\gr#1{$\langle$#1$\rangle$}
\def\key#1{{\tt#1}}
\def\alt{}\def\altt{}%this way in manstijl
\def\ldash{\unskip\ --\nobreak\ \ignorespaces}
\def\rdash{\unskip\nobreak\ --\ \ignorespaces}
% check these
\def\hex{{\tt"}}
\def\ascii{{\sc ascii}}
\def\ebcdic{{\sc ebcdic}}
\def\IniTeX{Ini\TeX}\def\LamsTeX{LAMS\TeX}\def\VirTeX{Vir\TeX}
\def\AmsTeX{Ams\TeX}
\def\TeXbook{the \TeX\ book}\def\web{{\sc web}}
% needs major thinking
\newenvironment{myquote}{\list{}{%
    \topsep=2pt \partopsep=0pt%
    \leftmargin=\parindent \rightmargin=\parindent
    }\item[]}{\endlist}
\newenvironment{disp}{\begin{myquote}}{\end{myquote}}
\newenvironment{Disp}{\begin{myquote}}{\end{myquote}}
\newenvironment{tdisp}{\begin{myquote}}{\end{myquote}}
\newenvironment{example}{\begin{myquote}\noindent\itshape 例子:}{\end{myquote}}
\newenvironment{inventory}{\begin{description}\raggedright}{\end{description}}
\newenvironment{glossinventory}{\begin{description}}{\end{description}}
\def\gram#1{\gr{#1}}%???
%
% index
%
\def\indexterm#1{\emph{#1}\index{#1}}
\def\indextermsub#1#2{\emph{#1 #2}\index{#1!#2}}
\def\indextermbus#1#2{\emph{#1 #2}\index{#2!#1}}
\def\term#1\par{\index{#1}}
\def\howto#1\par{}
\def\cstoidx#1\par{\index{#1@\cs{#1}@}}
\def\thecstoidx#1\par{\index{#1@\protect\csname #1\endcsname}}
\def\thecstoidxsub#1#2{\index{#1, #2@\protect\csname #1\endcsname, #2}\ignorespaces}
\def\csterm#1\par{\cstoidx #1\par\cs{#1}}
\def\csidx#1{\cstoidx #1\par\cs{#1}}

\def\tmc{\tracingmacros=2 \tracingcommands\tracingmacros}

%%%%%%%%%%%%%%%%%%%
\makeatletter
\def\snugbox{\hbox\bgroup\setbox\z@\vbox\bgroup
    \leftskip\z@
    \bgroup\aftergroup\make@snug
    \let\next=}
\def\make@snug{\par\sn@gify\egroup \box\z@\egroup}
\def\sn@gify
   {\skip\z@=\lastskip \unskip
    \advance\skip\z@\lastskip \unskip
    \unpenalty
    \setbox\z@\lastbox
    \ifvoid\z@ \nointerlineskip \else {\sn@gify} \fi
    \hbox{\unhbox\z@}\nointerlineskip
    \vskip\skip\z@
    }

\newdimen\fbh \fbh=60pt % dimension for easy scaling:
\newdimen\fbw \fbw=60pt % height and width of character box

\newdimen\dh \newdimen\dw % height and width of current character box
\newdimen\lh % height of previous character box
\newdimen\lw \lw=.4pt % line weight, instead of default .4pt

\def\hdotfill{\noindent
    \leaders\hbox{\vrule width 1pt height\lw
                  \kern4pt
                  \vrule width.5pt height\lw}\hfill\hbox{}
    \par}
\def\hlinefill{\noindent
    \leaders\hbox{\vrule width 5.5pt height\lw         }\hfill\hbox{}
    \par}
\def\stippel{$\qquad\qquad\qquad\qquad$}
\makeatother
%%%%%%%%%%%%%%%%%%%

%\def\SansSerif{\Typeface:macHelvetica }
%\def\SerifFont{\Typeface:macTimes }
%\def\SansSerif{\Typeface:bsGillSans }
%\def\SerifFont{\Typeface:bsBaskerville }
\let\SansSerif\relax \def\italic{\it}
\let\SerifFont\relax \def\MainFont{\rm}
\let\SansSerif\relax
\let\SerifFont\relax
\let\PopIndentLevel\relax \let\PushIndentLevel\relax
\let\ToVerso\relax \let\ToRecto\relax

%\def\stop@command@suffix{stop}
%\let\PopListLevel\PopIndentLevel
%\let\FlushRight\relax
%\let\flushright\FlushRight
%\let\SetListIndent\LevelIndent
%\def\awp{\ifhmode\vadjust{\penalty-10000 }\else
%    \penalty-10000 \fi}
\let\awp\relax
\let\PopIndentLevel\relax \let\PopListLevel\relax

\showboxdepth=-1

%\input figs
\def\endofchapter{\vfill\noindent}

\setcounter{chapter}{12}

\begin{document}

%\chapter{Conditionals}\label{if}
\chapter{Conditionals}\label{if}

%\emph{Conditionals}\index{conditionals}
%are an indispensible tool for powerful macros.
%\TeX\ has a large repertoire of conditionals for querying
%such things as category codes or processing modes.
%This chapter gives an inventory of the various conditionals,
%and it treats the evaluation of
%conditionals in detail.
\emph{Conditionals}\index{conditionals}
are an indispensible tool for powerful macros.
\TeX\ has a large repertoire of conditionals for querying
such things as category codes or processing modes.
This chapter gives an inventory of the various conditionals,
and it treats the evaluation of
conditionals in detail.

%\label{cschap:if}\label{cschap:ifcat}\label{cschap:ifx}\label{cschap:ifcase}\label{cschap:ifnum}\label{cschap:ifodd}\label{cschap:ifhmode}\label{cschap:ifvmode}\label{cschap:ifmmode}\label{cschap:ifinner}\label{cschap:ifdim}\label{cschap:ifvoid}\label{cschap:ifhbox2}\label{cschap:ifvbox2}\label{cschap:iftrue}\label{cschap:iffalse}\label{cschap:fi}\label{cschap:else}\label{cschap:or}\label{cschap:newif}
%\begin{inventory}
%\item [\cs{if}] 
%      Test equality of character codes. 
\label{cschap:if}\label{cschap:ifcat}\label{cschap:ifx}\label{cschap:ifcase}\label{cschap:ifnum}\label{cschap:ifodd}\label{cschap:ifhmode}\label{cschap:ifvmode}\label{cschap:ifmmode}\label{cschap:ifinner}\label{cschap:ifdim}\label{cschap:ifvoid}\label{cschap:ifhbox2}\label{cschap:ifvbox2}\label{cschap:iftrue}\label{cschap:iffalse}\label{cschap:fi}\label{cschap:else}\label{cschap:or}\label{cschap:newif}
\begin{inventory}
\item [\cs{if}] 
      Test equality of character codes. 

%\item [\cs{ifcat}] 
%      Test equality of category codes. 
\item [\cs{ifcat}] 
      Test equality of category codes. 

%\item [\cs{ifx}] 
%      Test equality of macro expansion, or equality of character code and
%      category code.
\item [\cs{ifx}] 
      Test equality of macro expansion, or equality of character code and
      category code.

%\item [\cs{ifcase}] 
%      Enumerated case statement.
\item [\cs{ifcase}] 
      Enumerated case statement.

%\item [\cs{ifnum}] 
%      Test relations between numbers.
\item [\cs{ifnum}] 
      Test relations between numbers.

%\item [\cs{ifodd}] 
%      Test whether a number is odd.
\item [\cs{ifodd}] 
      Test whether a number is odd.

%\item [\cs{ifhmode}] 
%      Test whether the current mode is (possibly restricted) horizontal mode.
\item [\cs{ifhmode}] 
      Test whether the current mode is (possibly restricted) horizontal mode.

%\item [\cs{ifvmode}] 
%      Test whether the current mode is (possibly internal) vertical mode.
\item [\cs{ifvmode}] 
      Test whether the current mode is (possibly internal) vertical mode.

%\item [\cs{ifmmode}] 
%     Test whether the current mode is (possibly display) math mode.
\item [\cs{ifmmode}] 
     Test whether the current mode is (possibly display) math mode.

%\item [\cs{ifinner}] 
%      Test whether the current mode is an internal mode.
\item [\cs{ifinner}] 
      Test whether the current mode is an internal mode.

%\item [\cs{ifdim}] 
%      Compare two dimensions. 
\item [\cs{ifdim}] 
      Compare two dimensions. 

%\item [\cs{ifvoid}] 
%      Test whether a box register is empty.
\item [\cs{ifvoid}] 
      Test whether a box register is empty.

%\item [\cs{ifhbox}] 
%      Test whether a box register contains a horizontal box.
\item [\cs{ifhbox}] 
      Test whether a box register contains a horizontal box.

%\item [\cs{ifvbox}] 
%      Test whether a box register contains a vertical box. 
\item [\cs{ifvbox}] 
      Test whether a box register contains a vertical box. 

%\item [\cs{ifeof}] 
%      Test for end of input stream or non-existence of file.
\item [\cs{ifeof}] 
      Test for end of input stream or non-existence of file.

%\item [\cs{iftrue}] 
%      A test that is always true.
%\item [\cs{iffalse}] 
%      A test that is always false.
%\item [\cs{fi}] 
%      Closing delimiter for all conditionals.
\item [\cs{iftrue}] 
      A test that is always true.
\item [\cs{iffalse}] 
      A test that is always false.
\item [\cs{fi}] 
      Closing delimiter for all conditionals.

%\item [\cs{else}] 
%      Select \gr{false text} of a conditional 
%      or default case of \cs{ifcase}.
\item [\cs{else}] 
      Select \gr{false text} of a conditional 
      or default case of \cs{ifcase}.

%\item [\cs{or}] 
%      Separator for entries of an \cs{ifcase}.
\item [\cs{or}] 
      Separator for entries of an \cs{ifcase}.

%\item [\cs{newif}] 
%     Create a new test.
\item [\cs{newif}] 
     Create a new test.

%\end{inventory}
\end{inventory}

%%\point The shape of conditionals
%\section{The shape of conditionals}
%\point The shape of conditionals
\section{The shape of conditionals}

%Conditionals in \TeX\ have one of the following two forms
%\cstoidx else\par\cstoidx fi\par
%\begin{disp}\cs{if...}\gr{test tokens}\gr{true text}\cs{fi}\nl
%     \cs{if...}\gr{test tokens}\gr{true text}\cs{else}%
%     \gr{false text}\cs{fi}\end{disp}
%where the \gr{test tokens} are zero or more tokens, depending on
%the particular conditional; the \gr{true text} is a series of tokens
%to be processed if the test turns out true, and the \gr{false text}
%is a series of tokens to be processed if the test turns out false.
%Both the \gr{true text} and the \gr{false text} can be empty.
Conditionals in \TeX\ have one of the following two forms
\cstoidx else\par\cstoidx fi\par
\begin{disp}\cs{if...}\gr{test tokens}\gr{true text}\cs{fi}\nl
     \cs{if...}\gr{test tokens}\gr{true text}\cs{else}%
     \gr{false text}\cs{fi}\end{disp}
where the \gr{test tokens} are zero or more tokens, depending on
the particular conditional; the \gr{true text} is a series of tokens
to be processed if the test turns out true, and the \gr{false text}
is a series of tokens to be processed if the test turns out false.
Both the \gr{true text} and the \gr{false text} can be empty.

%The exact process of how \TeX\ expands conditionals is treated
%below.
The exact process of how \TeX\ expands conditionals is treated
below.


%\section{Character and control sequence tests}
%\label{sec:charactertests}
\section{Character and control sequence tests}
\label{sec:charactertests}

%Three tests exist for testing character tokens and
%control sequence tokens.
Three tests exist for testing character tokens and
control sequence tokens.


%\subsection{\cs{if}}
\subsection{\cs{if}}

%Equality of character codes can be tested by
%\cstoidx if\par
%\begin{Disp}\cs{if}\gr{token$_1$}\gr{token$_2$}\end{Disp}
%In order to allow the tokens to be control sequences,
%\TeX\ assigns character code~256 to control sequences,
%the lowest positive number that is not the character code of a
%character token (remember that the legal character codes
%are~0--255).
Equality of character codes can be tested by
\cstoidx if\par
\begin{Disp}\cs{if}\gr{token$_1$}\gr{token$_2$}\end{Disp}
In order to allow the tokens to be control sequences,
\TeX\ assigns character code~256 to control sequences,
the lowest positive number that is not the character code of a
character token (remember that the legal character codes
are~0--255).

%Thus all control sequences are equal as far as \cs{if} is
%concerned, and they are unequal to all character tokens.
%As an example, this fact can be used to define
%\howto Test whether a token is a control sequence\par
%\begin{verbatim}
%\def\ifIsControlSequence#1{\if\noexpand#1\relax}
%\end{verbatim}
%which tests whether a token is a control sequence token
%instead of a character token (its result is unpredictable
%if the argument is a \verb>{...}> group).
Thus all control sequences are equal as far as \cs{if} is
concerned, and they are unequal to all character tokens.
As an example, this fact can be used to define
\howto Test whether a token is a control sequence\par
\begin{verbatim}
\def\ifIsControlSequence#1{\if\noexpand#1\relax}
\end{verbatim}
which tests whether a token is a control sequence token
instead of a character token (its result is unpredictable
if the argument is a \verb>{...}> group).

%After \cs{if} \TeX\ will expand until two unexpandable
%tokens are obtained, so it is necessary to prefix
%expandable control sequences and active characters
%with \cs{noexpand} when testing them with~\cs{if}.
After \cs{if} \TeX\ will expand until two unexpandable
tokens are obtained, so it is necessary to prefix
expandable control sequences and active characters
with \cs{noexpand} when testing them with~\cs{if}.

%\begin{example} After
%\begin{verbatim}
%\catcode`\b=13 \catcode`\c=13 \def b{a} \def c{a} \let\d=a
%\end{verbatim}
%we find that
%\begin{tdisp}
%\verb-\if bc- is true, because both \n b and \n c expand to \n a,\nl
%\verb-\if\noexpand b\noexpand c- is false, and\nl
%\verb-\if b\d- is true because \n{b} expands to the character~\n{a},
%   and \cs{d} is an implicit character token~\n{a}.
%\end{tdisp}
%\end{example}
\begin{example} After
\begin{verbatim}
\catcode`\b=13 \catcode`\c=13 \def b{a} \def c{a} \let\d=a
\end{verbatim}
we find that
\begin{tdisp}
\verb-\if bc- is true, because both \n b and \n c expand to \n a,\nl
\verb-\if\noexpand b\noexpand c- is false, and\nl
\verb-\if b\d- is true because \n{b} expands to the character~\n{a},
   and \cs{d} is an implicit character token~\n{a}.
\end{tdisp}
\end{example}

%%\spoint \cs{ifcat}
%\subsection{\cs{ifcat}}
%\spoint \cs{ifcat}
\subsection{\cs{ifcat}}

%The \cs{if} test ignores category codes; these can be tested
%\cstoidx ifcat\par
%by 
%\begin{Disp}\cs{ifcat}\gr{token$_1$}\gr{token$_2$}\end{Disp}
The \cs{if} test ignores category codes; these can be tested
\cstoidx ifcat\par
by 
\begin{Disp}\cs{ifcat}\gr{token$_1$}\gr{token$_2$}\end{Disp}

%This test is a lot like \cs{if}: \TeX\ expands after it
%until unexpandable tokens remain. For this test
%control sequences
%are considered to have category code~16\index{category!16}
%(ordinarily, category codes are in the range~0--15), which makes them
%all equal to each other, and different from all character
%tokens.
This test is a lot like \cs{if}: \TeX\ expands after it
until unexpandable tokens remain. For this test
control sequences
are considered to have category code~16\index{category!16}
(ordinarily, category codes are in the range~0--15), which makes them
all equal to each other, and different from all character
tokens.

%%\spoint \cs{ifx}
%\subsection{\protect\cs{ifx}}
%\spoint \cs{ifx}
\subsection{\protect\cs{ifx}}

%Equality of tokens is tested in a stronger sense than
%\cstoidx ifx\par
%the above by 
%\begin{Disp}\cs{ifx}\gr{token$_1$}\gr{token$_2$}\end{Disp}
Equality of tokens is tested in a stronger sense than
\cstoidx ifx\par
the above by 
\begin{Disp}\cs{ifx}\gr{token$_1$}\gr{token$_2$}\end{Disp}

%\begin{itemize}\item Character tokens are equal for \cs{ifx} if
%they have the same character code and category code.
%    \item Control sequence tokens are equal if they represent the
%same \TeX\ primitive, or have been similarly defined by
%\cs{font}, \cs{countdef}, or some such. For example,
%\begin{verbatim}
%\let\boxhor=\hbox \ifx\boxhor\hbox %is true
%\font\a=cmr10 \font\b=cmr10 \ifx\a\b %is true
%\end{verbatim}
%\item Control sequences are also equal if they are
%macros with the same parameter text and replacement text,
%and the same status with respect to \cs{outer} and~\cs{long}.
%For example,
%\begin{verbatim}
%\def\a{z} \def\b{z} \def\c1{z} \def\d{\a}
%\ifx\a\b %is true
%\ifx\a\c %is false
%\ifx\a\d %is false
%\end{verbatim}
%\end{itemize}
\begin{itemize}\item Character tokens are equal for \cs{ifx} if
they have the same character code and category code.
    \item Control sequence tokens are equal if they represent the
same \TeX\ primitive, or have been similarly defined by
\cs{font}, \cs{countdef}, or some such. For example,
\begin{verbatim}
\let\boxhor=\hbox \ifx\boxhor\hbox %is true
\font\a=cmr10 \font\b=cmr10 \ifx\a\b %is true
\end{verbatim}
\item Control sequences are also equal if they are
macros with the same parameter text and replacement text,
and the same status with respect to \cs{outer} and~\cs{long}.
For example,
\begin{verbatim}
\def\a{z} \def\b{z} \def\c1{z} \def\d{\a}
\ifx\a\b %is true
\ifx\a\c %is false
\ifx\a\d %is false
\end{verbatim}
\end{itemize}

%Tokens following this test are not expanded.
Tokens following this test are not expanded.

%By way of example of the use of \cs{ifx} consider string testing.
%A simple implementation of string testing in \TeX\ is as follows:
%\begin{verbatim}
%\def\ifEqString#1#2{\def\testa{#1}\def\testb{#2}%
%    \ifx\testa\testb}
%\end{verbatim}
%The two strings are used as the replacement text of two macros,
%and equality of these macros is tested.
%This is about as efficient as string testing can get:
%\TeX\ will traverse the definition texts of the
%macros \cs{testa} and \cs{testb}, which has precisely the
%right effect.
By way of example of the use of \cs{ifx} consider string testing.
A simple implementation of string testing in \TeX\ is as follows:
\begin{verbatim}
\def\ifEqString#1#2{\def\testa{#1}\def\testb{#2}%
    \ifx\testa\testb}
\end{verbatim}
The two strings are used as the replacement text of two macros,
and equality of these macros is tested.
This is about as efficient as string testing can get:
\TeX\ will traverse the definition texts of the
macros \cs{testa} and \cs{testb}, which has precisely the
right effect.

%As another example, one can test whether a control sequence is defined
%by\howto Test whether a control sequence is (un)defined\par
%\begin{verbatim}
%\def\ifUnDefinedCs#1{\expandafter
%    \ifx\csname#1\endcsname\relax}
%\ifUnDefinedCs{parindent} %is not true
%\ifUnDefinedCs{undefined} %is (one hopes) true
%\end{verbatim}
%This uses the fact that a \verb>\csname...\endcsname> command
%is equivalent to \cs{relax} if the control sequence
%has not been defined before. Unfortunately, this test also
%turns out true if a control sequence has been \cs{let} to
%\cs{relax}.
As another example, one can test whether a control sequence is defined
by\howto Test whether a control sequence is (un)defined\par
\begin{verbatim}
\def\ifUnDefinedCs#1{\expandafter
    \ifx\csname#1\endcsname\relax}
\ifUnDefinedCs{parindent} %is not true
\ifUnDefinedCs{undefined} %is (one hopes) true
\end{verbatim}
This uses the fact that a \verb>\csname...\endcsname> command
is equivalent to \cs{relax} if the control sequence
has not been defined before. Unfortunately, this test also
turns out true if a control sequence has been \cs{let} to
\cs{relax}.


%\section{Mode tests}
\section{Mode tests}

%In order to determine in which of the six modes 
%(see Chapter~\ref{hvmode}) \TeX\ 
%is currently operating, the tests \csidx{ifhmode},
%\csidx{ifvmode}, \csidx{ifmmode}, and~\csidx{ifinner}
%are available.
In order to determine in which of the six modes 
(see Chapter~\ref{hvmode}) \TeX\ 
is currently operating, the tests \csidx{ifhmode},
\csidx{ifvmode}, \csidx{ifmmode}, and~\csidx{ifinner}
are available.

%\begin{itemize}\item\cs{ifhmode} is true if \TeX\ is in horizontal mode
%or restricted horizontal mode.
%\item\cs{ifvmode} is true if \TeX\ is in vertical mode or
%internal vertical mode.
%\item\cs{ifmmode} is true if \TeX\ is in math mode or display
%math mode.\end{itemize}
\begin{itemize}\item\cs{ifhmode} is true if \TeX\ is in horizontal mode
or restricted horizontal mode.
\item\cs{ifvmode} is true if \TeX\ is in vertical mode or
internal vertical mode.
\item\cs{ifmmode} is true if \TeX\ is in math mode or display
math mode.\end{itemize}

%The \cs{ifinner} test is true if \TeX\ is in any of the three
%internal modes: restricted horizontal mode, internal vertical
%mode, and non-display math mode.
The \cs{ifinner} test is true if \TeX\ is in any of the three
internal modes: restricted horizontal mode, internal vertical
mode, and non-display math mode.

%See also chapter~\ref{chap:hvmode}.
See also chapter~\ref{chap:hvmode}.

%\section{Numerical tests}
\section{Numerical tests}

%Numerical relations between \gr{number}s can be tested
%\cstoidx ifnum\par
%with 
%\begin{disp}\cs{ifnum}\gr{number$_1$}\gr{relation}%
%\gr{number$_2$}\end{disp}
%where the relation is a character \n{<}, \n{=}, or~\n{>},
%of category~12\index{category!12}.
Numerical relations between \gr{number}s can be tested
\cstoidx ifnum\par
with 
\begin{disp}\cs{ifnum}\gr{number$_1$}\gr{relation}%
\gr{number$_2$}\end{disp}
where the relation is a character \n{<}, \n{=}, or~\n{>},
of category~12\index{category!12}.

%Quantities such as glue can be used as a number here
%through the conversion to scaled points, and \TeX\
%will expand in order to arrive at the two \gr{number}s.
Quantities such as glue can be used as a number here
through the conversion to scaled points, and \TeX\
will expand in order to arrive at the two \gr{number}s.

%Testing for odd or even numbers can be done with \csidx{ifodd}:
%the test
%\begin{disp}\cs{ifodd}\gr{number}\end{disp}
%is true if the \gr{number} is odd.
Testing for odd or even numbers can be done with \csidx{ifodd}:
the test
\begin{disp}\cs{ifodd}\gr{number}\end{disp}
is true if the \gr{number} is odd.

%%\point Other tests
%\section{Other tests}
%\point Other tests
\section{Other tests}

%%\spoint Dimension testing
%\subsection{Dimension testing}
%\spoint Dimension testing
\subsection{Dimension testing}

%Relations between \gr{dimen} values (Chapter~\ref{glue})
%can be tested with
%\csidx{ifdim} using the same three relations as in \cs{ifnum}.
Relations between \gr{dimen} values (Chapter~\ref{glue})
can be tested with
\csidx{ifdim} using the same three relations as in \cs{ifnum}.

%%\spoint Box tests
%\subsection{Box tests}
%\spoint Box tests
\subsection{Box tests}

%Contents of box registers (Chapter~\ref{boxes}) can be tested with
%\cstoidx ifhbox\par\cstoidx ifvbox\par\cstoidx ifvoid\par
%\begin{disp}\cs{ifvoid}\gr{8-bit number}\end{disp}
%which is true if the register contains no box,
%\begin{disp}\cs{ifhbox}\gr{8-bit number}\end{disp}
%which is true if the register contains a horizontal box, and
%\begin{disp}\cs{ifvbox}\gr{8-bit number}\end{disp}
%which is true if the register contains a vertical box.
Contents of box registers (Chapter~\ref{boxes}) can be tested with
\cstoidx ifhbox\par\cstoidx ifvbox\par\cstoidx ifvoid\par
\begin{disp}\cs{ifvoid}\gr{8-bit number}\end{disp}
which is true if the register contains no box,
\begin{disp}\cs{ifhbox}\gr{8-bit number}\end{disp}
which is true if the register contains a horizontal box, and
\begin{disp}\cs{ifvbox}\gr{8-bit number}\end{disp}
which is true if the register contains a vertical box.

%%\spoint I{/}O tests
%\subsection{I{/}O tests}
%\spoint I{/}O tests
\subsection{I{/}O tests}

%The status of input streams (Chapter~\ref{io}) can be tested with
%\cstoidx ifeof\par
%the end-of-file test
%\cs{ifeof}\gr{number}, which is only false
%if the number is in the range 0--15, and the corresponding
%stream is open and not fully read. In particular, this test
%is true if the file name connected
%to this stream (through \cs{openin})
%does not correspond to an existing file.
%See the example on page~\pageref{ex:eof}.
The status of input streams (Chapter~\ref{io}) can be tested with
\cstoidx ifeof\par
the end-of-file test
\cs{ifeof}\gr{number}, which is only false
if the number is in the range 0--15, and the corresponding
stream is open and not fully read. In particular, this test
is true if the file name connected
to this stream (through \cs{openin})
does not correspond to an existing file.
See the example on page~\pageref{ex:eof}.

%%\spoint Case statement
%\subsection{Case statement}
%\spoint Case statement
\subsection{Case statement}

%The \TeX\ case statement is called \cs{ifcase};
%\cstoidx ifcase\par\cstoidx or\par
%its syntax is
%\begin{disp}\cs{ifcase}\gr{number}\gr{case$_0$}\cs{or}%
%\n{...}\cs{or}\gr{case$_n$}\cs{else}\gr{other cases}\cs{fi}
%\end{disp} where for $n$ cases there are $n-1$ \cs{or}
%control sequences. Each of the \gr{case$_i$}
%parts can be empty,
%and the \cs{else}\gr{other cases} part is optional.
The \TeX\ case statement is called \cs{ifcase};
\cstoidx ifcase\par\cstoidx or\par
its syntax is
\begin{disp}\cs{ifcase}\gr{number}\gr{case$_0$}\cs{or}%
\n{...}\cs{or}\gr{case$_n$}\cs{else}\gr{other cases}\cs{fi}
\end{disp} where for $n$ cases there are $n-1$ \cs{or}
control sequences. Each of the \gr{case$_i$}
parts can be empty,
and the \cs{else}\gr{other cases} part is optional.

%%\spoint Special tests
%\subsection{Special tests}
%\spoint Special tests
\subsection{Special tests}

%The tests \cs{iftrue} and \cs{iffalse} are always
%\cstoidx iftrue\par\cstoidx iffalse\par
%true and false respectively.
%They are mainly useful as tools in macros.
The tests \cs{iftrue} and \cs{iffalse} are always
\cstoidx iftrue\par\cstoidx iffalse\par
true and false respectively.
They are mainly useful as tools in macros.

%For instance, the sequences
%\begin{verbatim}
%\iftrue{\else}\fi
%\end{verbatim}
%and
%\begin{verbatim}
%\iffalse{\else}\fi
%\end{verbatim}
%yield a left and right
%brace respectively, but they have balanced braces, so they
%can be used inside a macro replacement text.
For instance, the sequences
\begin{verbatim}
\iftrue{\else}\fi
\end{verbatim}
and
\begin{verbatim}
\iffalse{\else}\fi
\end{verbatim}
yield a left and right
brace respectively, but they have balanced braces, so they
can be used inside a macro replacement text.

%The \cs{newif} macro, treated below,
%provides another use of \cs{iftrue} and \cs{iffalse}.
%On page 260 of \TeXbook\ these control sequences
%are also used in an interesting manner.
The \cs{newif} macro, treated below,
provides another use of \cs{iftrue} and \cs{iffalse}.
On page 260 of \TeXbook\ these control sequences
are also used in an interesting manner.

%%\point[newif:def] The \cs{newif} macro
%\section{The \protect\cs{newif} macro}
%\label{newif:def}
%\point[newif:def] The \cs{newif} macro
\section{The \protect\cs{newif} macro}
\label{newif:def}

%The plain format defines an (outer) macro \csidx{newif} by
%which the user can define new conditionals.
%If the user defines
%\begin{verbatim}
%\newif\iffoo
%\end{verbatim}
%\TeX\ defines three new control sequences,
%\cs{footrue} and \cs{foofalse} with which the user can set
%the condition, and \cs{iffoo} which tests the `foo' condition.
The plain format defines an (outer) macro \csidx{newif} by
which the user can define new conditionals.
If the user defines
\begin{verbatim}
\newif\iffoo
\end{verbatim}
\TeX\ defines three new control sequences,
\cs{footrue} and \cs{foofalse} with which the user can set
the condition, and \cs{iffoo} which tests the `foo' condition.

%The macro call \verb-\newif\iffoo- expands to
%\begin{verbatim}
%\def\footrue{\let\iffoo=\iftrue} \def\foofalse{\let\iffoo=\iffalse}
%\foofalse
%\end{verbatim}
%The actual definition, especially the part that ensures that
%the \cs{iffoo} indeed starts with \cs{if}, is a pretty hack.
%An explanation follows here.
%This uses concepts from Chapters~\ref{macro}
%and~\ref{expand}.
The macro call \verb-\newif\iffoo- expands to
\begin{verbatim}
\def\footrue{\let\iffoo=\iftrue} \def\foofalse{\let\iffoo=\iffalse}
\foofalse
\end{verbatim}
The actual definition, especially the part that ensures that
the \cs{iffoo} indeed starts with \cs{if}, is a pretty hack.
An explanation follows here.
This uses concepts from Chapters~\ref{macro}
and~\ref{expand}.

%The macro \cs{newif} starts as follows:
%\begin{verbatim}
%\outer\def\newif#1{\count@\escapechar \escapechar\m@ne
%\end{verbatim}
%This saves the current escape character in \cs{count@}, and
%sets the value of \cs{escapechar} to~\n{-1}.
%The latter action has the
%effect that no escape character is used in the output
%of \cs{string}\gr{control sequence}.
The macro \cs{newif} starts as follows:
\begin{verbatim}
\outer\def\newif#1{\count@\escapechar \escapechar\m@ne
\end{verbatim}
This saves the current escape character in \cs{count@}, and
sets the value of \cs{escapechar} to~\n{-1}.
The latter action has the
effect that no escape character is used in the output
of \cs{string}\gr{control sequence}.

%An auxiliary macro \verb>\if@> is defined by
%\begin{verbatim}
%{\uccode`1=`i \uccode`2=`f \uppercase{\gdef\if@12{}}}
%\end{verbatim}
%Since the uppercase command changes only character codes, and
%not category codes, the macro \cs{if@} now has
%to be followed by the characters \n{if} of category~12\index{category!12}.
%Ordinarily, these characters have category code~11\index{category!11}.
%In effect this
%macro then eats these two characters, and \TeX\ complains if
%they are not present.
An auxiliary macro \verb>\if@> is defined by
\begin{verbatim}
{\uccode`1=`i \uccode`2=`f \uppercase{\gdef\if@12{}}}
\end{verbatim}
Since the uppercase command changes only character codes, and
not category codes, the macro \cs{if@} now has
to be followed by the characters \n{if} of category~12\index{category!12}.
Ordinarily, these characters have category code~11\index{category!11}.
In effect this
macro then eats these two characters, and \TeX\ complains if
they are not present.

%Next there is a macro \verb>\@if> defined by
%\begin{verbatim}
%\def\@if#1#2{\csname\expandafter\if@\string#1#2\endcsname}
%\end{verbatim}
%which will be called like \verb>\@if\iffoo{true}> and
%\verb>\@if\iffoo{false}>. 
Next there is a macro \verb>\@if> defined by
\begin{verbatim}
\def\@if#1#2{\csname\expandafter\if@\string#1#2\endcsname}
\end{verbatim}
which will be called like \verb>\@if\iffoo{true}> and
\verb>\@if\iffoo{false}>. 

%Let us examine the call \verb>\@if\iffoo{true}>.
%\begin{itemize}\item The \cs{expandafter} reaches over the \verb>\if@>
%\mdqon
%to expand \cs{string} first. The part \cs{string}""\cs{iffoo}
%\mdqoff
%expands to \n{iffoo} because the escape character is not printed,
%and all characters have category~12.
%\item The \verb>\if@> eats the first two characters
%\n i$_{12}$\n f$_{12}$ of this.
%\item As a result, the final expansion of \verb>\@if\iffoo{true}>
%is then
%\begin{verbatim}
%\csname footrue\endcsname
%\end{verbatim}
%\end{itemize}
Let us examine the call \verb>\@if\iffoo{true}>.
\begin{itemize}\item The \cs{expandafter} reaches over the \verb>\if@>
\mdqon
to expand \cs{string} first. The part \cs{string}""\cs{iffoo}
\mdqoff
expands to \n{iffoo} because the escape character is not printed,
and all characters have category~12.
\item The \verb>\if@> eats the first two characters
\n i$_{12}$\n f$_{12}$ of this.
\item As a result, the final expansion of \verb>\@if\iffoo{true}>
is then
\begin{verbatim}
\csname footrue\endcsname
\end{verbatim}
\end{itemize}

%Now we can treat the relevant parts of \cs{newif} itself:
%\begin{verbatim}
%\expandafter\expandafter\expandafter
%   \edef\@if#1{true}{\let\noexpand#1=\noexpand\iftrue}%
%\end{verbatim}
Now we can treat the relevant parts of \cs{newif} itself:
\begin{verbatim}
\expandafter\expandafter\expandafter
   \edef\@if#1{true}{\let\noexpand#1=\noexpand\iftrue}%
\end{verbatim}

%The three \cs{expandafter} commands may look intimidating, so let us
%take one step at a time.
%\begin{itemize}\item One \cs{expandafter} is necessary to reach over the \cs{edef},
%such that \verb>\@if> will expand:
%\begin{verbatim}
%\expandafter\edef\@if\iffoo{true}
%\end{verbatim}
%gives
%\begin{verbatim}
%\edef\csname footrue\endcsname
%\end{verbatim}
%\item Then another \cs{expandafter} is necessary to activate
%\altt
%the \cs{csname}:
%\begin{verbatim}
%\expandafter \expandafter \expandafter \edef \@if ...
%%   new          old          new
%\end{verbatim}
%\item This makes the final expansion 
%\begin{verbatim}
%\edef\footrue{\let\noexpand\iffoo=\noexpand\iftrue}
%\end{verbatim}
%\end{itemize}
The three \cs{expandafter} commands may look intimidating, so let us
take one step at a time.
\begin{itemize}\item One \cs{expandafter} is necessary to reach over the \cs{edef},
such that \verb>\@if> will expand:
\begin{verbatim}
\expandafter\edef\@if\iffoo{true}
\end{verbatim}
gives
\begin{verbatim}
\edef\csname footrue\endcsname
\end{verbatim}
\item Then another \cs{expandafter} is necessary to activate
\altt
the \cs{csname}:
\begin{verbatim}
\expandafter \expandafter \expandafter \edef \@if ...
%   new          old          new
\end{verbatim}
\item This makes the final expansion 
\begin{verbatim}
\edef\footrue{\let\noexpand\iffoo=\noexpand\iftrue}
\end{verbatim}
\end{itemize}

%After this follows a similar statement for the \n{false} case:
%\begin{verbatim}
%   \expandafter\expandafter\expandafter
%   \edef\@if#1{false}{\let\noexpand#1=\noexpand\iffalse}%
%\end{verbatim}
%The conditional starts out false, and the escape character
%has to be reset:
%\begin{verbatim}
%  \@if#1{false}\escapechar\count@} 
%\end{verbatim}
After this follows a similar statement for the \n{false} case:
\begin{verbatim}
   \expandafter\expandafter\expandafter
   \edef\@if#1{false}{\let\noexpand#1=\noexpand\iffalse}%
\end{verbatim}
The conditional starts out false, and the escape character
has to be reset:
\begin{verbatim}
  \@if#1{false}\escapechar\count@} 
\end{verbatim}


%\section{Evaluation of conditionals}
\section{Evaluation of conditionals}

%\TeX's conditionals behave differently from those
%in ordinary programming languages. In many instances
%one may not notice the difference, but in certain contexts
%it is important to know precisely the \indextermbus{evaluation of}{conditionals} proceeds.
\TeX's conditionals behave differently from those
in ordinary programming languages. In many instances
one may not notice the difference, but in certain contexts
it is important to know precisely the \indextermbus{evaluation of}{conditionals} proceeds.

%When \TeX\ evaluates a conditional, it first determines
%what is to be tested. This in itself may involve some
%expansion; as we saw in the previous chapter, 
%only after an \cs{ifx} test
%does \TeX\  not expand. After all other tests \TeX\ will
%expand tokens until the extent of the test and the tokens
%to be tested have been determined. On the basis of the outcome
%of this test the \gr{true text} and the \gr{false text}
%are either expanded or skipped.
When \TeX\ evaluates a conditional, it first determines
what is to be tested. This in itself may involve some
expansion; as we saw in the previous chapter, 
only after an \cs{ifx} test
does \TeX\  not expand. After all other tests \TeX\ will
expand tokens until the extent of the test and the tokens
to be tested have been determined. On the basis of the outcome
of this test the \gr{true text} and the \gr{false text}
are either expanded or skipped.

%For the processing of the parts of the conditional
%let us consider some cases separately.
%\begin{itemize}
%\item \verb>\if... ... \fi> and the result of the test is false.
% After the test \TeX\ will start skipping material
% without expansion, without counting braces, but balancing
% nested conditionals, until a \cs{fi} token is encountered.
% If the \cs{fi} is not found an error message results
% at the end of the file:
% 
%\begin{disp}\tt Incomplete \cs{if...}; all text was ignored after line \n{...}
% \end{disp} where the line number indicated is that of the line
% where \TeX\ started skipping, that is, where the conditional
% occurred.
For the processing of the parts of the conditional
let us consider some cases separately.
\begin{itemize}
\item \verb>\if... ... \fi> and the result of the test is false.
 After the test \TeX\ will start skipping material
 without expansion, without counting braces, but balancing
 nested conditionals, until a \cs{fi} token is encountered.
 If the \cs{fi} is not found an error message results
 at the end of the file:
 
\begin{disp}\tt Incomplete \cs{if...}; all text was ignored after line \n{...}
 \end{disp} where the line number indicated is that of the line
 where \TeX\ started skipping, that is, where the conditional
 occurred.

%\item \verb>\if... \else ... \fi> and the result of the test is false.
% Any material in between the condition and the \cs{else} is skipped
% without expansion, without counting braces, but balancing nested
% conditionals.
\item \verb>\if... \else ... \fi> and the result of the test is false.
 Any material in between the condition and the \cs{else} is skipped
 without expansion, without counting braces, but balancing nested
 conditionals.

% The \cs{fi} token can be the result of expansion; if it never
% turns up \TeX\ will give a diagnostic message
% 
%\begin{disp}\tt \cs{end} occurred when \cs{if...} on line \n{...}
%      was incomplete\end{disp}
% This sort of error is not visible in the output.
 The \cs{fi} token can be the result of expansion; if it never
 turns up \TeX\ will give a diagnostic message
 
\begin{disp}\tt \cs{end} occurred when \cs{if...} on line \n{...}
      was incomplete\end{disp}
 This sort of error is not visible in the output.

% This point plus the previous may jointly be described as follows:
% after a false condition \TeX\ skips until an \cs{else} or \cs{fi}
% is found; any material in between \cs{else} and \cs{fi} is processed.
% 
%\item \verb>\if... ... \fi> and the result of the test is true.
% \TeX\ will start processing the material following the condition.
% As above, the \cs{fi} token may be inserted by expansion of
% a macro.
% 
%\item \verb>\if... \else ... \fi> and the result of the test is true.
% Any material following the condition is processed until the \cs{else}
% is found; then \TeX\ skips everything until the matching \cs{fi}
% is found.
% 
% This point plus the previous may be described as follows:
% after a true test \TeX\ starts processing material until
% an \cs{else} or \cs{fi} is found; if an \cs{else} is found
% \TeX\ skips until it finds the matching \cs{fi}.
%\end{itemize}
 This point plus the previous may jointly be described as follows:
 after a false condition \TeX\ skips until an \cs{else} or \cs{fi}
 is found; any material in between \cs{else} and \cs{fi} is processed.
 
\item \verb>\if... ... \fi> and the result of the test is true.
 \TeX\ will start processing the material following the condition.
 As above, the \cs{fi} token may be inserted by expansion of
 a macro.
 
\item \verb>\if... \else ... \fi> and the result of the test is true.
 Any material following the condition is processed until the \cs{else}
 is found; then \TeX\ skips everything until the matching \cs{fi}
 is found.
 
 This point plus the previous may be described as follows:
 after a true test \TeX\ starts processing material until
 an \cs{else} or \cs{fi} is found; if an \cs{else} is found
 \TeX\ skips until it finds the matching \cs{fi}.
\end{itemize}


%%\point Assorted remarks
%\section{Assorted remarks}
%\point Assorted remarks
\section{Assorted remarks}

%%\spoint The test gobbles up tokens
%\subsection{The test gobbles up tokens}
%\spoint The test gobbles up tokens
\subsection{The test gobbles up tokens}

%A common mistake is to write the following:
%\begin{verbatim}
%\ifnum\x>0\someaction \else\anotheraction \fi
%\end{verbatim}
%which has the effect that the \verb.\someaction. is expanded,
%regardless of whether the test succeeds or not. 
%The reason for this is that \TeX\ evaluates the input stream until
%it is certain that it has found the arguments to be tested.
%In this case it is perfectly possible for the \verb.\someaction.
%to yield a digit, so it is expanded. The remedy is to insert
%\altt
%a space or a \cs{relax} control sequence
%after the last digit of the number to be tested.
A common mistake is to write the following:
\begin{verbatim}
\ifnum\x>0\someaction \else\anotheraction \fi
\end{verbatim}
which has the effect that the \verb.\someaction. is expanded,
regardless of whether the test succeeds or not. 
The reason for this is that \TeX\ evaluates the input stream until
it is certain that it has found the arguments to be tested.
In this case it is perfectly possible for the \verb.\someaction.
to yield a digit, so it is expanded. The remedy is to insert
\altt
a space or a \cs{relax} control sequence
after the last digit of the number to be tested.

%%\spoint The test wants to gobble up the \cs{else} or \cs{fi}
%\subsection{The test wants to gobble up the \cs{else} or \cs{fi}}
%\spoint The test wants to gobble up the \cs{else} or \cs{fi}
\subsection{The test wants to gobble up the \cs{else} or \cs{fi}}

%The same mechanism that underlies the phenomenon in the previous
%point can lead to even more surprising effects if \TeX\
%bumps into an \verb.\else., \verb.\or., or \verb.\fi. 
%while still busy determining the extent of the test itself.
The same mechanism that underlies the phenomenon in the previous
point can lead to even more surprising effects if \TeX\
bumps into an \verb.\else., \verb.\or., or \verb.\fi. 
while still busy determining the extent of the test itself.

%Recall that \verb.\pageno. is a synomym for \verb.\count0., and
%consider the following examples:
%\begin{verbatim}
%\newcount\nct \nct=1\ifodd\pageno\else 2\fi 1
%\end{verbatim}
%and
%\begin{verbatim}
%\newcount\nct \nct=1\ifodd\count0\else 2\fi 1
%\end{verbatim}
%The first example will assign either 11 or~121 to \cs{nct}, 
%but the second one will assign 1 or~121. 
%The explanation is that
%in cases like the second, where
%\altt
%an \verb.\else. is encountered while the
%test still has not been delimited, a \verb.\relax. is inserted.
%In the case that \verb.\count0. is odd the result will thus be \verb.\relax.,
%and the example will yield
%\begin{verbatim}
%\nct=1\relax2
%\end{verbatim}
%which will assign~1 to \cs{nct}, and print~2.
Recall that \verb.\pageno. is a synomym for \verb.\count0., and
consider the following examples:
\begin{verbatim}
\newcount\nct \nct=1\ifodd\pageno\else 2\fi 1
\end{verbatim}
and
\begin{verbatim}
\newcount\nct \nct=1\ifodd\count0\else 2\fi 1
\end{verbatim}
The first example will assign either 11 or~121 to \cs{nct}, 
but the second one will assign 1 or~121. 
The explanation is that
in cases like the second, where
\altt
an \verb.\else. is encountered while the
test still has not been delimited, a \verb.\relax. is inserted.
In the case that \verb.\count0. is odd the result will thus be \verb.\relax.,
and the example will yield
\begin{verbatim}
\nct=1\relax2
\end{verbatim}
which will assign~1 to \cs{nct}, and print~2.


%%\spoint[after:cond] Macros and conditionals; the use of \cs{expandafter}
%\subsection{Macros and conditionals; the use of \cs{expandafter}}
%\label{after:cond}
%\spoint[after:cond] Macros and conditionals; the use of \cs{expandafter}
\subsection{Macros and conditionals; the use of \cs{expandafter}}
\label{after:cond}

%Consider the following example:
%\begin{verbatim}
%\def\bold#1{{\bf #1}} \def\slant#1{{\sl #1}}
%\ifnum1>0 \bold \else \slant \fi {some text} ...
%\end{verbatim}
%This will make not only `some text', 
%but {\sl all\/} subsequent text bold.
%Also, at the end of the job there will be a notice that
%`end occurred inside a group at level~1'.
%Switching on \cs{tracingmacros} reveals that the argument
%of \verb.\bold. was \verb.\else..
%This means that, after expansion of \verb.\bold., 
%the input stream looked like
%\begin{verbatim}
%\ifnum1>0 {\bf \else }\fi {some text} rest of the text 
%\end{verbatim}
%so the closing brace was skipped as part of the \gram{false text}.
%Effectively, then, the resulting stream is
%\begin{verbatim}
%{\bf {some text} rest of the text
%\end{verbatim}
%which is unbalanced.
Consider the following example:
\begin{verbatim}
\def\bold#1{{\bf #1}} \def\slant#1{{\sl #1}}
\ifnum1>0 \bold \else \slant \fi {some text} ...
\end{verbatim}
This will make not only `some text', 
but {\sl all\/} subsequent text bold.
Also, at the end of the job there will be a notice that
`end occurred inside a group at level~1'.
Switching on \cs{tracingmacros} reveals that the argument
of \verb.\bold. was \verb.\else..
This means that, after expansion of \verb.\bold., 
the input stream looked like
\begin{verbatim}
\ifnum1>0 {\bf \else }\fi {some text} rest of the text 
\end{verbatim}
so the closing brace was skipped as part of the \gram{false text}.
Effectively, then, the resulting stream is
\begin{verbatim}
{\bf {some text} rest of the text
\end{verbatim}
which is unbalanced.

%One solution to this sort of problem would be to write
%\begin{verbatim}
%\ifnum1>0 \let\next=\bold \else \let\next=\slant \fi \next
%\end{verbatim}
%but  a solution using \cs{expandafter} is also possible:
%\begin{verbatim}
%\ifnum1>0 \expandafter \bold \else \expandafter \slant \fi
%\end{verbatim}
%This works, because the \cs{expandafter} commands let \TeX\ determine
%the boundaries of the \gram{true text} and the \gram{false text}.
One solution to this sort of problem would be to write
\begin{verbatim}
\ifnum1>0 \let\next=\bold \else \let\next=\slant \fi \next
\end{verbatim}
but  a solution using \cs{expandafter} is also possible:
\begin{verbatim}
\ifnum1>0 \expandafter \bold \else \expandafter \slant \fi
\end{verbatim}
This works, because the \cs{expandafter} commands let \TeX\ determine
the boundaries of the \gram{true text} and the \gram{false text}.

%In fact, the second solution may be preferred over the first,
%since conditionals are handled by the expansion processor,
%and the \cs{let} statements are tackled only by the execution
%processor; that is, they are not expandable.
%Thus the second solution will (and the first will not)
%work, for instance,
%inside an~\cs{edef}.
In fact, the second solution may be preferred over the first,
since conditionals are handled by the expansion processor,
and the \cs{let} statements are tackled only by the execution
processor; that is, they are not expandable.
Thus the second solution will (and the first will not)
work, for instance,
inside an~\cs{edef}.

%Another example with \cs{expandafter} is the sequence
%\begin{verbatim}
%\def\get#1\get{ ... }
%\expandafter \get \ifodd1 \ifodd3 5\fi \fi \get
%\end{verbatim}
%This gives
%\begin{verbatim}
%#1<- \ifodd3 5\fi \fi
%\end{verbatim}
%and
%\begin{verbatim}
%\expandafter \get \ifodd2 \ifodd3 5\fi\fi \get
%\end{verbatim}
%gives
%\begin{verbatim}
%#1<-
%\end{verbatim}
%This illustrates again that the result of evaluating a 
%conditional is not the final expansion, but the start
%of the expansion of the \gr{true text} or \gr{false text},
%depending on the outcome of the test.
Another example with \cs{expandafter} is the sequence
\begin{verbatim}
\def\get#1\get{ ... }
\expandafter \get \ifodd1 \ifodd3 5\fi \fi \get
\end{verbatim}
This gives
\begin{verbatim}
#1<- \ifodd3 5\fi \fi
\end{verbatim}
and
\begin{verbatim}
\expandafter \get \ifodd2 \ifodd3 5\fi\fi \get
\end{verbatim}
gives
\begin{verbatim}
#1<-
\end{verbatim}
This illustrates again that the result of evaluating a 
conditional is not the final expansion, but the start
of the expansion of the \gr{true text} or \gr{false text},
depending on the outcome of the test.

%A detail should be noted: with \cs{expandafter}
%it is possible that the \verb.\else. is encountered
%before the \gram{true text} has been expanded completely.
%This raises the question as to the exact timing of expansion
%and skipping.
%In the example
%\begin{verbatim}
%\def\hello{\message{Hello!}}
%\ifnum1>0 \expandafter \hello \else \message{goodbye} \bye
%\end{verbatim}
%the error message caused by the missing \verb.\fi. is given
%without \verb.\hello. ever having been expanded.
%The conclusion must be that the \gram{false text} is
%skipped as soon as it has been located, even if this is at a time
%when the \gram{true text} has not been expanded completely.
A detail should be noted: with \cs{expandafter}
it is possible that the \verb.\else. is encountered
before the \gram{true text} has been expanded completely.
This raises the question as to the exact timing of expansion
and skipping.
In the example
\begin{verbatim}
\def\hello{\message{Hello!}}
\ifnum1>0 \expandafter \hello \else \message{goodbye} \bye
\end{verbatim}
the error message caused by the missing \verb.\fi. is given
without \verb.\hello. ever having been expanded.
The conclusion must be that the \gram{false text} is
skipped as soon as it has been located, even if this is at a time
when the \gram{true text} has not been expanded completely.

%%\spoint Incorrect matching
%\subsection{Incorrect matching}
%\spoint Incorrect matching
\subsection{Incorrect matching}

%\TeX's matching of \verb.\if., \verb.\else., and \verb.\fi.
%is easily upset. For instance, \TeXbook\ warns you that
%you should not say
%\begin{verbatim}
% \let\ifabc=\iftrue
%\end{verbatim}
%inside a 
%conditional, because if this text is skipped \TeX\ sees
%at least one \verb.\if. to be matched. 
\TeX's matching of \verb.\if., \verb.\else., and \verb.\fi.
is easily upset. For instance, \TeXbook\ warns you that
you should not say
\begin{verbatim}
 \let\ifabc=\iftrue
\end{verbatim}
inside a 
conditional, because if this text is skipped \TeX\ sees
at least one \verb.\if. to be matched. 

%The reason for this is that when \TeX\ is skipping
%it recognizes all \cs{if...}, \cs{or}, \cs{else}, and \cs{fi}
%tokens, and everything that has been declared a synonym of
%such a token by \cs{let}. In \verb>\let\ifabc=\iftrue>
%\TeX\ will therefore at least see the \cs{iftrue} as
%the opening of a conditional, and, if the current meaning
%of \cs{ifabc} was for instance \cs{iffalse}, it will also
%be considered as the opening of a conditional statement.
The reason for this is that when \TeX\ is skipping
it recognizes all \cs{if...}, \cs{or}, \cs{else}, and \cs{fi}
tokens, and everything that has been declared a synonym of
such a token by \cs{let}. In \verb>\let\ifabc=\iftrue>
\TeX\ will therefore at least see the \cs{iftrue} as
the opening of a conditional, and, if the current meaning
of \cs{ifabc} was for instance \cs{iffalse}, it will also
be considered as the opening of a conditional statement.

%As another example, if
%\begin{verbatim}
% \csname if\sometest\endcsname \someaction \fi
%\end{verbatim}
%is skipped as part of conditional text,
%the \verb.\fi. will unintentionally close the
%outer conditional.
As another example, if
\begin{verbatim}
 \csname if\sometest\endcsname \someaction \fi
\end{verbatim}
is skipped as part of conditional text,
the \verb.\fi. will unintentionally close the
outer conditional.

%It does not help to enclose such potentially dangerous
%constructs inside a group, because grouping is independent of
%conditional structure. Burying such commands inside macros is
%the safest approach.
It does not help to enclose such potentially dangerous
constructs inside a group, because grouping is independent of
conditional structure. Burying such commands inside macros is
the safest approach.

%Sometimes another solution is possible, however.
%The \cs{loop} macro of plain \TeX\ (see page~\pageref{loop:ex})
%is used as
%\begin{verbatim}
%\loop ... \if ... \repeat
%\end{verbatim}
%where the \cs{repeat} is not an actually executable
%command, but is merely a delimiter:
%\begin{verbatim}
%\def\loop#1\repeat{ ... }
%\end{verbatim}
%Therefore,
%by declaring
%\begin{verbatim}
%\let\repeat\fi
%\end{verbatim}
%the \cs{repeat} balances the \cs{if...} that terminates
%the loop, and it becomes possible to have loops in
%skipped conditional text.
Sometimes another solution is possible, however.
The \cs{loop} macro of plain \TeX\ (see page~\pageref{loop:ex})
is used as
\begin{verbatim}
\loop ... \if ... \repeat
\end{verbatim}
where the \cs{repeat} is not an actually executable
command, but is merely a delimiter:
\begin{verbatim}
\def\loop#1\repeat{ ... }
\end{verbatim}
Therefore,
by declaring
\begin{verbatim}
\let\repeat\fi
\end{verbatim}
the \cs{repeat} balances the \cs{if...} that terminates
the loop, and it becomes possible to have loops in
skipped conditional text.

%%\spoint Conditionals and grouping
%\subsection{Conditionals and grouping}
%\spoint Conditionals and grouping
\subsection{Conditionals and grouping}

%It has already been mentioned above that group nesting in \TeX\
%is independent of conditional nesting.
%The reason for this is that conditionals are handled by the
%expansion part of \TeX; in that stage braces are just
%unexpandable tokens that require no special treatment.
%Grouping is only performed in the later stage of execution
%processing.
It has already been mentioned above that group nesting in \TeX\
is independent of conditional nesting.
The reason for this is that conditionals are handled by the
expansion part of \TeX; in that stage braces are just
unexpandable tokens that require no special treatment.
Grouping is only performed in the later stage of execution
processing.

%An example of this independence is now given.
%One may write a macro that yields part of
%a conditional:
%\begin{verbatim}
%\def\elsepart{\else \dosomething \fi}
%\end{verbatim}
%The other way around, the following macros
%yield a left brace and a right brace respectively:
%\begin{verbatim}
%\def\leftbrace{\iftrue{\else}\fi}
%\def\rightbrace{\iffalse{\else}\fi}
%\end{verbatim}
%Note that braces in these definitions are properly nested.
An example of this independence is now given.
One may write a macro that yields part of
a conditional:
\begin{verbatim}
\def\elsepart{\else \dosomething \fi}
\end{verbatim}
The other way around, the following macros
yield a left brace and a right brace respectively:
\begin{verbatim}
\def\leftbrace{\iftrue{\else}\fi}
\def\rightbrace{\iffalse{\else}\fi}
\end{verbatim}
Note that braces in these definitions are properly nested.

%%\spoint A trick
%\subsection{A trick}
%\spoint A trick
\subsection{A trick}

%In some contexts it may be hard to get rid of
%\cs{else} or \cs{fi} tokens in a proper
%manner. The above approach with \cs{expandafter}
%works only if there is a limited number of tokens involved.
%In other cases the following trick may provide a way out:
%\begin{verbatim}
%\def\hop#1\fi{\fi #1}
%\end{verbatim}
%Using this as
%\begin{disp}\verb>\if... \hop >\gr{lots of tokens}\verb>\fi>\end{disp}
%will place the tokens outside the conditional.
%This is for instance used in~\cite{E2}.
In some contexts it may be hard to get rid of
\cs{else} or \cs{fi} tokens in a proper
manner. The above approach with \cs{expandafter}
works only if there is a limited number of tokens involved.
In other cases the following trick may provide a way out:
\begin{verbatim}
\def\hop#1\fi{\fi #1}
\end{verbatim}
Using this as
\begin{disp}\verb>\if... \hop >\gr{lots of tokens}\verb>\fi>\end{disp}
will place the tokens outside the conditional.
This is for instance used in~\cite{E2}.

%As a further example of this sort of trick,
%consider the problem (suggested to me and solved by
%Alan Jeffrey) of implementing a conditional 
%\verb-\ifLessThan#1#2#3#4-
%such that the arguments corresponding to \verb-#3- or
%\verb-#4- result, depending on whether \verb-#1- is
%less than \verb-#2- or not. 
As a further example of this sort of trick,
consider the problem (suggested to me and solved by
Alan Jeffrey) of implementing a conditional 
\verb-\ifLessThan#1#2#3#4-
such that the arguments corresponding to \verb-#3- or
\verb-#4- result, depending on whether \verb-#1- is
less than \verb-#2- or not. 

%The problem here is how to get rid of the \cs{else} and the~\cs{fi}.
%The \ldash or at least, one \rdash solution is to scoop them up
%as delimiters for macros:
%\begin{verbatim}
%\def\ifLessThan#1#2{\ifnum#1<#2\relax\taketrue \else \takefalse \fi}
%\def\takefalse\fi#1#2{\fi#2}
%\def\taketrue\else\takefalse\fi#1#2{\fi#1}
%\end{verbatim}
%Note that \cs{ifLessThan} has only two parameters
%(the things to be tested); however, its
%result is a macro that
%chooses between the next two arguments.
The problem here is how to get rid of the \cs{else} and the~\cs{fi}.
The \ldash or at least, one \rdash solution is to scoop them up
as delimiters for macros:
\begin{verbatim}
\def\ifLessThan#1#2{\ifnum#1<#2\relax\taketrue \else \takefalse \fi}
\def\takefalse\fi#1#2{\fi#2}
\def\taketrue\else\takefalse\fi#1#2{\fi#1}
\end{verbatim}
Note that \cs{ifLessThan} has only two parameters
(the things to be tested); however, its
result is a macro that
chooses between the next two arguments.

%%\spoint More examples of expansion in conditionals
%\subsection{More examples of expansion in conditionals}
%\spoint More examples of expansion in conditionals
\subsection{More examples of expansion in conditionals}

%Above, the macro \cs{ifEqString} was given
%\alt
%that compares two strings:
%\howto Compare two strings\par
%\begin{verbatim}
%\def\ifEqString#1#2%
%   {\def\csa{#1}\def\csb{#2}\ifx\csa\csb }
%\end{verbatim}
%However, this macro relies on \cs{def},  which is not an
%expandable command. If we need a string tester that will
%work, for instance, inside an  \cs{edef}, we need some
%more ingenuity (this solution was taken from~\cite{E2}).
%The basic principle of this solution is to compare the strings
%one character at a time. Macro delimiting by \cs{fi} is used;
%this was explained above.
Above, the macro \cs{ifEqString} was given
\alt
that compares two strings:
\howto Compare two strings\par
\begin{verbatim}
\def\ifEqString#1#2%
   {\def\csa{#1}\def\csb{#2}\ifx\csa\csb }
\end{verbatim}
However, this macro relies on \cs{def},  which is not an
expandable command. If we need a string tester that will
work, for instance, inside an  \cs{edef}, we need some
more ingenuity (this solution was taken from~\cite{E2}).
The basic principle of this solution is to compare the strings
one character at a time. Macro delimiting by \cs{fi} is used;
this was explained above.

%First of all, the \cs{ifEqString} call is replaced by a
%sequence \verb>\ifAllChars ...\Are ...\TheSame>, and both
%strings are delimited by a dollar sign, which is not supposed
%to appear in the strings themselves.
%\begin{verbatim}
%\def\ifEqString
%    #1#2{\ifAllChars#1$\Are#2$\TheSame}
%\end{verbatim}
%The test for equality of characters first determines
%whether either string has ended. If both have ended, the original
%strings were equal; if only one has ended, they were of unequal
%length, hence unequal. If neither string has ended, we test
%whether the first characters are equal, and if so, we make a recursive
%call to test the remainder of the string.
%\begin{verbatim}
%\def\ifAllChars#1#2\Are#3#4\TheSame
%   {\if#1$\if#3$\say{true}%
%          \else \say{false}\fi
%    \else \if#1#3\ifRest#2\TheSame#4\else
%                 \say{false}\fi\fi}
%\def\ifRest#1\TheSame#2\else#3\fi\fi
%   {\fi\fi \ifAllChars#1\Are#2\TheSame}
%\end{verbatim}
%The \cs{say} macro is supposed to  give \cs{iftrue} for
%\verb>\say{true}> and \cs{iffalse} for \verb>\say{false}>.
%Observing that all  calls to this macro occur two conditionals deep,
%we use the `hop' trick explained above as follows.
%\begin{verbatim}
%\def\say#1#2\fi\fi
%   {\fi\fi\csname if#1\endcsname}
%\end{verbatim}
First of all, the \cs{ifEqString} call is replaced by a
sequence \verb>\ifAllChars ...\Are ...\TheSame>, and both
strings are delimited by a dollar sign, which is not supposed
to appear in the strings themselves.
\begin{verbatim}
\def\ifEqString
    #1#2{\ifAllChars#1$\Are#2$\TheSame}
\end{verbatim}
The test for equality of characters first determines
whether either string has ended. If both have ended, the original
strings were equal; if only one has ended, they were of unequal
length, hence unequal. If neither string has ended, we test
whether the first characters are equal, and if so, we make a recursive
call to test the remainder of the string.
\begin{verbatim}
\def\ifAllChars#1#2\Are#3#4\TheSame
   {\if#1$\if#3$\say{true}%
          \else \say{false}\fi
    \else \if#1#3\ifRest#2\TheSame#4\else
                 \say{false}\fi\fi}
\def\ifRest#1\TheSame#2\else#3\fi\fi
   {\fi\fi \ifAllChars#1\Are#2\TheSame}
\end{verbatim}
The \cs{say} macro is supposed to  give \cs{iftrue} for
\verb>\say{true}> and \cs{iffalse} for \verb>\say{false}>.
Observing that all  calls to this macro occur two conditionals deep,
we use the `hop' trick explained above as follows.
\begin{verbatim}
\def\say#1#2\fi\fi
   {\fi\fi\csname if#1\endcsname}
\end{verbatim}

%Similar to the above example, let us write a macro
%that will test lexicographic (`dictionary') precedence
%of two strings:
%\howto Compare two strings lexicographically\par
%\begin{verbatim}
%\let\ex=\expandafter
%\def\ifbefore
%    #1#2{\ifallchars#1$\are#2$\before}
%\def\ifallchars#1#2\are#3#4\before
%   {\if#1$\say{true\ex}\else
%     \if#3$\say{false\ex\ex\ex}\else
%      \ifnum`#1>`#3 \say{false%
%         \ex\ex\ex\ex\ex\ex\ex}\else
%       \ifnum`#1<`#3 \say{true%
%         \ex\ex\ex\ex\ex\ex\ex
%         \ex\ex\ex\ex\ex\ex\ex\ex}\else
%       \ifrest#2\before#4\fi\fi\fi\fi}
%\def\ifrest#1\before#2\fi\fi\fi\fi
%   {\fi\fi\fi\fi
%    \ifallchars#1\are#2\before}
%\def\say#1{\csname if#1\endcsname}
%\end{verbatim}
%In this macro a slightly
%different implementation of \cs{say} is used.
Similar to the above example, let us write a macro
that will test lexicographic (`dictionary') precedence
of two strings:
\howto Compare two strings lexicographically\par
\begin{verbatim}
\let\ex=\expandafter
\def\ifbefore
    #1#2{\ifallchars#1$\are#2$\before}
\def\ifallchars#1#2\are#3#4\before
   {\if#1$\say{true\ex}\else
     \if#3$\say{false\ex\ex\ex}\else
      \ifnum`#1>`#3 \say{false%
         \ex\ex\ex\ex\ex\ex\ex}\else
       \ifnum`#1<`#3 \say{true%
         \ex\ex\ex\ex\ex\ex\ex
         \ex\ex\ex\ex\ex\ex\ex\ex}\else
       \ifrest#2\before#4\fi\fi\fi\fi}
\def\ifrest#1\before#2\fi\fi\fi\fi
   {\fi\fi\fi\fi
    \ifallchars#1\are#2\before}
\def\say#1{\csname if#1\endcsname}
\end{verbatim}
In this macro a slightly
different implementation of \cs{say} is used.

%Simplified, a call to \cs{ifbefore} will eventually lead to a situation
%that looks (in the `true' case) like
%\begin{verbatim}
%\ifbefore{...}{...}
%       \if... %% some comparison that turns out true
%          \csname iftrue\expandafter\endcsname
%       \else .... \fi
%   ... %% commands for the `before' case
%\else
%   ... %% commands for the `not-before' case
%\fi
%\end{verbatim}
%When the comparison has turned out true, \TeX\ will start processing
%the \gr{true text}, and make a mental note to remove any
%\verb>\else ... \fi> part once an \cs{else} token is seen.
%Thus, the sequence
%\begin{verbatim}
%\csname iftrue\expandafter\endcsname \else ... \fi
%\end{verbatim}
%is replaced by
%\begin{verbatim}
% \csname iftrue\endcsname
%\end{verbatim}
%as the \cs{else} is seen while \TeX\ is still processing
%\verb>\csname...\endcsname>.
Simplified, a call to \cs{ifbefore} will eventually lead to a situation
that looks (in the `true' case) like
\begin{verbatim}
\ifbefore{...}{...}
       \if... %% some comparison that turns out true
          \csname iftrue\expandafter\endcsname
       \else .... \fi
   ... %% commands for the `before' case
\else
   ... %% commands for the `not-before' case
\fi
\end{verbatim}
When the comparison has turned out true, \TeX\ will start processing
the \gr{true text}, and make a mental note to remove any
\verb>\else ... \fi> part once an \cs{else} token is seen.
Thus, the sequence
\begin{verbatim}
\csname iftrue\expandafter\endcsname \else ... \fi
\end{verbatim}
is replaced by
\begin{verbatim}
 \csname iftrue\endcsname
\end{verbatim}
as the \cs{else} is seen while \TeX\ is still processing
\verb>\csname...\endcsname>.

%Calls to \cs{say} occur inside nested conditionals, so
%the number of \cs{expandafter} commands necessary may be 
%\alt
%larger than~1: for level two it is~3, for level three
%it is~7, and for level~4 it is 15. Slightly more compact
%implementations of this macro do exist.
Calls to \cs{say} occur inside nested conditionals, so
the number of \cs{expandafter} commands necessary may be 
\alt
larger than~1: for level two it is~3, for level three
it is~7, and for level~4 it is 15. Slightly more compact
implementations of this macro do exist.

%\endofchapter
%%%%% end of input file [ifelsefi]
\endofchapter
%%%% end of input file [ifelsefi]

\end{document}
