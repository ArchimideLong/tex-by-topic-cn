% -*- coding: utf-8 -*-
\documentclass[twoside,letterpaper,openright]{rapport3}

% -*- coding: utf-8 -*-

\usepackage[b5paper,text={5in,8in},centering]{geometry}

\usepackage[CJKchecksingle]{xeCJK}
\setmainfont[Mapping=tex-text]{TeX Gyre Schola}
%\setsansfont{URW Gothic L Book}
%\setmonofont{Nimbus Mono L}
\setCJKmainfont[BoldFont=FandolHei,ItalicFont=FandolKai]{FandolSong}
\setCJKsansfont{FandolHei}
\setCJKmonofont{FandolFang}
\xeCJKsetup{PunctStyle = kaiming}

\linespread{1.25}
\setlength{\parindent}{2em}

\usepackage{xcolor}
\definecolor{myblue}{rgb}{0,0.2,0.6}

\usepackage{titlesec}
\titleformat{\chapter}
    {\normalfont\Huge\sffamily\color{myblue}}
    {第\thechapter 章}
    {1em}
    {}
%\titlespacing{\chapter}{0pt}{50pt}{40pt}
\titleformat{\section}
    {\normalfont\Large\sffamily\color{myblue}}
    {\thesection}
    {1em}
    {}
%\titlespacing{\section}{0pt}{3.5ex plus 1ex minus .2ex}{2.3ex plus .2ex}
\titleformat{\subsection}
    {\normalfont\large\sffamily\color{myblue}}
    {\thesubsection}
    {1em}
    {}
%\titlespacing{\subsection}{0pt}{3.25ex plus 1ex minus .2ex}{1.5ex plus .2ex}
%
\newpagestyle{special}[\small\sffamily]{
  \headrule
  \sethead[\usepage][][\chaptertitle]
  {\chaptertitle}{}{\usepage}}
\newpagestyle{main}[\small\sffamily]{
  \headrule
  \sethead[\usepage][][第\thechapter 章\quad\chaptertitle]
  {\thesection\quad\sectiontitle}{}{\usepage}}

\usepackage{titletoc}
%\setcounter{tocdepth}{1}
%\titlecontents{标题层次}[左间距]{上间距和整体格式}{标题序号}{标题内容}{指引线和页码}[下间距]
\titlecontents{chapter}[1.5em]{\vspace{.5em}\bfseries\sffamily}{\color{myblue}\contentslabel{1.5em}}{}
    {\titlerule*[20pt]{$\cdot$}\contentspage}[]
\titlecontents{section}[4.5em]{\sffamily}{\color{myblue}\contentslabel{3em}}{}
    {\titlerule*[20pt]{$\cdot$}\contentspage}[]
%\titlecontents{subsection}[8.5em]{\sffamily}{\contentslabel{4em}}{}
%    {\titlerule*[20pt]{$\cdot$}\contentspage}

\usepackage{enumitem}
\setlist{topsep=2pt,itemsep=2pt,parsep=1pt,leftmargin=\parindent}

\usepackage{fancyvrb}
\DefineVerbatimEnvironment{verbatim}{Verbatim}
  {xleftmargin=2em,baselinestretch=1,formatcom=\color{teal}\upshape}

\usepackage{etoolbox}
\makeatletter
\preto{\FV@ListVSpace}{\topsep=2pt \partopsep=0pt }
\makeatother

\usepackage[colorlinks,plainpages,pagebackref]{hyperref}
\hypersetup{
   pdfstartview={FitH},
   citecolor=teal,
   linkcolor=myblue,
   urlcolor=black,
   bookmarksnumbered
}

\usepackage{comment,makeidx,multicol}

%\usepackage{german}
%% german
%\righthyphenmin=3
%\mdqoff
%\captionsenglish
\usepackage[english]{babel}
{\catcode`"=13 \gdef"#1{\ifx#1"\discretionary{}{}{}\fi\relax}}
\def\mdqon{\catcode`"=13\relax}
\def\mdqoff{\catcode`"=12\relax}
\makeindex
\hyphenation{ex-em-pli-fies}

\newdimen\tempdima \newdimen\tempdimb

% these are fine
\def\handbreak{\\ \message{^^JManual break!!!!^^J}}
\def\nl{\protect\\}\def\n#1{{\tt #1}}
\def\cs#1{\texorpdfstring{{\tt\char`\\#1}}{\textbackslash#1}} %\def\cs#1{{\tt\char`\\#1}}
\let\csc\cs
\def\lb{{\tt\char`\{}}\def\rb{{\tt\char`\}}}
\def\gr#1{\texorpdfstring{$\langle$#1$\rangle$}{<#1>}} %\def\gr#1{$\langle$#1$\rangle$}
\def\key#1{{\tt#1}}
\def\alt{}\def\altt{}%this way in manstijl
\def\ldash{\unskip\ --\nobreak\ \ignorespaces}
\def\rdash{\unskip\nobreak\ --\ \ignorespaces}
% check these
\def\hex{{\tt"}}
\def\ascii{{\sc ascii}}
\def\ebcdic{{\sc ebcdic}}
\def\IniTeX{Ini\TeX}\def\LamsTeX{LAMS\TeX}\def\VirTeX{Vir\TeX}
\def\AmsTeX{Ams\TeX}
\def\TeXbook{the \TeX\ book}\def\web{{\sc web}}
% needs major thinking
\newenvironment{myquote}{\list{}{%
    \topsep=2pt \partopsep=0pt%
    \leftmargin=\parindent \rightmargin=\parindent
    }\item[]}{\endlist}
\newenvironment{disp}{\begin{myquote}}{\end{myquote}}
\newenvironment{Disp}{\begin{myquote}}{\end{myquote}}
\newenvironment{tdisp}{\begin{myquote}}{\end{myquote}}
\newenvironment{example}{\begin{myquote}\noindent\itshape 例子:}{\end{myquote}}
\newenvironment{inventory}{\begin{description}\raggedright}{\end{description}}
\newenvironment{glossinventory}{\begin{description}}{\end{description}}
\def\gram#1{\gr{#1}}%???
%
% index
%
\def\indexterm#1{\emph{#1}\index{#1}}
\def\indextermsub#1#2{\emph{#1 #2}\index{#1!#2}}
\def\indextermbus#1#2{\emph{#1 #2}\index{#2!#1}}
\def\term#1\par{\index{#1}}
\def\howto#1\par{}
\def\cstoidx#1\par{\index{#1@\cs{#1}@}}
\def\thecstoidx#1\par{\index{#1@\protect\csname #1\endcsname}}
\def\thecstoidxsub#1#2{\index{#1, #2@\protect\csname #1\endcsname, #2}\ignorespaces}
\def\csterm#1\par{\cstoidx #1\par\cs{#1}}
\def\csidx#1{\cstoidx #1\par\cs{#1}}

\def\tmc{\tracingmacros=2 \tracingcommands\tracingmacros}

%%%%%%%%%%%%%%%%%%%
\makeatletter
\def\snugbox{\hbox\bgroup\setbox\z@\vbox\bgroup
    \leftskip\z@
    \bgroup\aftergroup\make@snug
    \let\next=}
\def\make@snug{\par\sn@gify\egroup \box\z@\egroup}
\def\sn@gify
   {\skip\z@=\lastskip \unskip
    \advance\skip\z@\lastskip \unskip
    \unpenalty
    \setbox\z@\lastbox
    \ifvoid\z@ \nointerlineskip \else {\sn@gify} \fi
    \hbox{\unhbox\z@}\nointerlineskip
    \vskip\skip\z@
    }

\newdimen\fbh \fbh=60pt % dimension for easy scaling:
\newdimen\fbw \fbw=60pt % height and width of character box

\newdimen\dh \newdimen\dw % height and width of current character box
\newdimen\lh % height of previous character box
\newdimen\lw \lw=.4pt % line weight, instead of default .4pt

\def\hdotfill{\noindent
    \leaders\hbox{\vrule width 1pt height\lw
                  \kern4pt
                  \vrule width.5pt height\lw}\hfill\hbox{}
    \par}
\def\hlinefill{\noindent
    \leaders\hbox{\vrule width 5.5pt height\lw         }\hfill\hbox{}
    \par}
\def\stippel{$\qquad\qquad\qquad\qquad$}
\makeatother
%%%%%%%%%%%%%%%%%%%

%\def\SansSerif{\Typeface:macHelvetica }
%\def\SerifFont{\Typeface:macTimes }
%\def\SansSerif{\Typeface:bsGillSans }
%\def\SerifFont{\Typeface:bsBaskerville }
\let\SansSerif\relax \def\italic{\it}
\let\SerifFont\relax \def\MainFont{\rm}
\let\SansSerif\relax
\let\SerifFont\relax
\let\PopIndentLevel\relax \let\PushIndentLevel\relax
\let\ToVerso\relax \let\ToRecto\relax

%\def\stop@command@suffix{stop}
%\let\PopListLevel\PopIndentLevel
%\let\FlushRight\relax
%\let\flushright\FlushRight
%\let\SetListIndent\LevelIndent
%\def\awp{\ifhmode\vadjust{\penalty-10000 }\else
%    \penalty-10000 \fi}
\let\awp\relax
\let\PopIndentLevel\relax \let\PopListLevel\relax

\showboxdepth=-1

%\input figs
\def\endofchapter{\vfill\noindent}

\setcounter{chapter}{28}

\begin{document}

%\chapter{Insertions}\label{insert}
\chapter{Insertions}\label{insert}

%Insertions are \TeX's way of handling floating information.
%\TeX's page builder calculates what insertions and how many
%of them will fit on the page; these insertion items are then
%placed in insertion boxes which are to be handled by the
%output routine.
Insertions are \TeX's way of handling floating information.
\TeX's page builder calculates what insertions and how many
of them will fit on the page; these insertion items are then
placed in insertion boxes which are to be handled by the
output routine.


%\label{cschap:insert}\label{cschap:newinsert}\label{cschap:insertpenalties}\label{cschap:floatingpenalty}\label{cschap:holdinginserts}\label{cschap:footins}\label{cschap:topins}\label{cschap:topinsert}\label{cschap:pageinsert}\label{cschap:midinsert}\label{cschap:endinsert}
%\begin{inventory}
%\item [\cs{insert}] 
%      Start an insertion item.
\label{cschap:insert}\label{cschap:newinsert}\label{cschap:insertpenalties}\label{cschap:floatingpenalty}\label{cschap:holdinginserts}\label{cschap:footins}\label{cschap:topins}\label{cschap:topinsert}\label{cschap:pageinsert}\label{cschap:midinsert}\label{cschap:endinsert}
\begin{inventory}
\item [\cs{insert}] 
      Start an insertion item.

%\item [\cs{newinsert}] 
%      Allocate a new insertion class. 
\item [\cs{newinsert}] 
      Allocate a new insertion class. 

%\item [\cs{insertpenalties}] 
%      Total of penalties for split insertions.
%      Inside the output routine, the number of held-over insertions.
\item [\cs{insertpenalties}] 
      Total of penalties for split insertions.
      Inside the output routine, the number of held-over insertions.

%\item [\cs{floatingpenalty}] 
%      Penalty added when an insertion is split.
\item [\cs{floatingpenalty}] 
      Penalty added when an insertion is split.

%\item [\cs{holdinginserts}]
%      (\TeX3 only)
%      If this is positive, insertions are not placed in their boxes 
%      at output time.
\item [\cs{holdinginserts}]
      (\TeX3 only)
      If this is positive, insertions are not placed in their boxes 
      at output time.

%\item [\cs{footins}]
%      Number of the footnote insertion class in plain \TeX.
\item [\cs{footins}]
      Number of the footnote insertion class in plain \TeX.

%\item [\cs{topins}]
%      Number of the top insertion class.
\item [\cs{topins}]
      Number of the top insertion class.

%\item [\cs{topinsert}]
%      Plain \TeX\ macro to start a top insert.
\item [\cs{topinsert}]
      Plain \TeX\ macro to start a top insert.

%\item [\cs{pageinsert}]
%      Plain \TeX\ macro to start an insert that will take
%      up a whole page.
\item [\cs{pageinsert}]
      Plain \TeX\ macro to start an insert that will take
      up a whole page.

%\item [\cs{midinsert}]
%      Plain \TeX\ macro that places its argument if there is space,
%      and converts it into a top insert otherwise.
\item [\cs{midinsert}]
      Plain \TeX\ macro that places its argument if there is space,
      and converts it into a top insert otherwise.

%\item [\cs{endinsert}]
%      Plain \TeX\ macro to wind up an insertion item
%      that started with \cs{topinsert}, \cs{midinsert},
%      or \cs{pageinsert}.
\item [\cs{endinsert}]
      Plain \TeX\ macro to wind up an insertion item
      that started with \cs{topinsert}, \cs{midinsert},
      or \cs{pageinsert}.

%\end{inventory}
\end{inventory}


%%\point Insertion items
%\section{Insertion items}
%\point Insertion items
\section{Insertion items}

%Floating information, items that are not bound to a fixed
%place in the vertical list, such as figures or footnotes,
%are handled by \indexterm{insertions}.
%The treatment of insertions is a strange interplay between the
%user, \TeX's internal workings, and the output routine.
%First the user specifies an insertion, which is
%a certain amount of vertical material; 
%then \TeX's page builder decides what insertions should go
%on the current page and puts these insertions in insertion boxes;
%finally, the output routine has to do something with these boxes.
Floating information, items that are not bound to a fixed
place in the vertical list, such as figures or footnotes,
are handled by \indexterm{insertions}.
The treatment of insertions is a strange interplay between the
user, \TeX's internal workings, and the output routine.
First the user specifies an insertion, which is
a certain amount of vertical material; 
then \TeX's page builder decides what insertions should go
on the current page and puts these insertions in insertion boxes;
finally, the output routine has to do something with these boxes.

%An insertion item looks like
%\cstoidx insert\par
%\begin{disp}\cs{insert}\gr{8-bit number}\lb\gr{vertical mode material}\rb
%\end{disp} where the 8-bit number should not be~255,
%because \cs{box255} is used by \TeX\ for passing the page to the output
%routine.
An insertion item looks like
\cstoidx insert\par
\begin{disp}\cs{insert}\gr{8-bit number}\lb\gr{vertical mode material}\rb
\end{disp} where the 8-bit number should not be~255,
because \cs{box255} is used by \TeX\ for passing the page to the output
routine.

%The braces around the vertical mode material in an insertion
%item can be implicit; they imply a new level of grouping.
%The vertical mode material is processed in internal
%vertical mode.
The braces around the vertical mode material in an insertion
item can be implicit; they imply a new level of grouping.
The vertical mode material is processed in internal
vertical mode.

%Values of \cs{splittopskip}, \cs{splitmaxdepth}, 
%and \cs{floatingpenalty} are relevant for split insertions
%(see below); the values that are current just before
%the end of the group are used.
Values of \cs{splittopskip}, \cs{splitmaxdepth}, 
and \cs{floatingpenalty} are relevant for split insertions
(see below); the values that are current just before
the end of the group are used.

%Insertion items can appear in vertical mode, horizontal
%mode, and math mode. For the latter two modes they have to
%migrate to the surrounding vertical list
%(see page~\pageref{migrate}).
%After an insertion item is put on  the vertical list the
%page builder is exercised.
Insertion items can appear in vertical mode, horizontal
mode, and math mode. For the latter two modes they have to
migrate to the surrounding vertical list
(see page~\pageref{migrate}).
After an insertion item is put on  the vertical list the
page builder is exercised.


%%\point Insertion class declaration
%\section{Insertion class declaration}
%\point Insertion class declaration
\section{Insertion class declaration}

%In the plain format 
%the  number for a new insertion class
%is allocated by \csidx{newinsert}:
%\begin{verbatim}
%\newinsert\myinsert % new insertion class
%\end{verbatim}
%which uses \cs{chardef} to assign a number to the control
%sequence.
In the plain format 
the  number for a new insertion class
is allocated by \csidx{newinsert}:
\begin{verbatim}
\newinsert\myinsert % new insertion class
\end{verbatim}
which uses \cs{chardef} to assign a number to the control
sequence.

%Insertion classes are allocated numbering from 254 downward.
%As box~255 is used for output, this allocation scheme leaves
%\cs{skip255}, \cs{dimen255}, and \cs{count255}
%free for scratch use.
Insertion classes are allocated numbering from 254 downward.
As box~255 is used for output, this allocation scheme leaves
\cs{skip255}, \cs{dimen255}, and \cs{count255}
free for scratch use.

%%\point Insertion parameters
%\section{Insertion parameters}
%\point Insertion parameters
\section{Insertion parameters}

%For each insertion class~$n$ four registers are allocated:
%\begin{itemize}
%\item \cs{box}$\,n$ When the output routine is active this
% box contains the insertion items of class~$n$ that should
% be placed on the current page.
%\item \cs{dimen}$\,n$ This is the maximum space allotted for
% insertions of class~$n$ per page. If this amount would
% be exceeded \TeX\ will split insertions.
%\item \cs{skip}$\,n$ Glue of this size is added the first
% time an insertion item of class~$n$ is added to the
% current page. This is useful for such phenomena as a rule
% separating the footnotes from the text of the page.
%\item \cs{count}$\,n$ Each insertion item is a vertical list,
% so it has a certain height. However, the effective height,
% the amount of influence it has on the text height of the
% page, may differ from this real height.
% The value of \cs{count}$\,n$
% is then 1000 times the factor by which the height should
% be multiplied to obtain the effective height.
% 
% Consider the following examples:
% 
%\begin{itemize}\item Marginal notes do not affect
% the text height, so the factor should be~0. \item Footnotes
% set in double column mode affect the page by half of their height:
% the count value should by~500. \item Conversely, footnotes
% set at page width underneath a page in double column mode
% affect both columns, so \ldash provided that the double column mode
% is implemented by applying \cs{vsplit} to a double-height column \rdash 
% the count value should be~2000.\end{itemize}
%\end{itemize}
For each insertion class~$n$ four registers are allocated:
\begin{itemize}
\item \cs{box}$\,n$ When the output routine is active this
 box contains the insertion items of class~$n$ that should
 be placed on the current page.
\item \cs{dimen}$\,n$ This is the maximum space allotted for
 insertions of class~$n$ per page. If this amount would
 be exceeded \TeX\ will split insertions.
\item \cs{skip}$\,n$ Glue of this size is added the first
 time an insertion item of class~$n$ is added to the
 current page. This is useful for such phenomena as a rule
 separating the footnotes from the text of the page.
\item \cs{count}$\,n$ Each insertion item is a vertical list,
 so it has a certain height. However, the effective height,
 the amount of influence it has on the text height of the
 page, may differ from this real height.
 The value of \cs{count}$\,n$
 is then 1000 times the factor by which the height should
 be multiplied to obtain the effective height.
 
 Consider the following examples:
 
\begin{itemize}\item Marginal notes do not affect
 the text height, so the factor should be~0. \item Footnotes
 set in double column mode affect the page by half of their height:
 the count value should by~500. \item Conversely, footnotes
 set at page width underneath a page in double column mode
 affect both columns, so \ldash provided that the double column mode
 is implemented by applying \cs{vsplit} to a double-height column \rdash 
 the count value should be~2000.\end{itemize}
\end{itemize}

%%\point Moving insertion items from the contributions list
%\section{Moving insertion items from the contributions list}
%\point Moving insertion items from the contributions list
\section{Moving insertion items from the contributions list}

%The most complicated issue with insertions is the algorithm
%that adds insertion items to the main vertical list,
%and calculates breakpoints if necessary.
The most complicated issue with insertions is the algorithm
that adds insertion items to the main vertical list,
and calculates breakpoints if necessary.

%\TeX\ never changes the \cs{vsize}, but it diminishes the
%\csidx{pagegoal} by the (effective) heights of the insertion
%items that will appear before a page break. Thus the output
%routine will receive a \cs{box255} that has height \cs{pagegoal},
%not necessarily \cs{vsize}.
\TeX\ never changes the \cs{vsize}, but it diminishes the
\csidx{pagegoal} by the (effective) heights of the insertion
items that will appear before a page break. Thus the output
routine will receive a \cs{box255} that has height \cs{pagegoal},
not necessarily \cs{vsize}.

%\begin{enumerate}
%\item When the first insertion of a certain class $n$ occurs
%  on the current page \TeX\ has to account for the quantity
%  \cs{skip}$\,n$. This step is executed only if no earlier
%  insertion item of this class occurs on the vertical list
%  \ldash this includes insertions that were split \rdash  but \cs{box}$\,n$
%  need not be empty at this time.
%  
%  If \cs{box}$\,n$ is not empty, its height plus depth is multiplied
%  by \cs{count}$\,n/1000$ and the result is subtracted
%  from \cs{pagegoal}. Then the \cs{pagegoal} is diminished
%  by the natural component of \cs{skip}$\,n$. Any stretch and
%  shrink of \cs{skip}$\,n$ are incorporated in \cs{pagestretch}
%  and \cs{pageshrink} respectively.
%\item If there is a split insertion of class $n$ on the page
%  \ldash this case and the previous step in the algorithm are
%  mutually exclusive \rdash  the \csidx{floatingpenalty} is added to
%  \csidx{insertpenalties}. A~split insertion is an insertion item
%  for which a breakpoint has been calculated as it will not
%  fit on the current page in its entirety. Thus the insertion
%  currently under consideration will certainly not wind up 
%  on the current page.
%\item After the preliminary action of the two previous points
%  \TeX\ will place the actual insertion item on the main vertical
%  list, at the end of the current contributions.
%  First it will check whether the item will fit without being split.
%  
%  There are two conditions to be checked:
%\begin{itemize}\item
%  adding the insertion item (plus all previous insertions of that class)
%  to \cs{box}$\,n$ should not let
%  the height plus depth of that box exceed \cs{dimen}$\,n$, and
%  \item either the effective height of the insertion is negative, or
%  \cs{pagetotal} plus \cs{pagedepth} minus \cs{pageshrink}
%  plus the effective size of the insertion should be less than
%  \cs{pagegoal}.\end{itemize}
%  If these conditions are satisfied, \cs{pagegoal} is diminished
%  by the effective size of the insertion item, that is,
%  by the height plus depth, multiplied by \cs{count}$n/1000$.
\begin{enumerate}
\item When the first insertion of a certain class $n$ occurs
  on the current page \TeX\ has to account for the quantity
  \cs{skip}$\,n$. This step is executed only if no earlier
  insertion item of this class occurs on the vertical list
  \ldash this includes insertions that were split \rdash  but \cs{box}$\,n$
  need not be empty at this time.
  
  If \cs{box}$\,n$ is not empty, its height plus depth is multiplied
  by \cs{count}$\,n/1000$ and the result is subtracted
  from \cs{pagegoal}. Then the \cs{pagegoal} is diminished
  by the natural component of \cs{skip}$\,n$. Any stretch and
  shrink of \cs{skip}$\,n$ are incorporated in \cs{pagestretch}
  and \cs{pageshrink} respectively.
\item If there is a split insertion of class $n$ on the page
  \ldash this case and the previous step in the algorithm are
  mutually exclusive \rdash  the \csidx{floatingpenalty} is added to
  \csidx{insertpenalties}. A~split insertion is an insertion item
  for which a breakpoint has been calculated as it will not
  fit on the current page in its entirety. Thus the insertion
  currently under consideration will certainly not wind up 
  on the current page.
\item After the preliminary action of the two previous points
  \TeX\ will place the actual insertion item on the main vertical
  list, at the end of the current contributions.
  First it will check whether the item will fit without being split.
  
  There are two conditions to be checked:
\begin{itemize}\item
  adding the insertion item (plus all previous insertions of that class)
  to \cs{box}$\,n$ should not let
  the height plus depth of that box exceed \cs{dimen}$\,n$, and
  \item either the effective height of the insertion is negative, or
  \cs{pagetotal} plus \cs{pagedepth} minus \cs{pageshrink}
  plus the effective size of the insertion should be less than
  \cs{pagegoal}.\end{itemize}
  If these conditions are satisfied, \cs{pagegoal} is diminished
  by the effective size of the insertion item, that is,
  by the height plus depth, multiplied by \cs{count}$n/1000$.

%\item Insertions that fail on one of the two conditions in the
%  previous step of the algorithm will be considered for splitting.
%  \TeX\ will calculate the size of the maximal portion to 
%  be split off the insertion item, such that
%  
%\begin{enumerate}\item adding this portion
%  together with earlier insertions of this class to \cs{box}$\,n$
%  will not let the size of the box exceed \cs{dimen}$\,n$,
%  and \item the effective size of this portion,
%  added to \cs{pagetotal} plus \cs{pagedepth}, will not
%  exceed \cs{pagegoal}. Note that \cs{pageshrink} is not taken
%  into account this time, as it was in the previous step.
%  \end{enumerate}
%  
%  Once this maximal size to be split off has been determined,
%  \TeX\ locates the least-cost breakpoint in the current 
%  insertion item that will result in a box with a  height
%  that is equal to this maximal size. The penalty associated
%  with this breakpoint is added to \cs{insertpenalties},
%  and \cs{pagegoal} is diminished by the effective height plus
%  depth of the box to be split off the insertion item.
\item Insertions that fail on one of the two conditions in the
  previous step of the algorithm will be considered for splitting.
  \TeX\ will calculate the size of the maximal portion to 
  be split off the insertion item, such that
  
\begin{enumerate}\item adding this portion
  together with earlier insertions of this class to \cs{box}$\,n$
  will not let the size of the box exceed \cs{dimen}$\,n$,
  and \item the effective size of this portion,
  added to \cs{pagetotal} plus \cs{pagedepth}, will not
  exceed \cs{pagegoal}. Note that \cs{pageshrink} is not taken
  into account this time, as it was in the previous step.
  \end{enumerate}
  
  Once this maximal size to be split off has been determined,
  \TeX\ locates the least-cost breakpoint in the current 
  insertion item that will result in a box with a  height
  that is equal to this maximal size. The penalty associated
  with this breakpoint is added to \cs{insertpenalties},
  and \cs{pagegoal} is diminished by the effective height plus
  depth of the box to be split off the insertion item.

%\end{enumerate}
\end{enumerate}



%%\point Insertions in the output routine
%\section{Insertions in the output routine}
%\point Insertions in the output routine
\section{Insertions in the output routine}

%When the output routine comes into action \ldash more precisely:
%when \TeX\ starts processing the tokens in the \cs{output}
%token list \rdash  all insertions that should be placed on the
%current page have been put in their boxes, and
%it is the responsibility of the output routine
%to put them somewhere in the box that is going to be shipped out.
When the output routine comes into action \ldash more precisely:
when \TeX\ starts processing the tokens in the \cs{output}
token list \rdash  all insertions that should be placed on the
current page have been put in their boxes, and
it is the responsibility of the output routine
to put them somewhere in the box that is going to be shipped out.

%\begin{example} The plain \TeX\ output routine
%handles top inserts and footnotes by packaging the following
%sequence:
%\begin{verbatim}
%\ifvoid\topins \else \unvbox\topins \fi
%\pagebody
%\ifvoid\footins \else \unvbox\footins \fi
%\end{verbatim}
%Unboxing the insertion boxes makes the glue on various parts
%of the page stretch or shrink in a uniform manner.
%\end{example}
\begin{example} The plain \TeX\ output routine
handles top inserts and footnotes by packaging the following
sequence:
\begin{verbatim}
\ifvoid\topins \else \unvbox\topins \fi
\pagebody
\ifvoid\footins \else \unvbox\footins \fi
\end{verbatim}
Unboxing the insertion boxes makes the glue on various parts
of the page stretch or shrink in a uniform manner.
\end{example}

%With \TeX3 the insertion mechanism has been extended slightly:
%\handbreak \cstoidx holdinginserts\par
%\thecstoidxsub{TeX}{version 3}
%the parameter \cs{holdinginserts} can be used to specify that
%insertions should not yet be placed in their boxes.
%This is very useful if the output routine wants to
%recalculate the \cs{vsize}, or if the output routine
%is called to do other intermediate calculations instead of
%ejecting a page.
With \TeX3 the insertion mechanism has been extended slightly:
\handbreak \cstoidx holdinginserts\par
\thecstoidxsub{TeX}{version 3}
the parameter \cs{holdinginserts} can be used to specify that
insertions should not yet be placed in their boxes.
This is very useful if the output routine wants to
recalculate the \cs{vsize}, or if the output routine
is called to do other intermediate calculations instead of
ejecting a page.

%During the output routine the parameter
%\csidx{insertpenalties} holds the number of insertion items that
%are being held over for the next page.
%In the plain \TeX\ output routine this is used after the
%last page:
%\begin{verbatim}
%\def\dosupereject{\ifnum\insertpenalties>0 
%    % something is being held over
%  \line{}\kern-\topskip\nobreak\vfill\supereject\fi}
%\end{verbatim}
During the output routine the parameter
\csidx{insertpenalties} holds the number of insertion items that
are being held over for the next page.
In the plain \TeX\ output routine this is used after the
last page:
\begin{verbatim}
\def\dosupereject{\ifnum\insertpenalties>0 
    % something is being held over
  \line{}\kern-\topskip\nobreak\vfill\supereject\fi}
\end{verbatim}

%%\point Plain \TeX\ insertions
%\section{Plain \TeX\ insertions}
%\point Plain \TeX\ insertions
\section{Plain \TeX\ insertions}

%The plain \TeX\ format has only two insertion classes:
%the footnotes and the top inserts.
%The macro \csidx{pageinsert} generates
%top inserts that are stretched to be exactly \cs{vsize} high.
%The \csidx{midinsert} macro tests whether the vertical material
%specified by the user fits on the page; if so, it is placed
%there; if not, it is converted to a top insert.
The plain \TeX\ format has only two insertion classes:
the footnotes and the top inserts.
The macro \csidx{pageinsert} generates
top inserts that are stretched to be exactly \cs{vsize} high.
The \csidx{midinsert} macro tests whether the vertical material
specified by the user fits on the page; if so, it is placed
there; if not, it is converted to a top insert.

%Footnotes are allowed to be split, but once one has been
%split no further footnotes should appear on the current
%page. This effect is attained by setting 
%\begin{verbatim}
%\floatingpenalty=20000
%\end{verbatim} 
%The \cs{floatingpenalty} is added to \cs{insertpenalties}
%if an insertion follows a split insertion of the same 
%class. However, \cs{floatingpenalty}${}>10\,000$ has infinite
%cost, so \TeX\ will take an earlier breakpoint for
%splitting off the page from the vertical list.
Footnotes are allowed to be split, but once one has been
split no further footnotes should appear on the current
page. This effect is attained by setting 
\begin{verbatim}
\floatingpenalty=20000
\end{verbatim} 
The \cs{floatingpenalty} is added to \cs{insertpenalties}
if an insertion follows a split insertion of the same 
class. However, \cs{floatingpenalty}${}>10\,000$ has infinite
cost, so \TeX\ will take an earlier breakpoint for
splitting off the page from the vertical list.

%Top inserts essentially contain only a vertical box
%which holds whatever the user specified. Thus such an insert
%cannot be split. However, the \csidx{endinsert} macro
%puts a \cs{penalty100} on top of the box, so the
%insertion can be split with an empty part before the split.
%The effect is that the whole insertion is carried over to
%the next page. As the \cs{floatingpenalty} for top inserts
%is zero, arbitrarily many of these inserts can be moved forward
%until there is a page with sufficient space.
Top inserts essentially contain only a vertical box
which holds whatever the user specified. Thus such an insert
cannot be split. However, the \csidx{endinsert} macro
puts a \cs{penalty100} on top of the box, so the
insertion can be split with an empty part before the split.
The effect is that the whole insertion is carried over to
the next page. As the \cs{floatingpenalty} for top inserts
is zero, arbitrarily many of these inserts can be moved forward
until there is a page with sufficient space.

%Further examples of insertion macros can be found
%in~\cite{Sal3}.
Further examples of insertion macros can be found
in~\cite{Sal3}.

%%\message{Maybe spaceleft example?}
%\message{Maybe spaceleft example?}


%\endofchapter
%%%%% end of input file [page]
\endofchapter
%%%% end of input file [page]

\end{document}
