% -*- coding: utf-8 -*-

%\chapter{Rules and Leaders}\label{rules}
\chapter{Rules and Leaders}\label{rules}

%Rules and leaders are two ways of getting \TeX\ to draw a line.
%Leaders are more general than rules: they can also fill
%available space with copies of a certain box. This chapter
%explain how rules and leaders work, and how they interact with modes.
Rules and leaders are two ways of getting \TeX\ to draw a line.
Leaders are more general than rules: they can also fill
available space with copies of a certain box. This chapter
explain how rules and leaders work, and how they interact with modes.

%\label{cschap:hrule}\label{cschap:vrule}\label{cschap:leaders}\label{cschap:cleaders}\label{cschap:xleaders}
%\begin{inventory}
%\item [\cs{hrule}] 
%      Rule that spreads in horizontal direction.
\label{cschap:hrule}\label{cschap:vrule}\label{cschap:leaders}\label{cschap:cleaders}\label{cschap:xleaders}
\begin{inventory}
\item [\cs{hrule}] 
      Rule that spreads in horizontal direction.

%\item [\cs{vrule}] 
%      Rule that spreads in vertical direction.
\item [\cs{vrule}] 
      Rule that spreads in vertical direction.

%\item [\cs{leaders}] 
%      Fill a specified amount of space with a rule or copies of box.
\item [\cs{leaders}] 
      Fill a specified amount of space with a rule or copies of box.

%\item [\cs{cleaders}] 
%      Like \verb=\leaders=, but with box leaders 
%      any excess space is split equally before and after the leaders.
\item [\cs{cleaders}] 
      Like \verb=\leaders=, but with box leaders 
      any excess space is split equally before and after the leaders.

%\item [\cs{xleaders}] 
%      Like \verb=\leaders=, but with box leaders any excess space is 
%      spread equally before, after, and between the boxes.
\item [\cs{xleaders}] 
      Like \verb=\leaders=, but with box leaders any excess space is 
      spread equally before, after, and between the boxes.

%\end{inventory}
\end{inventory}

%%\point Rules
%\section{Rules}
%\point Rules
\section{Rules}

%\TeX's rule commands give
%\indexterm{rules}:
%rectangular black patches with horizontal and vertical sides.
%Most of the times, a rule command will give output that
%looks like a rule, but~\hbox{\vrule height 1.5ex width 1.5ex}
%can also be produced by a rule.
\TeX's rule commands give
\indexterm{rules}:
rectangular black patches with horizontal and vertical sides.
Most of the times, a rule command will give output that
looks like a rule, but~\hbox{\vrule height 1.5ex width 1.5ex}
can also be produced by a rule.

%\TeX\ has both horizontal and vertical rules, 
%but the names do not necessarily imply anything about the shape.
%They do, however, imply something about modes:
%an \csidx{hrule} command can only be used in vertical mode,
%and a \csidx{vrule} only in horizontal mode.
%In fact, an \cs{hrule} is a \gr{vertical command}, and a \cs{vrule}
%is a \gr{horizontal command}, so \TeX\ may change
%modes when encountering these commands.
\TeX\ has both horizontal and vertical rules, 
but the names do not necessarily imply anything about the shape.
They do, however, imply something about modes:
an \csidx{hrule} command can only be used in vertical mode,
and a \csidx{vrule} only in horizontal mode.
In fact, an \cs{hrule} is a \gr{vertical command}, and a \cs{vrule}
is a \gr{horizontal command}, so \TeX\ may change
modes when encountering these commands.

%Why then is a \cs{vrule} called a {\em vertical\/} rule?
%The reason is that a \cs{vrule} can expand arbitrarily
%far in the vertical direction: if its height and depth are not
%specified explicitly it will take as much room as its
%surroundings allow\altt.
Why then is a \cs{vrule} called a {\em vertical\/} rule?
The reason is that a \cs{vrule} can expand arbitrarily
far in the vertical direction: if its height and depth are not
specified explicitly it will take as much room as its
surroundings allow\altt.

%\begin{example}
%\begin{verbatim}
%\hbox{\vrule\ text \vrule}
%\end{verbatim}
%looks like \begin{disp}\leavevmode\hbox{\vrule\ text \vrule}\end{disp}
%and
%\begin{verbatim}
%\hbox{\vrule\ A gogo! \vrule}
%\end{verbatim}
%looks like
%\begin{disp}\leavevmode\hbox{\vrule\ A gogo! \vrule}\end{disp}
%\end{example}
\begin{example}
\begin{verbatim}
\hbox{\vrule\ text \vrule}
\end{verbatim}
looks like \begin{disp}\leavevmode\hbox{\vrule\ text \vrule}\end{disp}
and
\begin{verbatim}
\hbox{\vrule\ A gogo! \vrule}
\end{verbatim}
looks like
\begin{disp}\leavevmode\hbox{\vrule\ A gogo! \vrule}\end{disp}
\end{example}

%For the \cs{hrule} command a similar statement is true:
%a horizontal rule can spread to assume the width of
%its surroundings. Thus 
%\begin{verbatim}
%\vbox{\hbox{One line of text}\hrule}
%\end{verbatim}
%looks like
%\begin{disp}\leavevmode\vtop{\hbox{One line of text}\hrule}\end{disp}
For the \cs{hrule} command a similar statement is true:
a horizontal rule can spread to assume the width of
its surroundings. Thus 
\begin{verbatim}
\vbox{\hbox{One line of text}\hrule}
\end{verbatim}
looks like
\begin{disp}\leavevmode\vtop{\hbox{One line of text}\hrule}\end{disp}


%%\spoint Rule dimensions
%\subsection{Rule dimensions}
%\spoint Rule dimensions
\subsection{Rule dimensions}

%Horizontal and vertical rules have a default thickness:
%\begin{Disp} \cs{hrule}\quad is the same as\quad \verb-\hrule height.4pt depth0pt-
%\end{Disp}
%and 
%\begin{Disp} \cs{vrule}\quad is the same as\quad \verb-\vrule width.4pt- \end{Disp}
%and if the remaining dimension remains unspecified, the rule
%extends in that direction to fill the enclosing box.
Horizontal and vertical rules have a default thickness:
\begin{Disp} \cs{hrule}\quad is the same as\quad \verb-\hrule height.4pt depth0pt-
\end{Disp}
and 
\begin{Disp} \cs{vrule}\quad is the same as\quad \verb-\vrule width.4pt- \end{Disp}
and if the remaining dimension remains unspecified, the rule
extends in that direction to fill the enclosing box.

%Here is the formal specification of how to indicate rule sizes:
%\begin{disp}\gr{vertical rule} $\longrightarrow$ 
%                        \cs{vrule}\gr{rule specification}\nl
%     \gr{horizontal rule} $\longrightarrow$
%                        \cs{hrule}\gr{rule specification}\nl
%     \gr{rule specification} $\longrightarrow$
%                        \gr{optional spaces} \nl \indent$|$
%                        \gr{rule dimensions}\gr{rule specification}\nl
%     \gr{rule dimension} $\longrightarrow$
%                        \n{width}\gr{dimen} $|$ \n{height}\gr{dimen} $|$
%                        \n{depth}\gr{dimen}
%     \end{disp}
%If a rule dimension is specified twice, the second instance
%takes precedence over the first. This makes it possible
%to override the default dimensions. For instance,
%after
%\alt
%\howto Change the default dimensions of rules\par
%\begin{verbatim}
%\let\xhrule\hrule  \def\hrule{\xhrule height .8pt}
%\end{verbatim}
%the macro \cs{hrule} gives a horizontal rule
%of double the original height, and it is still possible
%with
%\begin{verbatim}
%\hrule height 2pt
%\end{verbatim}
%to specify other heights.
Here is the formal specification of how to indicate rule sizes:
\begin{disp}\gr{vertical rule} $\longrightarrow$ 
                        \cs{vrule}\gr{rule specification}\nl
     \gr{horizontal rule} $\longrightarrow$
                        \cs{hrule}\gr{rule specification}\nl
     \gr{rule specification} $\longrightarrow$
                        \gr{optional spaces} \nl \indent$|$
                        \gr{rule dimensions}\gr{rule specification}\nl
     \gr{rule dimension} $\longrightarrow$
                        \n{width}\gr{dimen} $|$ \n{height}\gr{dimen} $|$
                        \n{depth}\gr{dimen}
     \end{disp}
If a rule dimension is specified twice, the second instance
takes precedence over the first. This makes it possible
to override the default dimensions. For instance,
after
\alt
\howto Change the default dimensions of rules\par
\begin{verbatim}
\let\xhrule\hrule  \def\hrule{\xhrule height .8pt}
\end{verbatim}
the macro \cs{hrule} gives a horizontal rule
of double the original height, and it is still possible
with
\begin{verbatim}
\hrule height 2pt
\end{verbatim}
to specify other heights.

%It is possible to specify all three dimensions; then
%\begin{verbatim}
%\vrule height1ex depth0pt width1ex
%\end{verbatim}
%and
%\begin{verbatim}
%\hrule height1ex depth0pt width1ex
%\end{verbatim}
%look the same.
%Still, each of them can be used only in the appropriate mode.
It is possible to specify all three dimensions; then
\begin{verbatim}
\vrule height1ex depth0pt width1ex
\end{verbatim}
and
\begin{verbatim}
\hrule height1ex depth0pt width1ex
\end{verbatim}
look the same.
Still, each of them can be used only in the appropriate mode.

%%\point Leaders
%\section{Leaders}
%\point Leaders
\section{Leaders}

%Rules are intimately connected to modes, which makes it easy
%to obtain some effects. For instance, a typical application
%of a vertical rule looks like
%\begin{verbatim}
%\hbox{\vrule width1pt\ Important text! \vrule width 1pt}
%\end{verbatim}
%which gives
%\begin{disp}\leavevmode\hbox{\vrule width 1pt\ Important text! 
%                      \vrule width 1pt}\end{disp}
%However, one might want to have a horizontal rule
%in horizontal mode for effects such as
%\begin{disp}\leavevmode
%\vbox{\hbox to 5cm{$\longleftarrow$\hfil 5cm\hfil$\longrightarrow$}
%    \hbox to 5cm{from here\leaders\hrule\hfil to there}}\end{disp}
%An \cs{hrule} can not be used in horizontal mode, and
%a vertical rule will not spread automatically.
Rules are intimately connected to modes, which makes it easy
to obtain some effects. For instance, a typical application
of a vertical rule looks like
\begin{verbatim}
\hbox{\vrule width1pt\ Important text! \vrule width 1pt}
\end{verbatim}
which gives
\begin{disp}\leavevmode\hbox{\vrule width 1pt\ Important text! 
                      \vrule width 1pt}\end{disp}
However, one might want to have a horizontal rule
in horizontal mode for effects such as
\begin{disp}\leavevmode
\vbox{\hbox to 5cm{$\longleftarrow$\hfil 5cm\hfil$\longrightarrow$}
    \hbox to 5cm{from here\leaders\hrule\hfil to there}}\end{disp}
An \cs{hrule} can not be used in horizontal mode, and
a vertical rule will not spread automatically.

%However, there is a way to use an \cs{hrule} command in
%horizontal mode and a \cs{vrule} in vertical mode,
%and that is with \indexterm{leaders}, so called because
%they lead your eye across the page. 
%A~leader command tells \TeX\
%to fill a~specified space, in whatever mode it is in,
%with as many copies of some box or rule specification
%as are needed. For instance, the above example
%was given as
%\begin{disp}\verb>\hbox to 5cm{from here\leaders\hrule\hfil to there}>\end{disp}
%that is, with an \cs{hrule} that was allowed to stretch along
%an \cs{hfil}.
%Note that the leader was given a horizontal skip,
%corresponding to the horizontal mode in which it appeared.
However, there is a way to use an \cs{hrule} command in
horizontal mode and a \cs{vrule} in vertical mode,
and that is with \indexterm{leaders}, so called because
they lead your eye across the page. 
A~leader command tells \TeX\
to fill a~specified space, in whatever mode it is in,
with as many copies of some box or rule specification
as are needed. For instance, the above example
was given as
\begin{disp}\verb>\hbox to 5cm{from here\leaders\hrule\hfil to there}>\end{disp}
that is, with an \cs{hrule} that was allowed to stretch along
an \cs{hfil}.
Note that the leader was given a horizontal skip,
corresponding to the horizontal mode in which it appeared.

%A general leader command looks like
%\begin{Disp} \gr{leaders}\gr{box or rule}%
%      \gr{vertical/horizontal/mathematical skip}\end{Disp}
%where \gr{leaders} is \cs{leaders}, \cs{cleaders}, 
%or~\cs{xleaders}, a \gr{box~or~rule}
%is a~\gr{box}, \cs{vrule}, or~\cs{hrule}, and the
%lists of horizontal and vertical skips appear in Chapter~\ref{hvmode};
%a~mathematical skip is either a horizontal skip or an~\cs{mskip}
%(see page~\pageref{muglue}).
%Leaders can thus be used in all three modes. Of course, the
%appropriate kind of skip must be specified. 
A general leader command looks like
\begin{Disp} \gr{leaders}\gr{box or rule}%
      \gr{vertical/horizontal/mathematical skip}\end{Disp}
where \gr{leaders} is \cs{leaders}, \cs{cleaders}, 
or~\cs{xleaders}, a \gr{box~or~rule}
is a~\gr{box}, \cs{vrule}, or~\cs{hrule}, and the
lists of horizontal and vertical skips appear in Chapter~\ref{hvmode};
a~mathematical skip is either a horizontal skip or an~\cs{mskip}
(see page~\pageref{muglue}).
Leaders can thus be used in all three modes. Of course, the
appropriate kind of skip must be specified. 

%A horizontal (vertical) box containing leaders has at least
%the height and depth (width) of the \gr{box~or~rule} used
%in the leaders, even if, as can happen in the case of box leaders,
%no actual leaders are placed.
A horizontal (vertical) box containing leaders has at least
the height and depth (width) of the \gr{box~or~rule} used
in the leaders, even if, as can happen in the case of box leaders,
no actual leaders are placed.

%%\spoint Rule leaders
%\subsection{Rule leaders}
%\spoint Rule leaders
\subsection{Rule leaders}

%\emph{Rule leaders}\index{leaders!rule}
%fill the specified amount of space with a rule
%\cstoidx leaders\par
%extending in the direction of the skip specified.
%The other dimensions of the resulting rule leader
%are determined by the sort of rule that is used:
%either dimensions can be specified explicitly, or
%the default values can be used.
\emph{Rule leaders}\index{leaders!rule}
fill the specified amount of space with a rule
\cstoidx leaders\par
extending in the direction of the skip specified.
The other dimensions of the resulting rule leader
are determined by the sort of rule that is used:
either dimensions can be specified explicitly, or
the default values can be used.

%For instance, 
%\begin{verbatim}
%\hbox{g\leaders\hrule\hskip20pt f}
%\end{verbatim}
%gives \begin{disp}\leavevmode\hbox{g\leaders\hrule\hskip20pt f}\end{disp}
%because a horizontal rule has a default height of~\n{.4pt}.
%On the other hand,
%\begin{verbatim}
%\hbox{g\leaders\vrule\hskip20pt f}
%\end{verbatim}
%gives \begin{disp}\leavevmode\hbox{g\leaders\vrule\hskip20pt f}\end{disp}
%because the height and depth of a vertical rule
%by default fill the surrounding box.
For instance, 
\begin{verbatim}
\hbox{g\leaders\hrule\hskip20pt f}
\end{verbatim}
gives \begin{disp}\leavevmode\hbox{g\leaders\hrule\hskip20pt f}\end{disp}
because a horizontal rule has a default height of~\n{.4pt}.
On the other hand,
\begin{verbatim}
\hbox{g\leaders\vrule\hskip20pt f}
\end{verbatim}
gives \begin{disp}\leavevmode\hbox{g\leaders\vrule\hskip20pt f}\end{disp}
because the height and depth of a vertical rule
by default fill the surrounding box.

%Spurious rule dimensions are ignored: in horizontal mode
%\begin{verbatim}
%\leaders\hrule width 10pt \hskip 20pt
%\end{verbatim}
%is equivalent to
%\begin{verbatim}
%\leaders\hrule \hskip 20pt
%\end{verbatim}
Spurious rule dimensions are ignored: in horizontal mode
\begin{verbatim}
\leaders\hrule width 10pt \hskip 20pt
\end{verbatim}
is equivalent to
\begin{verbatim}
\leaders\hrule \hskip 20pt
\end{verbatim}

%If the width or height-plus-depth
%of either the skip or the box is negative, 
%\TeX\ uses ordinary glue instead of leaders.
If the width or height-plus-depth
of either the skip or the box is negative, 
\TeX\ uses ordinary glue instead of leaders.

%%\spoint Box leaders
%\subsection{Box leaders}
%\spoint Box leaders
\subsection{Box leaders}

%Box leaders fill the available spaces with copies of
%a given box, instead of with a rule. 
Box leaders fill the available spaces with copies of
a given box, instead of with a rule. 

%\newbox\centerdot  \setbox\centerdot=\hbox{\hskip.7em.\hskip.7em}
\newbox\centerdot  \setbox\centerdot=\hbox{\hskip.7em.\hskip.7em}

%For all of the following examples, assume that a box register
%has been allocated:
%\begin{verbatim}
%\newbox\centerdot  \setbox\centerdot=\hbox{\hskip.7em.\hskip.7em}
%\end{verbatim}
%Now the output of
%\begin{verbatim}
%\hbox to 8cm {here\leaders\copy\centerdot\hfil there}
%\end{verbatim}
%is 
%\begin{disp}\leavevmode\hbox to 8cm {here\leaders\copy\centerdot\hfil there}
%\end{disp} That is, copies of the box register fill up the
%available space.
For all of the following examples, assume that a box register
has been allocated:
\begin{verbatim}
\newbox\centerdot  \setbox\centerdot=\hbox{\hskip.7em.\hskip.7em}
\end{verbatim}
Now the output of
\begin{verbatim}
\hbox to 8cm {here\leaders\copy\centerdot\hfil there}
\end{verbatim}
is 
\begin{disp}\leavevmode\hbox to 8cm {here\leaders\copy\centerdot\hfil there}
\end{disp} That is, copies of the box register fill up the
available space.

%Dot leaders, as in the above example, are often used for
%tables of contents. In such applications it is desirable that
%dots on subsequent lines are vertically aligned.
%The \cs{leaders} command does this automatically:
%\begin{verbatim}
%\hbox to 8cm {here\leaders\copy\centerdot\hfil there}
%\hbox to 8cm {over here\leaders\copy\centerdot\hfil over there}
%\end{verbatim}
%gives \begin{disp}\leavevmode
%\vtop{\hbox to 8cm {here\leaders\copy\centerdot\hfil there}
%\hbox to 8cm {over here\leaders\copy\centerdot\hfil over there\strut}}
%\end{disp}
%The mechanism behind this is the following:
%\TeX\ acts as if an infinite row of boxes starts (invisibly) at 
%the left edge of the surrounding box, 
%and the row of copies actually placed is 
%merely the part of this row that is not obscured by
%the other contents of the box.
Dot leaders, as in the above example, are often used for
tables of contents. In such applications it is desirable that
dots on subsequent lines are vertically aligned.
The \cs{leaders} command does this automatically:
\begin{verbatim}
\hbox to 8cm {here\leaders\copy\centerdot\hfil there}
\hbox to 8cm {over here\leaders\copy\centerdot\hfil over there}
\end{verbatim}
gives \begin{disp}\leavevmode
\vtop{\hbox to 8cm {here\leaders\copy\centerdot\hfil there}
\hbox to 8cm {over here\leaders\copy\centerdot\hfil over there\strut}}
\end{disp}
The mechanism behind this is the following:
\TeX\ acts as if an infinite row of boxes starts (invisibly) at 
the left edge of the surrounding box, 
and the row of copies actually placed is 
merely the part of this row that is not obscured by
the other contents of the box.

%Stated differently, box leaders are a window on an infinite
%row of boxes, and the row starts at the left edge of the
%surrounding box. Consider the following example:
%\begin{verbatim}
%\hbox to 8cm {\leaders\copy\centerdot\hfil}
%\hbox to 8cm {word\leaders\copy\centerdot\hfil}
%\end{verbatim}
%which gives
%\begin{disp}\leavevmode\vtop{\hbox to 8cm {\leaders\copy\centerdot\hfil}
%\hbox to 8cm {word\leaders\copy\centerdot\hfil\strut}}\end{disp}
%The row of leaders boxes becomes visible as soon as it
%does not coincide with other material.
Stated differently, box leaders are a window on an infinite
row of boxes, and the row starts at the left edge of the
surrounding box. Consider the following example:
\begin{verbatim}
\hbox to 8cm {\leaders\copy\centerdot\hfil}
\hbox to 8cm {word\leaders\copy\centerdot\hfil}
\end{verbatim}
which gives
\begin{disp}\leavevmode\vtop{\hbox to 8cm {\leaders\copy\centerdot\hfil}
\hbox to 8cm {word\leaders\copy\centerdot\hfil\strut}}\end{disp}
The row of leaders boxes becomes visible as soon as it
does not coincide with other material.

%The above discussion only talked about leaders in horizontal
%mode. Leaders can equally well be placed in vertical mode;
%for box leaders the `infinite row' then starts at the top
%of the surrounding box.
The above discussion only talked about leaders in horizontal
mode. Leaders can equally well be placed in vertical mode;
for box leaders the `infinite row' then starts at the top
of the surrounding box.

%%\spoint Evenly spaced leaders
%\subsection{Evenly spaced leaders}
%\spoint Evenly spaced leaders
\subsection{Evenly spaced leaders}

%Aligning subsequent box leaders in the way described above
%means that the white space before and after the
%leaders will in general be different.
%If vertical alignment is not
%an issue it may be aesthetically more pleasing to have
%the leaders evenly spaced.
%The \csidx{cleaders} command is like \cs{leaders},
%except that it splits excess space before and after the leaders
%into two equal parts, centring the row of boxes in the
%available space.
Aligning subsequent box leaders in the way described above
means that the white space before and after the
leaders will in general be different.
If vertical alignment is not
an issue it may be aesthetically more pleasing to have
the leaders evenly spaced.
The \csidx{cleaders} command is like \cs{leaders},
except that it splits excess space before and after the leaders
into two equal parts, centring the row of boxes in the
available space.

%\begin{example}\message{check verbatim indentation}
%\begin{verbatim}
%\hbox to 7.8cm {here\cleaders\copy\centerdot\hfil there}
%\hbox to 7.8cm {here is\cleaders\copy\centerdot\hfil there}
%\end{verbatim}
%gives \begin{disp}\leavevmode\vbox{
%\hbox to 7.8cm {here\cleaders\copy\centerdot\hfil there}
%\hbox to 7.8cm {here is\cleaders\copy\centerdot\hfil there\strut}
%}\end{disp}
%The `expanding leaders' \csidx{xleaders} spread excess space evenly
%between the boxes, with equal globs of glue before, after,
%and in between leader boxes.
%\end{example}
\begin{example}\message{check verbatim indentation}
\begin{verbatim}
\hbox to 7.8cm {here\cleaders\copy\centerdot\hfil there}
\hbox to 7.8cm {here is\cleaders\copy\centerdot\hfil there}
\end{verbatim}
gives \begin{disp}\leavevmode\vbox{
\hbox to 7.8cm {here\cleaders\copy\centerdot\hfil there}
\hbox to 7.8cm {here is\cleaders\copy\centerdot\hfil there\strut}
}\end{disp}
The `expanding leaders' \csidx{xleaders} spread excess space evenly
between the boxes, with equal globs of glue before, after,
and in between leader boxes.
\end{example}

%\begin{example}
%\begin{verbatim}
%\hbox to 7.8cm{here\hskip.7em
%      \xleaders\copy\centerdot\hfil  \hskip.7em there}
%\end{verbatim}
%gives \begin{disp}\leavevmode
%\hbox to 7.8cm {here\hskip.7em\xleaders\copy\centerdot\hfil\hskip.7em there}
%\end{disp} Note that the glue in the leader box is balanced here
%with explicit glue before and after the leaders;
%leaving out these glue items, as in
%\begin{verbatim}
%\hbox to 7.8cm {here\xleaders\copy\centerdot\hfil there}
%\end{verbatim}
%gives \begin{disp}\leavevmode
%\hbox to 7.8cm {here\xleaders\copy\centerdot\hfil there}
%\end{disp}
%which is clearly not what was intended.
%\end{example}
\begin{example}
\begin{verbatim}
\hbox to 7.8cm{here\hskip.7em
      \xleaders\copy\centerdot\hfil  \hskip.7em there}
\end{verbatim}
gives \begin{disp}\leavevmode
\hbox to 7.8cm {here\hskip.7em\xleaders\copy\centerdot\hfil\hskip.7em there}
\end{disp} Note that the glue in the leader box is balanced here
with explicit glue before and after the leaders;
leaving out these glue items, as in
\begin{verbatim}
\hbox to 7.8cm {here\xleaders\copy\centerdot\hfil there}
\end{verbatim}
gives \begin{disp}\leavevmode
\hbox to 7.8cm {here\xleaders\copy\centerdot\hfil there}
\end{disp}
which is clearly not what was intended.
\end{example}

%%\point Assorted remarks
%\section{Assorted remarks}
%\point Assorted remarks
\section{Assorted remarks}

%%\spoint Rules and modes
%\subsection{Rules and modes}
%\spoint Rules and modes
\subsection{Rules and modes}

%Above it was explained how rules can only occur in the 
%appropriate modes. Rules also influence mode-specific
%quantities:
%no baselineskip is added before rules in 
%vertical mode. In order to prevent glue after rules,
%\TeX\ sets \cs{prevdepth} to
%\n{\hbox{-}1000pt}
%(see Chapter~\ref{baseline}).
%Similarly the \cs{spacefactor} is set to 1000 after a \cs{vrule}
%in horizontal mode (see Chapter~\ref{line:break}).
Above it was explained how rules can only occur in the 
appropriate modes. Rules also influence mode-specific
quantities:
no baselineskip is added before rules in 
vertical mode. In order to prevent glue after rules,
\TeX\ sets \cs{prevdepth} to
\n{\hbox{-}1000pt}
(see Chapter~\ref{baseline}).
Similarly the \cs{spacefactor} is set to 1000 after a \cs{vrule}
in horizontal mode (see Chapter~\ref{line:break}).


%%\spoint[par:leaders:end] Ending a paragraph with leaders
%\subsection{Ending a paragraph with leaders}
%\label{par:leaders:end}
%\spoint[par:leaders:end] Ending a paragraph with leaders
\subsection{Ending a paragraph with leaders}
\label{par:leaders:end}

%An attempt to simulate an \cs{hrule} at the end of a paragraph by
%\howto End a paragraph with leaders\par
%\begin{verbatim}
%\nobreak\leaders\hrule\hfill\par
%\end{verbatim}
%does not work. The reason for this is that \TeX\
%performs an \cs{unskip} at the end of a paragraph,
%which removes the leaders. Normally this \cs{unskip} removes
%any space token inserted by the input processor after the
%last line. Remedy: stick an \verb.\hbox{}. at the end of
%the leaders.
An attempt to simulate an \cs{hrule} at the end of a paragraph by
\howto End a paragraph with leaders\par
\begin{verbatim}
\nobreak\leaders\hrule\hfill\par
\end{verbatim}
does not work. The reason for this is that \TeX\
performs an \cs{unskip} at the end of a paragraph,
which removes the leaders. Normally this \cs{unskip} removes
any space token inserted by the input processor after the
last line. Remedy: stick an \verb.\hbox{}. at the end of
the leaders.

%%\spoint Leaders and box registers
%\subsection{Leaders and box registers}
%\spoint Leaders and box registers
\subsection{Leaders and box registers}

%In the above examples the leader box was inserted with
%\cs{copy}. The output of
%\begin{verbatim}
%\hbox to 8cm {here\leaders\box\centerdot\hfil there}
%\hbox to 8cm {over here\leaders\box\centerdot\hfil 
%                   over there}
%\end{verbatim}
%is
%\begin{disp}\leavevmode
%     \vtop{\hbox to 8cm {here\leaders\box\centerdot\hfil there}
%           \hbox to 8cm {over here\leaders\box\centerdot\hfil over there}
%           }\end{disp}
%The box register is emptied after the first leader command,
%but more than one copy is placed in that first command.
In the above examples the leader box was inserted with
\cs{copy}. The output of
\begin{verbatim}
\hbox to 8cm {here\leaders\box\centerdot\hfil there}
\hbox to 8cm {over here\leaders\box\centerdot\hfil 
                   over there}
\end{verbatim}
is
\begin{disp}\leavevmode
     \vtop{\hbox to 8cm {here\leaders\box\centerdot\hfil there}
           \hbox to 8cm {over here\leaders\box\centerdot\hfil over there}
           }\end{disp}
The box register is emptied after the first leader command,
but more than one copy is placed in that first command.

%%\spoint Output in leader boxes
%\subsection{Output in leader boxes}
%\spoint Output in leader boxes
\subsection{Output in leader boxes}

%Any \cs{write}, \cs{openout}, or \cs{closeout} operation
%appearing in leader boxes is ignored. 
%Otherwise such an operation would be executed once for every
%copy of the box that would be shipped out.
Any \cs{write}, \cs{openout}, or \cs{closeout} operation
appearing in leader boxes is ignored. 
Otherwise such an operation would be executed once for every
copy of the box that would be shipped out.

%%\spoint Box leaders in trace output
%\subsection{Box leaders in trace output}
%\spoint Box leaders in trace output
\subsection{Box leaders in trace output}

%The dumped box representation obtained from,
%for instance, \cs{tracingoutput}
%does not write out box leaders in full: only the total size and
%one copy of the box used are dumped. In particular,
%the surrounding white space before and after the leaders
%is not indicated.
The dumped box representation obtained from,
for instance, \cs{tracingoutput}
does not write out box leaders in full: only the total size and
one copy of the box used are dumped. In particular,
the surrounding white space before and after the leaders
is not indicated.

%%\spoint Leaders and shifted margins
%\subsection{Leaders and shifted margins}
%\spoint Leaders and shifted margins
\subsection{Leaders and shifted margins}

%If margins have been shifted,
%leaders may look different
%depending on how the shift has been realized.
%For an illustration of how \cs{hangindent} and \cs{leftskip}
%influence the look of leaders, consider the following
%examples, where
%\begin{verbatim}
%\setbox0=\hbox{K o }
%\end{verbatim}
%The horizontal boxes above  the leaders
%\altt
%serve to indicate the starting point of the row of leaders.
If margins have been shifted,
leaders may look different
depending on how the shift has been realized.
For an illustration of how \cs{hangindent} and \cs{leftskip}
influence the look of leaders, consider the following
examples, where
\begin{verbatim}
\setbox0=\hbox{K o }
\end{verbatim}
The horizontal boxes above  the leaders
\altt
serve to indicate the starting point of the row of leaders.

%First
%\begin{verbatim}
%\hbox{\leaders\copy0\hskip5cm}
%\noindent\advance\leftskip 1em
%      \leaders\copy0\hskip5cm\hbox{}\par
%\end{verbatim}
%gives\message{examples on}
%\begin{disp}\leavevmode\vbox{\leftskip=0pt \hsize=7cm
%\setbox0=\hbox{K o }
%\hbox{\leaders\copy0\hskip5cm}
%\noindent\advance\leftskip 1em
%      \leaders\copy0\hskip5cm\hbox{}\par
%    }\end{disp}
%Then
%\begin{verbatim}
%\hbox{\kern1em\hbox{\leaders\copy0\hskip5cm}}
%\hangindent=1em \hangafter=-1 \noindent
%      \leaders\copy0\hskip5cm\hbox{}\par
%\end{verbatim}
%gives (note the shift with respect to the previous example)
%\begin{disp}\leavevmode\vbox{\leftskip=0pt \hsize=7cm
%\setbox0=\hbox{K o }
%\hbox{\kern1em\hbox{\leaders\copy0\hskip5cm}}
%\hangindent=1em \hangafter=-1 \noindent
%      \leaders\copy0\hskip5cm\hbox{}\par}\end{disp}
%\message{one page}
%In the first paragraph the \cs{leftskip} glue only obscures
%the first leader box; in the second paragraph the hanging
%indentation actually shifts the orientation point for the 
%row of leaders. Hanging indentation is performed in \TeX\
%by a \cs{moveright} of the boxes containing the lines
%of the paragraph.
First
\begin{verbatim}
\hbox{\leaders\copy0\hskip5cm}
\noindent\advance\leftskip 1em
      \leaders\copy0\hskip5cm\hbox{}\par
\end{verbatim}
gives\message{examples on}
\begin{disp}\leavevmode\vbox{\leftskip=0pt \hsize=7cm
\setbox0=\hbox{K o }
\hbox{\leaders\copy0\hskip5cm}
\noindent\advance\leftskip 1em
      \leaders\copy0\hskip5cm\hbox{}\par
    }\end{disp}
Then
\begin{verbatim}
\hbox{\kern1em\hbox{\leaders\copy0\hskip5cm}}
\hangindent=1em \hangafter=-1 \noindent
      \leaders\copy0\hskip5cm\hbox{}\par
\end{verbatim}
gives (note the shift with respect to the previous example)
\begin{disp}\leavevmode\vbox{\leftskip=0pt \hsize=7cm
\setbox0=\hbox{K o }
\hbox{\kern1em\hbox{\leaders\copy0\hskip5cm}}
\hangindent=1em \hangafter=-1 \noindent
      \leaders\copy0\hskip5cm\hbox{}\par}\end{disp}
\message{one page}
In the first paragraph the \cs{leftskip} glue only obscures
the first leader box; in the second paragraph the hanging
indentation actually shifts the orientation point for the 
row of leaders. Hanging indentation is performed in \TeX\
by a \cs{moveright} of the boxes containing the lines
of the paragraph.

%\endofchapter
%%%%% end of input file [rules]
\endofchapter
%%%% end of input file [rules]
