% -*- coding: utf-8 -*-
\documentclass{book}

% -*- coding: utf-8 -*-

\usepackage[b5paper,text={5in,8in},centering]{geometry}

\usepackage[CJKchecksingle]{xeCJK}
\setmainfont[Mapping=tex-text]{TeX Gyre Schola}
%\setsansfont{URW Gothic L Book}
%\setmonofont{Nimbus Mono L}
\setCJKmainfont[BoldFont=FandolHei,ItalicFont=FandolKai]{FandolSong}
\setCJKsansfont{FandolHei}
\setCJKmonofont{FandolFang}
\xeCJKsetup{PunctStyle = kaiming}

\linespread{1.25}
\setlength{\parindent}{2em}

\usepackage{xcolor}
\definecolor{myblue}{rgb}{0,0.2,0.6}

\usepackage{titlesec}
\titleformat{\chapter}
    {\normalfont\Huge\sffamily\color{myblue}}
    {第\thechapter 章}
    {1em}
    {}
%\titlespacing{\chapter}{0pt}{50pt}{40pt}
\titleformat{\section}
    {\normalfont\Large\sffamily\color{myblue}}
    {\thesection}
    {1em}
    {}
%\titlespacing{\section}{0pt}{3.5ex plus 1ex minus .2ex}{2.3ex plus .2ex}
\titleformat{\subsection}
    {\normalfont\large\sffamily\color{myblue}}
    {\thesubsection}
    {1em}
    {}
%\titlespacing{\subsection}{0pt}{3.25ex plus 1ex minus .2ex}{1.5ex plus .2ex}
%
\newpagestyle{special}[\small\sffamily]{
  \headrule
  \sethead[\usepage][][\chaptertitle]
  {\chaptertitle}{}{\usepage}}
\newpagestyle{main}[\small\sffamily]{
  \headrule
  \sethead[\usepage][][第\thechapter 章\quad\chaptertitle]
  {\thesection\quad\sectiontitle}{}{\usepage}}

\usepackage{titletoc}
%\setcounter{tocdepth}{1}
%\titlecontents{标题层次}[左间距]{上间距和整体格式}{标题序号}{标题内容}{指引线和页码}[下间距]
\titlecontents{chapter}[1.5em]{\vspace{.5em}\bfseries\sffamily}{\color{myblue}\contentslabel{1.5em}}{}
    {\titlerule*[20pt]{$\cdot$}\contentspage}[]
\titlecontents{section}[4.5em]{\sffamily}{\color{myblue}\contentslabel{3em}}{}
    {\titlerule*[20pt]{$\cdot$}\contentspage}[]
%\titlecontents{subsection}[8.5em]{\sffamily}{\contentslabel{4em}}{}
%    {\titlerule*[20pt]{$\cdot$}\contentspage}

\usepackage{enumitem}
\setlist{topsep=2pt,itemsep=2pt,parsep=1pt,leftmargin=\parindent}

\usepackage{fancyvrb}
\DefineVerbatimEnvironment{verbatim}{Verbatim}
  {xleftmargin=2em,baselinestretch=1,formatcom=\color{teal}\upshape}

\usepackage{etoolbox}
\makeatletter
\preto{\FV@ListVSpace}{\topsep=2pt \partopsep=0pt }
\makeatother

\usepackage[colorlinks,plainpages,pagebackref]{hyperref}
\hypersetup{
   pdfstartview={FitH},
   citecolor=teal,
   linkcolor=myblue,
   urlcolor=black,
   bookmarksnumbered
}

\usepackage{comment,makeidx,multicol}

%\usepackage{german}
%% german
%\righthyphenmin=3
%\mdqoff
%\captionsenglish
\usepackage[english]{babel}
{\catcode`"=13 \gdef"#1{\ifx#1"\discretionary{}{}{}\fi\relax}}
\def\mdqon{\catcode`"=13\relax}
\def\mdqoff{\catcode`"=12\relax}
\makeindex
\hyphenation{ex-em-pli-fies}

\newdimen\tempdima \newdimen\tempdimb

% these are fine
\def\handbreak{\\ \message{^^JManual break!!!!^^J}}
\def\nl{\protect\\}\def\n#1{{\tt #1}}
\def\cs#1{\texorpdfstring{{\tt\char`\\#1}}{\textbackslash#1}} %\def\cs#1{{\tt\char`\\#1}}
\let\csc\cs
\def\lb{{\tt\char`\{}}\def\rb{{\tt\char`\}}}
\def\gr#1{\texorpdfstring{$\langle$#1$\rangle$}{<#1>}} %\def\gr#1{$\langle$#1$\rangle$}
\def\key#1{{\tt#1}}
\def\alt{}\def\altt{}%this way in manstijl
\def\ldash{\unskip\ --\nobreak\ \ignorespaces}
\def\rdash{\unskip\nobreak\ --\ \ignorespaces}
% check these
\def\hex{{\tt"}}
\def\ascii{{\sc ascii}}
\def\ebcdic{{\sc ebcdic}}
\def\IniTeX{Ini\TeX}\def\LamsTeX{LAMS\TeX}\def\VirTeX{Vir\TeX}
\def\AmsTeX{Ams\TeX}
\def\TeXbook{the \TeX\ book}\def\web{{\sc web}}
% needs major thinking
\newenvironment{myquote}{\list{}{%
    \topsep=2pt \partopsep=0pt%
    \leftmargin=\parindent \rightmargin=\parindent
    }\item[]}{\endlist}
\newenvironment{disp}{\begin{myquote}}{\end{myquote}}
\newenvironment{Disp}{\begin{myquote}}{\end{myquote}}
\newenvironment{tdisp}{\begin{myquote}}{\end{myquote}}
\newenvironment{example}{\begin{myquote}\noindent\itshape 例子:}{\end{myquote}}
\newenvironment{inventory}{\begin{description}\raggedright}{\end{description}}
\newenvironment{glossinventory}{\begin{description}}{\end{description}}
\def\gram#1{\gr{#1}}%???
%
% index
%
\def\indexterm#1{\emph{#1}\index{#1}}
\def\indextermsub#1#2{\emph{#1 #2}\index{#1!#2}}
\def\indextermbus#1#2{\emph{#1 #2}\index{#2!#1}}
\def\term#1\par{\index{#1}}
\def\howto#1\par{}
\def\cstoidx#1\par{\index{#1@\cs{#1}@}}
\def\thecstoidx#1\par{\index{#1@\protect\csname #1\endcsname}}
\def\thecstoidxsub#1#2{\index{#1, #2@\protect\csname #1\endcsname, #2}\ignorespaces}
\def\csterm#1\par{\cstoidx #1\par\cs{#1}}
\def\csidx#1{\cstoidx #1\par\cs{#1}}

\def\tmc{\tracingmacros=2 \tracingcommands\tracingmacros}

%%%%%%%%%%%%%%%%%%%
\makeatletter
\def\snugbox{\hbox\bgroup\setbox\z@\vbox\bgroup
    \leftskip\z@
    \bgroup\aftergroup\make@snug
    \let\next=}
\def\make@snug{\par\sn@gify\egroup \box\z@\egroup}
\def\sn@gify
   {\skip\z@=\lastskip \unskip
    \advance\skip\z@\lastskip \unskip
    \unpenalty
    \setbox\z@\lastbox
    \ifvoid\z@ \nointerlineskip \else {\sn@gify} \fi
    \hbox{\unhbox\z@}\nointerlineskip
    \vskip\skip\z@
    }

\newdimen\fbh \fbh=60pt % dimension for easy scaling:
\newdimen\fbw \fbw=60pt % height and width of character box

\newdimen\dh \newdimen\dw % height and width of current character box
\newdimen\lh % height of previous character box
\newdimen\lw \lw=.4pt % line weight, instead of default .4pt

\def\hdotfill{\noindent
    \leaders\hbox{\vrule width 1pt height\lw
                  \kern4pt
                  \vrule width.5pt height\lw}\hfill\hbox{}
    \par}
\def\hlinefill{\noindent
    \leaders\hbox{\vrule width 5.5pt height\lw         }\hfill\hbox{}
    \par}
\def\stippel{$\qquad\qquad\qquad\qquad$}
\makeatother
%%%%%%%%%%%%%%%%%%%

%\def\SansSerif{\Typeface:macHelvetica }
%\def\SerifFont{\Typeface:macTimes }
%\def\SansSerif{\Typeface:bsGillSans }
%\def\SerifFont{\Typeface:bsBaskerville }
\let\SansSerif\relax \def\italic{\it}
\let\SerifFont\relax \def\MainFont{\rm}
\let\SansSerif\relax
\let\SerifFont\relax
\let\PopIndentLevel\relax \let\PushIndentLevel\relax
\let\ToVerso\relax \let\ToRecto\relax

%\def\stop@command@suffix{stop}
%\let\PopListLevel\PopIndentLevel
%\let\FlushRight\relax
%\let\flushright\FlushRight
%\let\SetListIndent\LevelIndent
%\def\awp{\ifhmode\vadjust{\penalty-10000 }\else
%    \penalty-10000 \fi}
\let\awp\relax
\let\PopIndentLevel\relax \let\PopListLevel\relax

\showboxdepth=-1

%\input figs
\def\endofchapter{\vfill\noindent}

\setcounter{chapter}{13}

\begin{document}

%\chapter{Token Lists}\label{token}
\chapter{Token Lists}\label{token}

%\TeX\ has only one type of data structure: the \indextermsub{token}{list}
%There are token list registers that are available to the user,
%and \TeX\ has some special token lists: the \cs{every...}
%variables, \cs{errhelp}, and \cs{output}.
\TeX\ has only one type of data structure: the \indextermsub{token}{list}
There are token list registers that are available to the user,
and \TeX\ has some special token lists: the \cs{every...}
variables, \cs{errhelp}, and \cs{output}.

%\label{cschap:toks}\label{cschap:toksdef}\label{cschap:newtoks}
%\begin{inventory}
%\item [\cs{toks}] 
%      Prefix for a token list register.
\label{cschap:toks}\label{cschap:toksdef}\label{cschap:newtoks}
\begin{inventory}
\item [\cs{toks}] 
      Prefix for a token list register.

%\item [\cs{toksdef}] 
%      Define a control sequence to be a synonym for
%      a~\cs{toks} register.
\item [\cs{toksdef}] 
      Define a control sequence to be a synonym for
      a~\cs{toks} register.

%\item [\cs{newtoks}] 
%      Macro that allocates a token list register.
\item [\cs{newtoks}] 
      Macro that allocates a token list register.

%\end{inventory}
\end{inventory}

%%\point Token lists
%\section{Token lists}
%\point Token lists
\section{Token lists}

%Token lists are the only type of data structure that \TeX\ knows.
%They can contain character tokens and control sequence tokens.
%Spaces in a token list are significant.
%The only operations on token lists are assignment and
%unpacking.
Token lists are the only type of data structure that \TeX\ knows.
They can contain character tokens and control sequence tokens.
Spaces in a token list are significant.
The only operations on token lists are assignment and
unpacking.

%\TeX\ has 256 token list registers \verb|\toks|$nnn$ that can be
%allocated using the macro \verb|\newtoks|, or explicitly
%assigned by \cs{toksdef}; see below.
\TeX\ has 256 token list registers \verb|\toks|$nnn$ that can be
allocated using the macro \verb|\newtoks|, or explicitly
assigned by \cs{toksdef}; see below.

%%\point Use of token lists
%\section{Use of token lists}
%\point Use of token lists
\section{Use of token lists}

%Token lists are  assigned by a \gr{variable assignment},
%which is in this case takes one of the forms
%\begin{disp}\gr{token variable}\gr{equals}\gr{general text}\nl
%     \gr{token variable}\gr{equals}\gr{filler}\gr{token variable}\end{disp}
%Here a \gr{token variable} is an explicit \cs{toks}$nnn$
%register, something that has been defined to such a register
%by \cs{toksdef} (probably hidden in \cs{newtoks}),
%or one of the special \gr{token parameter}
%lists below.
%A~\gr{general text} has an explicit closing brace, but the
%open brace can be implicit.
Token lists are  assigned by a \gr{variable assignment},
which is in this case takes one of the forms
\begin{disp}\gr{token variable}\gr{equals}\gr{general text}\nl
     \gr{token variable}\gr{equals}\gr{filler}\gr{token variable}\end{disp}
Here a \gr{token variable} is an explicit \cs{toks}$nnn$
register, something that has been defined to such a register
by \cs{toksdef} (probably hidden in \cs{newtoks}),
or one of the special \gr{token parameter}
lists below.
A~\gr{general text} has an explicit closing brace, but the
open brace can be implicit.

%Examples of token lists are (the first two lines are equivalent):
%\begin{verbatim}
%\toks0=\bgroup \a \b cd}
%\toks0={\a \b cd}
%\toks1=\toks2
%\end{verbatim}
Examples of token lists are (the first two lines are equivalent):
\begin{verbatim}
\toks0=\bgroup \a \b cd}
\toks0={\a \b cd}
\toks1=\toks2
\end{verbatim}

%Unpacking a token list is done by the command \cs{the}:
%the expansion of \cs{the}\gr{token variable} is the 
%sequence of tokens that was in the token list.
Unpacking a token list is done by the command \cs{the}:
the expansion of \cs{the}\gr{token variable} is the 
sequence of tokens that was in the token list.

%Token lists have a special behaviour in \cs{edef}:
%when prefixed by \verb|\the| they are unpacked, 
%but the resulting tokens
%are not evaluated further. Thus
%\begin{verbatim}
%\toks0={\a \b} \edef\SomeCs{\the\toks0}
%\end{verbatim}
%gives
%\begin{verbatim}
%\SomeCs: macro:-> \a \b
%\end{verbatim}
%This is in contrast to what happens ordinarily in an~\cs{edef};
%see page~\pageref{expand:edef}.
Token lists have a special behaviour in \cs{edef}:
when prefixed by \verb|\the| they are unpacked, 
but the resulting tokens
are not evaluated further. Thus
\begin{verbatim}
\toks0={\a \b} \edef\SomeCs{\the\toks0}
\end{verbatim}
gives
\begin{verbatim}
\SomeCs: macro:-> \a \b
\end{verbatim}
This is in contrast to what happens ordinarily in an~\cs{edef};
see page~\pageref{expand:edef}.


%%\point \gr{token parameter}
%\section{\gr{token parameter}}
%\point \gr{token parameter}
\section{\gr{token parameter}}

%There are in \TeX\ a number of token lists that are automatically
%inserted at certain points. These \gr{token parameter}s are
%the following:
%\begin{description} \item [\cs{output}]
%   this token list is inserted
%   whenever \TeX\ decides it has sufficient material for a page,
%   or when the user forces activation by a penalty~$\leq-10\,000$
%   in vertical mode
%   (see Chapter~\ref{output});
%\item [\cs{everypar}]
%   is inserted when \TeX\ switches from external or internal
%   vertical mode to unrestricted horizontal mode 
%   (see Chapter~\ref{par:start});
%\item [\cs{everymath}]
%   is inserted after a single math-shift character that starts
%   a formula;
%\item [\cs{everydisplay}]
%   is inserted after a double math-shift character that starts
%   a display formula;
%\item [\cs{everyhbox}]
%   is inserted when an \cs{hbox} begins (see Chapter~\ref{boxes});
%\item [\cs{everyvbox}]
%   is inserted when a vertical box begins (see Chapter~\ref{boxes});
%\item [\cs{everyjob}]
%   is inserted when a job begins (see Chapter~\ref{run});
%\item [\cs{everycr}]
%   is inserted in alignments after \cs{cr} or a non-redundant
%   \cs{crcr} (see Chapter~\ref{align});
%\item [\cs{errhelp}]
%   contains tokens to supplement an \cs{errmessage}
%    (see Chapter~\ref{error}).
%\end{description}
There are in \TeX\ a number of token lists that are automatically
inserted at certain points. These \gr{token parameter}s are
the following:
\begin{description} \item [\cs{output}]
   this token list is inserted
   whenever \TeX\ decides it has sufficient material for a page,
   or when the user forces activation by a penalty~$\leq-10\,000$
   in vertical mode
   (see Chapter~\ref{output});
\item [\cs{everypar}]
   is inserted when \TeX\ switches from external or internal
   vertical mode to unrestricted horizontal mode 
   (see Chapter~\ref{par:start});
\item [\cs{everymath}]
   is inserted after a single math-shift character that starts
   a formula;
\item [\cs{everydisplay}]
   is inserted after a double math-shift character that starts
   a display formula;
\item [\cs{everyhbox}]
   is inserted when an \cs{hbox} begins (see Chapter~\ref{boxes});
\item [\cs{everyvbox}]
   is inserted when a vertical box begins (see Chapter~\ref{boxes});
\item [\cs{everyjob}]
   is inserted when a job begins (see Chapter~\ref{run});
\item [\cs{everycr}]
   is inserted in alignments after \cs{cr} or a non-redundant
   \cs{crcr} (see Chapter~\ref{align});
\item [\cs{errhelp}]
   contains tokens to supplement an \cs{errmessage}
    (see Chapter~\ref{error}).
\end{description}

%A \gr{token parameter} behaves the same as an explicit \cs{toks}$nnn$
%list, or a quantity defined by \cs{toksdef}.
A \gr{token parameter} behaves the same as an explicit \cs{toks}$nnn$
list, or a quantity defined by \cs{toksdef}.

%%\point Token list registers
%\section{Token list registers}
%\point Token list registers
\section{Token list registers}

%Token lists can be stored in \csidx{toks} registers:
%\begin{Disp}\cs{toks}\gr{8-bit number}\end{Disp}
%which is a \gr{token variable}.
%Synonyms for token list registers can be made by the \gr{registerdef}
%command \csidx{toksdef} in a \gr{shorthand definition}:
%\begin{Disp}\cs{toksdef}\gr{control sequence}\gr{equals}\gr{8-bit number}
%\end{Disp} A control sequence defined this way is called
%a \gr{toksdef token}, and this is also a token variable
%(the remaining third kind of token variable is
%the \gr{token parameter}).
Token lists can be stored in \csidx{toks} registers:
\begin{Disp}\cs{toks}\gr{8-bit number}\end{Disp}
which is a \gr{token variable}.
Synonyms for token list registers can be made by the \gr{registerdef}
command \csidx{toksdef} in a \gr{shorthand definition}:
\begin{Disp}\cs{toksdef}\gr{control sequence}\gr{equals}\gr{8-bit number}
\end{Disp} A control sequence defined this way is called
a \gr{toksdef token}, and this is also a token variable
(the remaining third kind of token variable is
the \gr{token parameter}).

%The plain \TeX\ macro \csidx{newtoks} uses \cs{toksdef} to 
%allocate unused token list registers. This macro is \cs{outer}.
The plain \TeX\ macro \csidx{newtoks} uses \cs{toksdef} to 
allocate unused token list registers. This macro is \cs{outer}.

%\section{Examples}
\section{Examples}

%Token lists are probably among the least obvious components
%of \TeX: most \TeX\ users will never find occasion  for their use,
%but format designers and other macro writers
%can find interesting applications.
%Following are some examples of the sorts of things that can be
%done with token lists.
Token lists are probably among the least obvious components
of \TeX: most \TeX\ users will never find occasion  for their use,
but format designers and other macro writers
can find interesting applications.
Following are some examples of the sorts of things that can be
done with token lists.

%%\spoint Operations on token lists: stack macros
%\subsection{Operations on token lists: stack macros}
%\spoint Operations on token lists: stack macros
\subsection{Operations on token lists: stack macros}

%The number of primitive operations available for token lists is
%\howto Stack macros\par
%rather limited: assignment and unpacking. However, these are
%sufficient to implement other operations such as appending.
The number of primitive operations available for token lists is
\howto Stack macros\par
rather limited: assignment and unpacking. However, these are
sufficient to implement other operations such as appending.

%Let us say we have allocated a token register
%\begin{verbatim}
%\newtoks\list \list={\c} 
%\end{verbatim}
%and we want to add tokens to it,
%\alt
%using the syntax
%\begin{verbatim}
%\Prepend \a \b (to:)\list
%\end{verbatim}
%such that
%\begin{verbatim}
%\showthe\list
%\end{verbatim}
%gives
%\begin{verbatim}
%> \a \b \c .
%\end{verbatim}
%For this the original list has to be unpacked, and 
%\alt
%the new tokens followed by the old contents have to assigned
%again to the register. Unpacking can be done with \cs{the}
%inside an \cs{edef}, so we arrive at the following macro:
%\begin{verbatim}
%\def\Prepend#1(to:)#2{\toks0={#1}%
%    \edef\act{\noexpand#2={\the\toks0 \the#2}}%
%    \act}
%\end{verbatim}
%Note that the tokens that are to be added are first packed
%\alt
%into a temporary token list, which is then again unpacked
%inside the \cs{edef}. Including them directly would have
%led to their expansion.
Let us say we have allocated a token register
\begin{verbatim}
\newtoks\list \list={\c} 
\end{verbatim}
and we want to add tokens to it,
\alt
using the syntax
\begin{verbatim}
\Prepend \a \b (to:)\list
\end{verbatim}
such that
\begin{verbatim}
\showthe\list
\end{verbatim}
gives
\begin{verbatim}
> \a \b \c .
\end{verbatim}
For this the original list has to be unpacked, and 
\alt
the new tokens followed by the old contents have to assigned
again to the register. Unpacking can be done with \cs{the}
inside an \cs{edef}, so we arrive at the following macro:
\begin{verbatim}
\def\Prepend#1(to:)#2{\toks0={#1}%
    \edef\act{\noexpand#2={\the\toks0 \the#2}}%
    \act}
\end{verbatim}
Note that the tokens that are to be added are first packed
\alt
into a temporary token list, which is then again unpacked
inside the \cs{edef}. Including them directly would have
led to their expansion.

%Next we want to use token lists as a sort of stack:
%we want a `pop' operation that removes the first element
%from the list. Specifically,
%\begin{verbatim}
%\Pop\list(into:)\first
%\show\first \showthe\list
%\end{verbatim}
%should give
%\begin{verbatim}
%> \first=macro:
%->\a .
%\end{verbatim}
%and for the remaining list
%\begin{verbatim}
% 
%> \b \c .
%\end{verbatim}
Next we want to use token lists as a sort of stack:
we want a `pop' operation that removes the first element
from the list. Specifically,
\begin{verbatim}
\Pop\list(into:)\first
\show\first \showthe\list
\end{verbatim}
should give
\begin{verbatim}
> \first=macro:
->\a .
\end{verbatim}
and for the remaining list
\begin{verbatim}
 
> \b \c .
\end{verbatim}

%Here we make creative use of delimited and undelimited
%parameters. With an \cs{edef} we unpack the list,
%and the auxiliary macro \cs{SplitOff} scoops up the elements
%as one undelimited argument, the first element, and one
%delimited argument, the rest of the elements.
%\begin{verbatim}
%\def\Pop#1(into:)#2{%
%    \edef\act{\noexpand\SplitOff\the#1%
%              (head:)\noexpand#2(tail:)\noexpand#1}%
%    \act}
%\def\SplitOff#1#2(head:)#3(tail:)#4{\def#3{#1}#4={#2}}
%\end{verbatim}
Here we make creative use of delimited and undelimited
parameters. With an \cs{edef} we unpack the list,
and the auxiliary macro \cs{SplitOff} scoops up the elements
as one undelimited argument, the first element, and one
delimited argument, the rest of the elements.
\begin{verbatim}
\def\Pop#1(into:)#2{%
    \edef\act{\noexpand\SplitOff\the#1%
              (head:)\noexpand#2(tail:)\noexpand#1}%
    \act}
\def\SplitOff#1#2(head:)#3(tail:)#4{\def#3{#1}#4={#2}}
\end{verbatim}

%\subsection{Executing token lists}
\subsection{Executing token lists}

%The \cs{the} operation for unpacking token lists was used above
%only inside an \cs{edef}. Used on its own it has the effect
%of feeding the tokens of the list to \TeX's expansion mechanism.
%If the tokens have been added to the list in a uniform syntax,
%this gives rise to some interesting possibilities.
The \cs{the} operation for unpacking token lists was used above
only inside an \cs{edef}. Used on its own it has the effect
of feeding the tokens of the list to \TeX's expansion mechanism.
If the tokens have been added to the list in a uniform syntax,
this gives rise to some interesting possibilities.

%Imagine that we are implementing the bookkeeping of external
%files for a format. Such external files can be used for
%table of contents, list of figures, et cetera.
%If the presence
%of such objects is under the control of the user, we need some
%general routines for opening and closing files, and keeping
%track of what files we have opened at the user's request.
Imagine that we are implementing the bookkeeping of external
files for a format. Such external files can be used for
table of contents, list of figures, et cetera.
If the presence
of such objects is under the control of the user, we need some
general routines for opening and closing files, and keeping
track of what files we have opened at the user's request.

%Here only  some routines for bookkeeping will be described.
%Let us say there is a list of auxiliary files, and an auxiliary
%counter:
%\begin{verbatim}
%\newtoks\auxlist \newcount\auxcount
%\end{verbatim}
%First of all there must be an operation to add auxiliary files:
%\begin{verbatim}
%\def\NewAuxFile#1{\AddToAuxList{#1}%
%    % plus other actions
%    }
%\def\AddToAuxList#1{\let\\=\relax
%    \edef\act{\noexpand\auxlist={\the\auxlist \\{#1}}}%
%    \act}
%\end{verbatim}
%This adds the name to the list in a uniform format:
%\begin{verbatim}
%\NewAuxFile{toc} \NewAuxFile{lof}
%\showthe\auxlist
%> \\{toc}\\{lof}.
%\end{verbatim}
%using the control sequence \verb>\\> which is left undefined.
Here only  some routines for bookkeeping will be described.
Let us say there is a list of auxiliary files, and an auxiliary
counter:
\begin{verbatim}
\newtoks\auxlist \newcount\auxcount
\end{verbatim}
First of all there must be an operation to add auxiliary files:
\begin{verbatim}
\def\NewAuxFile#1{\AddToAuxList{#1}%
    % plus other actions
    }
\def\AddToAuxList#1{\let\\=\relax
    \edef\act{\noexpand\auxlist={\the\auxlist \\{#1}}}%
    \act}
\end{verbatim}
This adds the name to the list in a uniform format:
\begin{verbatim}
\NewAuxFile{toc} \NewAuxFile{lof}
\showthe\auxlist
> \\{toc}\\{lof}.
\end{verbatim}
using the control sequence \verb>\\> which is left undefined.

%Now this control sequence can be used for instance to
%count the number of elements in the list:
%\begin{verbatim}
%\def\ComputeLengthOfAuxList{\auxcount=0 
%    \def\\##1{\advance\auxcount1\relax}%
%    \the\auxlist}
%\ComputeLengthOfAuxList \showthe\auxcount
%> 2.
%\end{verbatim}
%Another use of this structure is the following:
%at the end of the job we can now close all auxiliary
%files at once, by
%\begin{verbatim}
%\def\CloseAuxFiles{\def\\##1{\CloseAuxFile{##1}}%
%    \the\auxlist}
%\def\CloseAuxFile#1{\message{closing file: #1. }%
%    % plus other actions
%    }
%\CloseAuxFiles
%\end{verbatim}
%which gives the output
%\begin{verbatim}
%closing file: toc.  closing file: lof. 
%\end{verbatim}
Now this control sequence can be used for instance to
count the number of elements in the list:
\begin{verbatim}
\def\ComputeLengthOfAuxList{\auxcount=0 
    \def\\##1{\advance\auxcount1\relax}%
    \the\auxlist}
\ComputeLengthOfAuxList \showthe\auxcount
> 2.
\end{verbatim}
Another use of this structure is the following:
at the end of the job we can now close all auxiliary
files at once, by
\begin{verbatim}
\def\CloseAuxFiles{\def\\##1{\CloseAuxFile{##1}}%
    \the\auxlist}
\def\CloseAuxFile#1{\message{closing file: #1. }%
    % plus other actions
    }
\CloseAuxFiles
\end{verbatim}
which gives the output
\begin{verbatim}
closing file: toc.  closing file: lof. 
\end{verbatim}

%% 
%%%% \begin{comment}
%%%% 
%%%% % %\spoint Dynamic macro definition
%%%% % \subsection{Dynamic macro definition}
%%%% 
%%%% % Unpacking token lists inside an \cs{edef} can be put to a
%%%% % rather ambitious use: dynamic definition of macros.
%%%% % Consider a simple example.
%%%% % \altt
%%%% % We set ourselves the goal of letting
%%%% % the user define macros, without ever having to use \cs{def}.
%%%% % The syntax for this could look like
%%%% \begin{verbatim}
%%%% % \startdefinition
%%%% % \do:this
%%%% % \do:that
%%%% % \define:MyMacro
%%%% % \end{verbatim}
%%%% % such that \verb>\show\MyMacro> gives
%%%% \begin{verbatim}
%%%% % > \MyMacro=macro:
%%%% % ->\this \that .
%%%% % \end{verbatim}
%%%% % An implementation of this uses a token list to collect
%%%% % the commands that the user specifies:
%%%% \begin{verbatim}
%%%% % \newtoks\actionlist
%%%% % \end{verbatim}
%%%% % The first command is easy:
%%%% \begin{verbatim}
%%%% % \def\startdefinition{\actionlist{}}
%%%% % \end{verbatim}
%%%% % Now the \cs{do} command has to hang control sequences
%%%% % in the \cs{actionlist}:
%%%% \begin{verbatim}
%%%% % \def\do:#1 {%
%%%% %     \edef\act{\noexpand\appendaction
%%%% %               \expandafter\noexpand\csname#1\endcsname}%
%%%% %     \act}
%%%% % \end{verbatim}
%%%% % The \cs{edef} is used solely to form the actual control sequence.
%%%% % The next macro uses \cs{edef} to unpack the \cs{actionlist} so far:
%%%% %
%%%% \begin{verbatim}
%%%% % \def\appendaction#1{%
%%%% %     \edef\act{\noexpand\actionlist=
%%%% %                  {\the\actionlist \noexpand#1}}%
%%%% %     \act}
%%%% % \end{verbatim}
%%%% % Finally, definition of the user macro also needs an \cs{edef}.
%%%% % Some \cs{expandafter} trickery is necessary here to form
%%%% % the control sequence of the user macro:
%%%% \begin{verbatim}
%%%% % \def\define:#1 {%
%%%% %     \expandafter\edef\csname#1\endcsname{\the\actionlist}}
%%%% % \end{verbatim}
%%%% 
%%%% % Of course, this is a very simple, rather pointless, example.
%%%% % However, it illustrates an important principle of how
%%%% % token lists can be used to implement another syntax level
%%%% % in \TeX\ (see~\cite{EL}). This principle underlies the
%%%% % \indexterm{Lollipop}
%%%% % `Lollipop' format that was used to typeset this book.
%%%% 
%%%% % \end{comment}
%%%% %%%% end of input file [token]
% 
%%% \begin{comment}
%%% 
%%% % %\spoint Dynamic macro definition
%%% % \subsection{Dynamic macro definition}
%%% 
%%% % Unpacking token lists inside an \cs{edef} can be put to a
%%% % rather ambitious use: dynamic definition of macros.
%%% % Consider a simple example.
%%% % \altt
%%% % We set ourselves the goal of letting
%%% % the user define macros, without ever having to use \cs{def}.
%%% % The syntax for this could look like
%%% \begin{verbatim}
%%% % \startdefinition
%%% % \do:this
%%% % \do:that
%%% % \define:MyMacro
%%% % \end{verbatim}
%%% % such that \verb>\show\MyMacro> gives
%%% \begin{verbatim}
%%% % > \MyMacro=macro:
%%% % ->\this \that .
%%% % \end{verbatim}
%%% % An implementation of this uses a token list to collect
%%% % the commands that the user specifies:
%%% \begin{verbatim}
%%% % \newtoks\actionlist
%%% % \end{verbatim}
%%% % The first command is easy:
%%% \begin{verbatim}
%%% % \def\startdefinition{\actionlist{}}
%%% % \end{verbatim}
%%% % Now the \cs{do} command has to hang control sequences
%%% % in the \cs{actionlist}:
%%% \begin{verbatim}
%%% % \def\do:#1 {%
%%% %     \edef\act{\noexpand\appendaction
%%% %               \expandafter\noexpand\csname#1\endcsname}%
%%% %     \act}
%%% % \end{verbatim}
%%% % The \cs{edef} is used solely to form the actual control sequence.
%%% % The next macro uses \cs{edef} to unpack the \cs{actionlist} so far:
%%% %
%%% \begin{verbatim}
%%% % \def\appendaction#1{%
%%% %     \edef\act{\noexpand\actionlist=
%%% %                  {\the\actionlist \noexpand#1}}%
%%% %     \act}
%%% % \end{verbatim}
%%% % Finally, definition of the user macro also needs an \cs{edef}.
%%% % Some \cs{expandafter} trickery is necessary here to form
%%% % the control sequence of the user macro:
%%% \begin{verbatim}
%%% % \def\define:#1 {%
%%% %     \expandafter\edef\csname#1\endcsname{\the\actionlist}}
%%% % \end{verbatim}
%%% 
%%% % Of course, this is a very simple, rather pointless, example.
%%% % However, it illustrates an important principle of how
%%% % token lists can be used to implement another syntax level
%%% % in \TeX\ (see~\cite{EL}). This principle underlies the
%%% % \indexterm{Lollipop}
%%% % `Lollipop' format that was used to typeset this book.
%%% 
%%% % \end{comment}
%%% %%%% end of input file [token]

%\endofchapter
\endofchapter

\end{document}
